% Options for packages loaded elsewhere
\PassOptionsToPackage{unicode}{hyperref}
\PassOptionsToPackage{hyphens}{url}
\PassOptionsToPackage{dvipsnames,svgnames*,x11names*}{xcolor}
%
\documentclass[]{article}
\usepackage{lmodern}
\usepackage{amssymb,amsmath}
\usepackage{ifxetex,ifluatex}
\ifnum 0\ifxetex 1\fi\ifluatex 1\fi=0 % if pdftex
  \usepackage[T1]{fontenc}
  \usepackage[utf8]{inputenc}
  \usepackage{textcomp} % provide euro and other symbols
\else % if luatex or xetex
  \usepackage{unicode-math}
  \defaultfontfeatures{Scale=MatchLowercase}
  \defaultfontfeatures[\rmfamily]{Ligatures=TeX,Scale=1}
\fi
% Use upquote if available, for straight quotes in verbatim environments
\IfFileExists{upquote.sty}{\usepackage{upquote}}{}
\IfFileExists{microtype.sty}{% use microtype if available
  \usepackage[]{microtype}
  \UseMicrotypeSet[protrusion]{basicmath} % disable protrusion for tt fonts
}{}
\makeatletter
\@ifundefined{KOMAClassName}{% if non-KOMA class
  \IfFileExists{parskip.sty}{%
    \usepackage{parskip}
  }{% else
    \setlength{\parindent}{0pt}
    \setlength{\parskip}{6pt plus 2pt minus 1pt}}
}{% if KOMA class
  \KOMAoptions{parskip=half}}
\makeatother
\usepackage{xcolor}\pagecolor[RGB]{28,30,38} \color[RGB]{213,216,218}
\IfFileExists{xurl.sty}{\usepackage{xurl}}{} % add URL line breaks if available
\IfFileExists{bookmark.sty}{\usepackage{bookmark}}{\usepackage{hyperref}}
\hypersetup{
  pdftitle={Fourier Analysis and the Theory of Distributions},
  pdfauthor={Kexing Ying},
  colorlinks=true,
  linkcolor=Maroon,
  filecolor=Maroon,
  citecolor=Blue,
  urlcolor=red,
  pdfcreator={LaTeX via pandoc}}
\urlstyle{same} % disable monospaced font for URLs
\usepackage[margin = 1.5in]{geometry}
\usepackage{graphicx}
\makeatletter
\def\maxwidth{\ifdim\Gin@nat@width>\linewidth\linewidth\else\Gin@nat@width\fi}
\def\maxheight{\ifdim\Gin@nat@height>\textheight\textheight\else\Gin@nat@height\fi}
\makeatother
% Scale images if necessary, so that they will not overflow the page
% margins by default, and it is still possible to overwrite the defaults
% using explicit options in \includegraphics[width, height, ...]{}
\setkeys{Gin}{width=\maxwidth,height=\maxheight,keepaspectratio}
% Set default figure placement to htbp
\makeatletter
\def\fps@figure{htbp}
\makeatother
\setlength{\emergencystretch}{3em} % prevent overfull lines
\providecommand{\tightlist}{%
  \setlength{\itemsep}{0pt}\setlength{\parskip}{0pt}}
\setcounter{secnumdepth}{5}
\usepackage{tikz}
\usepackage{physics}
\usepackage{amsthm}
\usepackage{mathtools}
\usepackage{esint}
\usepackage[ruled,vlined]{algorithm2e}
\theoremstyle{definition}
\newtheorem{theorem}{Theorem}
\newtheorem{definition*}{Definition}
\newtheorem{prop}{Proposition}
\newtheorem{corollary}{Corollary}[theorem]
\newtheorem*{remark}{Remark}
\theoremstyle{definition}
\newtheorem{definition}{Definition}[section]
\newtheorem{lemma}{Lemma}[section]
\newtheorem{proposition}{Proposition}[section]
\newtheorem{example}{Example}[section]
\newcommand{\diag}{\mathop{\mathrm{diag}}}
\newcommand{\Arg}{\mathop{\mathrm{Arg}}}
\newcommand{\hess}{\mathop{\mathrm{Hess}}}

\title{Fourier Analysis and the Theory of Distributions}
\author{Kexing Ying}

\begin{document}
\maketitle

{
\hypersetup{linkcolor=}
\setcounter{tocdepth}{2}
\tableofcontents
}
\newpage

\section{Orthonormal Systems}

We will in this section recall some results about orthonormal systems in Euclidean 
spaces\footnote{In this course, we shall call real inner product spaces Euclidean 
spaces.} and generalize them to complex spaces. 

\begin{definition}
  A system of nonzero vectors \(\{X_\alpha\} \subseteq R\) where \(R\) is an 
  Euclidean space is called orthogonal if \(\langle X_\alpha, X_\beta\rangle = 0\)
  for all \(\alpha \neq \beta\). 

  In addition, if for all \(\alpha\), \(\langle X_\alpha, X_\alpha\rangle = 1\), 
  we say the system is orthonormal.
\end{definition}

Clearly, given an orthogonal system \(\{X_\alpha\}\), we may normalize the vector 
such that \(\{X_\alpha / \|X_\alpha\|\}\) is an orthonormal system. Furthermore, 
recall that a system of orthogonal vectors is linearly independent.

\begin{definition}
  A complete (i.e. the smallest closed subspace containing the system is \(R\)) 
  orthogonal system \(\{X_\alpha\} \subseteq R\) is said to 
  be an orthogonal basis of \(R\). 
\end{definition}

Some important spaces we shall study in this course include \(\mathbb{R}^2\) 
(equipped with the Euclidean norm), \(l_2\), \(\mathcal{C}([-\pi , \pi])\) 
(the space of continuous functions on \([-\pi, \pi]\) equipped with the \(L_2\) norm). 

\begin{proposition}
  Let \(R\) be a separable Euclidean space. Then any orthogonal system in \(R\) 
  is countable. 
\end{proposition}
\begin{proof}
  By normalizing, we may assume the system \(\{X_\alpha\}\) is orthonormal. Then, 
  for \(\alpha \neq \beta\),
  \[\|X_\alpha - X_\beta\|^2 = \|X_\alpha\|^2 - 2\langle X_\alpha, X_\beta\rangle + 
  \|X_\beta\|^2 = \|X_\alpha\|^2 + \|X_\beta\|^2 = 2.\]
  Then, \(B_{1 / 2}(X_\alpha) \cap B_{1 / 2}(X_\beta) = \varnothing\) for all 
  \(\alpha \neq \beta\). Thus, if the system is not countable, we have found 
  a uncountable number of disjoint open balls, contradicting the separability of 
  \(R\).
\end{proof}

\begin{proposition}
  Let \(f_1, f_2, \cdots\) be a linearly independent system in a Euclidean space 
  \(R\). Then, there exists an orthonormal system \(\phi_1, \phi_2, \cdots\) such 
  that 
  \[\phi_n = a_{n_1} f_1 + \cdots + a_{n_n} f_n\]
  and 
  \[f_n = b_{n_1}\phi_1 + \cdots + b_{n_n} \phi_n\]
  for some \(a_{n_k}, b_{n_k} \in \mathbb{R}\) and \(a_{n_n}, b_{n_n} \neq 0\).
  Furthermore, the system \(\phi_1, \phi_2, \cdots\) is uniquely determined up 
  to a multiplication by \(\pm 1\).
\end{proposition}
\begin{proof}
  Use Gram-Schmidt. 
\end{proof}

\begin{corollary}
  A separable Euclidean space \(R\) possesses an orthonormal basis.
\end{corollary}
\begin{proof}
  Simply obtain the orthonormal system corresponding to the countable dense 
  system of \(R\). The resulting system is complete since the two systems have the 
  same linear closure.
\end{proof}

\begin{definition}[Fourier Coefficients]
  Let \(\phi_1, \phi_2, \cdots\) be an orthonormal system in \(R\) and let \(f \in R\). 
  Consider the sequence \(c_k = \langle f, \phi_k \rangle\) for all \(k = 1, 2, \cdots\). 
  Then \(c_k\) are called the coordinates or Fourier coefficients of \(f\) with respect 
  to the system \(\{\phi_k\}\) and \(\sum_{k = 1}^\infty c_k \phi_k\) is 
  called the Fourier series of \(f\).
  
  Note that this series in the definition is a formal series as we do not yet 
  know whether or not the series converges.
\end{definition}

In the finite case, it is not difficult to see that the sequence 
\(\alpha_k\) for \(k = 1, \cdots, n\) which minimizes \(\|f - S_n^{(\alpha)}\|\) 
where \(S_n^{(\alpha)} := \sum_{k = 1}^n \alpha_k \phi_k\) is the Fourier coefficients. 
Indeed, we have 
\[\begin{split}
  \|f - S_n^{(\alpha)}\|^2 & = \langle f, f \rangle - 2 \langle f, S_n^{(\alpha)} \rangle +
    \langle S_n^{(\alpha)}, S_n^{(\alpha)} \rangle\\
    & = \|f\|^2 - 2 \sum \alpha_k c_k + \sum \alpha_k^2\\
    & = \|f\|^2 - \sum c_k^2 + \sum (\alpha_k - c_k)^2.
\end{split}\]
Hence, \(\|f - S_n^{(\alpha)}\|\) is minimized when \(\alpha_k = c_k\) for all 
\(k = 1, \cdots, n\). With this in mind, choosing \(\alpha\) to be the Fourier 
coefficients, we have 
\[\|f - S_n^{(c)}\| = \|f\|^2 - \sum_{k = 1}^n c_k^2.\]
Geometrically, \(f - S_n^{(\alpha)}\) is orthogonal to the subspace generated by 
\(\phi_1, \cdots, \phi_n\) if and only if \(\alpha = c\).

Furthermore, by noting \(0 \le \|f - S_n^{(c)}\| = \|f\|^2 - \sum_{k = 1}^n c_k^2\), 
we have 
\[\sum_{k = 1}^n c_k^2 \le \|f\|^2 < \infty,\]
and hence, taking \(n \to \infty\), we have \(\sum_{k = 1}^\infty c_k^2\) exists 
and is bounded above by \(\|f\|^2\). This inequality is known as the Bessel inequality.

\begin{definition}[Closed Orthonormal System]
  The orthonormal system \(\{\phi_k\}\) is closed if for any \(f \in R\), we have 
  \[\sum_{k = 1}^\infty c_k^2 = \|f\|^2.\]
  This property is called the Parseval equality.
\end{definition}

Again, by observing \(\|f - S_n^{(c)}\| = \|f\|^2 - \sum_{k = 1}^n c_k^2\),
the system is closed if and only if for any \(f\), the partial sums of the 
Fourier series converge to \(f\), i.e. \(f = \sum_{k = 1}^\infty c_k \phi_k\).

\begin{proposition}
  In a separable Euclidean space \(R\), an orthonormal system is complete 
  if and only if it is closed.
\end{proposition}
\begin{proof}
  Suppose first that \(\{\phi_k\}\) is closed. Then, for all \(f \in R\), 
  \(f = \sum_{k = 1}^\infty c_k \phi_k\). Thus, the finite linear combinations of 
  \(\{\phi_k\}\) is dense in \(R\) and thus, \(\{\phi_k\}\) is complete.

  On the other hand, suppose that \(\{\phi_k\}\) is complete (it is countable 
  as \(R\) is separable), for any \(f \in R\), there exists some \(\alpha^k\) 
  such that \(\|f - S^{(\alpha^k)}_\infty\| \to 0\). As we have seen, for any partial 
  sum \(S^{(\alpha^k)}_n\), we have \(\|f - S^{(c)}_n\| \le \|f - S^{(\alpha^k)}_n\|\) 
  and so, 
  \[\|f - S^{(c)}_\infty\| \le \|f - S^{(\alpha^k)}_\infty\| \to 0\]
  implying \(\|f - S^{(c)}_\infty\| = 0\) and the system is closed.
\end{proof}


\begin{proposition}
  Given \(f, g \in R\) and a closed orthonormal system \(\{\phi_k\}\), 
  \[\langle f, g \rangle = \sum_{k = 1}^\infty c_k d_k\]
  where \((c_k), (d_k)\) are the Fourier coefficients of \(f\) and \(g\) with respect 
  to \(\{\phi_k\}\) respectively.
\end{proposition}
\begin{proof}
  We have, by Parseval's identity, \(\|f\|^2 = \sum c_k^2\), \(\|g\|^2 = \sum d_k^2\) 
  and \(\|f + g\|^2 = \sum (c_k + d_k)^2 = \sum c_k^2 + 2 \sum c_k d_k + \sum d_k^2\),
  we have 
  \[\sum c_k^2 + 2 \sum c_k d_k + \sum d_k^2 = \|f + g\|^2 = \|f\|^2 + 2\langle f, g\rangle + \|g\|^2.\]
  Thus, cancelling using \(\|f\|^2 = \sum c_k^2\) and \(\|g\|^2 = \sum d_k^2\), we have 
  \(\langle f, g \rangle = \sum_{k = 1}^\infty c_k d_k\) as required.
\end{proof}

In the case the system is only orthogonal but not necessary orthonormal, we may 
normalize the Fourier coefficients, i.e. given an orthogonal system \(\{\phi_k\}\), 
we have \(\{\phi / \|\phi_k\|\}\) is an orthonormal system, and so, we define
\[c_k = \left\langle f, \frac{\phi_k}{\|\phi_k\|} \right\rangle = \frac{1}{\|\phi_k\|} \langle f, \phi_k \rangle.\]
Similarly, the Fourier series of \(f\) is becomes 
\[\sum_{k = 1}^\infty c_k \frac{\phi_k}{\|\phi_k\|} = \sum \frac{\langle f, \phi_k\rangle}{\|\phi_k\|^2} \phi_k.\]
Substituting this definition of the Fourier coefficients into the Bessel inequality,
we obtain 
\[\sum_{k = 1}^\infty \frac{\langle f, \phi_k\rangle^2}{\|\phi_k\|^2} \le \|f\|^2,\]
for any orthogonal system \(\{\phi_k\}\).

\begin{theorem}[Riesz]
  Let \(\{\phi_k\}\) be a orthonormal system in a complete Euclidean space \(R\) 
  (i.e. a real Hilbert space) and let \(c \in \ell_2\) 
  (i.e. \(\sum_{k = 1}^\infty c_k^2 < \infty\)). Then, 
  there exists some \(f \in R\) such that \(c_k = \langle f, \phi_k \rangle\) and 
  Parseval's identity holds, i.e.
  \[\sum_{k = 1}^\infty c_k^2 = \|f\|^2.\]
\end{theorem}
\begin{proof}
  Let \(f_n := \sum_{k = 1}^n c_k \phi_k\). Then, by definition, we have 
  \(c_k = \langle f_n, \phi_k \rangle\) for all \(k = 1, \cdots, n\). Then, 
  for all \(p \ge 1\), we have
  \[\|f_{n + p} - f_n\|^2 = \|c_{n + 1} \phi_{n + 1} + \cdots + c_{n + p} \phi_{n + p}\|^2 
    = \sum_{k = n + 1}^{n + p} c_k^2.\]
  Now, as \(\sum c_k^2 < \infty\), we have \(\{f_n\}\) is Cauchy, and thus, as 
  \(R\) is complete, there exists some \(f \in R\) such that \(f_n \to f\). 
  Thus, by noting, 
  \[\langle f, \phi_k \rangle = \langle f_n \phi_k\rangle + \langle f - f_n, \phi_k\rangle
    = c_k + \langle f - f_n, \phi_k\rangle,\]
  where \(\langle f - f_n, \phi_k\rangle \to 0\) as \(n \to \infty\) since 
  \(|\langle f - f_n, \phi_k\rangle| \le \|f - f_n\| \|\phi_k\|\) by the 
  Cauchy-Schwarz inequality, we have \(c_k = \langle f, \phi_k\rangle\).

  Finally, Parseval's identity, follows as \(\|\cdot\|^2\) is continuous in a 
  normed space.
\end{proof}

Let us recall the following result from functional analysis.

\begin{proposition}
  Any separable Hilbert space is isomorphic to \(\ell_2\) (thus, any two separable 
  Hilbert spaces are isomorphic). 
\end{proposition}
\begin{proof}
  Let \(H\) be a separable Hilbert space and choose \(\{\phi_k\}\) a complete 
  orthonormal system (which exists as \(H\) is separable). Then, for any \(f \in H\), 
  we map \(f\) to the sequence corresponding to its Fourier coefficients, i.e.
  \[\psi : f \mapsto (c_1, c_2, \cdots)\]
  which is well-defined by Bessel's inequality. On the other hand, by Riesz's 
  theorem, for any \(x \in \ell_2\), \(\sum x_k^2 < \infty\) and so, there 
  exists a unique \(f \in H\),  such that \(\psi(f) = x\). Thus, as \(\psi\) is 
  clearly linear (as the inner products are linear with respect to the left 
  component), we have the isomorphism between \(H\) and \(\ell_2\).
\end{proof}

\end{document}