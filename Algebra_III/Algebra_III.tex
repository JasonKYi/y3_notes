% Options for packages loaded elsewhere
\PassOptionsToPackage{unicode}{hyperref}
\PassOptionsToPackage{hyphens}{url}
\PassOptionsToPackage{dvipsnames,svgnames*,x11names*}{xcolor}
%
\documentclass[]{article}
\usepackage{lmodern}
\usepackage{amssymb,amsmath}
\usepackage{ifxetex,ifluatex}
\ifnum 0\ifxetex 1\fi\ifluatex 1\fi=0 % if pdftex
  \usepackage[T1]{fontenc}
  \usepackage[utf8]{inputenc}
  \usepackage{textcomp} % provide euro and other symbols
\else % if luatex or xetex
  \usepackage{unicode-math}
  \defaultfontfeatures{Scale=MatchLowercase}
  \defaultfontfeatures[\rmfamily]{Ligatures=TeX,Scale=1}
\fi
% Use upquote if available, for straight quotes in verbatim environments
\IfFileExists{upquote.sty}{\usepackage{upquote}}{}
\IfFileExists{microtype.sty}{% use microtype if available
  \usepackage[]{microtype}
  \UseMicrotypeSet[protrusion]{basicmath} % disable protrusion for tt fonts
}{}
\makeatletter
\@ifundefined{KOMAClassName}{% if non-KOMA class
  \IfFileExists{parskip.sty}{%
    \usepackage{parskip}
  }{% else
    \setlength{\parindent}{0pt}
    \setlength{\parskip}{6pt plus 2pt minus 1pt}}
}{% if KOMA class
  \KOMAoptions{parskip=half}}
\makeatother
\usepackage{xcolor}
\IfFileExists{xurl.sty}{\usepackage{xurl}}{} % add URL line breaks if available
\IfFileExists{bookmark.sty}{\usepackage{bookmark}}{\usepackage{hyperref}}
\hypersetup{
  pdftitle={Algebra III},
  pdfauthor={Kexing Ying},
  colorlinks=true,
  linkcolor=Maroon,
  filecolor=Maroon,
  citecolor=Blue,
  urlcolor=red,
  pdfcreator={LaTeX via pandoc}}
\urlstyle{same} % disable monospaced font for URLs
\usepackage[margin = 1.5in]{geometry}
\usepackage{graphicx}
\makeatletter
\def\maxwidth{\ifdim\Gin@nat@width>\linewidth\linewidth\else\Gin@nat@width\fi}
\def\maxheight{\ifdim\Gin@nat@height>\textheight\textheight\else\Gin@nat@height\fi}
\makeatother
% Scale images if necessary, so that they will not overflow the page
% margins by default, and it is still possible to overwrite the defaults
% using explicit options in \includegraphics[width, height, ...]{}
\setkeys{Gin}{width=\maxwidth,height=\maxheight,keepaspectratio}
% Set default figure placement to htbp
\makeatletter
\def\fps@figure{htbp}
\makeatother
\setlength{\emergencystretch}{3em} % prevent overfull lines
\providecommand{\tightlist}{%
  \setlength{\itemsep}{0pt}\setlength{\parskip}{0pt}}
\setcounter{secnumdepth}{5}
\usepackage{tikz}
\usepackage{physics}
\usepackage{amsthm}
\usepackage{mathtools}
\usepackage{esint}
\usepackage[ruled,vlined]{algorithm2e}
\theoremstyle{definition}
\newtheorem{theorem}{Theorem}
\newtheorem{definition*}{Definition}
\newtheorem{prop}{Proposition}
\newtheorem{corollary}{Corollary}[theorem]
\newtheorem*{remark}{Remark}
\theoremstyle{definition}
\newtheorem{definition}{Definition}[section]
\newtheorem{lemma}{Lemma}[section]
\newtheorem{proposition}{Proposition}[section]
\newtheorem{example}{Example}[section]
\newcommand{\diag}{\mathop{\mathrm{diag}}}
\newcommand{\Arg}{\mathop{\mathrm{Arg}}}
\newcommand{\hess}{\mathop{\mathrm{Hess}}}

\title{Algebra III}
\author{Kexing Ying}
\date{July 24, 2021}

\begin{document}
\maketitle

{
\hypersetup{linkcolor=}
\setcounter{tocdepth}{2}
\tableofcontents
}
\newpage

\section{Definitions}

We will in this section recall and introduce some fundamental definitions which 
we will study throughout the course.

\begin{definition}[Ring]
  A ring \(R\) is a set together with two distinct elements \(0_R, 1_R\), and 
  two binary operations \(+_R, \times_R : R^2 \to R\) such that 
  \begin{itemize}
    \item \((R, +_R)\) is an additive abelian group with identity \(0_R\);
    \item \((R, \times_R)\) is a multiplicative abelian monoid with identity \(1_R\);
    \item \(\times_R\) distributes over \(+_R\), i.e. for all \(r, s, t \in R\), 
      \[(r +_R s) \times_R t = r \times_R t +_R s \times_R t,\] and 
      \[r \times_R (s +_R t) = r \times_R s +_R r \times_R t.\]
  \end{itemize}
\end{definition}

We note that there is some ambiguity in the literature in the definition of a 
ring, and in particular, some might call the definition above as a commutative 
unital ring. We will in this course mostly consider ourselves with this definition, 
though we might later consider non-commutative rings.

\begin{definition}[Field]
  A field \(F\) is a ring is for all \(f \in F \setminus \{0_F\}\), there 
  exists some \(f^{-1} \in F\) such that \(f \times_F f^{-1} = 1_F\).
\end{definition}

We will simply drop the subscript from the operations and the elements from 
these definitions whenever there is no confusion. 

Recall that one method of constructing a ring from another is the polynomial 
ring. Let \(R\) be ring, then a polynomial on \(X\) is a sum 
\[\sum_{n = 0}^\infty a_n X^n\]
for some \((a_n)_{n \in \mathbb{N}} \subseteq R\) where all but finitely 
many \(a_i\) are zero. We say \(P(X) = \sum_{n = 0}^\infty a_n X^n\) has 
degree \(d\) if \(d\) is the largest number such that \(a_d \neq 0\).

\begin{definition}[Polynomial Ring]
  Given a ring \(R\), the polynomial ring \(R[X]\) is the set of polynomials 
  equipped with the operations \(+_{R[X]}\) and \(\times_{R[X]}\) such that 
  \[\sum_{n = 0}^\infty a_n X^n +_{R[X]} \sum_{n = 0}^\infty b_n X^n = 
    \sum_{n = 0}^\infty (a_n + b_n) X^n,\]
  and,
  \[\sum_{n = 0}^\infty a_n X^n \times_{R[X]} \sum_{n = 0}^\infty b_n X^n =
    \sum_{n = 0}^\infty \left(\sum_{i = 0}^n a_i b_{n - i} \right) X^n.\]
\end{definition}

It is not difficult to see that the ring axioms are satisfied and in fact, it 
is possible to construct polynomial rings with infinite degrees, though this 
shall not be considered in this course. An equivalent way of considering elements 
of polynomial rings is to see them as sequences with finite non-zero elements. 

One may adjoin a polynomial ring with another variable, that is 
\(R[X][Y]\) and by writing out the elements, we see that \(R[X][Y] \cong R[Y][X]\) 
and we may instead write \(R[X, Y]\) with no ambiguity. 

\subsection{Subrings and Extensions}

\begin{definition}[Subring]
  A subring of the ring \(R\) is a subset of \(R\) containing \(0, 1\) and is 
  closed under \(+\) and \(\times\).
\end{definition}

It is clear that a subring of a ring is a ring itself with the inherited 
operations. 

\begin{proposition}
  If \(S, T\) are subrings of the ring \(R\), then so is \(S \cap T\).
\end{proposition}

\begin{definition}
  Given a subring \(S\) of \(R\), \(S[\alpha]\) for some \(\alpha \in R\) is the 
  subset of \(R\) consisting of all elements of \(R\) that can be expressed as 
  \(r_0 + r_1 \alpha + \cdots + r_n \alpha^n\) for \(r_i \in S\) and 
  \(n \in \mathbb{N}\). We call this process the adjoining of \(S\) with \(\alpha\).
\end{definition}

Clearly \(S[\alpha]\) contains \(0\) and \(1\) (as \(S \subseteq S[\alpha]\)) 
and is closed under \(+\) and \(\times\), and thus, is a subring of \(R\).

An important example of the above construction is the following. Consider 
\(\mathbb{Z} \subseteq \mathbb{C}\), we have \(\mathbb{Z}[i]\) constructed 
through the definition above is known as the Gaussian integers is a subring 
of \(\mathbb{C}\) consisting of all elements of the form \(a + bi\) for 
\(a, b \in \mathbb{Z}\). To see this, consider if \(X^2 - r X - s\) is a polynomial 
of integer coefficients with complex root \(\alpha \notin \mathbb{Z}\), then, 
we may consider \(\mathbb{Z}[\alpha]\). As \(\alpha^2 - r \alpha - s = 0\), 
we obtain \(\alpha^2 = r \alpha + s\) and thus, for all 
\(r_0 + r_1 \alpha + \cdots + r_n \alpha^n \in \mathbb{Z}[\alpha]\),
\[\begin{split}
  r_0 + r_1 \alpha + r_2 \alpha^2 + \cdots + r_n \alpha^n 
  & = r_0 + r_1 \alpha + r_2 (r \alpha + s) + \cdots \\
  & = (r_0 + r_2 s + \cdots) + (r_1 + r_2 r + \cdots) \alpha.
\end{split}\]
Hence, all elements of \(\mathbb{Z}[\alpha]\) are of the form \(a + b \alpha\) 
for \(a, b \in \mathbb{Z}\).

On the other hand, if we consider \(\mathbb{Z}[\pi] \subseteq \mathbb{C}\), 
as \(\pi\) is not an algebraic number, for all \(P(X) \in \mathbb{Z}[X] \setminus \{0\}\), 
\(P(\pi) \neq 0\). Thus, if \(P(X), Q(X)\) are polynomials such that
\(P(\pi) = r_0 + r_1 \pi + \cdots + r_n \pi^n = s_0 + s_1 \pi + \cdots + s_m \pi^m = Q(\pi)\), 
WLOG. \(n \le m\) we have \(0 = (s_0 - r_0) + (s_1 - r_1) \pi + \cdots + 
(s_n - r_n) \pi^n + s_{n + 1} \pi^{n + 1} + \cdots + s_m \pi^{m + 1}\), implying 
\(s_i = r_i\) for all \(i = 1, \cdots, n\) and \(s_i = 0\) for \(i > n\), we have 
\(P = Q\). Hence, \(\mathbb{Z}[\pi] \cong \mathbb{Z}[X]\).

\begin{proposition}
  If \(R\) is a subring of \(S\), then \(R[\alpha]\) for some \(\alpha \in S\) 
  is the intersection of all subrings of \(S\) containing \(R \cup \{\alpha\}\).
\end{proposition}
\begin{proof}
  Since \(R[\alpha]\) contains both \(R\) and \(\alpha\), we have 
  \[\bigcap \{U \mid R \cup \{\alpha\} \subseteq U \le S\} \subseteq R[\alpha].\]
  On the other hand, for all subrings \(U\) containing \(R \cup \{\alpha\}\), 
  \(R[\alpha] \subseteq U\) as \(U\) is closed under \(+\) and \(\times\). 
  Thus, 
  \[\bigcap \{U \mid R \cup \{\alpha\} \subseteq U \le S\} = R[\alpha].\]
\end{proof}

\begin{definition}[Integral Domain]
  A ring \(R\) is an integral domain if for all \(r, s \in R\), \(rs = 0\) 
  implies \(r = 0\) or \(s = 0\).
\end{definition}

In particular, we say \(r \in R\) is a zero divisor if there exists a 
\(s \in R \setminus \{0\}\) such that \(rs = 0\). Thus, an integral domain 
is simply a ring with no zero divisors. 

\begin{definition}[Field of Fractions]
  For \(R\) an integral domain, then the field of fractions of \(R\) denoted 
  \(\text{Frac}(R)\), is \(R^2\) quotiented by the equivalence class 
  \[(a, b) \sim (r, s) \iff as = br.\]
  We write \(a / b\) as a representative of the equivalence class \([a, b]\).
\end{definition}

We may equip the field of fractions of \(R\) with addition and multiplication 
such that for \(a / b, r / s \in \text{Frac}(R)\) 
\[\frac{a}{b} + \frac{r}{s} = \frac{ad + bc}{bd} \text{ and } 
  \frac{a}{b} \times \frac{r}{s} = \frac{ar}{bs}.\]
It is routine to check these operations are well-defined and that the ring 
axioms are satisfied. Furthermore, as the name suggests, \(\text{Frac}(R)\) 
is a field and for all \(a / b \neq 0\), \((a / b) \times (b / a) = 1\).

\begin{definition}[Multiplicative System]
  A set \(S \subseteq R\) is a multiplicative system if \(1 \in S, 0 \notin S\) 
  and is closed under multiplication.
\end{definition}

\begin{definition}
  Let \(R\) be a ring and \(S \subseteq R\) be a multiplicative system. Then 
  \(S^{-1}R\) is \(R \times S\) quotiented by the equivalence class 
  \[(a, b) \sim (r, s) \iff as = br\]
  for \(a, r \in R, b, s \in S\).
\end{definition}

Similarly, we may equip \(S^{-1}R\) with addition and multiplication such that 
\(S^{-1}R\) is a subring of \(\text{Frac}(R)\). 

It is possible to use this construction on rings which are not integral domains, 
though in that case, the equivalence class is more subtle as division by a 
zero divisor will introduces other elements into the subring. This will be explored 
later in this course.

\end{document}
