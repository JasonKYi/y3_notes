% Options for packages loaded elsewhere
\PassOptionsToPackage{unicode}{hyperref}
\PassOptionsToPackage{hyphens}{url}
\PassOptionsToPackage{dvipsnames,svgnames*,x11names*}{xcolor}
%
\documentclass[]{article}
\usepackage{lmodern}
\usepackage{amssymb,amsmath}
\usepackage{ifxetex,ifluatex}
\ifnum 0\ifxetex 1\fi\ifluatex 1\fi=0 % if pdftex
  \usepackage[T1]{fontenc}
  \usepackage[utf8]{inputenc}
  \usepackage{textcomp} % provide euro and other symbols
\else % if luatex or xetex
  \usepackage{unicode-math}
  \defaultfontfeatures{Scale=MatchLowercase}
  \defaultfontfeatures[\rmfamily]{Ligatures=TeX,Scale=1}
\fi
% Use upquote if available, for straight quotes in verbatim environments
\IfFileExists{upquote.sty}{\usepackage{upquote}}{}
\IfFileExists{microtype.sty}{% use microtype if available
  \usepackage[]{microtype}
  \UseMicrotypeSet[protrusion]{basicmath} % disable protrusion for tt fonts
}{}
\makeatletter
\@ifundefined{KOMAClassName}{% if non-KOMA class
  \IfFileExists{parskip.sty}{%
    \usepackage{parskip}
  }{% else
    \setlength{\parindent}{0pt}
    \setlength{\parskip}{6pt plus 2pt minus 1pt}}
}{% if KOMA class
  \KOMAoptions{parskip=half}}
\makeatother
\usepackage{xcolor}\pagecolor[RGB]{28,30,38} \color[RGB]{213,216,218}
\IfFileExists{xurl.sty}{\usepackage{xurl}}{} % add URL line breaks if available
\IfFileExists{bookmark.sty}{\usepackage{bookmark}}{\usepackage{hyperref}}
\hypersetup{
  pdftitle={Algebra III},
  pdfauthor={Kexing Ying},
  colorlinks=true,
  linkcolor=Maroon,
  filecolor=Maroon,
  citecolor=Blue,
  urlcolor=red,
  pdfcreator={LaTeX via pandoc}}
\urlstyle{same} % disable monospaced font for URLs
\usepackage[margin = 1.5in]{geometry}
\usepackage{graphicx}
\makeatletter
\def\maxwidth{\ifdim\Gin@nat@width>\linewidth\linewidth\else\Gin@nat@width\fi}
\def\maxheight{\ifdim\Gin@nat@height>\textheight\textheight\else\Gin@nat@height\fi}
\makeatother
% Scale images if necessary, so that they will not overflow the page
% margins by default, and it is still possible to overwrite the defaults
% using explicit options in \includegraphics[width, height, ...]{}
\setkeys{Gin}{width=\maxwidth,height=\maxheight,keepaspectratio}
% Set default figure placement to htbp
\makeatletter
\def\fps@figure{htbp}
\makeatother
\setlength{\emergencystretch}{3em} % prevent overfull lines
\providecommand{\tightlist}{%
  \setlength{\itemsep}{0pt}\setlength{\parskip}{0pt}}
\setcounter{secnumdepth}{5}
\usepackage{tikz}
\usepackage{physics}
\usepackage{amsthm}
\usepackage{mathtools}
\usepackage{esint}
\usepackage[ruled,vlined]{algorithm2e}
\usepackage{tikz-cd}
\theoremstyle{definition}
\newtheorem{theorem}{Theorem}
\newtheorem{definition*}{Definition}
\newtheorem{prop}{Proposition}
\newtheorem{corollary}{Corollary}[theorem]
\newtheorem*{remark}{Remark}
\theoremstyle{definition}
\newtheorem{definition}{Definition}[section]
\newtheorem{lemma}{Lemma}[section]
\newtheorem{proposition}{Proposition}[section]
\newtheorem{example}{Example}[section]
\newcommand{\diag}{\mathop{\mathrm{diag}}}
\newcommand{\Arg}{\mathop{\mathrm{Arg}}}
\newcommand{\hess}{\mathop{\mathrm{Hess}}}
% the redefinition for the missing \setminus must be delayed
\AtBeginDocument{\renewcommand{\setminus}{\mathbin{\backslash}}}

\title{Algebra III}
\author{Kexing Ying}
\date{July 24, 2021}

\begin{document}
\maketitle

\begin{center}
  \begin{minipage}{.75\textwidth}
    \textbf{N.B.} this course has large overlap with the second year course 
    \textit{Groups and Rings} in particular, the ring subsection. Thus, 
    most revisited proofs are simply omitted or replaced with a hint.
  \end{minipage}
\end{center}

{
\hypersetup{linkcolor=}
\setcounter{tocdepth}{2}
\tableofcontents
}
\newpage

\section{Rings}

We will in this section recall some fundamental definitions about rings which 
we will study throughout the course.

\begin{definition}[Ring]
  A ring \(R\) is a set together with two distinct elements \(0_R, 1_R\), and 
  two binary operations \(+_R, \times_R : R^2 \to R\) such that 
  \begin{itemize}
    \item \((R, +_R)\) is an additive abelian group with identity \(0_R\);
    \item \((R, \times_R)\) is a multiplicative abelian monoid with identity \(1_R\);
    \item \(\times_R\) distributes over \(+_R\), i.e. for all \(r, s, t \in R\), 
      \[(r +_R s) \times_R t = r \times_R t +_R s \times_R t,\] and 
      \[r \times_R (s +_R t) = r \times_R s +_R r \times_R t.\]
  \end{itemize}
\end{definition}

We note that there is some ambiguity in the literature in the definition of a 
ring, and in particular, some might call the definition above as a commutative 
unital ring. We will in this course mostly consider ourselves with this definition, 
though we might later consider non-commutative rings.

\begin{definition}[Field]
  A field \(F\) is a ring is for all \(f \in F \setminus \{0_F\}\), there 
  exists some \(f^{-1} \in F\) such that \(f \times_F f^{-1} = 1_F\).
\end{definition}

We will simply drop the subscript from the operations and the elements from 
these definitions whenever there is no confusion. 

Recall that one method of constructing a ring from another is the polynomial 
ring. Let \(R\) be ring, then a polynomial on \(X\) is a sum 
\[\sum_{n = 0}^\infty a_n X^n\]
for some \((a_n)_{n \in \mathbb{N}} \subseteq R\) where all but finitely 
many \(a_i\) are zero. We say \(P(X) = \sum_{n = 0}^\infty a_n X^n\) has 
degree \(d\) if \(d\) is the largest number such that \(a_d \neq 0\).

\begin{definition}[Polynomial Ring]
  Given a ring \(R\), the polynomial ring \(R[X]\) is the set of polynomials 
  equipped with the operations \(+_{R[X]}\) and \(\times_{R[X]}\) such that 
  \[\sum_{n = 0}^\infty a_n X^n +_{R[X]} \sum_{n = 0}^\infty b_n X^n = 
    \sum_{n = 0}^\infty (a_n + b_n) X^n,\]
  and,
  \[\sum_{n = 0}^\infty a_n X^n \times_{R[X]} \sum_{n = 0}^\infty b_n X^n =
    \sum_{n = 0}^\infty \left(\sum_{i = 0}^n a_i b_{n - i} \right) X^n.\]
\end{definition}

It is not difficult to see that the ring axioms are satisfied and in fact, it 
is possible to construct polynomial rings with infinite degrees, though this 
shall not be considered in this course. An equivalent way of considering elements 
of polynomial rings is to see them as sequences with finite non-zero elements. 

One may adjoin a polynomial ring with another variable, that is 
\(R[X][Y]\) and by writing out the elements, we see that \(R[X][Y] \cong R[Y][X]\) 
and we may instead write \(R[X, Y]\) with no ambiguity. 

\subsection{Subrings and Extensions}

\begin{definition}[Subring]
  A subring of the ring \(R\) is a subset of \(R\) containing \(0, 1\) and is 
  closed under \(+\) and \(\times\).
\end{definition}

It is clear that a subring of a ring is a ring itself with the inherited 
operations. 

\begin{proposition}
  If \(S, T\) are subrings of the ring \(R\), then so is \(S \cap T\).
\end{proposition}

\begin{definition}
  Given a subring \(S\) of \(R\), \(S[\alpha]\) for some \(\alpha \in R\) is the 
  subset of \(R\) consisting of all elements of \(R\) that can be expressed as 
  \(r_0 + r_1 \alpha + \cdots + r_n \alpha^n\) for \(r_i \in S\) and 
  \(n \in \mathbb{N}\). We call this process the adjoining of \(S\) with \(\alpha\).
\end{definition}

Clearly \(S[\alpha]\) contains \(0\) and \(1\) (as \(S \subseteq S[\alpha]\)) 
and is closed under \(+\) and \(\times\), and thus, is a subring of \(R\).

An important example of the above construction is the following. Consider 
\(\mathbb{Z} \subseteq \mathbb{C}\), we have \(\mathbb{Z}[i]\) constructed 
through the definition above is known as the Gaussian integers is a subring 
of \(\mathbb{C}\) consisting of all elements of the form \(a + bi\) for 
\(a, b \in \mathbb{Z}\). To see this, consider if \(X^2 - r X - s\) is a polynomial 
of integer coefficients with complex root \(\alpha \notin \mathbb{Z}\), then, 
we may consider \(\mathbb{Z}[\alpha]\). As \(\alpha^2 - r \alpha - s = 0\), 
we obtain \(\alpha^2 = r \alpha + s\) and thus, for all 
\(r_0 + r_1 \alpha + \cdots + r_n \alpha^n \in \mathbb{Z}[\alpha]\),
\[\begin{split}
  r_0 + r_1 \alpha + r_2 \alpha^2 + \cdots + r_n \alpha^n 
  & = r_0 + r_1 \alpha + r_2 (r \alpha + s) + \cdots \\
  & = (r_0 + r_2 s + \cdots) + (r_1 + r_2 r + \cdots) \alpha.
\end{split}\]
Hence, all elements of \(\mathbb{Z}[\alpha]\) are of the form \(a + b \alpha\) 
for \(a, b \in \mathbb{Z}\).

On the other hand, if we consider \(\mathbb{Z}[\pi] \subseteq \mathbb{C}\), 
as \(\pi\) is not an algebraic number, for all \(P(X) \in \mathbb{Z}[X] \setminus \{0\}\), 
\(P(\pi) \neq 0\). Thus, if \(P(X), Q(X)\) are polynomials such that
\(P(\pi) = r_0 + r_1 \pi + \cdots + r_n \pi^n = s_0 + s_1 \pi + \cdots + s_m \pi^m = Q(\pi)\), 
WLOG. \(n \le m\) we have \(0 = (s_0 - r_0) + (s_1 - r_1) \pi + \cdots + 
(s_n - r_n) \pi^n + s_{n + 1} \pi^{n + 1} + \cdots + s_m \pi^{m + 1}\), implying 
\(s_i = r_i\) for all \(i = 1, \cdots, n\) and \(s_i = 0\) for \(i > n\), we have 
\(P = Q\). Hence, \(\mathbb{Z}[\pi] \cong \mathbb{Z}[X]\).

\begin{proposition}
  If \(R\) is a subring of \(S\), then \(R[\alpha]\) for some \(\alpha \in S\) 
  is the intersection of all subrings of \(S\) containing \(R \cup \{\alpha\}\).
\end{proposition}
\begin{proof}
  Since \(R[\alpha]\) contains both \(R\) and \(\alpha\), we have 
  \[\bigcap \{U \mid R \cup \{\alpha\} \subseteq U \le S\} \subseteq R[\alpha].\]
  On the other hand, for all subrings \(U\) containing \(R \cup \{\alpha\}\), 
  \(R[\alpha] \subseteq U\) as \(U\) is closed under \(+\) and \(\times\). 
  Thus, 
  \[\bigcap \{U \mid R \cup \{\alpha\} \subseteq U \le S\} = R[\alpha].\]
\end{proof}

\begin{definition}[Integral Domain]
  A ring \(R\) is an integral domain if for all \(r, s \in R\), \(rs = 0\) 
  implies \(r = 0\) or \(s = 0\).
\end{definition}

In particular, we say \(r \in R\) is a zero divisor if there exists a 
\(s \in R \setminus \{0\}\) such that \(rs = 0\). Thus, an integral domain 
is simply a ring with no zero divisors. 

\begin{definition}[Field of Fractions]
  For \(R\) an integral domain, then the field of fractions of \(R\) denoted 
  \(\text{Frac}(R)\), is \(R \times R\setminus\{0\}\) quotiented by the 
  equivalence class 
  \[(a, b) \sim (r, s) \iff as = br.\]
  We write \(a / b\) as a representative of the equivalence class \([a, b]\).
\end{definition}

We may equip the field of fractions of \(R\) with addition and multiplication 
such that for \(a / b, r / s \in \text{Frac}(R)\) 
\[\frac{a}{b} + \frac{r}{s} = \frac{ad + bc}{bd} \text{ and } 
  \frac{a}{b} \times \frac{r}{s} = \frac{ar}{bs}.\]
It is routine to check these operations are well-defined and that the ring 
axioms are satisfied. Furthermore, as the name suggests, \(\text{Frac}(R)\) 
is a field and for all \(a / b \neq 0\), \((a / b) \times (b / a) = 1\).

\begin{definition}[Multiplicative System]
  A set \(S \subseteq R\) is a multiplicative system if \(1 \in S, 0 \notin S\) 
  and is closed under multiplication.
\end{definition}

\begin{definition}
  Let \(R\) be a ring and \(S \subseteq R\) be a multiplicative system. Then 
  \(S^{-1}R\) is \(R \times S\) quotiented by the equivalence class 
  \[(a, b) \sim (r, s) \iff as = br\]
  for \(a, r \in R, b, s \in S\).
\end{definition}

Similarly, we may equip \(S^{-1}R\) with addition and multiplication such that 
\(S^{-1}R\) is a subring of \(\text{Frac}(R)\). 

It is possible to use this construction on rings which are not integral domains, 
though in that case, the equivalence class is more subtle as division by a 
zero divisor will introduces other elements into the subring. This will be explored 
later in this course.

\subsection{Homomorphisms and Ideals}

We recall the definition of ring homomorphism and some related results 
(whose proofs omitted or shortened).

\begin{definition}[Ring Homomorphism]
  Given \(R, S\) rings, a ring homomorphism from \(R\) to \(S\) is a map 
  \(f : R \to S\) such that for all \(a, b \in R\), 
  \begin{itemize}
    \item \(f(1_R) = 1_S\);
    \item \(f(a +_R b) = f(a) +_S f(b)\);
    \item \(f(a \cdot_R b) = f(a) \cdot_S f(b)\).
  \end{itemize}
  If \(f\) is a bijection then we say \(f\) is an isomorphism.
\end{definition}

Automatically, it is not difficult to see that condition 2 implies \(f(0_R) = 0_S\) 
and from this we can deduece properties such as \(f(-x) = -f(x)\).

\begin{proposition}
  The image of a ring homomorphism \(f : R \to S\) is a subring of \(S\).
\end{proposition}

As we have seen in other contexts, the notion of an isomorphism is often defined 
to be a invertible structure preserving map. Though in some contexts, such as 
topological spaces, bijection is often not enough and we will require the inverse 
to be structure preserving. The following proposition shows that these two 
cases are equivalent for rings.

\begin{proposition}
  If \(f : R \to S\) is an isomorphism, then \(f^{-1} : S \to R\) is a ring 
  homomorphism.
\end{proposition}
\begin{proof}
  For all \(a, b \in S\), we have \(f^{-1}(a + b) = 
  f^{-1}(f(f^{-1}(a)) + f(f^{-1}(b))) = f^{-1}(f(f^{-1}(a) + f^{-1}(b)))
  = f^{-1}(a) + f^{-1}(b)\). Similar argument for the other conditions.
\end{proof}

\begin{proposition}
  There exist a unique homomorphism from \(\mathbb{Z}\) to \(R\) for all ring 
  \(R\).
\end{proposition}
\begin{proof}
  Clear by considering if \(f : \mathbb{Z} \to R\) is a homomorphism, 
  \(f(n_{\mathbb{Z}}) = n_{\mathbb{Z}} \cdot 1_R\). 
\end{proof}

\begin{proposition}
  Given a ring \(R\) and \(\alpha \in R\), there exists a unique homomorphism 
  \(f : R[X] \to R\) such that \(f(X) = \alpha\) and \(f\mid_R = \text{id}_R\).
  This homomorphism is called the evaluation map at \(\alpha\) and we denote it 
  as \(\text{ev}_\alpha\).
\end{proposition}
\begin{proof}
  Clear and as the name suggests, the unique map is 
  \[\text{ev}_\alpha(P(X)) = P(\alpha),\]
  for all \(P \in R[X]\).
\end{proof}

More generally, if \(f : R \to S\) is a homomorphism and \(\alpha \in S\), there 
exists a unique \(\text{ev}_{f, \alpha} : R[X] \to S\) such that
\(\text{ev}_{f, \alpha} \mid_R = f\) and \(\text{ev}_{f, \alpha}(X) = \alpha\). 
Furthermore, if \(f\) is simply the inclusion map from \(R \to S\), image of 
\(\text{ev}_{f, \alpha}(X) = \alpha\) is \(R[\alpha]\).

\begin{definition}[Kernel]
  Let \(R, S\) be rings and \(f : R \to S\) a ring homomorphism. Then the 
  kernel of \(f\) is 
  \[\ker f := \{r \in R \mid f(r) = 0_S\}.\]
\end{definition}

\begin{proposition}
  A ring homomorphism \(f : R \to S\) is injective if and only if 
  \(\ker f = \{0\}\).
\end{proposition}

\begin{definition}[Ideal]
  Given a subset \(I\) of a ring \(R\), then \(I\) is said to be an ideal if 
  \begin{itemize}
    \item \(0_R \in I\);
    \item for all \(a, b \in I\) then \(a + b \in I\);
    \item for all \(a \in I\), \(r \in R\), \(ra \in I\).
  \end{itemize}
\end{definition}

\begin{definition}
  The following ideals are important enough to warrant a definition.
  \begin{itemize}
    \item \(\{0_R\} \subseteq R\) is the zero ideal;
    \item \(R \subset R\) is the unit idea;
    \item for all \(r \in R\), \(\langle r \rangle := \{rs \mid s \in R\}\) 
      is the principal ideal generated by \(r\).
  \end{itemize}
\end{definition}

\begin{proposition}
  Every ideal of \(\mathbb{Z}\) is principle.
\end{proposition}

\begin{proposition}
  In intersection of ideals is an ideal. Similarly, the sum of two ideals, 
  i.e. if \(I, J\) are ideals, then \(\{i + j \mid i \in I, j \in J\}\) is an 
  ideal.
\end{proposition}

\begin{definition}
  Let \(R\) be a ring and \(r_1, \cdots, r_n \in R\). Then the ideal generated 
  by \(r_1, \cdots, r_n\) is 
  \[\langle r_1, \cdots, r_n \rangle := 
    \{r_1 s_1 + \cdots r_n s_n \mid s_i \in R\}.\]
\end{definition}

It is clear that the ideal generated by \(r_1, \cdots, r_n\) is the smallest 
ideal containing \(r_1, \cdots, r_n\).

\begin{definition}
  The produce of ideals \(I\) and \(J\) is the ideal which elements are
  of the form \(i_1j_1 + \cdots + i_nj_n\) for all \(i_1, \cdots i_n \in I\), 
  \(j_1, \cdots, j_n \in J\).
\end{definition}

For ideals \(I, J\), we see that \(IJ \subseteq I \cap J\) though they are not 
necessary equal (consider \(\langle 2 \rangle \langle 2 \rangle = \langle 4 
\rangle\) thought \(\langle 2 \rangle \cap \langle 2 \rangle = \langle 2 \rangle\)).

\begin{proposition}
  If ideals \(I, J\) satisfy \(I + J = \langle 1 \rangle\), then 
  \(I \cap J = IJ\).
\end{proposition}

As with other mathematical objects, we would like to construct a quotient 
object for the rings. The equivalence relation we shall quotient on it the 
following. Let \(I \subseteq R\) be an ideal and we define say
\(r \equiv s \mod I\) for \(r, s \in R\) if \(r - s \in I\). It is not difficult 
to check that \(\equiv_I\) is a equivalence relation and thus, we may take 
a quotient of \(R\) with respect to this equivalence relation and we denote 
the equivalence classes with \(r + I\).

\begin{definition}[Quotient Ring]
  Given \(R\) a ring and \(I\) an ideal of \(R\), then the quotient ring 
  of \(R\) by \(I\) is the ring with the underlying set 
  \[R / I := R / \equiv_I = \{r + I \mid r \in R\},\]
  where \(0_{R/I} = 0_R + I\), \(1_{R/I} = 1_R + I\), and for all 
  \(r + I, s + I \in R / I\), \((r + I) + (s + I) = (r + s) + I\) 
  and \((r + I) \cdot (s + I) = rs + I\).
\end{definition}

\begin{definition}[Quotient Map]
  Given \(R\) a ring and \(I\) an ideal of \(R\), the quotient map is then the 
  surjective ring homomorphism \(q : R \to R / I : r \mapsto r + I\).
\end{definition}

It is clear that \(\ker q = I\).

A more modern interpretation of the quotient ring is by defining it as an 
object satisfying its universal property. In particular, the ring \(R / I\),
taken together with a ring homomorphism \(q : R \to R / I\), has the following 
universal property.

\begin{proposition}
  If \(f : R \to S\) is a ring homomorphism such that \(I \subseteq \ker f\), 
  then there exists a unique ring homomorphism \(\tilde f : R / I \to S\)
  such that for all \(r \in R\), \(\tilde f(r + I) = f(r)\).
\end{proposition}

Essentially, the universal property states that there exists a unique \(\tilde f\) 
such that the following diagram commutes.
\[\begin{tikzcd}
  R \arrow[r, "f"] \arrow[swap, d, "q"] & S \\
  R / I \arrow[swap, dashed, ru, "\tilde f"]
  \end{tikzcd}\]
\begin{proof}
  Uniqueness is clear and thus we will show \(\tilde f\) is well-defined and 
  is a ring homomorphism. Let \(r \equiv s \mod I\), and will show \(f(r) = f(s)\). 
  Indeed, since \(r - s \in I\), we have \(r - s \in \ker f\) and so, 
  \(f(r) - f(s) = f(r - s) = 0\), hence \(f(r) = f(s)\) and \(\tilde f\) is 
  well-defined. Now, let \(r + I, s + I \in R / I\), we have 
  \[\tilde f((r + I) + (s + I)) = \tilde f((r + s) + I) = f(r + s) = 
    f(r) + f(s) = \tilde f(r + s) + \tilde f(s + I),\]
  hence by similar argument for multiplication, we have \(\tilde f\) is a ring 
  homomorphism.
\end{proof}

As an example consider the surjective map \(\mathbb{R}[X] \to \mathbb{C}\) 
which is \(\text{id}\) on \(\mathbb{R}\) and sends \(X\) to \(i\). Then 
this map have kernel \(\{P \in \mathbb{R}[X] \mid P(i) = 0\} = 
\langle X^2 + 1 \rangle\). Thus, we have the diagram 
\[\begin{tikzcd}
  \mathbb{R}[X] \arrow[d, swap, "q"] \arrow[r] & \mathbb{C}\\
  \mathbb{R}[X] / \langle X^2 + 1 \rangle \arrow[swap, leftrightarrow, ru]
\end{tikzcd}\]
where the pull-back map is an isomorphism as the map itself is surjective while 
injectivity follows as we have quotiented out its kernel. As we shall see, 
whenever we have one field inside another, there is a construction similar this 
such that we can construct the larger field from the smaller field.

By recalling the evaluation map, if \(\alpha \in R\), by the above process, we 
see that 
\[R[X] / I \cong R[\alpha],\]
where \(I\) is the kernel of the evaluation map at \(\alpha\).

\begin{definition}
  Let \(R\) be a ring and \(I\) an ideal of \(R\). Then we say \(I\) is a 
  prime ideal if \(R / I\) is an integral domain. Furthermore, we say 
  \(I\) is a maximal ideal if \(R / I\) is a field.
\end{definition}

Since fields are integral domains, maximal ideals are prime.

\begin{proposition}
  An ideal \(I\) of \(R\) is prime if and only if for all \(rs \in I\), either 
  \(r \in I\) or \(s \in I\).
\end{proposition}

\begin{proposition}
  An ideal \(I\) of \(R\) is maximal if and only if the only ideal of \(R\) 
  containing \(I\) is \(I\) or the unit ideal \(R\).
\end{proposition}
\begin{proof}
  Follows by considering that a ring is a field if and only if its only ideals 
  are the zero or the unit ideal, and the image of an ideal by a surjective 
  homomorphism is also an ideal.
\end{proof}

\subsection{Factorization and PIDs}

\begin{definition}[Unit]
  Let \(R\) be a integral domain, then \(R^\times\) is the set of elements \(r\) 
  of \(R\) such that there exists some \(r' \in R\) such that \(rr' = 1\). If 
  \(r \in R^\times\), then we call \(r\) a unit.
\end{definition}

\begin{definition}[Divides]
  Let \(r, s \in R\), we say \(r\) divides \(s\) if \(s \in \langle r \rangle\).
\end{definition}

It is clear that a unit divides any element. Indeed, if \(u \in R^\times\) 
and \(s \in R\) such that \(uu' = 1\), then \(s = (su')u\) implying 
\(s \in \langle u \rangle\).

\begin{definition}[Associate]
  An associate of \(r \in R\) is an element \(ur\) of \(r\) with \(u \in R^\times\).
\end{definition}

\begin{definition}[Irreducible]
  An element \(r \in R\) is irreducible if \(r \neq 0\), \(r \not\in R^\times\) 
  and the only dividors of \(r\) are units and associates of \(r\).
\end{definition}

\begin{definition}[Unique Factorization Domain]
  A ring \(R\) is a unique factorization domain (UFD) if it is a integral domain 
  and 
  \begin{itemize}
    \item for all non-zero, non-unit element of \(R\) is a product of finitely 
      many irreducibles.
    \item for all \(r \in R\) non-zero, non-unit such that 
    \[r = p_1 \cdots p_s = q_1 \cdots q_t,\]
    where \(p_i, q_i\) are irreducibles, then \(s = t\) and after reordering, 
    \(p_i\) is an associate of \(q_i\).
  \end{itemize}
\end{definition}

Some typical examples of UFDs are \(\mathbb{Z}, \mathbb{F}[X], 
\mathbb{Z}[X], \cdots\) (where \(\mathbb{F}\) is a field), though it is more 
challenging to come up with counter-examples. Consider the ring 
\(\mathbb{Z}[\sqrt{-5}] = \{a + b\sqrt{-5} \mid a, b \in \mathbb{Z}\}\), define 
\[N : \mathbb{Z}[\sqrt{-5}] \to \mathbb{Z} : z \mapsto z \overline{z}.\]
It is easy to see that \(N\) is multiplicative, and thus, if 
\(u \in \mathbb{Z}[\sqrt{-5}]\) is a unit such that \(uu' = 1\), we have
\[N(u)N(u') = N(uu') = N(1) = 1,\]
implying \(N(u) = \pm 1\) and so \(u = \pm 1\). Then, as \(\pm 1\) are the 
only units of \(\mathbb{Z}[\sqrt{-5}]\), we have 
\(3 \cdot 2 = (1 - \sqrt{-5})(1 + \sqrt{-5})\) are products of non-units which 
are not associate with each other. Hence, to show that \(\mathbb{Z}[\sqrt{-5}]\) 
is not a UFD it suffices to show that the factors are irreducibles. To show this, 
one again use \(N\) by plugging the factors. 

Let us construct a ring such that the first 
condition of UFD fails, i.e. a ring for which a non-zero, non-unit element is 
not a product of finitely many irreducibles. Define 
\[\mathbb{C}[t^{\mathbb{Q}_\ge 0}] := 
  \left\{\sum_{i = 0}^r c_i t^{a_i} \mid c_i \in \mathbb{C}, a_i \in \mathbb{Q} 
  \right\} = \bigcup_{n = 1} \{f^{1/n} \mid f \in \mathbb{C}[X]\}.\]
Then, \(\mathbb{C}[t^{\mathbb{Q}_\ge 0}]^\times = \mathbb{C}^\times\) and 
in fact, \(\mathbb{C}[t^{\mathbb{Q}_\ge 0}]\) does not have any irreducible 
elements. Let \(f \in \mathbb{C}[t^{\mathbb{Q}_\ge 0}]^\times\) such that 
\(f = P(t^{1/n})\) and \(f^{-1} = Q(t^{1/m})\), then we may write 
\(f = P'(t^{1/(nm)})\) and \(f^{-1} = Q'(t^{1/(nm)})\). Hence, 
\[1 = P'(t^{1/(nm)}) Q'(t^{1/(nm)}) \implies P'Q' = 1 \implies P', Q' 
\text{ are constants,}\]
and so \(f \in \mathbb{C}^\times\). On the other hand, if 
\(P(t^{1 / n}) \in \mathbb{C}[t^{\mathbb{Q}_\ge 0}]\) is irreducible, by the 
fundamental theorem of algebra, it is a product of linear polynomials implying 
\(P(t^{1 / n}) = t^{1 / n} - a\) for some \(a \in \mathbb{C}\). But, 
\(t^{1 / n} - a = (t^{1 / (2n)} + \sqrt{a})(t^{1 / (2n)} - \sqrt{a})\), a 
contradiction.

\begin{definition}[Prime]
  An element \(r\) of a ring \(R\) is prime if \(\langle r \rangle\) is a prime 
  ideal. Equivalently, \(r\) is prime if for all \(s, t \in R\), \(r \mid st\) 
  implies either \(r \mid s\) or \(r \mid t\).
\end{definition}

\begin{proposition}\label{ufdiff}
  Let \(R\) be an integral domain in which every element is a finite product of 
  irreducibles. Then every irreducible element of \(R\) is prime if 
  and only if for all 
  \[p_1 \cdots p_s = q_1 \cdots q_t,\]
  where \(p_i, q_i\) are irreducible, then \(s = t\) and after reordering, 
  \(p_i\) is an associate of \(q_i\).
\end{proposition}
\begin{proof}
  Suppose every irreducible element of \(R\) is prime. Then, if 
  \[p_1 \cdots p_s = q_1 \cdots q_t,\]
  where \(p_i, q_i\) are irreducible, we have 
  \(p_1 \mid q_1,\cdots, q_t\) and so, \(p_1 \mid q_i\) for some 
  \(i = 1, \cdots, t\), and hence \(p_i\) is an associate of \(q_1\). Then, 
  by reordering, we have \(p_1\) and \(q_1\) are associates. Repeating this 
  argument, we may cancel all associates with some terms remaining if \(s > t\),
  \[p_{t + 1} \cdots p_s = 1.\]
  But this is a contradiction since then \(p_{t + 1}\) is a unit and so 
  \(s = t\) as required.  

  Conversely, suppose \(r\) is irreducible and \(r \mid st\) and so there exists 
  some \(rx = st\) for some \(x \in R\). Then, we may factor \(x, s, t\) into 
  irreducibles such that 
  \[r p_1 \cdots p_l = q_1 \cdots q_m n_1 \cdots n_k.\]
  Then, as such factorizations are unique, \(r\) must be an associate of some 
  \(q_i\) or \(n_i\) which implies that \(r \mid s\) or \(r \mid t\), 
  so \(r\) is prime.
\end{proof}

\begin{proposition}
  In an integral domain \(R\), if \(r \in R\) is prime, then \(r\) is irreducible.
\end{proposition}
\begin{proof}
  Suppose otherwise, \(r = st\). Then \(r \mid st\) but neither \(r \mid s\) 
  nor \(r \mid t\).
\end{proof}

A counter-example of the reverse is that 2 is irreducible in 
\(\mathbb{Z}[\sqrt{-5}]\) but is not a prime.

\begin{definition}[Principal Ideal Domain]
  A ring \(R\) is a principal ideal domain (PID) if \(R\) is an integral 
  domain and every ideal \(I\) is principal.
\end{definition}

\begin{lemma}
  If \(R\) is a PID, then any increasing tower of ideals 
  \[I_1 \subseteq I_2 \subseteq I_3 \subseteq \cdots\]
  is eventually constant, i.e. there exists an \(N \in \mathbb{N}\) such that 
  for all \(n \geq N\), \(I_n = I_N\).
\end{lemma}
\begin{proof}
  Let \(I = \bigcup_{j = 1}^\infty I_j\). Since \(x, y \in I\), there exists 
  \(j, k \in \mathbb{N}\), such that \(x \in I_j, y \in I_j\), and so 
  \(x, y \in I_{\max\{j, k\}} \implies x + y \in I_{\max\{j, k\}} \subseteq I\). 
  Similarly, by the same argument we have \(I\) is closed under multiplication 
  by elements of \(R\). Thus, \(I\) is an ideal, and so 
  \(I = \langle r \rangle\) is principal. Finally, as \(r \in I\), there exists 
  some \(N \in \mathbb{N}\) such that \(r \in I_N\), and so 
  \(I = \langle r \rangle \subseteq I_N\) implying \(I = I_N\).
\end{proof}

\begin{lemma}
  Let \(R\) be a PID, \(r \in R\) which is non-zero, non-unit. Then there 
  exists some irreducible \(s \in R\) which divides \(r\).
\end{lemma}
\begin{proof}
  If \(r\) is irreducible, then simply take \(s = r\). On the other hand, if 
  \(r\) is not irreducible, then there exists \(r_0, s_0\) non-zero, non-associates 
  of \(r\) such that \(r = r_0s_0\). If \(r_0\) is irreducible, then we are 
  done while otherwise, we may repeat the process such that \(r_0 = r_1s_1\). 
  This process must terminate since if otherwise, we have 
  \[\langle r \rangle \subsetneq \langle r_0 \rangle \subsetneq 
    \langle r_1 \rangle \subsetneq \cdots\]
  which is non-terminating strictly increasing tower of ideals contradicting 
  our previous lemma. 
\end{proof}

\begin{lemma}
  Let \(R\) be a PID, then any non-zero, non-unit of \(r\) factors into 
  irreducibles.
\end{lemma}
\begin{proof}
  Similar to before, all factors of \(r\) must terminate since otherwise we 
  have produces an non-terminating increasing tower of ideals.
\end{proof}

\begin{theorem}
  Let \(R\) be a PID, then \(R\) is a UFD.
\end{theorem}
\begin{proof}
  We already shown the existence of factorizations and so, it remains to show 
  uniqueness. By proposition \ref{ufdiff}, it suffices to show that every 
  irreducible element of \(R\) is prime. Let \(r \in R\) be irreducible, 
  \(r \mid st\) and \(r \not\mid s\). Then, since \(\langle r, s \rangle\) is 
  principal, there exists some \(q \in R\), such that 
  \(\langle r, s \rangle = \langle q \rangle\) implying \(q \mid r\) and 
  \(q \mid s\). But since \(r\) is irreducible, either \(q\) is an associate 
  of \(r\) or a unit. If \(q\) is an associate of \(r\), there exists a 
  unit \(u\) such that \(uq = r\). But \(q \mid s\) and so, there exists some 
  \(a \in R\), \(aq = s\) implying \((au^{-1})r = au^{-1}uq = aq = s\) 
  contradicting \(r \not\mid s\). Thus, \(q\) is a unit and so 
  \(\langle r, s \rangle = R\) and there exists some \(a, b \in R\) such that 
  \(ar + bs = 1\), and so \(t = art + bst\). Finally, as \(r \mid st\), we have 
  \(r \mid art + bst = t\) implying \(r\) is prime.
\end{proof}

\begin{corollary}
  Let \(R\) be a PID, then every non-zero prime ideal of \(R\) is maximal.
\end{corollary}
\begin{proof}
  Let \(I = \langle r \rangle \trianglelefteq R\) be a non-zero prime ideal. 
  Then \(r\) is a prime and so, it is irreducible. Now, if 
  \(I \le J = \langle s \rangle\) for some element \(s\), there exists some 
  \(t \in R\) such that \(st = r\) implying \(s\) is a unit or an associate of 
  \(r\). If \(s\) is an associate, then there exists some unit \(u\) such that 
  \(us = r\) and thus, \(s = u^{-1}r\) and so 
  \(\langle s \rangle \subseteq \langle r \rangle\) implying
  \(\langle s \rangle =\langle r \rangle\). On the other hand if 
  \(s\) is a unit, then \(s^{-1}s = 1 \in \langle s \rangle\) and so 
  \(\langle s \rangle = R\). 
\end{proof}

\subsection{Euclidean Domains}

So far we have developed a nice theory about PIDs though we have yet to have 
any tools to prove that ring is a PID. We will now develop the notion called 
Euclidean domains to aid us in this matter.

\begin{definition}[Euclidean Norm]
  For integral domain \(R\), a Euclidean norm on \(R\) is a function 
  \(N : R \to \mathbb{N}\) such that for all \(a, b \in R\), \(b \neq 0\), 
  there exists \(q, r \in R\) such that \(a = qb + r\) and either 
  \(N(r) < N(b)\) or \(r = 0\).
\end{definition}

\begin{definition}[Euclidean Domain]
  A Euclidean domain is a ring \(R\) that admits a Euclidean norm.
\end{definition}

\begin{proposition}
  Any Euclidean domain \(R\) is a PID.
\end{proposition}
\begin{proof}
  Let \(N\) be the Euclidean norm on \(R\) and suppose \(I \neq 0\) is an ideal of 
  \(R\). Now since \(N(I) \subseteq \mathbb{N}\) is non-empty, \(N(I)\) admits 
  some minimal element \(k = N(r)\) for some \(r \in I\). Suppose 
  \(I \neq \langle r \rangle\), then there exists some \(s \in I\), 
  \(s \not\in \langle r \rangle\). Then, by the definition of the Euclidean norm, 
  there exists some \(a, b \in R\) such that \(s = ar + b\) and either \(b = 0\) 
  or \(N(b) < N(r)\). But if \(b = 0\) then \(s = ar\) implying 
  \(s \in \langle r \rangle\), so \(N(b) < N(r)\). But this contradicts the 
  minimality of \(N(r)\) and hence, \(I = \langle r \rangle\).
\end{proof}

We see that the notion of an Euclidean norm is very similar to the quotient 
remainder for the integers, and so it is not very surprising that \(\mathbb{Z}\) 
is a Euclidean domain. In particular, we define \(N(n) = |n|\) and for 
\(a, b \in \mathbb{Z}\), \(b \neq 0\), we can define 
\[q := \left\lfloor\frac{a}{b}\right\rfloor,\]
so \(0 \le \frac{a}{b} - q < 1\), and thus, \(0 \le a - bq < b\) which 
implies \(|a - bq| < |b|\).

A more complicated example is the Gaussian integers \(\mathbb{Z}[i]\). 
Let \(N(n + mi) = n^2 + m^2\) and given \(x, y \in \mathbb{Z}[i]\), 
\(y \neq 0\), we can define \(q' = x / y = a + bi\) 
for some \(a, b \in \mathbb{R}\). Then, defining \(q = a' + b'i\) where 
\(a', b' \in \mathbb{Z}\) such that \(|a - a'|, |b - b'| < 1 / 2\). Finally, 
by noticing \(N\) is multiplicative (as \(N(z) = z \overline{z}\)),
we have 
\[\begin{split}
  N(r) & = N(x - qy) = N(y)N\left(\frac{x}{y} - q\right) = 
  N(y)N(q' - q)\\ & = N(y)(|a - a'|^2 + |b - b'|^2) < \frac{N(y)}{2} < N(y).
\end{split}\]
Thus, \(\mathbb{Z}[i]\) is a Euclidean domain.

Given a field \(\mathbb{F}\), we can define \(N(P) = \deg P\) for 
\(P \in \mathbb{F}[X]\). Then, for \(P, S \in \mathbb{F}[X]\), it is not 
difficult to see that the Euclidean norm conditions hold by long division and 
induction. A direct corollary of this is that, since \(\mathbb{F}[X]\) is 
a Euclidean domain, it is a PID, and thus, for any irreducible polynomial 
\(P \in \mathbb{F}[X]\), \(\langle P \rangle\) is a prime ideal, hence maximal, 
i.e. by definition \(\mathbb{F}[X] / \langle P \rangle\) is a field.

\subsection{Chinese Remainder Theorem}

\begin{definition}[Product ring]
  Let \(R, S\) be rings. Then \(R \times S\) is the Cartesian product of 
  \(R\) and \(S\) equipped with the ring structure where addition and 
  multiplication are defined pairwise.
\end{definition}

As one might expect, the projection maps from a product ring are ring 
homomorphisms. On the other hand, the inclusion maps are not ring homomorphisms.
In particular, the map
\[i_1 : R \to R \times S : r \mapsto (r, 0), i_2 : S \to R \times S : s \mapsto (0, s),\]
are not ring homomorphisms as they do not map the multiplicative identity 
to the multiplicative identity of \(R \times S\).

We see that the product ring operation is associative, i.e. 
\((R \times S) \times T \cong R \times (S \times T)\) via the isomorphism 
\(((r, s), t) \mapsto (r, (s, t))\), and we may simply omit the brackets. 
This argument can be applied analogously to \(n\)-fold products. 

In general, given a collection of rings \(\{R_i\}_{i \in I}\) for some index set 
\(I\), we may define \(\prod_{i \in I} R_i\)
to be the set of elements of the form \(\{x_i\}_{i \in I}\) such that 
\(x_i \in R_i\). Similarly, we may equip \(\prod R_i\) with the ring structure 
where addition and multiplication are defined pairwise. As before, the projection 
maps are ring homomorphisms.

This definition of product rings is characterised by the following universal property. 

\begin{proposition}
  Given a ring \(S\) and a collection of rings \(\{R_i\}_{i \in I}\). If 
  \(f_i : S \to R_1\) is a ring homomorphism for all \(i \in I\), then there exists 
  a unique ring homomorphism 
  \[\prod_{i \in I} f_i : S \to \prod_{i \in I} R_i,\]
  such that \(f_i = \pi_i \circ (\prod f_i)\) where \(p_i\) is the projection map 
  from \(\prod R_i\) to \(R_i\).
\end{proposition}
\begin{proof}
  Define \(\prod_{i \in I} f_i(s) := \{f_i(s)\}_{i \in I} \in \prod_{i \in I} R_i\).
  It is clear that \(\prod_{i \in I} f_i\) is a ring homomorphism and 
  \((\pi_i \circ (\prod_{i \in I} f_i))(s) = \pi_i(\{f_i(s)\}_{i \in I}) = f_i(s)\).

  Now, if \(g : S\ to \prod_{i \in I} R_i\) is a ring homomorphism such that 
  \(\pi_i \circ g = f_i\), then \(g(s) = \{f_i(s)\}_{i \in I} = 
  (\prod_{i \in I} f_i)(s)\). Thus, \(g = \prod_{i \in I} f_i\).
\end{proof}

The Chinese remainder theorem seeks to solve the question that, if \(I, J\) 
are ideals of the ring \(R\), if \(a \in R / I\) and \(b \in R / J\), then 
does there exists some \(c \in R\), \(c + I = a + I\) and \(c + J = b + J\).
Furthermore, if such an element exists, how unique is it.

Phrased in a more ring theoretic sense, the existence question asks, 
if \(q_1 : R \to R / I\) and \(q_2 : R \to R / J\) are the quotient maps, 
is \((a, b)\) in the image of 
\(q_1 \times q_2 : R \to R / I \times R / J : r \mapsto (q_1(r), q_2(r))\). 
On the other hand, the uniqueness question asks if there asks whether or not 
\(q_1 \times q_2\) is injective, and so, it essentially asks what is the 
kernel of \(q_1 \times q_2\).

It is clear that if \(q_1 \times q_2(s) = 0\) if and only if 
\(q_1(s) = 0\) and \(q_2(s) = 0\) and so, \(I \cap J = \ker q_1 \times q_2\).
With this in mind, we see that 
\[R / I \cap J \hookrightarrow R / I \times R / J\]
is a injection. Sadly, this injection is not a surjection by simply considering 
the case where \(I = J\) and \(a \neq b\). 

More generally, given \(I_1, \cdots, I_r\) is a finite collection of ideals, 
we have 
\[R / \bigcap_{i = 1}^r I_i \hookrightarrow R / I_1 \times \cdots \times R / I_r,\]
is an injection.

\begin{definition}[Relatively Prime Ideal]
  Ideals \(I, J\) are relatively prime ideals of \(R\) if \(I + J = R\).
\end{definition}

\begin{theorem}[Chinese Remainder Theorem]
  Let \(\{I_i\}_{i = 1}^r\) be a finite collection of ideals of \(R\) such 
  that \(\{I_i\}\) is pairwise relatively prime. Then the map  
  \[R / \bigcap_{i = 1}^r I_i \hookrightarrow R / I_1 \times \cdots \times R / I_r,\]
  is an isomorphism.
\end{theorem}
\begin{proof}
  It suffices to prove surjectivity. 

  For all \(i, j\), \(i \neq j\), as \(I_i\) and \(I_j\) are pairwise relatively 
  prime, there exists \(r_i, \in I_i\) and \(r_j, \in I_j\) such that 
  \(r_i + r_j = 1\). Thus, \(r_i = 1 \mod I_j\) and \(r_j = 1 \mod I_i\). Then, 
  for each \(j\), let \(f_j = \prod_{k \neq j} r_k\) where \(r_k\) is defined 
  as above, we have \(f_j = 1 \mod I_j\) since individual \(r_k = 1 \mod I_j\). 
  On the other hand, for all \(k \neq j\), \(r_k = 0 \mod I_k\), we have 
  \(f_j = 0 \mod I_k\). Finally, for all 
  \(s = (s_1, \cdots, s_r) \in R / I_1 \times \cdots \times R / I_r\), choose 
  \(\tilde s_i\) such that \(q_i(\tilde s_i) = s_i\). Then, setting 
  \(t = f_1 \tilde s_1 + \cdots + f_r \tilde s_r\), we have 
  \[q_i(t) = q_i(f_1)q_i(\tilde s_1) + \cdots + q_i(f_r)q_i(\tilde s_r) 
    = 0 + \cdots + 0 + 1 \cdot s_i + 0 + \cdots + 0 = s_i,\]
  and so \(t \mapsto s\) under the aforementioned map.
\end{proof}

\newpage
\section{Fields}

\subsection{Field Extensions}

Let \(K\) be a field. As we have seen before, as fields are rings, there exists a 
unique ring homomorphism from \(f : \mathbb{Z} \to K\). Then the kernel of this map 
is an ideal of \(\mathbb{Z}\), and in particular, as \(\mathbb{Z} / \ker f\) is a 
sub-ring of \(K\), it is an integral domain and so, \(\ker f\) is prime. 
Now, since the prime ideals of \(\mathbb{Z}\) are either the zero ideal or 
the principal ideal of a prime element, we see there are two cases. 
In the case that \(\ker f = \{0\}\), we obtain a injection 
\[\iota : \mathbb{Q} \hookrightarrow K : \frac{a}{b} \mapsto f(a)f(b)^{-1}.\]
On the other hand if \(\ker f = \langle p \rangle\) for some prime \(p \in \mathbb{Z}\),
we obtain an injection 
\[\iota : \mathbb{F}_p := \mathbb{Z} / p\mathbb{Z} \hookrightarrow K.\]
Similar to rings, we refer fields of the first type to have characteristic zero, 
while fields of the second type to have characteristic \(p\). We call 
the domain of \(\iota\) (i.e. \(\mathbb{Q}\) or \(\mathbb{F}_p\)) the prime field 
of \(K\).

\begin{definition}[Extension]
  If \(K, L\) are fields such that \(K \subset L\), then we say \(L\) is an 
  extension of \(K\).
\end{definition}

We observe that if \(L\) is an extension of \(K\), then \(L\) is a 
\(K\)-vector space with the natural operations.

\begin{definition}[Finite Extension]
  Let \(L\) be an extension of \(K\). Then \(L\) is a finite extension of \(K\) 
  if the \(L\) is a finite dimensional vector space of \(K\). In this case the 
  degree of the extension \(L / K\), denoted \([L : K]\) is the 
  dimension of \(L\) as a \(K\)-vector space.
\end{definition}

\begin{proposition}
  Let \(M, K, L\) be fields such that \(M \subseteq K \subseteq L\), then 
  the following are equivalent: 
  \begin{itemize}
    \item \(L\) is a finite extension of \(M\);
    \item \(L\) is a finite extension of \(K\) and \(K\) is a finite extension of 
      \(M\).
  \end{itemize}
  In the case that this holds, then \([L : M] = [L : K][K : M]\).
\end{proposition}
\begin{proof}
  \((\implies)\) Suppose \(L\) is a finite extension of \(M\), and let 
  \(\{e_1, \cdots, e_r\}\) span \(L\) over \(M\). Then, since \(M \subseteq K\), 
  \(\{e_1, \cdots, e_r\}\) also spans \(L\) over \(K\), implying 
  \(L / K\) is finite. On the other hand, as \(K\) is a \(M\)-subspace of 
  \(L\) which is finite dimensional, \(K\) must also be finite dimensional, 
  implying \(K / M\) is finite.

  \((\impliedby)\) Let \(\{e_1, \cdots, e_r\}\) be a basis of \(L\) over \(K\),
  and let \(\{f_1, \cdots, f_s\}\) be a basis of \(K\) over \(M\). Then, 
  for all \(l \in L\), \(l = \sum \lambda_i e_i\) for some \(\lambda_i \in K\).
  Furthermore, for all \(i\), \(\lambda_i = \sum \mu_j^i f_j\) for some 
  \(\mu_j^i \in M\), and so 
  \[l = \sum_i (\sum_j \mu_j^i f_j) e_i = \sum_{i,j} \mu_j^i f_j e_i,\]
  implying \(\{f_j e_i\}_{i, j}\) spans \(L\) over \(M\). This is a finite spanning 
  set and thus \(L / M\) is finite. 

  For the last claim, it suffices to show that \(\{f_j e_i\}_{i, j}\) as defined 
  above is linearly independent. Suppose 
  \(\sum_{i, j} \lambda_{i, j} f_j e_i = 0\), then 
  \(0 = \sum_i (\sum_j \lambda_{i, j} f_j) e_i\) implying 
  \(\sum_j \lambda_{i, j} f_j = 0\) as \(\{e_i\}\) is linearly independent. 
  Now as \(\{f_i\}\) is also linearly independent, this implies 
  \(\lambda_{i, j} = 0\) for all \(i, j\) implying \(\{f_j e_i\}_{i, j}\) 
  is a basis of \(L\) over \(M\). Thus, 
  \([L : M] = \dim_M L = |\{f_j e_i\}_{i, j}| = r s = \dim_M K \dim_K L = 
    [L : K][K : M]\) as required.
\end{proof}

\subsubsection{One Element Extensions}

\begin{definition}
  Let \(L\) be a field extension of \(K\) and let \(\alpha \in L\). We denote 
  \(K(\alpha)\) denote the smallest subfield of \(L\) containing both 
  \(K\) and \(\alpha\).  
\end{definition}

In particular, \(K(\alpha)\) consists of all elements of \(L\) expressible as 
\(P(\alpha) / Q(\alpha)\) for \(P, Q \in K[X]\) and \(Q(\alpha) \neq 0\).

\begin{proposition}
  Let \(L\) be a field extension of \(K\) and let \(\alpha \in L\). There 
  exists a unique homomorphism \(\text{ev}_\alpha : K[X] \to L\) such that 
  \(\text{ev}_\alpha \mid_K = \iota : K \hookrightarrow L\) and 
  \(\text{ev}_\alpha(X) = \alpha\).
\end{proposition}

There are two cases for the kernel of \(\text{ev}_\alpha\). In the first case, 
\(\text{ev}_\alpha\) is injective and so, for all \(P \in K[X] \setminus \{0\}\) 
\(P(\alpha) \neq 0\). If this is the case, we say \(\alpha\) is a transcendental 
number. In this case, the map extends to a map 
\[K(X) \hookrightarrow L : \frac{P(X)}{Q(X)} \mapsto \frac{P(\alpha)}{Q(\alpha)}.\]
By inspection, the image of this map is \(K(\alpha)\). Hence, this is an 
isomorphism between \(K(X)\) and \(K(\alpha)\).

In the case the kernel is non-zero, we have the kernel must be prime as the 
quotient \(K[X] / \ker \text{ev}_\alpha\) is isomorphic to a subring in \(K\) 
which are integral domains. Thus, there exists a unique monic irreducible polynomial 
\(P \in K[X]\) generating \(\ker \text{ev}_\alpha\) (recall that 
\(K[X]\) is a PID as it is a Euclidean domain). We call this polynomial 
\(P\) the minimal polynomial of \(\alpha\) and we say \(\alpha\) is 
algebraic over \(K\). In this case, we obtain an injective homomorphism 
\[K[X] / \langle P(X) \rangle \hookrightarrow L,\]
with its image being a field containing \(K, \alpha\) in which every element 
is expressible as a polynomial in \(\alpha\) with coefficients in \(K\). Namely 
the image is precisely \(K(\alpha) = K[\alpha]\) and we have an isomorphism 
\(K[X] / \langle P(X) \rangle \simeq K(\alpha)\).

Note that if \(P\) has degree \(d\), any element of \(K[X]\) is expressible as 
\(P(X)Q(X) + R(X)\) with \(\deg R < d\). So \(1, X, \cdots, X^{d - 1}\) spans 
\(K[X] / \langle P(X) \rangle\) over \(K\). Furthermore, they are linearly independent 
since if otherwise, there exists \(\lambda_i \in K\) such that 
\(\sum_{i = 0}^{d - 1} \lambda_i X_i = 0\) implying 
\(P \mid \sum_{i = 0}^{d - 1} \lambda_i X_i\), a contradiction. Thus, 
\(1, X, \cdots, X^{d - 1}\) is a basis of \(K[X] / \langle P(X) \rangle\) 
over \(K\) and thus, \(1, \alpha, \cdots, \alpha^{d - 1}\) is a basis 
of \(K(\alpha)\). With this, we conclude \([K(\alpha) : K] = \deg P\) 
where \(P\) is the minimal polynomial of \(\alpha\).

\begin{proposition}
  If \(L / K\) is finite and \(\alpha \in L\), then \(\alpha\) is 
  algebraic over \(K\).
\end{proposition}
\begin{proof}
  Let \(d = [L : K]\) so \(1, \alpha, \cdots, \alpha^d\) is \(d + 1\) elements 
  in \(L\). Then, they are linearly dependent and there exists \(\lambda_i\)
  not all of which are zero such that, \(\sum_{i = 0}^{d} \lambda_i \alpha^i = 0\) 
  implying \(\alpha\) is algebraic.
\end{proof}

Since \(K(\alpha) \subseteq L\), we have \([K(\alpha) : K]\) divides \([L : K]\).

In the absence of an ambient field, we write \(K(\alpha)\) where \(\alpha\) is
a root of irreducible \(P \in K[X]\) for the quotient 
\(K[X] / \langle P(X) \rangle\).

\begin{proposition}
  Let \(L\) be a field extension of \(K\) and let \(\alpha, \beta \in L\) be 
  algebraic over \(K\). Then \(\alpha + \beta, \alpha \beta\) and \(\alpha^{-1}\) 
  (if \(\alpha \neq 0\)) are all algebraic over \(K\).
\end{proposition}
\begin{proof}
  By the above proposition, it suffices to show that \([K(\alpha)](\beta)\) 
  is finite over \(K\) since \(\alpha + \beta, \alpha \beta, \alpha^{-1} 
  \in [K(\alpha)](\beta)\). As \(\beta\) is algebraic over \(K\), there exists 
  some \(P \in K[X]\) such that \(P(\alpha) = 0\). Then, as 
  \(P \in K[X] \subseteq K(\alpha)[X]\), we have \(\beta\) is algebraic over 
  \(K(\alpha)\) implying \([K(\alpha)](\beta)\) is finite over \(K(\alpha)\) 
  and so \(K\) by transitivity.
\end{proof}

With the above proposition in mind, we will write \(K(\alpha, \beta)\) for 
\([K(\alpha)](\beta)\).

\begin{corollary}
  Let \(\overline{\mathbb{Q}} \subseteq \mathbb{C}\) denote the subset of elements of 
  \(\mathbb{C}\) algebraic over \(\mathbb{Q}\). Then \(\overline{\mathbb{Q}}\) is 
  a field.
\end{corollary}

\end{document}
