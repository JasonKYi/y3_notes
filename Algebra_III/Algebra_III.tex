% Options for packages loaded elsewhere
\PassOptionsToPackage{unicode}{hyperref}
\PassOptionsToPackage{hyphens}{url}
\PassOptionsToPackage{dvipsnames,svgnames*,x11names*}{xcolor}
%
\documentclass[]{article}
\usepackage{lmodern}
\usepackage{amssymb,amsmath}
\usepackage{ifxetex,ifluatex}
\ifnum 0\ifxetex 1\fi\ifluatex 1\fi=0 % if pdftex
  \usepackage[T1]{fontenc}
  \usepackage[utf8]{inputenc}
  \usepackage{textcomp} % provide euro and other symbols
\else % if luatex or xetex
  \usepackage{unicode-math}
  \defaultfontfeatures{Scale=MatchLowercase}
  \defaultfontfeatures[\rmfamily]{Ligatures=TeX,Scale=1}
\fi
% Use upquote if available, for straight quotes in verbatim environments
\IfFileExists{upquote.sty}{\usepackage{upquote}}{}
\IfFileExists{microtype.sty}{% use microtype if available
  \usepackage[]{microtype}
  \UseMicrotypeSet[protrusion]{basicmath} % disable protrusion for tt fonts
}{}
\makeatletter
\@ifundefined{KOMAClassName}{% if non-KOMA class
  \IfFileExists{parskip.sty}{%
    \usepackage{parskip}
  }{% else
    \setlength{\parindent}{0pt}
    \setlength{\parskip}{6pt plus 2pt minus 1pt}}
}{% if KOMA class
  \KOMAoptions{parskip=half}}
\makeatother
\usepackage{xcolor}\pagecolor[RGB]{28,30,38} \color[RGB]{213,216,218}
\IfFileExists{xurl.sty}{\usepackage{xurl}}{} % add URL line breaks if available
\IfFileExists{bookmark.sty}{\usepackage{bookmark}}{\usepackage{hyperref}}
\hypersetup{
  pdftitle={Algebra III},
  pdfauthor={Kexing Ying},
  colorlinks=true,
  linkcolor=Maroon,
  filecolor=Maroon,
  citecolor=Blue,
  urlcolor=red,
  pdfcreator={LaTeX via pandoc}}
\urlstyle{same} % disable monospaced font for URLs
\usepackage[margin = 1.5in]{geometry}
\usepackage{graphicx}
\makeatletter
\def\maxwidth{\ifdim\Gin@nat@width>\linewidth\linewidth\else\Gin@nat@width\fi}
\def\maxheight{\ifdim\Gin@nat@height>\textheight\textheight\else\Gin@nat@height\fi}
\makeatother
% Scale images if necessary, so that they will not overflow the page
% margins by default, and it is still possible to overwrite the defaults
% using explicit options in \includegraphics[width, height, ...]{}
\setkeys{Gin}{width=\maxwidth,height=\maxheight,keepaspectratio}
% Set default figure placement to htbp
\makeatletter
\def\fps@figure{htbp}
\makeatother
\setlength{\emergencystretch}{3em} % prevent overfull lines
\providecommand{\tightlist}{%
  \setlength{\itemsep}{0pt}\setlength{\parskip}{0pt}}
\setcounter{secnumdepth}{5}
\usepackage{tikz}
\usepackage{physics}
\usepackage{amsthm}
\usepackage{mathtools}
\usepackage{esint}
\usepackage[ruled,vlined]{algorithm2e}
\usepackage{tikz-cd}
\theoremstyle{definition}
\newtheorem{theorem}{Theorem}
\newtheorem{definition*}{Definition}
\newtheorem{prop}{Proposition}
\newtheorem{corollary}{Corollary}[theorem]
\newtheorem*{remark}{Remark}
\theoremstyle{definition}
\newtheorem{definition}{Definition}[section]
\newtheorem{lemma}{Lemma}[section]
\newtheorem{proposition}{Proposition}[section]
\newtheorem{example}{Example}[section]
\newcommand{\diag}{\mathop{\mathrm{diag}}}
\newcommand{\Arg}{\mathop{\mathrm{Arg}}}
\newcommand{\hess}{\mathop{\mathrm{Hess}}}

\title{Algebra III}
\author{Kexing Ying}
\date{July 24, 2021}

\begin{document}
\maketitle

\begin{center}
  \begin{minipage}{.75\textwidth}
    \textbf{N.B.} this course has large overlap with the second year course 
    \textit{Groups and Rings} in particular, the ring subsection. Thus, 
    most revisited proofs are simply omitted or replaced with a hint.
  \end{minipage}
\end{center}

{
\hypersetup{linkcolor=}
\setcounter{tocdepth}{2}
\tableofcontents
}
\newpage

\section{Fundamental Definitions}

We will in this section recall some fundamental definitions which 
we will study throughout the course.

\begin{definition}[Ring]
  A ring \(R\) is a set together with two distinct elements \(0_R, 1_R\), and 
  two binary operations \(+_R, \times_R : R^2 \to R\) such that 
  \begin{itemize}
    \item \((R, +_R)\) is an additive abelian group with identity \(0_R\);
    \item \((R, \times_R)\) is a multiplicative abelian monoid with identity \(1_R\);
    \item \(\times_R\) distributes over \(+_R\), i.e. for all \(r, s, t \in R\), 
      \[(r +_R s) \times_R t = r \times_R t +_R s \times_R t,\] and 
      \[r \times_R (s +_R t) = r \times_R s +_R r \times_R t.\]
  \end{itemize}
\end{definition}

We note that there is some ambiguity in the literature in the definition of a 
ring, and in particular, some might call the definition above as a commutative 
unital ring. We will in this course mostly consider ourselves with this definition, 
though we might later consider non-commutative rings.

\begin{definition}[Field]
  A field \(F\) is a ring is for all \(f \in F \setminus \{0_F\}\), there 
  exists some \(f^{-1} \in F\) such that \(f \times_F f^{-1} = 1_F\).
\end{definition}

We will simply drop the subscript from the operations and the elements from 
these definitions whenever there is no confusion. 

Recall that one method of constructing a ring from another is the polynomial 
ring. Let \(R\) be ring, then a polynomial on \(X\) is a sum 
\[\sum_{n = 0}^\infty a_n X^n\]
for some \((a_n)_{n \in \mathbb{N}} \subseteq R\) where all but finitely 
many \(a_i\) are zero. We say \(P(X) = \sum_{n = 0}^\infty a_n X^n\) has 
degree \(d\) if \(d\) is the largest number such that \(a_d \neq 0\).

\begin{definition}[Polynomial Ring]
  Given a ring \(R\), the polynomial ring \(R[X]\) is the set of polynomials 
  equipped with the operations \(+_{R[X]}\) and \(\times_{R[X]}\) such that 
  \[\sum_{n = 0}^\infty a_n X^n +_{R[X]} \sum_{n = 0}^\infty b_n X^n = 
    \sum_{n = 0}^\infty (a_n + b_n) X^n,\]
  and,
  \[\sum_{n = 0}^\infty a_n X^n \times_{R[X]} \sum_{n = 0}^\infty b_n X^n =
    \sum_{n = 0}^\infty \left(\sum_{i = 0}^n a_i b_{n - i} \right) X^n.\]
\end{definition}

It is not difficult to see that the ring axioms are satisfied and in fact, it 
is possible to construct polynomial rings with infinite degrees, though this 
shall not be considered in this course. An equivalent way of considering elements 
of polynomial rings is to see them as sequences with finite non-zero elements. 

One may adjoin a polynomial ring with another variable, that is 
\(R[X][Y]\) and by writing out the elements, we see that \(R[X][Y] \cong R[Y][X]\) 
and we may instead write \(R[X, Y]\) with no ambiguity. 

\subsection{Subrings and Extensions}

\begin{definition}[Subring]
  A subring of the ring \(R\) is a subset of \(R\) containing \(0, 1\) and is 
  closed under \(+\) and \(\times\).
\end{definition}

It is clear that a subring of a ring is a ring itself with the inherited 
operations. 

\begin{proposition}
  If \(S, T\) are subrings of the ring \(R\), then so is \(S \cap T\).
\end{proposition}

\begin{definition}
  Given a subring \(S\) of \(R\), \(S[\alpha]\) for some \(\alpha \in R\) is the 
  subset of \(R\) consisting of all elements of \(R\) that can be expressed as 
  \(r_0 + r_1 \alpha + \cdots + r_n \alpha^n\) for \(r_i \in S\) and 
  \(n \in \mathbb{N}\). We call this process the adjoining of \(S\) with \(\alpha\).
\end{definition}

Clearly \(S[\alpha]\) contains \(0\) and \(1\) (as \(S \subseteq S[\alpha]\)) 
and is closed under \(+\) and \(\times\), and thus, is a subring of \(R\).

An important example of the above construction is the following. Consider 
\(\mathbb{Z} \subseteq \mathbb{C}\), we have \(\mathbb{Z}[i]\) constructed 
through the definition above is known as the Gaussian integers is a subring 
of \(\mathbb{C}\) consisting of all elements of the form \(a + bi\) for 
\(a, b \in \mathbb{Z}\). To see this, consider if \(X^2 - r X - s\) is a polynomial 
of integer coefficients with complex root \(\alpha \notin \mathbb{Z}\), then, 
we may consider \(\mathbb{Z}[\alpha]\). As \(\alpha^2 - r \alpha - s = 0\), 
we obtain \(\alpha^2 = r \alpha + s\) and thus, for all 
\(r_0 + r_1 \alpha + \cdots + r_n \alpha^n \in \mathbb{Z}[\alpha]\),
\[\begin{split}
  r_0 + r_1 \alpha + r_2 \alpha^2 + \cdots + r_n \alpha^n 
  & = r_0 + r_1 \alpha + r_2 (r \alpha + s) + \cdots \\
  & = (r_0 + r_2 s + \cdots) + (r_1 + r_2 r + \cdots) \alpha.
\end{split}\]
Hence, all elements of \(\mathbb{Z}[\alpha]\) are of the form \(a + b \alpha\) 
for \(a, b \in \mathbb{Z}\).

On the other hand, if we consider \(\mathbb{Z}[\pi] \subseteq \mathbb{C}\), 
as \(\pi\) is not an algebraic number, for all \(P(X) \in \mathbb{Z}[X] \setminus \{0\}\), 
\(P(\pi) \neq 0\). Thus, if \(P(X), Q(X)\) are polynomials such that
\(P(\pi) = r_0 + r_1 \pi + \cdots + r_n \pi^n = s_0 + s_1 \pi + \cdots + s_m \pi^m = Q(\pi)\), 
WLOG. \(n \le m\) we have \(0 = (s_0 - r_0) + (s_1 - r_1) \pi + \cdots + 
(s_n - r_n) \pi^n + s_{n + 1} \pi^{n + 1} + \cdots + s_m \pi^{m + 1}\), implying 
\(s_i = r_i\) for all \(i = 1, \cdots, n\) and \(s_i = 0\) for \(i > n\), we have 
\(P = Q\). Hence, \(\mathbb{Z}[\pi] \cong \mathbb{Z}[X]\).

\begin{proposition}
  If \(R\) is a subring of \(S\), then \(R[\alpha]\) for some \(\alpha \in S\) 
  is the intersection of all subrings of \(S\) containing \(R \cup \{\alpha\}\).
\end{proposition}
\begin{proof}
  Since \(R[\alpha]\) contains both \(R\) and \(\alpha\), we have 
  \[\bigcap \{U \mid R \cup \{\alpha\} \subseteq U \le S\} \subseteq R[\alpha].\]
  On the other hand, for all subrings \(U\) containing \(R \cup \{\alpha\}\), 
  \(R[\alpha] \subseteq U\) as \(U\) is closed under \(+\) and \(\times\). 
  Thus, 
  \[\bigcap \{U \mid R \cup \{\alpha\} \subseteq U \le S\} = R[\alpha].\]
\end{proof}

\begin{definition}[Integral Domain]
  A ring \(R\) is an integral domain if for all \(r, s \in R\), \(rs = 0\) 
  implies \(r = 0\) or \(s = 0\).
\end{definition}

In particular, we say \(r \in R\) is a zero divisor if there exists a 
\(s \in R \setminus \{0\}\) such that \(rs = 0\). Thus, an integral domain 
is simply a ring with no zero divisors. 

\begin{definition}[Field of Fractions]
  For \(R\) an integral domain, then the field of fractions of \(R\) denoted 
  \(\text{Frac}(R)\), is \(R \times R\setminus\{0\}\) quotiented by the 
  equivalence class 
  \[(a, b) \sim (r, s) \iff as = br.\]
  We write \(a / b\) as a representative of the equivalence class \([a, b]\).
\end{definition}

We may equip the field of fractions of \(R\) with addition and multiplication 
such that for \(a / b, r / s \in \text{Frac}(R)\) 
\[\frac{a}{b} + \frac{r}{s} = \frac{ad + bc}{bd} \text{ and } 
  \frac{a}{b} \times \frac{r}{s} = \frac{ar}{bs}.\]
It is routine to check these operations are well-defined and that the ring 
axioms are satisfied. Furthermore, as the name suggests, \(\text{Frac}(R)\) 
is a field and for all \(a / b \neq 0\), \((a / b) \times (b / a) = 1\).

\begin{definition}[Multiplicative System]
  A set \(S \subseteq R\) is a multiplicative system if \(1 \in S, 0 \notin S\) 
  and is closed under multiplication.
\end{definition}

\begin{definition}
  Let \(R\) be a ring and \(S \subseteq R\) be a multiplicative system. Then 
  \(S^{-1}R\) is \(R \times S\) quotiented by the equivalence class 
  \[(a, b) \sim (r, s) \iff as = br\]
  for \(a, r \in R, b, s \in S\).
\end{definition}

Similarly, we may equip \(S^{-1}R\) with addition and multiplication such that 
\(S^{-1}R\) is a subring of \(\text{Frac}(R)\). 

It is possible to use this construction on rings which are not integral domains, 
though in that case, the equivalence class is more subtle as division by a 
zero divisor will introduces other elements into the subring. This will be explored 
later in this course.

\subsection{Homomorphisms and Ideals}

We recall the definition of ring homomorphism and some related results 
(whose proofs omitted or shortened).

\begin{definition}[Ring Homomorphism]
  Given \(R, S\) rings, a ring homomorphism from \(R\) to \(S\) is a map 
  \(f : R \to S\) such that for all \(a, b \in R\), 
  \begin{itemize}
    \item \(f(1_R) = 1_S\);
    \item \(f(a +_R b) = f(a) +_S f(b)\);
    \item \(f(a \cdot_R b) = f(a) \cdot_S f(b)\).
  \end{itemize}
  If \(f\) is a bijection then we say \(f\) is an isomorphism.
\end{definition}

Automatically, it is not difficult to see that condition 2 implies \(f(0_R) = 0_S\) 
and from this we can deduece properties such as \(f(-x) = -f(x)\).

\begin{proposition}
  The image of a ring homomorphism \(f : R \to S\) is a subring of \(S\).
\end{proposition}

As we have seen in other contexts, the notion of an isomorphism is often defined 
to be a invertible structure preserving map. Though in some contexts, such as 
topological spaces, bijection is often not enough and we will require the inverse 
to be structure preserving. The following proposition shows that these two 
cases are equivalent for rings.

\begin{proposition}
  If \(f : R \to S\) is an isomorphism, then \(f^{-1} : S \to R\) is a ring 
  homomorphism.
\end{proposition}
\begin{proof}
  For all \(a, b \in S\), we have \(f^{-1}(a + b) = 
  f^{-1}(f(f^{-1}(a)) + f(f^{-1}(b))) = f^{-1}(f(f^{-1}(a) + f^{-1}(b)))
  = f^{-1}(a) + f^{-1}(b)\). Similar argument for the other conditions.
\end{proof}

\begin{proposition}
  There exist a unique homomorphism from \(\mathbb{Z}\) to \(R\) for all ring 
  \(R\).
\end{proposition}
\begin{proof}
  Clear by considering if \(f : \mathbb{Z} \to R\) is a homomorphism, 
  \(f(n_{\mathbb{Z}}) = n_{\mathbb{Z}} \cdot 1_R\). 
\end{proof}

\begin{proposition}
  Given a ring \(R\) and \(\alpha \in R\), there exists a unique homomorphism 
  \(f : R[X] \to R\) such that \(f(X) = \alpha\) and \(f\mid_R = \text{id}_R\).
  This homomorphism is called the evaluation map at \(\alpha\) and we denote it 
  as \(\text{ev}_\alpha\).
\end{proposition}
\begin{proof}
  Clear and as the name suggests, the unique map is 
  \[\text{ev}_\alpha(P(X)) = P(\alpha),\]
  for all \(P \in R[X]\).
\end{proof}

More generally, if \(f : R \to S\) is a homomorphism and \(\alpha \in S\), there 
exists a unique \(\text{ev}_{f, \alpha} : R[X] \to S\) such that
\(\text{ev}_{f, \alpha} \mid_R = f\) and \(\text{ev}_{f, \alpha}(X) = \alpha\). 
Furthermore, if \(f\) is simply the inclusion map from \(R \to S\), image of 
\(\text{ev}_{f, \alpha}(X) = \alpha\) is \(R[\alpha]\).

\begin{definition}[Kernel]
  Let \(R, S\) be rings and \(f : R \to S\) a ring homomorphism. Then the 
  kernel of \(f\) is 
  \[\ker f := \{r \in R \mid f(r) = 0_S\}.\]
\end{definition}

\begin{proposition}
  A ring homomorphism \(f : R \to S\) is injective if and only if 
  \(\ker f = \{0\}\).
\end{proposition}

\begin{definition}[Ideal]
  Given a subset \(I\) of a ring \(R\), then \(I\) is said to be an ideal if 
  \begin{itemize}
    \item \(0_R \in I\);
    \item for all \(a, b \in I\) then \(a + b \in I\);
    \item for all \(a \in I\), \(r \in R\), \(ra \in I\).
  \end{itemize}
\end{definition}

\begin{definition}
  The following ideals are important enough to warrant a definition.
  \begin{itemize}
    \item \(\{0_R\} \subseteq R\) is the zero ideal;
    \item \(R \subset R\) is the unit idea;
    \item for all \(r \in R\), \(\langle r \rangle := \{rs \mid s \in R\}\) 
      is the principal ideal generated by \(r\).
  \end{itemize}
\end{definition}

\begin{proposition}
  Every ideal of \(\mathbb{Z}\) is principle.
\end{proposition}

\begin{proposition}
  In intersection of ideals is an ideal. Similarly, the sum of two ideals, 
  i.e. if \(I, J\) are ideals, then \(\{i + j \mid i \in I, j \in J\}\) is an 
  ideal.
\end{proposition}

\begin{definition}
  Let \(R\) be a ring and \(r_1, \cdots, r_n \in R\). Then the ideal generated 
  by \(r_1, \cdots, r_n\) is 
  \[\langle r_1, \cdots, r_n \rangle := 
    \{r_1 s_1 + \cdots r_n s_n \mid s_i \in R\}.\]
\end{definition}

It is clear that the ideal generated by \(r_1, \cdots, r_n\) is the smallest 
ideal containing \(r_1, \cdots, r_n\).

\begin{definition}
  The produce of ideals \(I\) and \(J\) is the ideal which elements are
  of the form \(i_1j_1 + \cdots + i_nj_n\) for all \(i_1, \cdots i_n \in I\), 
  \(j_1, \cdots, j_n \in J\).
\end{definition}

For ideals \(I, J\), we see that \(IJ \subseteq I \cap J\) though they are not 
necessary equal (consider \(\langle 2 \rangle \langle 2 \rangle = \langle 4 
\rangle\) thought \(\langle 2 \rangle \cap \langle 2 \rangle = \langle 2 \rangle\)).

\begin{proposition}
  If ideals \(I, J\) satisfy \(I + J = \langle 1 \rangle\), then 
  \(I \cap J = IJ\).
\end{proposition}

As with other mathematical objects, we would like to construct a quotient 
object for the rings. The equivalence relation we shall quotient on it the 
following. Let \(I \subseteq R\) be an ideal and we define say
\(r \equiv s \mod I\) for \(r, s \in R\) if \(r - s \in I\). It is not difficult 
to check that \(\equiv_I\) is a equivalence relation and thus, we may take 
a quotient of \(R\) with respect to this equivalence relation and we denote 
the equivalence classes with \(r + I\).

\begin{definition}[Quotient Ring]
  Given \(R\) a ring and \(I\) an ideal of \(R\), then the quotient ring 
  of \(R\) by \(I\) is the ring with the underlying set 
  \[R / I := R / \equiv_I = \{r + I \mid r \in R\},\]
  where \(0_{R/I} = 0_R + I\), \(1_{R/I} = 1_R + I\), and for all 
  \(r + I, s + I \in R / I\), \((r + I) + (s + I) = (r + s) + I\) 
  and \((r + I) \cdot (s + I) = rs + I\).
\end{definition}

\begin{definition}[Quotient Map]
  Given \(R\) a ring and \(I\) an ideal of \(R\), the quotient map is then the 
  surjective ring homomorphism \(q : R \to R / I : r \mapsto r + I\).
\end{definition}

It is clear that \(\ker q = I\).

A more modern interpretation of the quotient ring is by defining it as an 
object satisfying its universal property. In particular, the ring \(R / I\),
taken together with a ring homomorphism \(q : R \to R / I\), has the following 
universal property.

\begin{proposition}
  If \(f : R \to S\) is a ring homomorphism such that \(I \subseteq \ker f\), 
  then there exists a unique ring homomorphism \(\tilde f : R / I \to S\)
  such that for all \(r \in R\), \(\tilde f(r + I) = f(r)\).
\end{proposition}

Essentially, the universal property states that there exists a unique \(\tilde f\) 
such that the following diagram commutes.
\[\begin{tikzcd}
  R \arrow[r, "f"] \arrow[swap, d, "q"] & S \\
  R / S \arrow[swap, dashed, ru, "\tilde f"]
  \end{tikzcd}\]
\begin{proof}
  Uniqueness is clear and thus we will show \(\tilde f\) is well-defined and 
  is a ring homomorphism. Let \(r \equiv s \mod I\), and will show \(f(r) = f(s)\). 
  Indeed, since \(r - s \in I\), we have \(r - s \in \ker f\) and so, 
  \(f(r) - f(s) = f(r - s) = 0\), hence \(f(r) = f(s)\) and \(\tilde f\) is 
  well-defined. Now, let \(r + I, s + I \in R / I\), we have 
  \[\tilde f((r + I) + (s + I)) = \tilde f((r + s) + I) = f(r + s) = 
    f(r) + f(s) = \tilde f(r + s) + \tilde f(s + I),\]
  hence by similar argument for multiplication, we have \(\tilde f\) is a ring 
  homomorphism.
\end{proof}

As an example consider the surjective map \(\mathbb{R}[X] \to \mathbb{C}\) 
which is \(\text{id}\) on \(\mathbb{R}\) and sends \(X\) to \(i\). Then 
this map have kernel \(\{P \in \mathbb{R}[X] \mid P(i) = 0\} = 
\langle X^2 + 1 \rangle\). Thus, we have the diagram 
\[\begin{tikzcd}
  \mathbb{R}[X] \arrow[d, swap, "q"] \arrow[r] & \mathbb{C}\\
  \mathbb{R}[X] / \langle X^2 + 1 \rangle \arrow[swap, leftrightarrow, ru]
\end{tikzcd}\]
where the pull-back map is an isomorphism as the map itself is surjective while 
injectivity follows as we have quotiented out its kernel. As we shall see, 
whenever we have one field inside another, there is a construction similar this 
such that we can construct the larger field from the smaller field.

By recalling the evaluation map, if \(\alpha \in R\), by the above process, we 
see that 
\[R[X] / I \cong R[\alpha],\]
where \(I\) is the kernel of the evaluation map at \(\alpha\).

\begin{definition}
  Let \(R\) be a ring and \(I\) an ideal of \(R\). Then we say \(I\) is a 
  prime ideal if \(R / I\) is an integral domain. Furthermore, we say 
  \(I\) is a maximal ideal if \(R / I\) is a field.
\end{definition}

Since fields are integral domains, maximal ideals are prime.

\begin{proposition}
  An ideal \(I\) of \(R\) is prime if and only if for all \(rs \in I\), either 
  \(r \in I\) or \(s \in I\).
\end{proposition}

\begin{proposition}
  An ideal \(I\) of \(R\) is maximal if and only if the only ideal of \(R\) 
  containing \(I\) is \(I\) or the unit ideal \(R\).
\end{proposition}
\begin{proof}
  Follows by considering that a ring is a field if and only if its only ideals 
  are the zero or the unit ideal, and the image of an ideal by a surjective 
  homomorphism is also an ideal.
\end{proof}

\end{document}
