% Options for packages loaded elsewhere
\PassOptionsToPackage{unicode}{hyperref}
\PassOptionsToPackage{hyphens}{url}
\PassOptionsToPackage{dvipsnames,svgnames*,x11names*}{xcolor}
%
\documentclass[]{article}
\usepackage{lmodern}
\usepackage{amssymb,amsmath}
\usepackage{ifxetex,ifluatex}
\ifnum 0\ifxetex 1\fi\ifluatex 1\fi=0 % if pdftex
  \usepackage[T1]{fontenc}
  \usepackage[utf8]{inputenc}
  \usepackage{textcomp} % provide euro and other symbols
\else % if luatex or xetex
  \usepackage{unicode-math}
  \defaultfontfeatures{Scale=MatchLowercase}
  \defaultfontfeatures[\rmfamily]{Ligatures=TeX,Scale=1}
\fi
% Use upquote if available, for straight quotes in verbatim environments
\IfFileExists{upquote.sty}{\usepackage{upquote}}{}
\IfFileExists{microtype.sty}{% use microtype if available
  \usepackage[]{microtype}
  \UseMicrotypeSet[protrusion]{basicmath} % disable protrusion for tt fonts
}{}
\makeatletter
\@ifundefined{KOMAClassName}{% if non-KOMA class
  \IfFileExists{parskip.sty}{%
    \usepackage{parskip}
  }{% else
    \setlength{\parindent}{0pt}
    \setlength{\parskip}{6pt plus 2pt minus 1pt}}
}{% if KOMA class
  \KOMAoptions{parskip=half}}
\makeatother
\usepackage{xcolor}\pagecolor[RGB]{28,30,38} \color[RGB]{213,216,218}
\IfFileExists{xurl.sty}{\usepackage{xurl}}{} % add URL line breaks if available
\IfFileExists{bookmark.sty}{\usepackage{bookmark}}{\usepackage{hyperref}}
\hypersetup{
  pdftitle={Algebraic Topology},
  pdfauthor={Kexing Ying},
  colorlinks=true,
  linkcolor=Maroon,
  filecolor=Maroon,
  citecolor=Blue,
  urlcolor=red,
  pdfcreator={LaTeX via pandoc}}
\urlstyle{same} % disable monospaced font for URLs
\usepackage[margin = 1.5in]{geometry}
\usepackage{graphicx}
\makeatletter
\def\maxwidth{\ifdim\Gin@nat@width>\linewidth\linewidth\else\Gin@nat@width\fi}
\def\maxheight{\ifdim\Gin@nat@height>\textheight\textheight\else\Gin@nat@height\fi}
\makeatother
% Scale images if necessary, so that they will not overflow the page
% margins by default, and it is still possible to overwrite the defaults
% using explicit options in \includegraphics[width, height, ...]{}
\setkeys{Gin}{width=\maxwidth,height=\maxheight,keepaspectratio}
% Set default figure placement to htbp
\makeatletter
\def\fps@figure{htbp}
\makeatother
\setlength{\emergencystretch}{3em} % prevent overfull lines
\providecommand{\tightlist}{%
  \setlength{\itemsep}{0pt}\setlength{\parskip}{0pt}}
\setcounter{secnumdepth}{5}
\usepackage{tikz}
\usepackage{physics}
\usepackage{amsthm}
\usepackage{mathtools}
\usepackage{esint}
\usepackage[ruled,vlined]{algorithm2e}
\theoremstyle{definition}
\newtheorem{theorem}{Theorem}
\newtheorem{definition*}{Definition}
\newtheorem{prop}{Proposition}
\newtheorem{corollary}{Corollary}[theorem]
\newtheorem*{remark}{Remark}
\theoremstyle{definition}
\newtheorem{definition}{Definition}[section]
\newtheorem{lemma}{Lemma}[section]
\newtheorem{proposition}{Proposition}[section]
\newtheorem{example}{Example}[section]
\newcommand{\diag}{\mathop{\mathrm{diag}}}
\newcommand{\Arg}{\mathop{\mathrm{Arg}}}
\newcommand{\hess}{\mathop{\mathrm{Hess}}}

\title{Algebraic Topology}
\author{Kexing Ying}

\begin{document}
\maketitle

{
\hypersetup{linkcolor=}
\setcounter{tocdepth}{2}
\tableofcontents
}
\newpage

\section{Introduction}

Let us introduce/recall some basic definitions which will be used throughout this 
course. 

\begin{definition}[Path]
  A path in a topological space \(X\) is a continuous map \(\gamma : [0, 1] \subseteq \mathbb{R} \to X\).
  In the case that \(\gamma(0) = \gamma(1)\), we call \(\gamma\) a loop/closed path.
\end{definition}

\begin{definition}[Homotopy]
  Given two paths \(\gamma_0, \gamma_1 : [0, 1] \to X\) with the same end points 
  (i.e. \(\gamma_0(0) = \gamma_1(0)\) and \(\gamma_0(1) = \gamma_1(1)\) are 
  said to be homotopic with fixed endpoints if there exists a continuous map 
  \[H : [0, 1] \times [0, 1] \to X\]
  such that 
  \begin{itemize}
    \item \(H(t, 0) = \gamma_0(t)\) for all \(t \in [0, 1]\);
    \item \(H(t, 1) = \gamma_1(t)\) for all \(t \in [0, 1]\);
    \item for all \(u \in [0, 1]\), 
    \(H(0, u) = \gamma_0(0) = \gamma_1(0)\) and \(H(1, u) = \gamma_0(1) = \gamma_1(1)\).
  \end{itemize}
\end{definition}

Thus, graphically, two paths are homotopic if you can continuously deform a path 
into the other without moving the starting and ending points (see second year 
complex analysis for more details).

\begin{definition}[Free Homotopy]
  The loops \(\gamma_0, \gamma_1\) in \(X\) is said to be freely homotopic if 
  there exists a continuous \(H : [0, 1] \times [0, 1] \to X\) such that 
  \begin{itemize}
    \item \(H(t, 0) = \gamma_0(t)\) for all \(t \in [0, 1]\);
    \item \(H(t, 1) = \gamma_1(t)\) for all \(t \in [0, 1]\);
    \item for all \(u \in [0, 1]\), \(H(0, u) = H(1, u)\).
  \end{itemize}
\end{definition}

\begin{definition}[Simply Connected]
  \(X\) is said to be simply connected if any loop in \(X\) is freely homotopic 
  to a constant loop. 
\end{definition}

Thus, informally, in a simply connected space, any loop can be contracted into 
a single point.

\begin{proposition}
  \(S^2\) is simply connected.
\end{proposition}

Simply connectedness is a important notion and relates to many difficult problems 
in geometry. 

\begin{theorem}
  \(S^2\) and \(\mathbb{R}^2\) are, up to homeomorphism, the only two simply-connected 
  2-dimensional manifolds.
\end{theorem}

\begin{theorem}[Poincaré Conjecture]
  The only compact, simply connected 3-dimensional manifold is the sphere \(S^3\) 
  (up to homeomorphism).
\end{theorem}

\end{document}
