% Options for packages loaded elsewhere
\PassOptionsToPackage{unicode}{hyperref}
\PassOptionsToPackage{hyphens}{url}
\PassOptionsToPackage{dvipsnames,svgnames*,x11names*}{xcolor}
%
\documentclass[
]{article}
\usepackage{lmodern}
\usepackage{amssymb,amsmath}
\usepackage{ifxetex,ifluatex}
\ifnum 0\ifxetex 1\fi\ifluatex 1\fi=0 % if pdftex
  \usepackage[T1]{fontenc}
  \usepackage[utf8]{inputenc}
  \usepackage{textcomp} % provide euro and other symbols
\else % if luatex or xetex
  \usepackage{unicode-math}
  \defaultfontfeatures{Scale=MatchLowercase}
  \defaultfontfeatures[\rmfamily]{Ligatures=TeX,Scale=1}
\fi
% Use upquote if available, for straight quotes in verbatim environments
\IfFileExists{upquote.sty}{\usepackage{upquote}}{}
\IfFileExists{microtype.sty}{% use microtype if available
  \usepackage[]{microtype}
  \UseMicrotypeSet[protrusion]{basicmath} % disable protrusion for tt fonts
}{}
\makeatletter
\@ifundefined{KOMAClassName}{% if non-KOMA class
  \IfFileExists{parskip.sty}{%
    \usepackage{parskip}
  }{% else
    \setlength{\parindent}{0pt}
    \setlength{\parskip}{6pt plus 2pt minus 1pt}}
}{% if KOMA class
  \KOMAoptions{parskip=half}}
\makeatother
\usepackage{xcolor}
\IfFileExists{xurl.sty}{\usepackage{xurl}}{} % add URL line breaks if available
\IfFileExists{bookmark.sty}{\usepackage{bookmark}}{\usepackage{hyperref}}
\hypersetup{
  pdftitle={Functional Analysis},
  pdfauthor={Kexing Ying},
  colorlinks=true,
  linkcolor=Maroon,
  filecolor=Maroon,
  citecolor=Blue,
  urlcolor=red,
  pdfcreator={LaTeX via pandoc}}
\urlstyle{same} % disable monospaced font for URLs
\usepackage[margin = 1.5in]{geometry}
\usepackage{graphicx}
\makeatletter
\def\maxwidth{\ifdim\Gin@nat@width>\linewidth\linewidth\else\Gin@nat@width\fi}
\def\maxheight{\ifdim\Gin@nat@height>\textheight\textheight\else\Gin@nat@height\fi}
\makeatother
% Scale images if necessary, so that they will not overflow the page
% margins by default, and it is still possible to overwrite the defaults
% using explicit options in \includegraphics[width, height, ...]{}
\setkeys{Gin}{width=\maxwidth,height=\maxheight,keepaspectratio}
% Set default figure placement to htbp
\makeatletter
\def\fps@figure{htbp}
\makeatother
\setlength{\emergencystretch}{3em} % prevent overfull lines
\providecommand{\tightlist}{%
  \setlength{\itemsep}{0pt}\setlength{\parskip}{0pt}}
\setcounter{secnumdepth}{5}
\usepackage{tikz}
\usepackage{physics}
\usepackage{amsthm}
\usepackage{mathtools}
\usepackage{esint}
\usepackage[ruled,vlined]{algorithm2e}
\theoremstyle{definition}
\newtheorem{theorem}{Theorem}
\newtheorem{prop}{Proposition}
\newtheorem{corollary}{Corollary}[theorem]
\newtheorem*{remark}{Remark}
\theoremstyle{definition}
\newtheorem{definition}{Definition}[section]
\newtheorem{lemma}{Lemma}[section]
\newtheorem{proposition}{Proposition}[section]
\newtheorem{example}{Example}[section]
\newcommand{\diag}{\mathop{\mathrm{diag}}}
\newcommand{\Arg}{\mathop{\mathrm{Arg}}}
\newcommand{\hess}{\mathop{\mathrm{Hess}}}

\title{Functional Analysis}
\author{Kexing Ying}
\date{May 30, 2021}

\begin{document}
\maketitle

{
\hypersetup{linkcolor=}
\setcounter{tocdepth}{2}
\tableofcontents
}
\newpage

\section{Basic Concepts}

\subsection{Basic Definitions}

Unless stated otherwise, we assume all topologies are Hausdorff and we denote 
\(\mathbb{K}\) for the field \(\mathbb{R}\) or \(\mathbb{C}\).

\begin{definition}[Banach Space]
  A Banach space is a normed linear space \((X, \|\cdot\|)\) that is complete 
  with respect to the induced metric.
\end{definition}

\begin{definition}[Linear Hull]
  Let \((X, \|\cdot\|)\) be a normed space and \(M \subseteq X\). Then, the 
  linear hull of \(M\), denoted by \(\text{span}(M)\), is the intersection of 
  all linear subspaces containing \(M\).
\end{definition}

\begin{definition}[Closed and Open Segment]
  If \(E\) is a vector space and \(x, y \in E\), then 
  \[[x, y] := \{\lambda x + (1 - \lambda) y \mid \lambda \in [0, 1]\},\] 
  is called the closed segment defined by \(x\) and \(y\). Similarly, 
  \[(x, y) := \{\lambda x + (1 - \lambda) y \mid \lambda \in (0, 1)\},\] 
  is called the open segment defined by \(x\) and \(y\).
\end{definition}

\begin{definition}[Convex]
  Give a vector space \(E\), a set \(C \subseteq E\) is convex if 
  \([x, y] \subseteq C\) for all \(x, y \in C\).
\end{definition}

\begin{definition}[Convex Hull]
  If \(M \subseteq X\), the convex hull of \(M\), denoted \(\text{conv}(M)\) is 
  smallest convex subset of \(X\) containing \(M\), i.e. the intersection of 
  all convex subsets of \(X\) containing \(M\).
\end{definition}

\begin{definition}[Convex Function]
  Given a convex subset \(U\) of the vector space \(V\), the function 
  \(f : U \to \mathbb{R}\) is said to be convex if 
  \[f(\lambda x + (1 - \lambda) y) \le \lambda f(x) + (1 - \lambda) f(y),\]
  for all \(x, y \in U\) and \(\lambda \in [0, 1]\). Furthermore, we say 
  \(f\) is strictly convex if 
  \[f(\lambda x + (1 - \lambda) y) < \lambda f(x) + (1 - \lambda) f(y),\]
  for all \(x, y \in U\), \(x \neq y\), \(\lambda \in (0, 1)\).
\end{definition}

By the triangle inequality, we observe that the norm of a normed space is 
a convex function on that space. 

\begin{definition}[Symmetric and Balanced]
  A set \(M \subseteq X\) is called symmetric if \((-1) M \subseteq M\) and 
  balanced if \(\alpha M \subseteq M\) for all \(\alpha \in \mathbb{K}, 
  |\alpha| \le 1\).
\end{definition}

\begin{proposition}
  Let \(Y \le X\) where \((X, \|\cdot\|)\) is a Banach space. Then \(Y\) is 
  a Banach space if and only if \(Y\) is closed in \(X\). 
\end{proposition}
\begin{proof}
  Clearly, if \(Y\) is closed, it contains all the limits of all Cauchy 
  sequences, and thus, is Banach. On the other hand, as all convergent sequences 
  are Cauchy, \(Y\) contains its limit points, and thus is closed.
\end{proof}

\begin{definition}[Bounded]
  A subset \(M\) of a normed space \((X, \|\cdot\|)\) is called bounded if 
  there exists some \(r > 0\) such that \(M \subseteq rB_X\) (where \(B_X\) is 
  the unit ball in \(X\)). \(M\) is called totally bounded if for every 
  \(\epsilon > 0\), \(M\) can be can be covered by a finite number of 
  \(\epsilon\) balls. A sequence \((x_n) \subseteq X\) is bounded if the 
  set \(\{x_n \mid n \in \mathbb{N}\}\) is bounded.
\end{definition}

\begin{proposition}
  Every totally bounded set is bounded.
\end{proposition}
\begin{proof}
  Take \(\epsilon = 1\) and suppose \(M\) is covered by \(n\) balls of radius 
  \(1\), \((B_1(x_i))_{i = 1}^n\). Then, we have for all \(x \in M\), 
  there exists some \(i \in \{1, \cdots, n\}\) such that \(x \in B_1(x_i)\), 
  and thus,
  \[\|x\| = \|x - x_i + x_i\| \le \|x - x_i\| + \|x_i\| < 1 + \|x_i\| \le 
    1 + \max_{i = 1, \cdots, n} \|x_i\|.\]
  Hence, choosing \(r = 1 + \max_{i = 1, \cdots, n} \|x_i\|\) suffices.
\end{proof}

Alternatively, total boundedness can be described using \(\epsilon\)-nets. 

\begin{definition}[\(\epsilon\)-net and \(\epsilon\)-separated]
  Let \(M\) be a subset of the normed space \((X, \|\cdot\|)\). Then, 
  \(A \subseteq M\) is an \(\epsilon\)-net in \(M\) if 
  \[M \subseteq \bigcup_{x \in A} B_\epsilon(x).\]
  That is to say, all elements of \(M\) are at most \(\epsilon\) away from 
  \(A\). 
  
  On the other hand \(A \subseteq M\) is said to be \(\epsilon\)-separated in 
  \(M\) if \(\|x - y\| \ge \epsilon\) for all \(x, y \in A\), \(x \neq y\). 
\end{definition}

Clearly, we see that a set \(M \subseteq X\) is totally bounded if and only if 
for every epsilon there exists a finite \(\epsilon\)-net.
Furthermore, we see that a maximal \(\epsilon\)-separated set is an \(\epsilon\)-net.

\begin{proposition}
  Let \((X, \|\cdot\|)\) be a \(n\)-dimensional real normed space and suppose
  \((x_j)_{j = 1}^N\) is an \(\epsilon\)-net of \(B_X\). Then, 
  \(N \ge \epsilon^{-n}\).
\end{proposition}
\begin{proof}
  Let \(T\) be the natural linear isomorphism between \(X\) and \(\mathbb{R}^n\) 
  and denote \(\lambda\) the standard Lebesgue measure on \(\mathbb{R}^n\). Then, 
  \[\lambda(T(B_X)) \le \lambda\left(T\left(\bigcup_{i = 1}^N B_\epsilon(x_i)\right)\right)
  = \lambda\left(\bigcup_{i = 1}^N T(B_\epsilon(x_i))\right) 
  \le \sum_{i = 1}^N \lambda(T(B_\epsilon(x_i))).\]
  Now, since \(T(B_\epsilon(x_i)) = T(B_\epsilon(0)) + T(x_i)\), we have 
  \[\sum_{i = 1}^N \lambda(T(B_\epsilon(x_i))) = \sum \lambda(T(B_\epsilon(0) + T(x_i))) 
  = N \lambda(\epsilon T(B_X)) = N \epsilon^{-n} \lambda(T(B_X)),\]
  and the result follows.
\end{proof}

\begin{proposition}
  On the other hand, there is an \(\epsilon\)-net \((x_j)_{j = 1}^N\) with 
  \(N \le (1 + 2 / \epsilon)^n\).
\end{proposition}
\begin{proof}
  Since the maximal \(\epsilon\)-separated set is an \(\epsilon\)-net we will 
  show the existence of a maximal \(\epsilon\)-separated set with 
  \(N \le (1 + 2 / \epsilon)^n\) elements. Let \((x_j)_{j = 1}^N\) be a 
  maximal \(\epsilon\)-separated set (in which we add points sequentially 
  until it is impossible to do so), then, it is clear that 
  \(B_{\epsilon / 2}(x_i) \cap B_{\epsilon / 2}(x_j) = \varnothing\) for all 
  \(i \neq j\). Thus, as \(B_{\epsilon / 2}(x_i) \subseteq (1 + \epsilon / 2)B_X\), 
  \[N (2 / \epsilon)^{n} \lambda(T(B_X)) = 
    \lambda\left(T\left(\bigcup_{i = 1}^N B_{\epsilon / 2}(x_i)\right)\right) 
    \lambda((1 + \epsilon / 2)T(B_X)) = (1 + \epsilon / 2)^n \lambda(T(B_X)).\]
  With that, by rearranging, the result follows.
\end{proof}

\subsection{Classical Spaces}

We recall some results from measure theory including Hölder and Minkowski's 
inequalities. 

\begin{proposition}
  \((C[0, 1], \|\cdot\|_\infty)\) is a Banach space (where \(C_[0, 1]\) is the 
  vector space of continuous functions \(f : [0, 1] \to \mathbb{K}\) and 
  \(\|\cdot\|_\infty\) is the supremum norm).
\end{proposition}
\begin{proof}
  See first year analysis.
\end{proof}

By the same argument, the space \(C(K)\) where \(K\) is compact, endowed with 
the supremum norm is also a Banach space.

It is easy to see \(C[0, 1]\) is infinite dimensional (consider the standard 
basis for the polynomials) and more generally, \(C(K)\) for compact \(K\) is 
infinite dimensional as soon as \(K\) is infinite (this follows by applying 
the Tietze-Urysohn theorem on distinct Kronecker delta functions).

\begin{definition}[Sequence Space]
  \(\ell^n_\infty\) is the \(n\)-dimensional vector space of all 
  \(n\)-tuples of \(\mathbb{K}\) endowed with the supremum norm 
  \(\|\cdot\|_\infty\).
\end{definition}

We note that \(\ell^n_\infty\) is a special case of \(C(K)\) endowed with the 
discrete topology.

By considering Hölder and Minkowski's inequalities for the sequence spaces, 
it is clear that we may define the classical \(\ell^n_p\) spaces.

\begin{definition}[Sequence Space]
  \(\ell^n_p\) for some \(p \in [1, \infty)\) is the \(n\)-dimensional vector 
  space of all \(n\)-tuples of \(\mathbb{K}\) endowed with the norm 
  \(\|\cdot\|_p\) such that 
  \[\|x\|_p := \left(\sum_{i = 1}^n |x_i|^p \right)^{\frac{1}{p}}.\]
  Furthermore, we denote \(\ell_p = \ell_p(\mathbb{N})\) for the vector space 
  of all scalar valued sequences \(x = (x_i)_{i = 1}^\infty\) satisfying 
  \(\sum |x_i| < \infty\) endowed with the norm  
  \[\|x\|_p := \left(\sum_{i = 1}^\infty |x_i|^p \right)^{\frac{1}{p}}.\]
  In the case \(p = \infty\), we define 
  \[\|x\|_\infty = \sup\{|x_i| \mid i \in \mathbb{N}\}.\]
  We denote \(c_{00} = c_{00}(\mathbb{N})\) for the subspace of \(\ell_\infty\) 
  consisting of the sequences with finite support. We also \(c = c(\mathbb{N})\) 
  for the subspace of convergent sequences. Lastly, we denote 
  \(c_0 = c_0(\mathbb{N})\) for the subspace of sequences convergent to 0.

  It is clear that \(c_{00} \subseteq c_0 \subseteq c\).
\end{definition}

\begin{proposition}~
  \begin{itemize}
    \item For \(p \in [1, \infty]\), \(\ell_p\) is a Banach space;
    \item \(c\) and \(c_0\) are closed subspaces of \(\ell_\infty\) (and hence, 
      also Banach);
    \item \(c_{00}\) is not complete.
  \end{itemize}
\end{proposition}
\begin{proof}
  The first claim is similar to that for the function spaces while the other 
  two are simple analysis proofs.
\end{proof}

We introduce the following generalisation for the above notion. We denote 
\(\Gamma\) an arbitrary non-empty set and \(\ell_p(\Gamma)\) the set of functions 
\(f : \Gamma \to \mathbb{K}\) such that \(\sum_{\gamma \in \Gamma} 
|f(\gamma)|^p < \infty\) equipped with the norm 
\[\|f\|_p := \sum_{\gamma \in \Gamma} |f(\gamma)|^p,\]
where we define 
\[\sum_{\gamma \in \Gamma} |f(\gamma)|^p := \sup \left\{\sum_{\gamma \in F} 
  |f(\gamma)|^p \mid F \text{ a finite subset of } \Gamma\right\}.\]
Furthermore, we define \(l_\infty(\Gamma)\) the set of all bounded functions 
\(f : \Gamma \to \mathbb{K}\) endowed with the supremum norm \(\|\cdot\|_\infty\). 
Similarly, we define the subspaces \(c_{0}(\Gamma)\) for the set of functions 
such that \(\{\gamma \in \Gamma \mid |f(\gamma)| \ge \epsilon\}\) is finite for 
all \(\epsilon > 0\), and \(c_{00}(\Gamma)\) for the set of functions with finite 
support. Note that \(c_0(\Gamma) = \overline{c_{00}(\Gamma)}\) (clearly 
\(c_0(\Gamma) \subseteq c_{00}(\Gamma)\) and we may construct a sequence in 
\(c_0\) converging to each element of \(c_{00}\)).
 
From the same argument as all the others, one can show \(\ell_p(\Gamma)\) and 
\(c_{0}(\Gamma)\) are Banach spaces.

\begin{definition}[Function Space]
  Given \(p \in [1, \infty)\), we define \(L^p := \mathcal{L}^p / \sim\) where 
  \(\sim\) is the equivalence relation
  \[f \sim g \iff f = g \text{ a.e.}\]
  and \(\mathcal{L}^p\) the set of measurable functions \(f\) such that 
  \[\int |f|^p \dd \mu < \infty.\]
\end{definition}

We recall that \(L^p\) spaces are Banach spaces.

\begin{proposition}
  A normed space \((X, \|\cdot\|)\) is a Banach space if and only if every 
  absolutely convergent series in \(X\) is convergent. 
\end{proposition}
\begin{proof}
  Exercise.
\end{proof}

\begin{definition}[Essential Supremum]
  Given \(f : [0, 1] \to \mathbb{R}\) measurable sets, we define the essential 
  supremum of \(f\) to be 
  \[\text{ess} \sup(f) := \inf_{\substack{N \subseteq [0, 1],\\ \lambda(N) = 1}} 
    (\sup f(N)).\]
  Equivalently, 
  \[\text{ess} \sup(f) = \inf \{\alpha \mid \lambda(f^{-1}(\alpha, \infty)) = 0\}.\]
\end{definition}

We recall that the essential supremum forms a norm on the function space resulting 
in the Banach space \(L^\infty\), i.e. \(\|f\|_\infty = \text{ess} \sup |f|\).

\subsection{Operators}

\begin{proposition}
  Let \((X, \|\cdot\|_X)\) and \((Y, \|\cdot\|_Y)\) be normed spaces and let 
  \(T : X \to Y\) be a linear map. Then the following are equivalent,
  \begin{enumerate}
    \item \(T\) is continuous on \(X\);
    \item \(T\) is continuous at \(0_X\);
    \item there exists some \(C > 0\) such that \(\|T(x)\|_Y \le C \|x\|_X\) for all \(x \in X\);
    \item \(T\) is Lipschitz;
    \item \(T(B_X)\) is a bounded set in \(Y\).
  \end{enumerate}
\end{proposition}
\begin{proof}
  Clearly \(1 \implies 2\), \(4 \implies 1\) and \(3, 4, 5\) are equivalent, it 
  suffices to show \(2 \implies 1\) and \(1 \implies 5\).

  Suppose \(T\) is continuous at \(0_X\), then, we will show \(T\) is continuous 
  at \(x\) for all \(x \in X\). Fix \(\epsilon > 0\), as \(T\) is continuous at 
  \(0_X\), there exists some \(\delta > 0\) such that for all 
  \(y \in X, \|y\|_X < \delta\), \(\|T(y)\|_Y < \epsilon\). Then, for all 
  \(z \in X, \|z - x\|_X < \delta\), we have \(\|T(z) - T(x)\|_Y = 
  \|T(z - x)\|_Y < \epsilon\), and so \(T\) is continuous at \(x\).

  Suppose \(T\) is continuous on \(X\), then, there exists some \(\delta > 0\) 
  such that for all \(\|x\|_X < \delta\), \(T(x) \in B_Y\). Thus, for all 
  \(x \in B_X\), as \(\|\delta x\|_X = \delta \|x\| < \delta\), \(\delta T(x) = 
  T(\delta x) \in B_X\), and hence, \(T(B_X) \subseteq \delta^{-1} B_Y\) implying 
  \(T(B_X)\) is bounded.
\end{proof}

\begin{definition}[Operator]
  Let \(X, Y\) be normed spaces. An operator from \(X\) to \(Y\) is simply a 
  linear map \(T : X \to Y\) and is called bounded if \(T(B_X)\) is bounded. 

  We endow the space of all bounded operators from \(X\) to \(Y\), 
  \(\mathcal{B}(X, Y)\) with the operator norm \(\|\cdot\|\) where 
  \[\|T\| := \sup \{\|T(x)\|_Y \mid x \in B_X\}.\]
\end{definition}

It is easy to see that \(\mathcal{B}(X, Y)\) is a normed space. 

\begin{proposition}
  Let \(X, Y\) be normed spaces. Then, if \(Y\) is Banach then so is 
  \(\mathcal{B}(X, Y)\).
\end{proposition}
\begin{proof}
  Follows similarly to the proof that \(C(X, Y)\) is Banach.
\end{proof}

\begin{definition}[Dual]
  Let \((X, \|\cdot\|_X)\) be a normed space and we denote \(X^*\) for the 
  dual of \(X\) consisting the linear functionals of \(X\), i.e. 
  \(X^* := \mathcal{B}(X, \mathbb{K})\).
\end{definition}

\begin{remark}
  As \(X^*\) forms an additive group, it is easy to see that there exists a 
  natural group action of \(X^*\) on \(X\) where given \(f \in X^*\), 
  \(x \in X\), we define \(f \cdot x = f(x)\).
\end{remark}

We recall from second year linear algebra that for finite dimensional spaces 
\(X\), \(\dim X = \dim X^*\). On the other hand, it is clear that if \(X\) is 
infinite dimensional, so is \(X^*\). Indeed, if \(X^*\) is finite dimensional, 
then \(\dim X^{**} = \dim X^* < \infty\). But, by considering the canonical 
isomorphism from \(X^{**}\) to \(X\), it follows \(\dim X = \dim X^{**} < \infty\).

\begin{proposition}
  \(X^*\) is a Banach space for every normed space \(X\).
\end{proposition}
\begin{proof}
  This is clear since \(\mathbb{K}\) is complete.
\end{proof}

\begin{definition}[Isomorphism]
  An operator \(T \in \mathcal{B}(X, Y)\) is an linear isomorphism if it is 
  bijective and \(T^{-1} \in \mathcal{B}(Y, X)\). We denote the set of all 
  linear isomorphisms between \(X\) and \(Y\) by \(GL(X, Y)\).
\end{definition}

It is clear that a linear map \(T \in \mathcal{L}(X, Y)\) is a linear isomorphism 
if and only if there exist some \(C_1, C_2\) such that for all \(x \in X\), 
\[C_1\|x\| \le \|T(x)\| \le C_2 \|x\|.\]
Indeed, we see that \(C_2\) bounds \(T\) while \(C_1\) bounds \(T^{-1}\).

\begin{definition}[Equivalent Norms]
  Let \(\|\cdot\|_1, \|\cdot\|_2\) be two norms on the vector space \(X\). Then 
  we say \(\|\cdot\|_1, \|\cdot\|_2\) are equivalent if the identity map 
  \(I_X : X \to X : x \mapsto x\) from \((X, \|\cdot\|_1)\) to \((X, \|\cdot\|_2)\) 
  is a linear isomorphism.
\end{definition}

By the above property we see that the two norms are equivalent if and only if 
there exists some \(C_1, C_2\) such that for all \(x \in X\), 
\[C_1 \|x\|_1 \le \|x\|_2 \le C_2 \|x\|_1.\]

Normed spaces \(X, Y\), are called linearly isomorphic if there exists a linear 
isomorphism between \(X\) and \(Y\). It is clear that since linear isomorphisms 
are both vector space isomorphisms and topological homeomorphisms, a linear 
isomorphism preserves both vector space and topological properties. In particular, 
if \(X, Y\) are linearly isomorphic and \(X\) is Banach, so is \(Y\).

\begin{definition}[Isometry]
  An operator \(T \in \mathcal{B}(X, Y)\) is an linear isometry if for all 
  \(x \in X\), \(\|x\| = \|T(x)\|\).
\end{definition}

\begin{definition}[Banach-Mazur Distance]
  Let \(X, Y\) be linearly isomorphic normed spaces. The Banach-Mazur distance 
  between \(X\) and \(Y\) is defined to be 
  \[d(X, Y) := \inf \{\|T\| \|T^{-1}\| \mid T \in GL(X, Y)\}.\]
\end{definition}

By taking the logarithm of the above, it is easy to see that \(\log \circ d\) 
forms a metric on the space of linearly isomorphic normed spaces. In particular, 
it turns out, the isometry classes of \(n\)-dimensional normed spaces becomes 
a compact space with respect to this metric. This fact is called the 
\textit{Banach-Mazur Compactum}.

\begin{definition}[Compact Operator]
  An operator \(\mathcal{B}(X, Y)\) is called a compact operator if 
  \(\overline{T(B_X)}\) is compact. We denote the space of all compact operators 
  from \(X\) to \(Y\) by \(\mathcal{K}(X, Y)\).
\end{definition}

\begin{definition}[Finite Rank Operator]
  An operator \(\mathcal{B}(X, Y)\) is called a finite rank operator if 
  \(\dim T(X) < \infty\). We denote the space of all finite rank operators from 
  \(X\) to \(Y\) by \(\mathcal{F}(X, Y)\).
\end{definition}

\begin{definition}[Linear Projection]
  Let \(V\) be a vector space. Then a linear projection \(P\) on \(V\) is 
  simply a idempotent \(V\)-endomorphism, i.e. \(P : V \to V\) is a linear map 
  such that \(P \circ P = P\). 
\end{definition}

It is not difficult to see that a linear projection provides a natural algebraic 
decomposition of \(V\). In particular, we observe \(V = \ker P \oplus P(V)\) 
(consider \(x = P(x) + (x - P(x))\)). Furthermore, given a decomposition 
\(V = X \oplus Y\), we may define a projection on \(V\) by defining \(P\) 
such that for each \(x + y \in V, P(x + y) = x\).

\begin{definition}[Topological Direct Sum]
  Given \((X, \|\cdot\|_X)\) and \((Y, \|\cdot\|_Y)\) normed spaces, we may 
  endow the outer direct sum \(X \oplus Y\) with the norm 
  \[\|(x, y)\| := \|x\|_X + \|y\|_Y,\]
  with the resulting normed space called the topological direct sum of \(X\) and 
  \(Y\).
\end{definition}

\subsection{Quotient Spaces}

We recall the quotient vector space from year 2 linear algebra and apply its 
definitions directly to normed spaces.

As normed spaces are additive groups, we may quotient them by its normal subgroups. 

\begin{definition}[Quotient Subspace]
  Given a normed space \(X\) and \(Y \le X\) a closed subspace, \(X / Y\) is 
  the space \(X / \sim\) where \(\sim\) is the equivalence relation 
  \[x_1 \sim x_2 \iff x_1 + Y = x_2 + Y,\]
  endowed with the norm \(\|[x]\| := \inf \{\|x\| \mid x \in [x]\}\).
\end{definition}

\begin{definition}[Quotient Map]
  The quotient map is the canonical mapping \(q : X \to X / Y : x \mapsto [x]\).
\end{definition}

It is clear that the quotient map is linear and continuous.

\begin{proposition}
  Let \(Y\) be a closed subspace of a Banach space \(X\). Then \(X / Y\) is also 
  Banach.
\end{proposition}
\begin{proof}
  It suffices to show every absolutely convergent series in \(X / Y\) converges. 
  Suppose \(\sum \|[x_n]\| < \infty\). Then, since for all \(n\), there exists 
  some \(x_n \in [x_n]\) such that \(\|x_n\| \le \|[x_n]\| + 2^{-n}\), we have 
  \(\sum \|x_n\| \le \sum (\|[x_n]\| + 2^{-n}) = \sum \|[x_n]\| + 1 < \infty\). 
  Thus, as \(X\) is Banach, \(\sum x_n\) converges. With that, by sequential 
  continuity, 
  \[\sum [x_n] = \sum q(x_n) = q\left(\sum x_n\right)\]
  also converges.
\end{proof}

\begin{theorem}\label{equiv}
  Let \(X\) be a finite dimensional vector space. Any two norms on \(X\) are 
  equivalent. In particular, all finite dimensional normed spaces are Banach 
  spaces and every normed space of dimension \(n\) is isomorphic to \(\ell^n_2\).
\end{theorem}
\begin{proof}
  Let \(B = \{e_1, \cdots, e_n\}\) be a basis of \(X\) and for each 
  \(x = \sum_{i = 1}^n \lambda_i e_i \in X\), we define 
  \(\|x\|_1 := \sum_{i = 1}^n |\lambda_i|\). By checking the axioms, it is 
  clear that \(\|\cdot\|_1\) is a norm on \(X\). Let \(S_1 := \{x \in X \mid \|x\|_1 = 1\}\) 
  and I claim it suffices to show \(S_1\) is compact. Indeed, if \(S_1\) is 
  compact, as \(\|\cdot\|\) is continuous on \(S_1\), \(\|\cdot\|\) achieves its 
  its maximum and minimum on \(S_1\). Hence, there exists some \(c_1, c_2 > 0\) 
  such that 
  \[c_1 \le \left\| \frac{x}{\|x\|_1}\right\| \le c_2.\]
  Now, to show \(S_1\) is compact, take \(x^m = \sum_{i = 1}^n \lambda_i^m e_i\) 
  a sequence in \(S_1\) and we will show it has a convergent subsequence. By 
  definition \(\sum_i |\lambda_i^m| = 1\) and thus, \(\lambda_i^m\) is bounded 
  for all \(m\). Thus, by Bolzanno-Weierstrass, \(\lambda_i^m\) has a convergent 
  subsequence (with respect to \(m\)), i.e. \(\lambda_i^{m_j} \to \lambda_i\) 
  as \(j \to \infty\) for some \(\lambda_i \in \mathbb{K}\). With that, 
  defining \(x := \sum_i \lambda_i e_i\), we see 
  \[\|x^{m_j} - x\| = \left\|\sum_i (\lambda_i^{m_j} - \lambda_i) e_i \right\| 
    \to 0 \text{ as } j \to \infty.\] 
  To show that \(\ell_2^n\) is linear isomorphic to \(X\) simply take \(T\) to 
  be a vector space isomorphism between \(X\) and \(\ell_2^n\), then we may 
  define \(\|\cdot\|_2\) where \(\|x\|_2 := \|T(x)\|_{\ell_2^n}\). With that 
  \(T\) forms an linear isometry between \((X, \|\cdot\|_2)\) and \(\ell_2^n\) 
  implying \(\|\cdot\|_{\ell_2^n} \simeq \|\cdot\|_2 \simeq \|\cdot\|_X\).
\end{proof}

By the above proposition, we see that given \(X, Y\) finite dimensional 
normed spaces, \(\|(x, y)\|_p := (\|x\|_X^p + \|y\|_Y^p)^{1 / p}\) is a 
equivalent renorming of \(X \oplus Y\). We denote this renormed space by 
\((X \oplus Y)_p\).

\begin{lemma}[Riesz's Lemma]
  Let \(X\) be a normed space. If \(Y\) is a proper closed subspace of \(X\), 
  then for every \(\epsilon > 0\), there exists some \(x \in X\), \(\|x\| = 1\), 
  such that \(\text{dist} (x, Y) \ge 1 - \epsilon\).
\end{lemma}
\begin{proof}
  Since \(Y\) is proper, \(X / Y\) is non-trivial and thus, there exists some 
  \([x] \in X / Y\), \(\|[x]\| = 1\). Thus, as \(\|\cdot\|\) is continuous, 
  by the intermediate value theorem, there exists some \([z] \in X / Y\), 
  \(1 > \|[z]\| > 1 - \epsilon\). By definition, 
  \[1 > \inf \{\|z\| \mid z \in [z]\} > 1 - \epsilon,\]
  and thus, there exists some \(z \in [z]\), \(\|z\| < 1\). Now, defining 
  \(x := z / \|z\|\), we have \(\|x\| = 1\) and 
  \[\text{dist}(x, Y) = \inf_{y \in Y} \|x - y\| = 
    \frac{1}{\|z\|} \inf_{y \in Y} \|z - y\| = \frac{\|[z]\|}{\|z\|} \ge 
    \frac{1 - \epsilon}{1} = 1 - \epsilon.\]
\end{proof}

From Riesz's lemma, we obtain that, for proper subspaces \(Y\), the 
quotient map \(q : X \to X / Y\) satisfies \(\|q\| = 1\) (since 
we may construct a sequence \(x_n \in B_X\) such that \(\|q(x_n)\| \ge 
1 - \frac{1}{n}\)).    

\begin{proposition}
  A normed space \(X\) is finite dimensional if and only if \(B_X\) is compact.
\end{proposition}
\begin{proof}
  If \(X\) is finite dimensional, then \(B_X\) is compact by the same proof as 
  theorem~\ref{equiv}. On the other hand, if \(X\) is infinitely dimensional, 
  we construct the following sequence inductively. Choose \(x_1 \in B_X\) and 
  for \(n \in \mathbb{N}\), choose \(x_{n + 1} \in B_X\) such that 
  \[\text{dist}(x_{n + 1}, \text{span}\{x_1, \cdots, x_n\}) > \frac{1}{2},\]
  which existence is guaranteed by Riesz's lemma. It is clear that \((x_n)\) 
  does not have a convergent subsequence and thus, \(B_X\) is not compact. 
\end{proof}

\begin{proposition}
  Every operator \(T\) from a finite-dimensional normed space \(X\) into a 
  normed space \(Y\) is continuous.
\end{proposition}
\begin{proof}
  This follows by simply bounding \(T(B_X)\) which is bounded naturally.
\end{proof}

From this, we see that every finite dimensional normed space is linearly isomorphic 
if they have the same dimension. Indeed, as they are already algebraically 
isomorphic, we obtain the isomorphisms (and its inverse) are continuous by 
above. Alternatively, the same result follows as they are linearly isomorphic to 
\(\ell^n_2\)

\begin{proposition}
  Let \(X, Y\) be normed spaces. Then \(\mathcal{F}(X, Y)\) is a linear subspace 
  of \(\mathcal{K}(X, Y)\). Furthermore, \(\mathcal{K}(X, Y)\) is a closed subspace 
  of \(\mathcal{B}(X, Y)\), and thus, if \(Y\) is a Banach space, so is 
  \(\mathcal{K}(X, Y)\).
\end{proposition}
\begin{proof}
  For the first claim, it suffices to show \(\mathcal{F}(X, Y) \subseteq 
  \mathcal{K}(X, Y)\). Let \(T \in \mathcal{F}(X, Y)\), then, 
  \(\dim T(X) < \infty\) and \(T(X)\) is homeomorphic to \(\mathbb{R}^n\) 
  where \(n = \dim T(X)\). Thus, as the Heine-Borel property is topologically 
  invariant, it follows \(\overline{T(B_X)}\) is compact as it is closed and 
  bounded. Hence, \(T \in \mathcal{K}(X, Y)\) and \(\mathcal{F}(X, Y) \le 
  \mathcal{K}(X, Y)\).

  To show \(\mathcal{K}(X, Y)\) is closed let \((T_n) \subseteq \mathcal{K}(X, Y)\) 
  such that \(T_n \to T \in \mathcal{B}(X, Y)\), then it suffices to show 
  \(T \in \mathcal{K}(X, Y)\), in particular, we will show \(T(B_X)\) is totally 
  bounded. Fix \(\epsilon > 0\), then as \(T_n \to T\) (where the convergence is 
  uniform), there exists some \(N\) such that for all \(x \in B_X\), \(n \ge N\), 
  \(\|T_n(x) - T(x)\| < \epsilon / 2\). Now, since \(T_n(B_X)\) is totally 
  bounded, there exists a finite \(\epsilon / 2\)-net \(F\) for \(T_n(B_X)\). 
  Clearly, \(F\) is a finite \(\epsilon\)-net for \(T(B_X)\). Indeed, for all 
  \(x \in B_X\), there exists some \(y \in F\) such that \(\|T_n(x) - y\| < 
  \epsilon / 2\) and so, 
  \[\|T(x) - y\| \le \|T(x) - T_n(x)\| + \|T_n(x) - y\| < \frac{\epsilon}{2}
    + \frac{\epsilon}{2} = \epsilon.\]
\end{proof}

\subsection{Hilbert Spaces}

We recall that Hilbert spaces are inner product spaces which is complete with 
respect to the induced norm. We will in this section let \(H\) equipped with 
the inner produce \(\langle \cdot, \cdot \rangle\) be a Hilbert space.

\begin{lemma}
  For all \(y \in H\), \(\langle \cdot, y \rangle : H \to \mathbb{F}\) is 
  continuous.
\end{lemma}
\begin{proof}
  This follows straight away from the Cauchy-Schwarz inequality.
\end{proof}

\begin{proposition}
  If \(F\) be a subspace of \(H\), then \(F^\perp\) is closed.
\end{proposition}
\begin{proof}
  Let \((x_n) \subseteq F^\perp\) such that \(x_n \to x \in H\), then for all 
  \(f \in F\), \(0 = \langle x_n, f\rangle \to \langle x, f\rangle\), and so 
  \(\langle x, f\rangle = 0\) for all \(f \in F\), i.e. \(x \in F^\perp\). 
\end{proof}

\begin{theorem}
  If \(F\) be a closed subspace of the Hilbert space \(H\), then
  \(F \oplus F^\perp = H\). In particular, the map 
  \(T : F \oplus F^\perp \to H : (x, y) \mapsto x + y\) is a topological, 
  and algebraic isomorphism.
\end{theorem}
\begin{proof}
  Clearly, \(F \cap F^\perp = \{0\}\) so it suffices to show 
  \(H \subseteq F + F^\perp\).

  Let \(h \in H\) and define 
  \[x_1 := \arg \min_{f \in F} \|h - f\|,\]
  which exists in \(F\) as it is closed. Furthermore, define \(x_2 := h - x_1\)
  (noting that \(\|x_2\|^2 = \text{dist}(x, F)^2\)), 
  it remains to show that \(x_2 \in F^\perp\). Suppose otherwise, then 
  there exists some \(f \in F\) such that \(\langle x_2, f \rangle > 0\). 
  Now, for all \(\epsilon > 0\), we have, 
  \begin{align*}
    \|h - (x_1 + \epsilon f)\|^2 & = \|x_2 - \epsilon f\|^2 = \langle x_2 - \epsilon f, x_2 - \epsilon f \rangle\\
    & = \|x_2\|^2 + \epsilon^2\|f\| - 2\epsilon \langle x_2, f \rangle\\
    & = \text{dist}(x, F)^2 - \epsilon (2 \langle x_2, f \rangle - \epsilon \|f\|).
  \end{align*}
  Thus, as for sufficiently small \(\epsilon\), we have 
  \(2 \langle x_2, f \rangle - \epsilon \|f\| > 0\), we found an element of \(F\) 
  achieving better distance, a contradiction! Hence, \(x_2 \in F^\perp\) and 
  \(F \oplus F^\perp = H\). 
\end{proof}

A consequence of this is that \(F = (F^\perp)^\perp\) for closed subspace \(F\). 
Indeed, it is clear that \(F \subseteq (F^\perp)^\perp\) and for 
\(h \in (F^\perp)^\perp\), there exists unique \(f \in F, g \in F^\perp\) such 
that \(h = f + g\) and so, \(f = h - g\). Now, since \(F^\perp\) is closed, we 
also have \(F^\perp \oplus (F^\perp)^\perp = H\) and so, the representation 
\(f = h + (-g)\) is unique. But, since \(f \in F \subseteq (F^\perp)^\perp\), 
we have \(f = f + 0\) and thus, \(g = 0\) and finally, \(h = f \in F\).

\begin{definition}[Orthonormal]
  Let \(H\) be a Hilbert space and \(S \subseteq H\). Then \(S\) is called 
  orthonormal if for all \(s_i, s_j \in S\), \(\langle s_i, s_j\rangle = \delta_{ij}\).

  We call an orthonormal set an orthonormal basis if it is maximal.
\end{definition}

\begin{theorem}
  Every Hilbert space has an orthonormal basis.
\end{theorem}
\begin{proof}
  Clearly, the union of a chain of orthonormal sets is orthonormal, thus, by 
  Zorn's lemma, there exists a maximal orthonormal set.
\end{proof}

\begin{proposition}
  Let \(H\) be a Hilbert space and \(H_0\) a closed subspace of \(H\). Then, every 
  orthonormal basis of \(H_0\) can be extended to an orthonormal basis of \(H_0\).
\end{proposition}
\begin{proof}
  This follows straight away by Zorn's lemma again since every chain of the set 
  of orthonormal sets has an upper bound.
\end{proof}

Consequently, if \(M\) is an orthonormal basis of \(H\), then 
\(\overline{\text{span}}M = H\). Indeed, if otherwise, we may simply take a 
vector \(v\) of norm 1 in \(\overline{\text{span}}M^\perp\), constructing a larger 
orthonormal set \(M \cup {v}\), contradicting the maximality of \(M\).

\begin{theorem}
  Every separable infinite-dimensional Hilbert space \(H\) has a countable 
  orthonormal basis \((e_i)_{i = 1}^\infty\). Furthermore, if \((e_i)_{i = 1}^\infty\) 
  is a countable basis of \(H\), then for all \(x \in H\), 
  \[x = \sum_{i = 1}^\infty \langle x, e_i \rangle e_i,\]
  where we call the coefficients \(\langle x, e_i \rangle\) the Fourier 
  coefficients and the sum itself as the Fourier expansion of \(x\).
\end{theorem}
\begin{proof}
  Let \((e_\mu)_{\mu \in M}\) be a orthonormal basis of \(H\) and suppose \(M\) 
  is uncountable. Then, for all \(e_\mu \neq e_\lambda\), by orthogonality, we 
  have 
  \[\| e_\mu - e_\lambda \| = \sqrt{2}.\]
  Thus, we have \(\{ B_{1/2}(e_\mu) \mid \mu \in M \}\) is a set of pairwise disjoint 
  balls, each containing one and only one basis vector. Now, as \(H\) is separable, 
  there exists some \((h_i)_{i = 1}^\infty\) which is dense in \(H\). Hence, 
  each ball contain at least one \(h_i\). But then, we have a injective function 
  from a countable set to an uncountable set, which is a contradiction. Thus, 
  the basis must be countable.
\end{proof}

\begin{proposition}
  Let \((e_i)_{i = 1}^\infty\) be an orthonormal set in a Hilbert space \(H\) and 
  \(x \in H\). Then 
  \begin{itemize}
    \item (The Bessel inequality) \(\sum \langle x, e_i \rangle^2 \le \|x\|^2\);
    \item (The Parseval equality) if \((e_i)_{i = 1}^\infty\) is an orthonormal 
      basis of \(H\), then \(\sum \langle x, e_i \rangle^2 = \|x\|^2\);
    \item if the Parseval equality holds for every \(x \in H\), then 
      \((e_i)_{i = 1}^\infty\) is an orthonormal basis of \(H\);
    \item if \(\overline{\text{span}}(e_i)_{i = 1}^\infty = H\), then 
      \((e_i)_{i = 1}^\infty\) is an orthonormal basis of \(H\).
  \end{itemize}
\end{proposition}

\begin{theorem}[Riesz-Fischer theorem]
  Every separable infinite-dimensional Hilbert space \(H\) is linearly isometric 
  to \(\ell_2\).
\end{theorem}

\newpage
\section{Hahn-Banach Theorem and Open Mapping Theorems}

\begin{definition}
  A real-valued function \(p : X \to \mathbb{R}\) is said to be subadditive if for all 
  \(x, y \in X\), \(p(x + y) \le p(x) + p(y)\); positively homogenous if for all 
  \(x \in X, a \ge 0\), we have \(p(a x) = a p(x)\). Finally, if \(p\) is 
  subadditive, and for all scalars \(a\), \(p(a x) = |a| p(x)\), we call \(p\)
  a seminorm.
\end{definition}

It is clear that any positively homogenous subadditive function is convex. 

\begin{theorem}[Hahn-Banach]
 Let \(Y\) be a subspace of the real-linear space \(X\), and \(p\) a positively 
 homogenous subadditive functional on \(X\). Then, if \(f\) is a linear functional 
 on \(Y\), such that \(f(x) \le p(x)\) for all \(x \in Y\), there exists a linear 
 functional \(F\) on \(X\), such that for all \(x \in Y\), \(F(x) = f(x)\) and 
 \(F(x) \le p(x)\) for all \(x \in F\).
\end{theorem}

\end{document}
