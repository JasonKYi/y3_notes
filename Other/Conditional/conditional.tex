% Options for packages loaded elsewhere
\PassOptionsToPackage{unicode}{hyperref}
\PassOptionsToPackage{hyphens}{url}
\PassOptionsToPackage{dvipsnames,svgnames*,x11names*}{xcolor}
%
\documentclass[]{article}
\usepackage{lmodern}
\usepackage{amssymb,amsmath}
\usepackage{ifxetex,ifluatex}
\ifnum 0\ifxetex 1\fi\ifluatex 1\fi=0 % if pdftex
  \usepackage[T1]{fontenc}
  \usepackage[utf8]{inputenc}
  \usepackage{textcomp} % provide euro and other symbols
\else % if luatex or xetex
  \usepackage{fontspec}
  \setmainfont{Bitstream Charter}
  % Cabin -- round
  % Bitstream Charter -- almost computer modern
  % \usepackage{unicode-math}
  % \defaultfontfeatures{Scale=MatchLowercase}
  % \defaultfontfeatures[\rmfamily]{Ligatures=TeX,Scale=1}
\fi
% Use upquote if available, for straight quotes in verbatim environments
\IfFileExists{upquote.sty}{\usepackage{upquote}}{}
\IfFileExists{microtype.sty}{% use microtype if available
  \usepackage[]{microtype}
  \UseMicrotypeSet[protrusion]{basicmath} % disable protrusion for tt fonts
}{}
\makeatletter
\@ifundefined{KOMAClassName}{% if non-KOMA class
  \IfFileExists{parskip.sty}{%
    \usepackage{parskip}
  }{% else
    \setlength{\parindent}{0pt}
    \setlength{\parskip}{6pt plus 2pt minus 1pt}}
}{% if KOMA class
  \KOMAoptions{parskip=half}}
\makeatother
\usepackage{xcolor}
\IfFileExists{xurl.sty}{\usepackage{xurl}}{} % add URL line breaks if available
\IfFileExists{bookmark.sty}{\usepackage{bookmark}}{\usepackage{hyperref}}
\hypersetup{
  pdftitle={Conditional Expectation},
  pdfauthor={Kexing Ying},
  colorlinks=true,
  linkcolor=Maroon,
  filecolor=Maroon,
  citecolor=Blue,
  urlcolor=red,
  pdfcreator={LaTeX via pandoc}}
\urlstyle{same} % disable monospaced font for URLs
\usepackage[margin = 1.5in]{geometry}
\usepackage{graphicx}
\makeatletter
\def\maxwidth{\ifdim\Gin@nat@width>\linewidth\linewidth\else\Gin@nat@width\fi}
\def\maxheight{\ifdim\Gin@nat@height>\textheight\textheight\else\Gin@nat@height\fi}
\makeatother
% Scale images if necessary, so that they will not overflow the page
% margins by default, and it is still possible to overwrite the defaults
% using explicit options in \includegraphics[width, height, ...]{}
\setkeys{Gin}{width=\maxwidth,height=\maxheight,keepaspectratio}
% Set default figure placement to htbp
\makeatletter
\def\fps@figure{htbp}
\makeatother
\setlength{\emergencystretch}{3em} % prevent overfull lines
\providecommand{\tightlist}{%
  \setlength{\itemsep}{0pt}\setlength{\parskip}{0pt}}
\setcounter{secnumdepth}{5}
\usepackage{tikz}
\usepackage{physics}
\usepackage{amsthm}
\usepackage{mathtools}
\usepackage{esint}
\usepackage[ruled,vlined]{algorithm2e}
\theoremstyle{definition}
\newtheorem{theorem}{Theorem}
\newtheorem{definition*}{Definition}
\newtheorem{prop}{Proposition}
\newtheorem{corollary}{Corollary}[theorem]
\newtheorem*{remark}{Remark}
\theoremstyle{definition}
\newtheorem{definition}{Definition}[section]
\newtheorem{lemma}{Lemma}[section]
\newtheorem{proposition}{Proposition}[section]
\newtheorem{example}{Example}[section]
\newcommand{\diag}{\mathop{\mathrm{diag}}}
\newcommand{\Arg}{\mathop{\mathrm{Arg}}}
\newcommand{\hess}{\mathop{\mathrm{Hess}}}
\newcommand\eqae{\mathrel{\overset{\makebox[0pt]{\mbox{\normalfont\tiny\sffamily a.e.}}}{=}}}

\title{A Bit About Conditional Expectation}
\author{Kexing Ying}
\date{July 24, 2021}

\begin{document}
\maketitle

Measure-theoretically, conditional expectation is defined in terms of conditioning 
on a sub-\(\sigma\) algebra. 

\begin{definition*}[Conditional Expectation]
  Let \(f : \Omega \to \mathbb{R}^+\) be a random variable on some probability 
  space \((\Omega, \mathcal{F}, \mathbb{P})\) such that \(\int f \dd \mathbb{P} < \infty\). 
  Then, given a sub-\(\sigma\)-algebra \(\mathcal{G}\) of \(\mathcal{F}\), the 
  conditional expectation of \(f\) with respect to \(\mathcal{G}\) is the 
  essentially unique \(\mathcal{G}\)-random variable \(f'\), such that 
  \begin{equation}\label{eq:cond_expect}
    \int_A f \dd \mathbb{P} = \int_A f' \dd \mathbb{P},
  \end{equation}
  for all \(A \in \mathcal{G}\). We denote \(f'\) by \(\mathbb{E}(f \mid \mathcal{G})\).
\end{definition*}

We note that the integrals within this definition are computed with respect to 
the larger \(\mathcal{F}\) rather than the smaller \(\mathcal{G}\). Indeed, 
while \(f'\) is measurable with respect to \(\mathcal{F}\), \(f\) is not 
measurable with respect to \(\mathcal{G}\).

We also note that, if \(f\) is already measurable with respect to \(\mathcal{G}\), 
then \(f' = f\), \(\mathbb{P}\)-a.e. This follows since the Radon-Nikodym derivative 
is essentially unique and clearly, \(f\) satisfy the integral equation~\ref{eq:cond_expect}.
Written, using the Radon-Nikodym derivative notation, we have 
\[f' = \dv{(f \mathbb{P})|_\mathcal{G}}{\mathbb{P}|_\mathcal{G}}.\]
Thus, if \(f\) is \(\mathcal{G}\)-measurable, we simply exploiting the fact that,
\[(f \mathbb{P})|_\mathcal{G} = f (\mathbb{P}|_\mathcal{G}).\]

Geometrically, we may think about conditioning on a sub-\(\sigma\)-algebra as projection 
onto the sub-\(\sigma\)-algebra. This notion is made rigorous in the case of 
\(L^2\)-random variables. 

\begin{definition*}[Conditional Expectation in \(L^2\)]
  Let \(f : \Omega \to \mathbb{R}^+\) be a \(L^2\)-random variable on some probability 
  space \((\Omega, \mathcal{F}, \mathbb{P})\). Then, given a sub-\(\sigma\)-algebra 
  \(\mathcal{G}\) of \(\mathcal{F}\), the conditional expectation of 
  \(f\) with respect to \(\mathcal{G}\) is the orthogonal projection of \(f\) into 
  \(L^2(\Omega, \mathcal{G}, \mathbb{P}|_\mathcal{G})\).
\end{definition*}

In this definition, the conditional expectation \(f'\) is the \(\mathcal{G}\)-random variable 
which minimizes 
\[\mathbb{E}[(f - f')^2] = \int (f - f')^2 \dd \mathbb{P}.\]

By now, one might wonder how does this definition relate to the classical definition 
of conditional expectation where we condition on an event. In particular, how 
does this notion relate to \(\mathbb{E}(X \mid B) = \mathbb{E}(X1|_B) = 
\frac{1}{\mathbb{P}(B)}\int_B X \dd \mathbb{P}\) where \(B \in \mathcal{F}, \mathbb{P}(B) > 0\).

\begin{prop}
  Given \(B \in \mathcal{F}\), 
  \[\mathbb{E}(X \mid \sigma(\{B\})) \eqae \omega \mapsto 
    \begin{cases}
      \mathbb{E}(X \mid B) & \text{if } \omega \in B \\
      \mathbb{E}(X \mid B^c) & \text{if } \omega \notin B.
    \end{cases}\]
\end{prop}
\begin{proof}
  Denote \(Y\) as the random variable defined by the right hand side.

  Since \(\sigma(\{B\}) = \{\varnothing, B, B^c, \Omega\}\), it is clear that the 
  \(Y\) is \(\sigma(\{B\})\)-measurable. Furthermore, it is also clear that, for all 
  \(A \in \sigma(\{B\})\), \(\int_A Y \dd \mathbb{P} = \int_A X \dd \mathbb{P}\). 
  Thus, the two random variables are equal almost everywhere.
\end{proof}

To obtain the other elementary definitions learnt during an introduction to probability 
course regarding conditional probabilities, we may simply define, 
\begin{definition*}\label{eq:cond_prop}
  Given \(A \in \mathcal{F}\), we write
  \(\mathbb{P}(A \mid \mathcal{G}) := \mathbb{E}(1|_A \mid \mathcal{G})\).
\end{definition*}
\begin{definition*}\label{eq:cond_expect_rv}
  Given \(f, g\) \(\mathcal{F}\)-random variables, we write 
  \(\mathbb{E}(f \mid g) := \mathbb{E}(f \mid \sigma(g))\).
\end{definition*}

Using this measure-theoretic definition of conditional expectation, some standard 
results are extremely simple to derive. 

\begin{prop}
  If \(\{A_i \mid i \in \mathbb{N}\}\) form a partition of \(\mathcal{F}\), then 
  \[\mathbb{E}(X \mid \mathcal{G}) = \sum_{i \in\mathbb{N}} \mathbb{E}(X \mid A_i) 1|_{A_i},\]
  where \(\mathcal{G} = \sigma(\{A_i \mid i \in \mathbb{N}\})\).
\end{prop}
\begin{proof}
  Clear by considering for all \(G \in \mathcal{G}\), there exists some 
  \(I \subseteq \mathbb{N}\) such that \(G = \bigcup_{i \in I} A_i\).
\end{proof}



\begin{prop}[Law of Total Expectation]
  Let \(\mathcal{H} \le \mathcal{G} \le \mathcal{F}\), and \(X\) a 
  \(\mathcal{F}\)-random variable, then 
  \[\mathbb{E}(\mathbb{E}(X \mid \mathcal{G})) = \mathbb{E}(X)\]
  and 
  \[\mathbb{E}(\mathbb{E}(X \mid \mathcal{G}) \mid \mathcal{H}) = 
    \mathbb{E}(X \mid \mathcal{H}),\mathbb{P}\text{-a.e.}\]
\end{prop}
\begin{proof}
  The first result is immediate as,
  \[\mathbb{E}(\mathbb{E}(X \mid \mathcal{G})) 
    = \int_\Omega \mathbb{E}(X \mid \mathcal{G}) \dd \mathbb{P} 
    = \int_\Omega X \dd \mathbb{P} = \mathbb{E}(X).\]
  For the second result, we see, for all \(A \in \mathcal{H}\),
  \[\int_A \mathbb{E}(\mathbb{E}(X \mid \mathcal{G}) \mid \mathcal{H}) \dd \mathbb{P}
    = \int_A \mathbb{E}(X \mid \mathcal{G}) \dd \mathbb{P}
    = \int_A X \dd \mathbb{P}.\]
  Thus, as the conditional expectation is essentially unique, we have 
  \[\mathbb{E}(\mathbb{E}(X \mid \mathcal{G}) \mid \mathcal{H}) = 
    \mathbb{E}(X \mid \mathcal{H}), \mathbb{P}\text{-a.e.}\]
\end{proof}

\end{document}
