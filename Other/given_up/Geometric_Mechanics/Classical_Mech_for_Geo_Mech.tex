% Options for packages loaded elsewhere
\PassOptionsToPackage{unicode}{hyperref}
\PassOptionsToPackage{hyphens}{url}
\PassOptionsToPackage{dvipsnames,svgnames*,x11names*}{xcolor}
%
\documentclass[]{article}
\usepackage{lmodern}
\usepackage{amssymb,amsmath}
\usepackage{ifxetex,ifluatex}
\ifnum 0\ifxetex 1\fi\ifluatex 1\fi=0 % if pdftex
  \usepackage[T1]{fontenc}
  \usepackage[utf8]{inputenc}
  \usepackage{textcomp} % provide euro and other symbols
\else % if luatex or xetex
  \usepackage{unicode-math}
  \defaultfontfeatures{Scale=MatchLowercase}
  \defaultfontfeatures[\rmfamily]{Ligatures=TeX,Scale=1}
\fi
% Use upquote if available, for straight quotes in verbatim environments
\IfFileExists{upquote.sty}{\usepackage{upquote}}{}
\IfFileExists{microtype.sty}{% use microtype if available
  \usepackage[]{microtype}
  \UseMicrotypeSet[protrusion]{basicmath} % disable protrusion for tt fonts
}{}
\makeatletter
\@ifundefined{KOMAClassName}{% if non-KOMA class
  \IfFileExists{parskip.sty}{%
    \usepackage{parskip}
  }{% else
    \setlength{\parindent}{0pt}
    \setlength{\parskip}{6pt plus 2pt minus 1pt}}
}{% if KOMA class
  \KOMAoptions{parskip=half}}
\makeatother
\usepackage{xcolor}\pagecolor[RGB]{28,30,38} \color[RGB]{213,216,218}
\IfFileExists{xurl.sty}{\usepackage{xurl}}{} % add URL line breaks if available
\IfFileExists{bookmark.sty}{\usepackage{bookmark}}{\usepackage{hyperref}}
\hypersetup{
  pdftitle={Classical Mechanics for Geometric Mechanics},
  pdfauthor={Kexing Ying},
  colorlinks=true,
  linkcolor=Maroon,
  filecolor=Maroon,
  citecolor=Blue,
  urlcolor=red,
  pdfcreator={LaTeX via pandoc}}
\urlstyle{same} % disable monospaced font for URLs
\usepackage[margin = 1.5in]{geometry}
\usepackage{graphicx}
\makeatletter
\def\maxwidth{\ifdim\Gin@nat@width>\linewidth\linewidth\else\Gin@nat@width\fi}
\def\maxheight{\ifdim\Gin@nat@height>\textheight\textheight\else\Gin@nat@height\fi}
\makeatother
% Scale images if necessary, so that they will not overflow the page
% margins by default, and it is still possible to overwrite the defaults
% using explicit options in \includegraphics[width, height, ...]{}
\setkeys{Gin}{width=\maxwidth,height=\maxheight,keepaspectratio}
% Set default figure placement to htbp
\makeatletter
\def\fps@figure{htbp}
\makeatother
\setlength{\emergencystretch}{3em} % prevent overfull lines
\providecommand{\tightlist}{%
  \setlength{\itemsep}{0pt}\setlength{\parskip}{0pt}}
\setcounter{secnumdepth}{5}
\usepackage{tikz}
\usepackage{physics}
\usepackage{amsthm}
\usepackage{mathtools}
\usepackage{esint}
\usepackage[ruled,vlined]{algorithm2e}
\theoremstyle{definition}
\newtheorem{theorem}{Theorem}
\newtheorem{definition*}{Definition}
\newtheorem{prop}{Proposition}
\newtheorem{corollary}{Corollary}[theorem]
\newtheorem*{remark}{Remark}
\theoremstyle{definition}
\newtheorem{definition}{Definition}[section]
\newtheorem{lemma}{Lemma}[section]
\newtheorem{proposition}{Proposition}[section]
\newtheorem{example}{Example}[section]
\newcommand{\diag}{\mathop{\mathrm{diag}}}
\newcommand{\Arg}{\mathop{\mathrm{Arg}}}
\newcommand{\hess}{\mathop{\mathrm{Hess}}}
% the redefinition for the missing \setminus must be delayed
\AtBeginDocument{\renewcommand{\setminus}{\mathbin{\backslash}}}

\title{Classical Mechanics for Geometric Mechanics}
\author{Kexing Ying}
\date{July 24, 2021}

\begin{document}
\maketitle

This document serves as a review for classical mechanics in order to appreciate 
geometric mechanics. In particular, we will recall...

\section*{Definitions}

We define the notion of motion on smooth manifolds with particular focus on 
its interpretation for mechanics. For a more mathematical approach to these 
notions, see the manifolds course.

\begin{definition}[Space]
  Throughout this course, we will take space to be a smooth manifold \(Q\) with 
  points \(q \in Q\). 
\end{definition}

Intuitively, we may imagine smooth manifolds as general spaces on which we may 
do calculus on. As we shall see later, the manifold \(Q\) may also be identified 
with a Lie group \(G\) especially whenever we would like to consider rotation 
and translation. 

\begin{definition}[Time]
  Time is a manifold \(T\) with \(t \in T\). Usually \(T = \mathbb{R}\) although 
  we will also consider \(T = \mathbb{R}^2\) or even more exotic cases in which 
  \(T\) and \(Q\) are taken to be complex manifolds. 
\end{definition}

\begin{definition}[Motion]
  Motion is a map \(\phi : T \to Q \to Q\) where we write \(\phi(t) = \phi_t\) 
  whenever there is no confusion. In the case that \(T = \mathbb{R}\), the motion 
  is a curve such that \(q(t) = \phi_t(q(0))\). A motion \(\phi\) is called a 
  flow if \(\phi_{t + s} = \phi_t \circ \phi_s\) for all \(s, t \in \mathbb{R}\) 
  and \(\phi_0 = \text{Id}\).
\end{definition}

\begin{definition}[Tangent Space]
  The tangent space of a manifold \(Q\) at \(q \in Q\), denoted by \(T_q Q\), 
  is the set of vectors \(v_q = \dot q(t) \in T_q Q\) tangent to the curve 
  \(q(t) \in Q\) at the point \(q\). Furthermore, we call \(v_q\) the tangent 
  lift vector/velocity at \(q\).
\end{definition}

Mathematically, the tangent space is defined to be the set of curves through the 
point \(q\) quotiented by the tangency of two curves at \(q\). In particular, 
two curves \(\sigma\) and \(\tau\) are said to be tangent as \(q\) if 
\(D\sigma\mid_0 = D\tau\mid_0\). This definition is independent of the chart 
though, if we fix a chart as \(q\), we may transport the tangent space into 
\(\mathbb{R}^{\dim Q}\). 

\begin{definition}[Tangent Bundle]
  The tangent bundle is the union of all tangent spaces, i.e. 
  \[TQ = \bigcup_{q \in Q} T_q Q.\]
\end{definition}

The tangent bundle is a useful concept as \(\dot q(t)\) lives inside of \(TQ\).

\begin{definition}[Motion Equation]
  The motion equation \(f\) is a vector field over \(Q\) such that 
  \(\dot q_t = f(q_t)\).
\end{definition}

\subsection*{The Metric}

The metric is a useful notion generally found in tensor calculus. We will quickly 
introduce them here with the goal of defining the Riemannian metric. We will 
work with \(\mathbb{R}^n\) in this section.

Let \(\{v^i\}\) be a coordinate system for \(\mathbb{R}^n\) and (employing 
Einstein's notation) let 
\[\mathbf{r} = x_i(v^i, ...) \mathbf{x}^i,\]
where \(\{\mathbf{x}^i\}\) are the standard basis of \(\mathbb{R}^n\) and 
\(x_i\) are functions of \(\{v^i\}\). Then we define the set of vectors 
\[\mathbf{e}_i = \pdv{\mathbf{r}}{v^i}, \mathbf{e}^i = \grad v^i.\]
By calculation, we find \(\mathbf{e}^i \cdot \mathbf{e}_j = \delta_{ij}\) where 
\(\delta_{ij}\) is the Kronecker delta function. This establishes a orthogonality 
relation between the vectors and it follows both \(\{\mathbf{e}^i\}\) and 
\(\{\mathbf{e}_i\}\) are linearly independent and hence are both basis of 
\(\mathbb{R}^n\). Thus, for all vectors \(\mathbf{X} \in \mathbb{R}^n\), there 
exists uniquely \(\{X^i\}\) and \(\{X_i\}\) such that 
\(\mathbf{X} = X^i \mathbf{e}_i = X_i \mathbf{e}^i\). We call \(X^i\) the 
contravariant components of \(\mathbf{X}\) and \(X_i\) to covariant components 
of \(\mathbf{X}\) and it is not hard to see 
\[\mathbf{e}_i \cdot \mathbf{X} = \mathbf{e}_i \cdot X_j \mathbf{e}^j = 
  X_j \delta_{ij} = X_i.\]
Similarly, \(X^i= \mathbf{e}^i \cdot \mathbf{X}\).

\begin{definition}[Metric]
  We define the metrics \(g_{ij} = \mathbf{e}_i \cdot \mathbf{e}_j\) and 
  \(g^{ij} = \mathbf{e}^i \cdot \mathbf{e}^j\).
\end{definition}

\begin{proposition}
  We have \(g_{ij} = g_{ji}, g^{ij} = g^{ji}\) and for 
  \(\mathbf{X}, \mathbf{Y} \in \mathbb{R}^n\), 
  \[\mathbf{X} \cdot \mathbf{Y} = g_{ij} X^i Y^j = g^{ij}X_i Y_j = X^i Y_i = X_i Y^i.\]
  From this, since \(\mathbf{X}\) is arbitrary, we deduce \(Y^i = g^{ij} Y_j\) 
  and \(Y_i = g_{ij} Y^j\). Furthermore, as \(Y_i = g^{ij}Y_j = g^{ij}g_{jk}Y^k\), 
  we have \(g^{ij}g_{jk} = \delta_{ik}\).
\end{proposition}

\subsection*{Variational Principles}

\begin{definition}[Kinetic Energy]
  The kinetic energy is defined to be the function 
  \[KE : TM \to \mathbb{R} : \dot q \mapsto \frac{1}{2}g_q(\dot q, \dot q),\]
  where \(g_q : TM^2 \to \mathbb{R}\) is a Riemannian metric.
\end{definition}

\begin{definition}[Lagrangian]
  The Lagrangian is a function \(L : TM \to \mathbb{R}\) which is commonly 
  chosen to be the kinetic energy.
\end{definition}

\begin{definition}[Hamilton's Principle]
  Given a family of curves \(q_\epsilon(t)\) for \((\epsilon, t) \in \mathbb{R}^2\) 
  such that \(q\) is smooth with respect to both parameters, we say the system 
  satisfies Hamilton's principle if \(\delta S = 0\) where 
  \(S = \int_\gamma L(q, \dot q) \dd t\), \(\gamma = q(\mathbb{R}^2)\) and 
  \(\delta = \pdv*{\epsilon}\mid_{\epsilon = 0}\).
\end{definition}

\(\delta S\) is known as the variational derivative of \(S\) near the identity 
\(\epsilon = 0\). Intuitively, we interpret Hamilton's principle as the 
notion that some quantity along some curve does not change given a small 
perturbation to said curve.

\begin{theorem}
  If Hamilton's principle is satisfied, i.e. \(\delta S = 0\), then so is 
  the Euler-Lagrange equation 
  \[\dv{t}\pdv{L(q, \dot q)}{\dot q} = \pdv{L(q, \dot q)}{q}.\]
\end{theorem}

\begin{definition}[Legendre Transform]
  The Legendre transform is a function \(LT_q : TQ \to T^*Q : \dot q \mapsto p\)
  that defined the momentum \(p\) as the fibre derivative of \(L\), namely, 
  \[p := \pdv{L(q, \dot q)}{\dot q}.\]
\end{definition}

The Legendre transform is invertible for \(\dot q = f(q, p)\) provided the 
Hessian \(\pdv*[2]{L(q, \dot q)}{\dot q}\) has a non-zero determinant.

\begin{definition}[Hamiltonian]
  The Hamiltonian is the function 
  \[H_q : T^* Q \to \mathbb{R} : p \mapsto \langle p, \dot q \rangle - L(q, \dot q).\]
\end{definition}

\end{document}
