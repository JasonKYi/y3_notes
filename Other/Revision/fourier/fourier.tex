% Options for packages loaded elsewhere
\PassOptionsToPackage{unicode}{hyperref}
\PassOptionsToPackage{hyphens}{url}
\PassOptionsToPackage{dvipsnames,svgnames*,x11names*}{xcolor}
%
\documentclass[]{article}
\usepackage{lmodern}
\usepackage{amssymb,amsmath}
\usepackage{ifxetex,ifluatex}
\ifnum 0\ifxetex 1\fi\ifluatex 1\fi=0 % if pdftex
  \usepackage[T1]{fontenc}
  \usepackage[utf8]{inputenc}
  \usepackage{textcomp} % provide euro and other symbols
\else % if luatex or xetex
  \usepackage{fontspec}
  \setmainfont{Bitstream Charter}
  % Cabin -- round
  % Bitstream Charter -- almost computer modern
  % \usepackage{unicode-math}
  % \defaultfontfeatures{Scale=MatchLowercase}
  % \defaultfontfeatures[\rmfamily]{Ligatures=TeX,Scale=1}
\fi
% Use upquote if available, for straight quotes in verbatim environments
\IfFileExists{upquote.sty}{\usepackage{upquote}}{}
\IfFileExists{microtype.sty}{% use microtype if available
  \usepackage[]{microtype}
  \UseMicrotypeSet[protrusion]{basicmath} % disable protrusion for tt fonts
}{}
\makeatletter
\@ifundefined{KOMAClassName}{% if non-KOMA class
  \IfFileExists{parskip.sty}{%
    \usepackage{parskip}
  }{% else
    \setlength{\parindent}{0pt}
    \setlength{\parskip}{6pt plus 2pt minus 1pt}}
}{% if KOMA class
  \KOMAoptions{parskip=half}}
\makeatother
\usepackage{xcolor}
\IfFileExists{xurl.sty}{\usepackage{xurl}}{} % add URL line breaks if available
\IfFileExists{bookmark.sty}{\usepackage{bookmark}}{\usepackage{hyperref}}
\hypersetup{
  pdftitle={Fourier Analysis Revision Notes},
  pdfauthor={Kexing Ying},
  colorlinks=true,
  linkcolor=Maroon,
  filecolor=Maroon,
  citecolor=Blue,
  urlcolor=red,
  pdfcreator={LaTeX via pandoc}}
\urlstyle{same} % disable monospaced font for URLs
\usepackage[margin = 1.5in]{geometry}
\usepackage{graphicx}
\makeatletter
\def\maxwidth{\ifdim\Gin@nat@width>\linewidth\linewidth\else\Gin@nat@width\fi}
\def\maxheight{\ifdim\Gin@nat@height>\textheight\textheight\else\Gin@nat@height\fi}
\makeatother
% Scale images if necessary, so that they will not overflow the page
% margins by default, and it is still possible to overwrite the defaults
% using explicit options in \includegraphics[width, height, ...]{}
\setkeys{Gin}{width=\maxwidth,height=\maxheight,keepaspectratio}
% Set default figure placement to htbp
\makeatletter
\def\fps@figure{htbp}
\makeatother
\setlength{\emergencystretch}{3em} % prevent overfull lines
\providecommand{\tightlist}{%
  \setlength{\itemsep}{0pt}\setlength{\parskip}{0pt}}
\setcounter{secnumdepth}{5}
\usepackage{tikz}
\usepackage{physics}
\usepackage{amsthm}
\usepackage{mathtools}
\usepackage{esint}
\usepackage[ruled,vlined]{algorithm2e}
\theoremstyle{definition}
\newtheorem*{theorem}{Theorem}
\newtheorem*{corollary}{Corollary}
\newtheorem*{remark}{Remark}
\newtheorem*{definition}{Definition}
\newtheorem*{lemma}{Lemma}
\newtheorem*{proposition}{Proposition}
\newtheorem*{example}{Example}
\newcommand{\diag}{\mathop{\mathrm{diag}}}
\newcommand{\Arg}{\mathop{\mathrm{Arg}}}
\newcommand{\hess}{\mathop{\mathrm{Hess}}}
\newcommand\eqae{\mathrel{\overset{\makebox[0pt]{\mbox{\normalfont\tiny\sffamily a.e.}}}{=}}}

\title{Fourier Analysis Revision Notes}
\author{Kexing Ying}

\begin{document}
\maketitle

\section*{Inner Product Spaces}

We denote \(R\) a (real or complex) inner product space (Euclidean space).

\begin{definition}[Complete System]
  A system \(\{X_\alpha\}_{\alpha \in A}\) is said to be complete if its linear 
  closure is \(R\), namely \(\langle X_\alpha \mid \alpha \in A\rangle = R\).
\end{definition}

\begin{definition}[Orthogonal Basis]
  A system is an orthogonal basis if it is orthogonal and complete.
\end{definition}

\begin{proposition}
  If \(R\) is separable, then any orthogonal system  of \(R\) is countable.
\end{proposition}

\begin{proposition}
  Any separable real inner product space possesses a orthonormal basis.
\end{proposition}

\begin{definition}[Fourier Coefficients]
  Given an orthonormal system \(\{\phi_n\}_{n = 1}^\infty\) of \(R\). The Fourier
  coefficients of any \(f \in R\) is defined to be 
  \[c_k := \langle f, \phi_k\rangle\]
  for all \(k\). The formal sum \(\sum_{k = 1}^\infty c_k \phi_k\) is called the 
  Fourier series of \(f\).
\end{definition}

\begin{definition}[Closed System]
  An othonormal system \(\{\phi_n\}\) is closed if 
  \[\sum_{k = 1}^\infty c_k^2 = \|f\|^2\]
  for all \(f \in R\). We call this property Parseval's identity.
\end{definition}

\begin{proposition}[Bessel's Inequality]
  Given the orthonormal system \(\{\phi_n\}\) of \(R\), we have 
  \[\sum_{k = 1}^\infty |c_k|^2 \le \|f\|^2\]
  for all \(f \in R\).
\end{proposition}

\begin{theorem}
  In a separable inner product space \(R\), an orthonormal system is complete 
  if and only if it is closed.
\end{theorem}

\begin{proposition}
  Given \(f, g\) and a closed system \(\{\phi_n\}\) of \(R\), \(\langle f, g\rangle = 
  \sum_{k = 1}^\infty c_k^f c_k^g\) where \(c_k^f, c_k^g\) are the Fourier coefficients
  of \(f, g\) respectively.
\end{proposition}

\section*{Contour Integration}

\begin{proposition}[Jordan's Lemma]
  If \(f\) is holomorphic except for finitely many singularities, and \(f(z) \to 0\) 
  as \(|z| \to \infty\), then 
  \[\int_{\gamma_R} f(z)e^{i\lambda z}\dd z \to 0\]
  for all \(\lambda > 0\) where \(\gamma_R\) is the upper half circle of radius \(R\) 
  centred at 0 oriented counter-clockwise.
\end{proposition}

In the case Jordan's lemma fails due to \(\lambda < 0\), try integrating on the 
lower half circle.

\begin{proposition}
  \(e^{iz} = 1 + O(|z|)\). Useful for integrating on small contours.
\end{proposition}

\section*{Fourier Series}

Smoother functions have quicker decaying of Fourier coefficients.

\begin{proposition}
  For all \(f \in L^2[-\pi, \pi]\), \(\|f - S_n\|_2 \to 0\) where \(S_n\) is the 
  \(n\)-th partial sum of the Fourier series of \(f\).
\end{proposition}

\begin{theorem}[Dini's Condition for Pointwise Convergence]
  If \(f \in L^1[-\pi, \pi]\) and for any \(x \in [-\pi, \pi]\), there exists some 
  \(\delta > 0\) such that 
  \[\int_{[-\delta, \delta]} \left|\frac{f(x + t) - f(x)}{t}\right| \lambda(\dd t) < \infty\]
  exists, then \(S_n(x) \to f(x)\) as \(n \to \infty\) for all \(x\).
\end{theorem}

Dini's condition is in some sense as strong as possible. Indeed, if 
\(\frac{f(x + t) - f(x)}{t}\) is not locally integrable at some \(x\), we can find 
a continuous function \(g\) with \(|g| \le f\) with non-convergence Fourier series 
at \(x\).

If \(f\) is continuous at \(x\) and has a derivative at \(x\) (or the limit exists 
from either the left or the right), then Dini's condition is satisfied at \(x\).

A continuous function with period \(2\pi\) is uniquely determined by its Fourier series. 
Furthermore, we can reconstruct a continuous function from its Fourier series by 
using the Fejer sums. Indeed, denoting \(\sigma_n\) the \(n\)-th Fejer sum, 
\(\sigma_n \to f\) uniformly.

\begin{proposition}[Poisson Summation Formula]
  Given \(f \in L^1\), 
  \[2\pi t \sum_{n = -\infty}^{\infty}f(2\pi t n) = \sum_{n = -\infty}^\infty \mathcal{F}[f](n / t).\]
\end{proposition}

\section*{Fourier Transform}

\begin{proposition}
  Let \(f \in L^1\). Then \(\mathcal{F}[f] = 0\) implies \(f = 0\) almost everywhere.
\end{proposition}

\begin{proposition}
  For \(f, f_n \in L^1\) such that \(f_n \to f\) in \(L^1\), then 
  \(\mathcal{F}[f_n] \to \mathcal{F}[f]\) uniformly.
\end{proposition}

\begin{proposition}
  For \(f \in L^1\), \(\mathcal{F}[f](y) \to 0\) as \(|y| \to \infty\).
\end{proposition}

\begin{corollary}
  For \(f \in L^1\), \(\mathcal{F}[f]\) is uniformly continuous.
\end{corollary}

\begin{proposition}
  For \(f \in L^1\) differentiable with \(f' \in L^1\) and \(f\) absolutely continuous 
  on any finite interval, \(\mathcal{F}[f'](y) = iy\mathcal{F}[f](y)\).
\end{proposition}

\begin{proposition}
  For \(f \in L^1\) such that \(xf(x) \in L^1\), we have \(\mathcal{F}[f]\) is 
  differentiable and \(D_y\mathcal{F}[f](y) = \mathcal{F}[-ixf(x)]\).
\end{proposition}

We also have the following properties: for \(f, g \in L^1\), \(c, c_1, c_2 \in \mathbb{R}\),
\begin{itemize}
  \item Linearity: \(\mathcal{F}[c_1 f + c_2 g] = c_1\mathcal{F}[f] + c_2 \mathcal{F}[g]\).
  \item Translation: \(\mathcal{F}[x \mapsto f(x - a)](y) = e^{-iay} \mathcal{F}[f](y)\).
  \item Rephasing: \(\mathcal{F}[x \mapsto e^{-icx}f(x)](y) = \mathcal{F}[f](y + c)\).
  \item Scaling: \(\mathcal{F}[x \mapsto f(cx)](y) = \frac{1}{|c|} \mathcal{F}[f](y / c)\).
  \item Convolution: \(\mathcal{F}[f * g] = \mathcal{F}[f] \mathcal{F}[g]\).
\end{itemize}

\section*{Distribution}

\(|x|' = \text{sgn}(x), \text{sgn}(x)' = 2\delta, \text{sgn}(x - a)' = 2\delta(x - a)\).

Suppose \(f \in S'\). Then,
\[\langle \mathcal{F}[f'], \phi\rangle = \langle f', \mathcal{F}[\phi]\rangle 
  = - \langle f, \mathcal{F}[\phi]'\rangle
  = \langle f, \mathcal{F}[it\phi(t)]\rangle = \langle \mathcal{F}[f], it\phi(t)\rangle
  = \langle ix \mathcal{F}[f], \phi\rangle\]
implying \(\mathcal{F}[f'] = ix \mathcal{F}[f]\). Hence, \(\mathcal{F}[f] = -ix^{-1}\mathcal{F}[f']\).

Similarly,
\[\begin{split} 
    \langle \mathcal{F}[f]', \phi \rangle & = - \langle \mathcal{F}[f], \phi'\rangle 
    = - \langle f, \mathcal{F}[\phi'] \rangle\\
    & = -\langle f, it \mathcal{F}[\phi](t)\rangle
    = - \langle ixf(x), \mathcal{F}[\phi] \rangle = -i \langle \mathcal{F}[xf(x)], \phi \rangle
\end{split}\]
so \(\mathcal{F}[xf(x)] = i \mathcal{F}[f]'\).

By a similar process, the Fourier transform of a tempered distribution satisfy 
all the normal properties as mentioned in the previous section.

\end{document}
