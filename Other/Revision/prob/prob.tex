% Options for packages loaded elsewhere
\PassOptionsToPackage{unicode}{hyperref}
\PassOptionsToPackage{hyphens}{url}
\PassOptionsToPackage{dvipsnames,svgnames*,x11names*}{xcolor}
%
\documentclass[]{article}
\usepackage{lmodern}
\usepackage{amssymb,amsmath}
\usepackage{ifxetex,ifluatex}
\ifnum 0\ifxetex 1\fi\ifluatex 1\fi=0 % if pdftex
  \usepackage[T1]{fontenc}
  \usepackage[utf8]{inputenc}
  \usepackage{textcomp} % provide euro and other symbols
\else % if luatex or xetex
  \usepackage{fontspec}
  \setmainfont{Bitstream Charter}
  % Cabin -- round
  % Bitstream Charter -- almost computer modern
  % \usepackage{unicode-math}
  % \defaultfontfeatures{Scale=MatchLowercase}
  % \defaultfontfeatures[\rmfamily]{Ligatures=TeX,Scale=1}
\fi
% Use upquote if available, for straight quotes in verbatim environments
\IfFileExists{upquote.sty}{\usepackage{upquote}}{}
\IfFileExists{microtype.sty}{% use microtype if available
  \usepackage[]{microtype}
  \UseMicrotypeSet[protrusion]{basicmath} % disable protrusion for tt fonts
}{}
\makeatletter
\@ifundefined{KOMAClassName}{% if non-KOMA class
  \IfFileExists{parskip.sty}{%
    \usepackage{parskip}
  }{% else
    \setlength{\parindent}{0pt}
    \setlength{\parskip}{6pt plus 2pt minus 1pt}}
}{% if KOMA class
  \KOMAoptions{parskip=half}}
\makeatother
\usepackage{xcolor}
\IfFileExists{xurl.sty}{\usepackage{xurl}}{} % add URL line breaks if available
\IfFileExists{bookmark.sty}{\usepackage{bookmark}}{\usepackage{hyperref}}
\hypersetup{
  pdftitle={Probability Theory Revision Notes},
  pdfauthor={Kexing Ying},
  colorlinks=true,
  linkcolor=Maroon,
  filecolor=Maroon,
  citecolor=Blue,
  urlcolor=red,
  pdfcreator={LaTeX via pandoc}}
\urlstyle{same} % disable monospaced font for URLs
\usepackage[margin = 1.5in]{geometry}
\usepackage{graphicx}
\makeatletter
\def\maxwidth{\ifdim\Gin@nat@width>\linewidth\linewidth\else\Gin@nat@width\fi}
\def\maxheight{\ifdim\Gin@nat@height>\textheight\textheight\else\Gin@nat@height\fi}
\makeatother
% Scale images if necessary, so that they will not overflow the page
% margins by default, and it is still possible to overwrite the defaults
% using explicit options in \includegraphics[width, height, ...]{}
\setkeys{Gin}{width=\maxwidth,height=\maxheight,keepaspectratio}
% Set default figure placement to htbp
\makeatletter
\def\fps@figure{htbp}
\makeatother
\setlength{\emergencystretch}{3em} % prevent overfull lines
\providecommand{\tightlist}{%
  \setlength{\itemsep}{0pt}\setlength{\parskip}{0pt}}
\setcounter{secnumdepth}{5}
\usepackage{tikz}
\usepackage{physics}
\usepackage{amsthm}
\usepackage{mathtools}
\usepackage{esint}
\usepackage[ruled,vlined]{algorithm2e}
\theoremstyle{definition}
\newtheorem*{theorem}{Theorem}
\newtheorem*{corollary}{Corollary}
\newtheorem*{remark}{Remark}
\newtheorem*{definition}{Definition}
\newtheorem*{lemma}{Lemma}
\newtheorem*{proposition}{Proposition}
\newtheorem*{example}{Example}
\newcommand{\diag}{\mathop{\mathrm{diag}}}
\newcommand{\Arg}{\mathop{\mathrm{Arg}}}
\newcommand{\hess}{\mathop{\mathrm{Hess}}}
\newcommand\eqae{\mathrel{\overset{\makebox[0pt]{\mbox{\normalfont\tiny\sffamily a.e.}}}{=}}}

\title{Probability Theory Revision Notes}
\author{Kexing Ying}

\begin{document}
\maketitle

\begin{theorem}
  For non-negative random variable \(\xi\), 
  \[\mathbb{E}\xi^k = k \int t^{k - 1} \mathbb{P}(\xi \ge t) \lambda(\dd t).\]
\end{theorem}

\begin{theorem}
  If \(\mathbb{E}|\xi_n|^k\) converges to 0 for some \(k\), then \(\xi_n \to 0\) 
  in probability.
\end{theorem}

\begin{theorem}
  If \(\sum \mathbb{P}(|\xi_n| \ge \epsilon) < \infty\), then \(\xi_n \to 0\) almost everywhere 
  (consider what it means for \(\omega \in \{\xi_n \not \to 0\}\)).  
\end{theorem}

\begin{corollary}
  If \(\sum \mathbb{E}|\xi_n|^k < \infty\) for some \(k\), then \(\xi_n \to 0\) almost everywhere.
\end{corollary}

\begin{theorem}
  \(\xi_n \to \xi\) almost every where if and only if \(\mathbb{P}(\sup_{k \ge n}|\xi_k - \xi| \ge \epsilon) \to 0\) 
  for all \(\epsilon > 0\) (Exercise sheet 5).
\end{theorem}

Method for showing SLLN without KSLLN (requires independence, equal mean (WLOG, mean 0), 
not necessary identically distributed):
\begin{itemize}
  \item By relabelling, we have by Kolmogorov's inequality 
  \[\mathbb{P}\left(\max_{2^n \le k \le 2^{n + 1}} |S_k| \ge \epsilon\right) \le 
    \frac{1}{\epsilon^2} \sum_{k = 2^n}^{2^{n + 1}} V_{\xi_k}.\]
  \item Choosing \(\epsilon\) to be \(n\epsilon\), we have 
  \[\mathbb{P}\left(\max_{2^n \le k \le 2^{n + 1}} \left|\frac{1}{n}S_k\right| \ge \epsilon\right) \le 
    \frac{1}{n^2\epsilon^2} \sum_{k = 2^n}^{2^{n + 1}} V_{\xi_k}.\]
  \item Summing over \(n\), we have 
  \[\sum_{n = 1}^\infty \mathbb{P}\left(\max_{2^n \le k \le 2^{n + 1}} \left|\frac{1}{n}S_k\right| \ge \epsilon\right) \le 
    \sum_{n = 1}^\infty \frac{1}{n^2\epsilon^2} V_{\xi_n} < \infty\]
  if \(V_{\xi_n} \le 1\) or some other conditions.
  \item Hence, by the first Borel-Cantelli lemma, 
  \[\mathbb{P}\left(\left|\frac{1}{n}S_k\right| \ge \epsilon \text{ i.o.}\right) = 
    \mathbb{P}\left(\max_{2^n \le k \le 2^{n + 1}} \left|\frac{1}{n}S_k\right| \ge \epsilon \text{ i.o.}\right) = 0.\]
  \item Thus, for all \(\epsilon > 0\)
  \[\mathbb{P}\left(\bigcup_{n = 1}^\infty \bigcap_{m \ge n} \left|\frac{1}{n} S_k\right| < \epsilon\right)
    = \mathbb{P}\left\{\left|\frac{1}{n}S_k\right| \ge \epsilon \text{ i.o.}\right\}^c = 1\]
  which implies convergence to 0 almost everywhere by intersecting over \(\epsilon\).
\end{itemize}

\section*{Characteristic Function}

Let \(\xi\) be a random variable and let \(\phi\) be its characteristic function
\begin{itemize}
  \item \(\phi(t) =\mathbb{E}e^{it\xi}\);
  \item \(|\phi(t)| \le \phi(0) = 1\);
  \item \(\phi(-t) = \overline{\phi(t)}\);
  \item \(\phi\) is uniformly continuous on \(\mathbb{R}\);
  \item \(\mathbb{E}\xi^r = i^{-r}\phi^{(r)}(0)\) if \(\mathbb{E}|\xi|^n < \infty\) where \(r \le n\);
  \item for random variables \(\xi_1, \cdots, \xi_n\), the characteristic function of 
    \(\sum \xi_i\) is \(\prod \phi_i\) if and only if \(\xi_i\) are independent;
  \item convex linear combinations of characteristic functions is a characteristic function;
  \item given \(\alpha \in \mathbb{R}\), \(\overline{\phi}, \text{Re}(\phi), |\phi|^2\) and \(\phi(\alpha t)\) are all characteristic functions;
  \item if \(|\phi(t_0)| = 1\) for some \(t_0 \neq 0\), then \(\xi\) is a pure point random variable;
  \item if \(|\phi(t)| = 1\) for all \(|t| \in (-\epsilon, \epsilon)\) then \(\xi\) is degenerate;
  \item see also Bochner, Polya and Marcinkievicz theorems.
\end{itemize}

\begin{theorem}
  If \(\xi\) is a discrete random variable with characteristic function \(\phi\), then 
  \[\mathbb{P}(\xi = k) = \frac{1}{2\pi}\int_{[-\pi, \pi]} e^{-ikt}\phi(t) \lambda(\dd t).\]
\end{theorem}

\end{document}
