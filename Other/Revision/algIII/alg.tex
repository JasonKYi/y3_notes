% Options for packages loaded elsewhere
\PassOptionsToPackage{unicode}{hyperref}
\PassOptionsToPackage{hyphens}{url}
\PassOptionsToPackage{dvipsnames,svgnames*,x11names*}{xcolor}
%
\documentclass[]{article}
\usepackage{lmodern}
\usepackage{amssymb,amsmath}
\usepackage{ifxetex,ifluatex}
\ifnum 0\ifxetex 1\fi\ifluatex 1\fi=0 % if pdftex
  \usepackage[T1]{fontenc}
  \usepackage[utf8]{inputenc}
  \usepackage{textcomp} % provide euro and other symbols
\else % if luatex or xetex
  \usepackage{fontspec}
  \setmainfont{Bitstream Charter}
  % Cabin -- round
  % Bitstream Charter -- almost computer modern
  % \usepackage{unicode-math}
  % \defaultfontfeatures{Scale=MatchLowercase}
  % \defaultfontfeatures[\rmfamily]{Ligatures=TeX,Scale=1}
\fi
% Use upquote if available, for straight quotes in verbatim environments
\IfFileExists{upquote.sty}{\usepackage{upquote}}{}
\IfFileExists{microtype.sty}{% use microtype if available
  \usepackage[]{microtype}
  \UseMicrotypeSet[protrusion]{basicmath} % disable protrusion for tt fonts
}{}
\makeatletter
\@ifundefined{KOMAClassName}{% if non-KOMA class
  \IfFileExists{parskip.sty}{%
    \usepackage{parskip}
  }{% else
    \setlength{\parindent}{0pt}
    \setlength{\parskip}{6pt plus 2pt minus 1pt}}
}{% if KOMA class
  \KOMAoptions{parskip=half}}
\makeatother
\usepackage{xcolor}
\IfFileExists{xurl.sty}{\usepackage{xurl}}{} % add URL line breaks if available
\IfFileExists{bookmark.sty}{\usepackage{bookmark}}{\usepackage{hyperref}}
\hypersetup{
  pdftitle={Algebra III Revision Notes},
  pdfauthor={Kexing Ying},
  colorlinks=true,
  linkcolor=Maroon,
  filecolor=Maroon,
  citecolor=Blue,
  urlcolor=red,
  pdfcreator={LaTeX via pandoc}}
\urlstyle{same} % disable monospaced font for URLs
\usepackage[margin = 1.5in]{geometry}
\usepackage{graphicx}
\makeatletter
\def\maxwidth{\ifdim\Gin@nat@width>\linewidth\linewidth\else\Gin@nat@width\fi}
\def\maxheight{\ifdim\Gin@nat@height>\textheight\textheight\else\Gin@nat@height\fi}
\makeatother
% Scale images if necessary, so that they will not overflow the page
% margins by default, and it is still possible to overwrite the defaults
% using explicit options in \includegraphics[width, height, ...]{}
\setkeys{Gin}{width=\maxwidth,height=\maxheight,keepaspectratio}
% Set default figure placement to htbp
\makeatletter
\def\fps@figure{htbp}
\makeatother
\setlength{\emergencystretch}{3em} % prevent overfull lines
\providecommand{\tightlist}{%
  \setlength{\itemsep}{0pt}\setlength{\parskip}{0pt}}
\setcounter{secnumdepth}{5}
\usepackage{tikz}
\usepackage{physics}
\usepackage{amsthm}
\usepackage{mathtools}
\usepackage{esint}
\usepackage[ruled,vlined]{algorithm2e}
\theoremstyle{definition}
\newtheorem*{theorem}{Theorem}
\newtheorem*{corollary}{Corollary}
\newtheorem*{remark}{Remark}
\newtheorem*{definition}{Definition}
\newtheorem*{lemma}{Lemma}
\newtheorem*{proposition}{Proposition}
\newtheorem*{example}{Example}
\newcommand{\diag}{\mathop{\mathrm{diag}}}
\newcommand{\Arg}{\mathop{\mathrm{Arg}}}
\newcommand{\hess}{\mathop{\mathrm{Hess}}}
\newcommand\eqae{\mathrel{\overset{\makebox[0pt]{\mbox{\normalfont\tiny\sffamily a.e.}}}{=}}}

\title{Algebra III Revision Notes}
\author{Kexing Ying}

\begin{document}
\maketitle

\section*{Ring Theory}

In a PID
\[\text{prime} \iff \text{irreducible} \iff \text{maximal ideal} \iff \text{quotient is a field}
  \iff \text{quotient ID.}\]
To find irreducible elements in a Euclidean domain, note that
\begin{itemize}
  \item invertible elements have norm 1
  \item elements have prime norm iff irreducible (Gaussian integers \(a + bi\), \(ab \neq 0\))
\end{itemize}
Finding GCD in (multiplicative) Euclidean domain:
\begin{itemize}
  \item compute the norms
  \item find GCD of the norms \(l\), the norm of the GCD must divide \(l\)
  \item find the elements which has norm dividing \(l\)
\end{itemize}
Given ideals \(I = \langle a_i \mid i \in \mathcal{I}\rangle, J = \langle b_j \mid j \in \mathcal{J}\rangle\),
\[IJ = \langle a_ib_j \mid i \in \mathcal{I}, j \in \mathcal{J}\rangle.\]

In a Euclidean domain (requires \(N(ab) = N(a)N(b) \ge N(a)\)) \(R\), \(N(a) = N(1)\) iff \(a \in R^\times\).

Show ideal not principle in \(\mathbb{Z}[\sqrt{-n}]\): Define the multiplicative function 
\(N : \mathbb{Z}[\sqrt{-n}] \to \mathbb{N} : a + b \sqrt{-n} \mapsto a^2 + n b^2\). Then,
if \(\langle \alpha, \beta \rangle = \langle \gamma \rangle\), \(N(\gamma) \mid N(\alpha), N(\beta)\),
from this show contradiction.

\section*{Field Extensions}

We will in this section denote \(K \subseteq L\) fields and assume \(L\) as a vector space 
over \(K\) has finite dimension.

\begin{definition}[Degree]
  Let \(K \subseteq L\) be fields. Then \([L : K] := \dim_K L\) is the degree of \(L\) over \(K\).
\end{definition}

\begin{theorem}
  Given field extensions \(K \subseteq L \subseteq M\), 
  \[[M : L][L : K] = [M : K].\]
\end{theorem}

Given \(\alpha \in L\) we define the evaluation map \(\text{ev}_\alpha : K[X] \to L : P(X) \mapsto P(\alpha)\).
We have the following observations.
\begin{itemize}
  \item assume \(\ker \text{ev}_\alpha \neq \{0\}\) and so, as \(K[X]\) is a PID (as it is an Euclidean domain), 
    there exists some \(P(X) \in K[X]\) such that \(\ker \text{ev}_\alpha = \langle P(X) \rangle\). We call 
    \(P(X)\) is \textit{minimal polynomial} of \(\alpha\);
  \item as \(\text{Im}(\text{ev}_\alpha)\) is a subring of \(K\), \(K[X] / \langle P(X) \rangle \simeq \text{Im}(\text{ev}_\alpha)\)
    is an integral domain, and so, \(\langle P(X) \rangle\) is a prime ideal;
  \item recalling that non-zero prime ideals in a PID are maximal, \(K[X] / \langle P(X) \rangle\) 
    is in fact a field;
  \item we denote \(K(\alpha)\) as \(K[X] / \langle P(X) \rangle\).
\end{itemize}

\begin{theorem}
  Applying quotient remainder using the Euclidean norm on \(K[X]\), 
  \([K(\alpha) : K] = \deg P(X)\).
\end{theorem}

\begin{theorem}
  If \([L : K] < \infty\), then every element of \(L\) is algebraic over \(K\), 
  i.e. for all \(\alpha \in L\), there exists \(P(X) \in K[X]\) such that \(P(\alpha) = 0\).
  This follows as \(1, \alpha, \cdots, \alpha^{[L : K]}\) is linear dependent as vectors 
  in \(L\) over \(K\).
\end{theorem}

\begin{theorem}
  If \(K\) has characteristics \(p\), then the map \(x \mapsto x^q\) is a field endomorphism 
  and is known as the Frobenious endomorphism.
\end{theorem}

\begin{theorem}
  If \(|K| = q = p^r\), then the map \(x \mapsto x^q\) is a field automorphism.
\end{theorem}

\begin{theorem}
  If \(A\) is an abelian group such that \(|A| = n\) and for all \(d \mid n\), \(A\) has exactly 
  \(d\) elements of order dividing \(d\), then \(A\) is cyclic. 
\end{theorem}

\begin{corollary}
  \(K^\times\) is cyclic since \(X^{p^r - 1} - 1\) factors into \(p^r - 1\) distinct roots in 
  \(K\) and has divisor \(X^d - 1\) for every \(d \mid |K|\).
\end{corollary}

\begin{theorem}
  If \(K\) has characteristics \(p\), then \(K = \mathbb{F}_p(\alpha)\) for some \(\alpha\). 
  Namely, we consider \(K\) as a field extension of \(\mathbb{F}_p\) by recalling the theory of 
  prime fields.
\end{theorem}

\begin{theorem}
  All irreducible polynomials of \(K[X]\) where \(|K| = q = p^s\) of degree \(r\) appears exactly once 
  in the divisors of \(X^{q^r} - X\).
\end{theorem}

\begin{corollary}
  Any two finite fields of the same cardinality are isomorphic.
\end{corollary}

Over \(\mathbb{F}_p\), \(P(X^p) = P(X)^p\) as \(X \mapsto X^p\) is a field automorphism.

Product of roots of a monic polynomial is the \((-1)^d c_0\) where \(c_0\) is the constant coefficient. 
So, if the \(c_0 = 1\) of a monic polynomial in \(\mathbb{Z}[X]\), the roots must be \(\pm 1\) which 
can be manually checked.

One may apply the quadratic formula for a degree 2 polynomial in \(\mathbb{C}[X, Y]\).

Given \(P(X), Q(X) \in K[X]\), \(P\) irreducible. Then, if \(P, Q\) share a root \(P \mid Q\) 
(proof by considering their GCD cannot be 1 by Bezout).

Use \textbf{derivatives} to show polynomial has no repeated factors (show it has no shared roots with 
its derivative).

\textbf{Product of roots = constant coeff. of monic} (useful for checking the degree 3 polynomial is irreducible).

\subsection*{Factorisation in UFD}

\begin{theorem}[Gauss's Lemma]
  Let \(R\) be a UFD and let \(P(X), Q(X) \in R[X]\) be primitive (i.e. coefficients have GCD 1), 
  then the produce \(P(X)Q(X)\) is also primitive.
\end{theorem}

\begin{corollary}
  If \(P(X) \in R[X]\) has a divisor \(A(X) \in K[X]\) where \(K\) is the field of fraction of \(R\), 
  then there exists some \(\alpha \in K^\times\) such that \(\alpha A(X) \in R[X]\) and 
  \(\alpha A(X) \mid P(X)\).
\end{corollary}

\begin{corollary}
  If \(R\) is a UFD then so is \(R[X]\).
\end{corollary}

\begin{proposition}
  Let \(P(X) \in R[X]\). \(P(X)\) is irreducible if and only if \(Q_r(X) := r^d P(X / r)\) is 
  irreducible for any \(r \in R\) where \(d = \deg P(X)\).
\end{proposition}

\begin{proposition}
  Let \(P(X) \in R[X]\) be monic and let \(I\) be a prime ideal of \(R\). Then, if 
  \(P(X)\) is irreducible as a polynomial in \(R / I[X]\), then \(P(X)\) is irreducible 
  in \(R[X]\).
\end{proposition}

\begin{proposition}[Eisenstein's Criterion]
  Let \(P(X) = c_0 + c_1X + \cdots + X^n\) be a monic polynomial in \(R[X]\) and let
  \(I\) be a prime ideal of \(R\). Then, if for all \(0 \le i \le n - 1\), 
  \(c_i \in I\) and \(c_0 \not\in I^2\), then \(P(X)\) is irreducible in \(R[X]\).
\end{proposition}

Showing \(\sum_{i = 0}^n X^i\) is irreducible in \(\mathbb{Q}[X]\): write \(T = X - 1\) and we 
observe 
\[\sum_{i = 0}^n X^i = \frac{X^{n + 1} - 1}{X - 1} = \frac{(T + 1)^{n + 1} - 1}{T} = T^n + \cdots =: P(T)\]
where by the binomial formula, all non-leading coefficients of \(P(T)\) are divisible by \(n + 1\) 
while the constant term is \(n + 1\) and so not divisible by \((n + 1)^2\). Thus, if \(n + 1\) is 
prime, we may conclude irreducibility by Eisentein's. Alternative adjustments can be made if 
\(n + 1\) is not prime.

\section*{Module Theory}

Basic concepts: module, submodule, generating set, finitely generated, quotient, direct sum, 
module homomorphism, kernel, image, universal property for modules, free module, universal 
property of free modules.

Let \(M\) be any \(R\)-module with generating set \(S\). Then, 
by the universal property of free modules, we have the natural homomorphism 
\(\tilde \iota : R[S] \to M\) induced by the inclusion map \(\iota : S \hookrightarrow M\).
Now, denoting \(K = \ker \tilde \iota\), let \(T\) be a generating set of \(K\), 
we have the short exact sequence 
\[0 \to R[T] \to R[S] \to M \to 0.\]
Hence, denoting the first map by \(\phi\), we have the isomorphism 
\[M \simeq R[S] / \phi(R[T])\]
and this is called a presentation of \(M\). If both modules in the quotient have finite 
rank, \(M\) is called finitely presented. In the case both \(|S| = n, |T| = m\) are finite, 
one can encode the information of \(M\) into a presentation matrix \(\Phi\) where 
\[\phi(t_i) = \sum_{j = 1}^n \Phi_{ji}s_j.\]
Namely, \(\Phi : R^m \to R^n\) is a homomorphism such that 
\[M \simeq R^n / \Phi R^m.\]
Elementary column operations changes the relation to an equivalent set of relations 
while elementary row operations changes the generating set.

Smith normal form algorithm on the matrix \(A\) over a Euclidean domain:

The main step is to write \(A\) in a form such that \(a_{11}\) is the only 
non-zero entry in the first row and the first column such that \(a_{11}\) divides every entry of \(A\).
\begin{itemize}
  \item Exchange rows and column such that \(a_{11}\) has the smallest Euclidean norm.
  \item Applying Euclidean algorithm to make each entry of the first row and column 
    have norm strictly less than \(a_{11}\) by adding multiples of the first row and 
    column to the other rows and columns.
  \item Repeat the above two steps until all entries of the first row and column are zero. 
  \item Now, if there is some \(a_{ij}\) not divisible by \(a_11\), add the \(i\)-th column 
    to the first (this does not change \(a_{11}\) as \(a_{1i} = 0\)).
  \item Repeat from step 1 (this will decrease the norm of \(a_{11}\)).
\end{itemize}
Then, applying this algorithm to the minors of \(A\) inductively, we end up with the 
Smith normal form of \(A\).  

\begin{theorem}[Classification of finitely generated modules over a Euclidean domain]
  Let \(M\) be a finitely generated module over a Euclidean domain \(R\). Then, there 
  exists a unique integer \(r\) and non-units \(a_1, \cdots, a_n\) of \(R\) such that 
  \(a_i \mid a_{i + 1}\) and 
  \[M \simeq R^r \oplus R / \langle a_1 \rangle \oplus R / \langle a_2 \rangle 
    \oplus \cdots \oplus R / \langle a_n \rangle.\] 
\end{theorem}
The above classification holds for PID as well though we did not prove it.

\begin{corollary}
  Let \(A\) be a finitely generated abelian group. Then there is a unique integer 
  \(r\) and integers \(a_1, \cdots, a_t > 1\) with \(a_1 \mid a_2 \mid \cdots \mid a_t\) 
  such that 
  \[A \simeq \mathbb{Z}^r \oplus \mathbb{Z} / a_1\mathbb{Z} \oplus \mathbb{Z} / a_2\mathbb{Z} 
    \oplus \cdots \oplus \mathbb{Z} / a_t\mathbb{Z}.\]
\end{corollary}

Let \(K\) be a field, \(V\) a finite dimensional \(K\)-vector space and let \(L : V \to V\) 
a linear map, then we define the \(K[T]\)-module (or \(K[X]\)-module) \(M_L\) by taking 
the underlying set \(V\) and defining scalar multiplication by 
\[P(T) \cdot v := P(L)(v)\]
for all \(P(T) \in K[T]\) and \(v \in V\) where \(P(L) : V \to V\) is the linear map 
obtained by applying \(P\) on \(L\).

It is clear that a basis \(\{v_1, \cdots, v_m\}\) of \(V\) generates \(M_L\) as a 
\(K[T]\)-module. Furthermore, writing \(A\) the matrix corresponding to \(L\) with 
respect to this basis so that \(Lv_i = \sum_{j = 1}^m a_{ji} v_j\), we have the relations 
\(T \in K[T]\), \(T \cdot v_i = L v_i = \sum_{j = 1}^m a_{ji} v_j\). These relations 
generate all relations. Thus, \(M_L\) has presentation matrix \(T \text{id}_m - A\).

Now, as \(M_L\) is finitely generated as a \(K[T]\)-module, by the classification of 
finitely generated modules, we have \(P_1(T), \cdots, P_t(T) \in K[T]\) such that 
\(P_1(T) \mid \cdots \mid P_t(T)\) and 
\[M_L \simeq K[T] / \langle P_1(T) \rangle \oplus \cdots \oplus K[T] / \langle P_t(T) \rangle.\]
where the rank is 0 as the number of generators of \(M_L\) equals the number of 
generators of the relations.

Now, by considering \(K[T] / \langle P(T) \rangle\) has basis \(1, T, \cdots, T^{d - 1}\) where 
\(d = \deg P(T)\), \(L\) restricted on \(K[T] / \langle P(T) \rangle\) has action 
``multiplication by \(T\)'' represented by the companion matrix of \(P(T)\) with respect 
to this basis. Hence, we have the rational canonical form of \(L\) given by 
block matrices for which the \(i\)-th block is the companion matrix of \(P_i(T)\).

Thus, if \(K\) is algebraically closed, each \(P_i(T)\) can be factored into a product of 
linear factors which allows us to conclude the Jordan normal form theorem.

Writting 
\[M_L \simeq K[T] / \langle P_1(T) \rangle \oplus \cdots \oplus K[T] / \langle P_t(T) \rangle,\]
the minimal polynomial of \(L\) is \(P_t\) and the characteristic polynomial of \(L\) 
is \(\prod P_i\).

The minimal polynomial of a linear map \(L\) is the unique monic generator of the ideal 
\(\{P(T) \mid P(L) = 0\}\) of \(K[T]\).

\subsection*{Noetherian}

\begin{definition}[Notherian]
  An \(R\)-module \(M\) is Noetherian if every increasing chain of \(M\)-submodules 
  stabilizes.
\end{definition}

\begin{theorem}
  An \(R\)-module \(M\) is Noetherian if and only if every \(M\)-submodule is finitely 
  generated.
\end{theorem}

\begin{corollary}
  Any PID is Noetherian over itself.
\end{corollary}

Some properties of Noetherian modules:
\begin{itemize}
  \item The quotient of Noetherian modules is Noetherian.
  \item If \(N \le M\) is Noetherian and \(M / N\) is Noetherian, then so is \(M\).
  \item Direct sums of Noetherian modules is Noetherian.
  \item \(R\) Noetherian implies \(R^k\) Noetherian as it is a direct sum of \(k\)-copies of \(R\).
  \item If \(R\) is a Noetherian ring and \(f : R \to S\) is a surjective ring homomorphism, then \(S\) is Noetherian.
\end{itemize}

\begin{theorem}
  Any finitely generated module over a Noetherian ring is Noetherian.
\end{theorem}

\begin{theorem}[Hilbert Basis Theorem]
  \(R\) is Noetherian if and only if \(R[X]\) is Noetherian.
\end{theorem}

\end{document}
