% Options for packages loaded elsewhere
\PassOptionsToPackage{unicode}{hyperref}
\PassOptionsToPackage{hyphens}{url}
\PassOptionsToPackage{dvipsnames,svgnames*,x11names*}{xcolor}
%
\documentclass[]{article}
\usepackage{lmodern}
\usepackage{amssymb,amsmath}
\usepackage{ifxetex,ifluatex}
\ifnum 0\ifxetex 1\fi\ifluatex 1\fi=0 % if pdftex
  \usepackage[T1]{fontenc}
  \usepackage[utf8]{inputenc}
  \usepackage{textcomp} % provide euro and other symbols
\else % if luatex or xetex
  \usepackage{unicode-math}
  \defaultfontfeatures{Scale=MatchLowercase}
  \defaultfontfeatures[\rmfamily]{Ligatures=TeX,Scale=1}
\fi
% Use upquote if available, for straight quotes in verbatim environments
\IfFileExists{upquote.sty}{\usepackage{upquote}}{}
\IfFileExists{microtype.sty}{% use microtype if available
  \usepackage[]{microtype}
  \UseMicrotypeSet[protrusion]{basicmath} % disable protrusion for tt fonts
}{}
\makeatletter
\@ifundefined{KOMAClassName}{% if non-KOMA class
  \IfFileExists{parskip.sty}{%
    \usepackage{parskip}
  }{% else
    \setlength{\parindent}{0pt}
    \setlength{\parskip}{6pt plus 2pt minus 1pt}}
}{% if KOMA class
  \KOMAoptions{parskip=half}}
\makeatother
\usepackage{xcolor}\pagecolor[RGB]{28,30,38} \color[RGB]{213,216,218}
\IfFileExists{xurl.sty}{\usepackage{xurl}}{} % add URL line breaks if available
\IfFileExists{bookmark.sty}{\usepackage{bookmark}}{\usepackage{hyperref}}
\hypersetup{
  pdftitle={Group Representation Theory},
  pdfauthor={Kexing Ying},
  colorlinks=true,
  linkcolor=Maroon,
  filecolor=Maroon,
  citecolor=Blue,
  urlcolor=red,
  pdfcreator={LaTeX via pandoc}}
\urlstyle{same} % disable monospaced font for URLs
\usepackage[margin = 1.5in]{geometry}
\usepackage{graphicx}
\makeatletter
\def\maxwidth{\ifdim\Gin@nat@width>\linewidth\linewidth\else\Gin@nat@width\fi}
\def\maxheight{\ifdim\Gin@nat@height>\textheight\textheight\else\Gin@nat@height\fi}
\makeatother
% Scale images if necessary, so that they will not overflow the page
% margins by default, and it is still possible to overwrite the defaults
% using explicit options in \includegraphics[width, height, ...]{}
\setkeys{Gin}{width=\maxwidth,height=\maxheight,keepaspectratio}
% Set default figure placement to htbp
\makeatletter
\def\fps@figure{htbp}
\makeatother
\setlength{\emergencystretch}{3em} % prevent overfull lines
\providecommand{\tightlist}{%
  \setlength{\itemsep}{0pt}\setlength{\parskip}{0pt}}
\setcounter{secnumdepth}{5}
\usepackage{tikz}
\usepackage{physics}
\usepackage{amsthm}
\usepackage{mathtools}
\usepackage{esint}
\usepackage{pst-node}
\usepackage{auto-pst-pdf}
\usepackage{tikz-cd} 
\usepackage[ruled,vlined]{algorithm2e}
\theoremstyle{definition}
\newtheorem{theorem}{Theorem}
\newtheorem{definition*}{Definition}
\newtheorem{prop}{Proposition}
\newtheorem{corollary}{Corollary}[theorem]
\newtheorem*{remark}{Remark}
\theoremstyle{definition}
\newtheorem{definition}{Definition}[section]
\newtheorem{lemma}{Lemma}[section]
\newtheorem{proposition}{Proposition}[section]
\newtheorem{example}{Example}[section]
\newcommand{\diag}{\mathop{\mathrm{diag}}}
\newcommand{\Arg}{\mathop{\mathrm{Arg}}}
\newcommand{\hess}{\mathop{\mathrm{Hess}}}

\title{Group Representation Theory}
\author{Kexing Ying}
\date{July 24, 2021}

\begin{document}
\maketitle

{
\hypersetup{linkcolor=}
\setcounter{tocdepth}{2}
\tableofcontents
}
\newpage

\section{Introduction}

Group representation theory is a field of mathematics that applies linear algebra 
to study properties of groups. The field itself originated through a letter 
from Dedekind to Frobenius in which he noted that, given \(f = \det A\), where 
\(A\) is the Cayley table of a group of \(n\) elements, by factorising \(f\) 
into irreducible polynomials, \(f = \prod_i f_i^{d_i}\), we have \(d_i = \deg f_i\). 
And this led Frobenius to invent group representation theory.

Group representation theory is applicable in many different areas.
\begin{itemize}
  \item Group theory arises in Klein's "Erlangen program" as symmetries of 
    geometric spaces.
  \item Burnside in 1904 proves the following using representation theory 
    (and so shall we later on)
    \begin{proposition}
      Let \(G\) be a group such that \(|G| = p^r q^s\) where \(p, q\) are 
      prime and \(r + s \ge 2\), then \(G\) is not simple.
    \end{proposition}
  \item In number theory, representations of Galois groups arises in the 
    number field case 
    \[\overline{F} / F, \mathbb{Q} \subseteq F, [F : \mathbb{Q}] < \infty,\]
    which has implications in Wiles' proof of Fermat's last theorem.
  \item In chemistry the symmetry and rotation of molecules can be represented 
    by group actions.
  \item In quantum mechanics, spherical symmetry gives rise to discrete energy 
    levels, orbitals, etc.
  \item In differential geometry, the vector space of solutions is a representation 
    of the symmetry group of an equation.
\end{itemize}

Recalling the definition of a group, informally, the representation of a group \(G\) 
is a way if writing group elements as linear transformations of a vector space 
such that the natural group properties are satisfied. 

Some examples of group representations are the following:
\begin{itemize}
  \item For all group \(G\), the trivial representation of \(G\) is \(\rho\) such 
    that \(\rho(g) = \text{id}\) for all \(g \in G\).
  \item Let \(\zeta \in \mathbb{C}\) be a \(n\)-th root of \(1\) and let 
    \(G = C_n = \{1, g, \cdots, g^{n - 1}\}\). Then \(\rho : g^i \mapsto (\zeta^i)\)
    is a representation of \(G\).
  \item In the case \(G = S_n\), the mapping of \(\sigma \in S_n\) to its 
    corresponding permutation matrix \(P_\sigma\) is a representation of \(G\).
  \item Another representation of \(S_n\) is 
    \(\sigma \in S_n \mapsto (\text{sign}(\sigma))\)\footnote{\(\text{sign}(\sigma) = \det P_\sigma\)}.
  \item Let \(G = D_n\) the dihedral group of order \(2n\). Then, a representation 
    \(D_n\) maps elements of \(D_n\) to the corresponding \(2 \times 2\) matrices 
    which rotates/reflects \(\mathbb{R}^2\) by the appropriate amount.
\end{itemize}

We shall in this module study and construct representations, and furthermore, 
classify up to isomorphism finite-dimensional complex representations of 
every finite group \(G\). 

\section{Fundamentals of Group Representation}

\begin{definition}[Representation]
  Let \(G\) be a group, then a representation of \(G\) is the pair \((V, \rho)\) 
  where \(V\) is a (finite-dimensional) vector space and \(\rho : G \mapsto GL(V)\) 
  is a group homomorphism.
\end{definition}

Alternatively, we may consider a group representation of \(G\) is a group action 
\((\cdot) : G \times V \to V : (g, v) \mapsto v\) such that \((\cdot)\) is 
linear with respect to the second parameter. In particular, we recall a group 
action \((\cdot)\) satisfies \(e \cdot v = v\) and 
\(g \cdot (h \cdot v) = gh \cdot v\). 

\begin{definition}[Dimension of a Representation]
  If \((V, \rho)\) is a representation of \(G\), then the dimension of 
  \((V, \rho)\) is \(\dim(V, \rho) = \dim V\).
\end{definition}

Similar to other objects in mathematics, we introduce a notion of morphisms 
between representations.

\begin{definition}[Homomorphism of Representation]
  Let \(G\) be a group and \((V, \rho_V)\) and \((W, \rho_W)\) be two representations 
  of \(G\). Then a homomorphism of representations is a linear map 
  \(T : V \to W\) such that for all \(g \in G\),
  \[T \circ \rho_V(g) = \rho_W(g) \circ T.\]
  Furthermore, we say \(T\) is an isomorphism is bijective (or equivalently, 
  it has an inverse which is also a homomorphism).
\end{definition}

In particular, one might imagine the homomorphism as a linear map such that 
the following diagram commute.
\[\begin{tikzcd}
  V \arrow{r}{T} \arrow[swap]{d}{\rho_V(g)} & W \arrow{d}{\rho_W(g)} \\
  V \arrow{r}{T}& W
  \end{tikzcd}\]
As with any definitions which work with finite-dimensional vector spaces, 
there are equivalent but ``worse'' (as we will have to choose a basis) corresponding 
definitions in terms of matrices. Nonetheless, these definitions with matrices 
are easier computationally and we shall recall the contrast here.

Clearly, if \(G\) is a group and \((\mathbb{C}^n, \rho)\) is a representation, 
we have \(\rho(e) = I_n\). Furthermore, we have a natural isomorphism between 
\(GL_n(\mathbb{C}) \cong GL(\mathbb{C}^n)\) and more generally 
\(\text{Mat}_{n, m}(\mathbb{C}) \cong \text{Hom}(\mathbb{C}^n, \mathbb{C}^m)\).
Similarly, given a representation \((V, \rho)\), with \(\dim V < \infty\), we may 
choose a basis \(B\) of \(V\) and write the representation as a 
matrix which we denote \(\rho^B(g) = [\rho(g)]_B\). Thus, we may use first year 
linear algebra methods to manipulate representations. 

\begin{definition}
  Given two matrix representations \(\rho, \rho' : G \mapsto GL_n(\mathbb{C})\), 
  we say \(\rho\) and \(\rho'\) are equivalent/isomorphic if there exists 
  \(P \in GL_n(\mathbb{C})\) such that for all \(g \in G\), 
  \(\rho'(g) = P^{-1} \rho(g) P\).
\end{definition}
  
This definition is motivated by the following.

\begin{proposition}
  Given \((V, \rho_V)\) and \((W, \rho_W)\) representations of \(G\),
  we have \(\rho_V \cong \rho_W\) if and only if there exists some 
  \(P \in GL_n(\mathbb{C})\) such that for all \(g \in G\), 
  \(\rho_W^C(g) = P^{-1} \rho_V^BP(g) P\) for some basis \(B, C\) of \(V\) and 
  \(W\) respectively.
\end{proposition}
\begin{proof}
  Exercise.
\end{proof}

\begin{proposition}
  Given a cyclic group \(C_n = \langle g \rangle\) with representations 
  \((V, \rho_V)\) and \((W, \rho_W)\) of equal dimensions, we have 
  \(\rho_V \cong \rho_W\) if and only if \(\rho_V^B(g)\) is conjugate to 
  \(\rho_W^C(g)\) for some basis \(B, C\) of \(V\) and \(W\) respectively.
\end{proposition}
\begin{proof}
  Exercise.
\end{proof}

In fact the proposition above holds for the infinite cyclic group 
\(C_\infty \cong \mathbb{Z}\).

\end{document} 

\end{document}
