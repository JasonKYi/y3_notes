% Options for packages loaded elsewhere
\PassOptionsToPackage{unicode}{hyperref}
\PassOptionsToPackage{hyphens}{url}
\PassOptionsToPackage{dvipsnames,svgnames*,x11names*}{xcolor}
%
\documentclass[]{article}
\usepackage{lmodern}
\usepackage{amssymb,amsmath}
\usepackage{ifxetex,ifluatex}
\ifnum 0\ifxetex 1\fi\ifluatex 1\fi=0 % if pdftex
  \usepackage[T1]{fontenc}
  \usepackage[utf8]{inputenc}
  \usepackage{textcomp} % provide euro and other symbols
\else % if luatex or xetex
  \usepackage{unicode-math}
  \defaultfontfeatures{Scale=MatchLowercase}
  \defaultfontfeatures[\rmfamily]{Ligatures=TeX,Scale=1}
\fi
% Use upquote if available, for straight quotes in verbatim environments
\IfFileExists{upquote.sty}{\usepackage{upquote}}{}
\IfFileExists{microtype.sty}{% use microtype if available
  \usepackage[]{microtype}
  \UseMicrotypeSet[protrusion]{basicmath} % disable protrusion for tt fonts
}{}
\makeatletter
\@ifundefined{KOMAClassName}{% if non-KOMA class
  \IfFileExists{parskip.sty}{%
    \usepackage{parskip}
  }{% else
    \setlength{\parindent}{0pt}
    \setlength{\parskip}{6pt plus 2pt minus 1pt}}
}{% if KOMA class
  \KOMAoptions{parskip=half}}
\makeatother
\usepackage{xcolor}\pagecolor[RGB]{28,30,38} \color[RGB]{213,216,218}
\IfFileExists{xurl.sty}{\usepackage{xurl}}{} % add URL line breaks if available
\IfFileExists{bookmark.sty}{\usepackage{bookmark}}{\usepackage{hyperref}}
\hypersetup{
  pdftitle={Group Representation Theory},
  pdfauthor={Kexing Ying},
  colorlinks=true,
  linkcolor=Maroon,
  filecolor=Maroon,
  citecolor=Blue,
  urlcolor=red,
  pdfcreator={LaTeX via pandoc}}
\urlstyle{same} % disable monospaced font for URLs
\usepackage[margin = 1.5in]{geometry}
\usepackage{graphicx}
\makeatletter
\def\maxwidth{\ifdim\Gin@nat@width>\linewidth\linewidth\else\Gin@nat@width\fi}
\def\maxheight{\ifdim\Gin@nat@height>\textheight\textheight\else\Gin@nat@height\fi}
\makeatother
% Scale images if necessary, so that they will not overflow the page
% margins by default, and it is still possible to overwrite the defaults
% using explicit options in \includegraphics[width, height, ...]{}
\setkeys{Gin}{width=\maxwidth,height=\maxheight,keepaspectratio}
% Set default figure placement to htbp
\makeatletter
\def\fps@figure{htbp}
\makeatother
\setlength{\emergencystretch}{3em} % prevent overfull lines
\providecommand{\tightlist}{%
  \setlength{\itemsep}{0pt}\setlength{\parskip}{0pt}}
\setcounter{secnumdepth}{5}
\usepackage{tikz}
\usepackage{physics}
\usepackage{amsthm}
\usepackage{mathtools}
\usepackage{esint}
\usepackage{pst-node}
\usepackage{auto-pst-pdf}
\usepackage{tikz-cd} 
\usepackage[ruled,vlined]{algorithm2e}
\theoremstyle{definition}
\newtheorem{theorem}{Theorem}
\newtheorem{definition*}{Definition}
\newtheorem{prop}{Proposition}
\newtheorem{corollary}{Corollary}[theorem]
\newtheorem*{remark}{Remark}
\theoremstyle{definition}
\newtheorem{definition}{Definition}[section]
\newtheorem{lemma}{Lemma}[section]
\newtheorem{proposition}{Proposition}[section]
\newtheorem{example}{Example}[section]
\newcommand{\diag}{\mathop{\mathrm{diag}}}
\newcommand{\Arg}{\mathop{\mathrm{Arg}}}
\newcommand{\hess}{\mathop{\mathrm{Hess}}}

\title{Group Representation Theory}
\author{Kexing Ying}
\date{July 24, 2021}

\begin{document}
\maketitle

{
\hypersetup{linkcolor=}
\setcounter{tocdepth}{2}
\tableofcontents
}
\newpage

\section{Introduction}

Group representation theory is a field of mathematics that applies linear algebra 
to study properties of groups. The field itself originated through a letter 
from Dedekind to Frobenius in which he noted that, given \(f = \det A\), where 
\(A\) is the Cayley table of a group of \(n\) elements, by factorising \(f\) 
into irreducible polynomials, \(f = \prod_i f_i^{d_i}\), we have \(d_i = \deg f_i\). 
And this led Frobenius to invent group representation theory.

Group representation theory is applicable in many different areas.
\begin{itemize}
  \item Group theory arises in Klein's "Erlangen program" as symmetries of 
    geometric spaces.
  \item Burnside in 1904 proves the following using representation theory 
    (and so shall we later on)
    \begin{proposition}
      Let \(G\) be a group such that \(|G| = p^r q^s\) where \(p, q\) are 
      prime and \(r + s \ge 2\), then \(G\) is not simple.
    \end{proposition}
  \item In number theory, representations of Galois groups arises in the 
    number field case 
    \[\overline{F} / F, \mathbb{Q} \subseteq F, [F : \mathbb{Q}] < \infty,\]
    which has implications in Wiles' proof of Fermat's last theorem.
  \item In chemistry the symmetry and rotation of molecules can be represented 
    by group actions.
  \item In quantum mechanics, spherical symmetry gives rise to discrete energy 
    levels, orbitals, etc.
  \item In differential geometry, the vector space of solutions is a representation 
    of the symmetry group of an equation.
\end{itemize}

Recalling the definition of a group, informally, the representation of a group \(G\) 
is a way if writing group elements as linear transformations of a vector space 
such that the natural group properties are satisfied. 

Some examples of group representations are the following:
\begin{itemize}
  \item For all group \(G\), the trivial representation of \(G\) is \(\rho\) such 
    that \(\rho(g) = \text{id}\) for all \(g \in G\).
  \item Let \(\zeta \in \mathbb{C}\) be a \(n\)-th root of \(1\) and let 
    \(G = C_n = \{1, g, \cdots, g^{n - 1}\}\). Then \(\rho : g^i \mapsto (\zeta^i)\)
    is a representation of \(G\).
  \item In the case \(G = S_n\), the mapping of \(\sigma \in S_n\) to its 
    corresponding permutation matrix \(P_\sigma\) is a representation of \(G\).
  \item Another representation of \(S_n\) is 
    \(\sigma \in S_n \mapsto (\text{sign}(\sigma))\)\footnote{\(\text{sign}(\sigma) = \det P_\sigma\)}.
  \item Let \(G = D_n\) the dihedral group of order \(2n\). Then, a representation 
    \(D_n\) maps elements of \(D_n\) to the corresponding \(2 \times 2\) matrices 
    which rotates/reflects \(\mathbb{R}^2\) by the appropriate amount.
\end{itemize}

We shall in this module study and construct representations, and furthermore, 
classify up to isomorphism finite-dimensional complex representations of 
every finite group \(G\). 

\section{Fundamentals of Group Representation}

\begin{definition}[Representation]
  Let \(G\) be a group, then a representation of \(G\) is the pair \((V, \rho)\) 
  where \(V\) is a (finite-dimensional) vector space and \(\rho : G \mapsto GL(V)\) 
  is a group homomorphism.
\end{definition}

Alternatively, we may consider a group representation of \(G\) is a group action 
\((\cdot) : G \times V \to V : (g, v) \mapsto v\) such that \((\cdot)\) is 
linear with respect to the second parameter. In particular, we recall a group 
action \((\cdot)\) satisfies \(e \cdot v = v\) and 
\(g \cdot (h \cdot v) = gh \cdot v\). 

\begin{definition}[Dimension of a Representation]
  If \((V, \rho)\) is a representation of \(G\), then the dimension of 
  \((V, \rho)\) is \(\dim(V, \rho) = \dim V\).
\end{definition}

Similar to other objects in mathematics, we introduce a notion of morphisms 
between representations.

\begin{definition}[Homomorphism of Representation]
  Let \(G\) be a group and \((V, \rho_V)\) and \((W, \rho_W)\) be two representations 
  of \(G\). Then a homomorphism of representations is a linear map 
  \(T : V \to W\) such that for all \(g \in G\),
  \[T \circ \rho_V(g) = \rho_W(g) \circ T.\]
  Furthermore, we say \(T\) is an isomorphism is bijective (or equivalently, 
  it has an inverse which is also a homomorphism).
\end{definition}

In particular, one might imagine the homomorphism as a linear map such that 
the following diagram commute.
\[\begin{tikzcd}
  V \arrow{r}{T} \arrow[swap]{d}{\rho_V(g)} & W \arrow{d}{\rho_W(g)} \\
  V \arrow{r}{T}& W
  \end{tikzcd}\]
As with any definitions which work with finite-dimensional vector spaces, 
there are equivalent but ``worse'' (as we will have to choose a basis) corresponding 
definitions in terms of matrices. Nonetheless, these definitions with matrices 
are easier computationally and we shall recall the contrast here.

Clearly, if \(G\) is a group and \((\mathbb{C}^n, \rho)\) is a representation, 
we have \(\rho(e) = I_n\). Furthermore, we have a natural isomorphism between 
\(GL_n(\mathbb{C}) \cong GL(\mathbb{C}^n)\) and more generally 
\(\text{Mat}_{n, m}(\mathbb{C}) \cong \text{Hom}(\mathbb{C}^n, \mathbb{C}^m)\).
Similarly, given a representation \((V, \rho)\), with \(\dim V < \infty\), we may 
choose a basis \(B\) of \(V\) and write the representation as a 
matrix which we denote \(\rho^B(g) = [\rho(g)]_B\). Thus, we may use first year 
linear algebra methods to manipulate representations. 

\begin{definition}
  Given two matrix representations \(\rho, \rho' : G \mapsto GL_n(\mathbb{C})\), 
  we say \(\rho\) and \(\rho'\) are equivalent/isomorphic if there exists 
  \(P \in GL_n(\mathbb{C})\) such that for all \(g \in G\), 
  \(\rho'(g) = P^{-1} \rho(g) P\).
\end{definition}
  
This definition is motivated by the following.

\begin{proposition}
  Given \((V, \rho_V)\) and \((W, \rho_W)\) representations of \(G\),
  we have \(\rho_V \cong \rho_W\) if and only if there exists some 
  \(P \in GL_n(\mathbb{C})\) such that for all \(g \in G\), 
  \(\rho_W^C(g) = P^{-1} \rho_V^BP(g) P\) for some basis \(B, C\) of \(V\) and 
  \(W\) respectively.
\end{proposition}
\begin{proof}
  Exercise.
\end{proof}

\begin{proposition}
  Given a cyclic group \(C_n = \langle g \rangle\) with representations 
  \((V, \rho_V)\) and \((W, \rho_W)\) of equal dimensions, we have 
  \(\rho_V \cong \rho_W\) if and only if \(\rho_V^B(g)\) is conjugate to 
  \(\rho_W^C(g)\) for some basis \(B, C\) of \(V\) and \(W\) respectively.
\end{proposition}
\begin{proof}
  Exercise.
\end{proof}

In fact the proposition above holds for the infinite cyclic group 
\(C_\infty \cong \mathbb{Z}\).

\subsection{Regular Representation}

Let us first recall some definition about group actions though we will omit 
stabilizers, the orbit-stabilizer theorem and transitive actions 
(though it might be helpful to recall them from last year).

\begin{definition}[Group Action]
  Let \(G\) be a group and \(X\) a set, then a group action \((\cdot)\) of 
  \(G\) on \(X\) is a function \(G \times X \to X\) such that for all 
  \(g, h \in G\), \(x \in X\), we have 
  \begin{itemize}
    \item \(g \cdot (h \cdot x) = gh \cdot x\),
    \item \(1 \cdot x = x\).
  \end{itemize}
\end{definition}

Equivalently, a group action can be represented by a group homomorphism between 
\(G\) to \(S_n\) if \(|X| = n < \infty\). We note that there exists an 
bijection between \(\text{Perm}(X)\) (a.k.a \(\text{Aut}(X)\) though we will 
avoid this term in case \(X\) has additional structures) and \(S_n\) with 
depends on a choice of \(X \simeq \{1, \cdots, |X|\}\). 

\begin{definition}[Kernel]
  A kernel of a representation (or group action) is simply the kernel of the 
  corresponding group homomorphism, i.e. if \(\rho\) is a representation 
  (or group action), 
  \[\ker \rho := \{ g \in G \mid \rho(g) = \text{id}\}. \]
  We say a representation (or group action) is faithful if \(\ker \rho = \{e\}\), 
  i.e. \(\rho\) is injective.
\end{definition}

\begin{definition}[Morphism of Group Actions]
  A morphism \(T : X \to Y\) of group actions on \(X\) and \(Y\) is a map 
  such that \(T(g \cdot x) = g \cdot T(x)\) for all \(g \in G\), \(x \in X\).

  This is also called a ``\(G\)-equivariant map'' from \(X\) to \(Y\) and 
  one can see the resemblance of this definition and the definition 
  for homomorphisms between representations.
\end{definition}

For any group \(G\), it acts on itself in three different ways. In particular, 
we have the left regular action \(g \cdot h = gh\), the right regular action 
\(g \cdot h = h g^{-1}\) (where the inverse is required for associativity) 
and the adjoint action \(g \cdot h = g h g^{-1}\). One can see that the left 
and right regular actions are isomorphic via \(T(g) = g^{-1}\). On the other 
hand, they are not isomorphic to the adjoint action (consider 
\(\rho_{\text{ad}}(g)(e) = e\) for all \(g \in G\)).

\begin{proposition}
  Given two actions (or representations) \(\rho, \rho'\) on \(G\), 
  \(g \mapsto \rho(g)\rho'(g)\) is an action (or representation) if and only if 
  \(\rho(g)\rho'(g) = \rho'(g)\rho(g)\), that is \(\rho\) and \(\rho'\) are 
  commuting actions.
\end{proposition}

\begin{definition}
  A subset \(Y \subseteq X\) is said to be stable under an action 
  \((\cdot)\) of \(G\) on \(X\) if \(g \cdot y \in Y\) for all 
  \(y \in Y, g \in G\).
\end{definition}

In the case that \(Y \subseteq X\) is stable, then we may restrict the action 
on \(Y\) to obtain a new action of \(G\) on \(Y\). 

\begin{definition}[Orbit]
  Let \(x \in X\), then \(G \cdot x := \{ g \cdot x \mid g \in G \}\) is called 
  an orbit of \(x\) and we denote this by \(\text{orb}(x)\). 
\end{definition}

It is not difficult to see that orbits are stable and in fact, as an exercise, 
one might show that \(Y \subseteq X\) is stable if and only if it is a union of 
orbits.

In a group \(G\) under the adjoint action, we see that the orbits are the 
conjugacy classes\footnote{What are the orbits of the left action?}. Thus, 
for every conjugacy class, we obtain a action on that class from the adjoint 
action on the whole group. 

\begin{example}
  Let \(G = S_4\) and let \(c = \{(12)(34), (13)(24), (14)(23)\}\). Then 
  as \(c\) is a conjugacy class, we have the adjoint action on \(c\) 
  \[\phi : S_4 \to \text{Perm}(c) \cong S_3.\]
  It is not difficult to show that \(\phi\) is surjective and 
  \(\ker \phi = c \cup \{e\} \cong K_4 \cong C_2 \times C_2\). Thus, 
  by the first isomorphism theorem we have 
  \[S_3 \cong S_4 / K_4.\]
\end{example}

\begin{definition}
  Given a finite set \(X\), let 
  \[\mathbb{C}[X] := \left\{\sum_{x \in X} a_x x \mid a_x 
    \in \mathbb{C}\right\},\]
  equipped with the addition 
  \(\sum_{x \in X} a_x x + \sum_{x \in X} b_x x = 
   \sum_{x \in X} (a_x + b_x) x\) and scalar multiplication 
   \(c \cdot \sum_{x \in X} a_x x = \sum_{x \in X} (c a_x) x\).
  This sum here does not represent some addition operation on \(X\) but a 
  notational trick. One might instead consider elements of \(\mathbb{C}[X]\) 
  as functions \(a : X \to \mathbb{C}\) equipped with point-wise addition and 
  scalar multiplication.
\end{definition}

We observe that \(\mathbb{C}[X] \cong \mathbb{C}^{|X|}\) depending on a 
choice of \(X \cong \{1, \cdots, |X|\}\). Furthermore, we have 
\(X \subseteq \mathbb{C}[X]\) and is a basis (if we interpret \(\mathbb{C}[X]\) 
as a space of functions, the canonical basis is 
\(\{a_x : y \mapsto \chi_{\{x\}} \mid x \in X\}\)). 
In the case that \(X\) is infinite we can still define \(\mathbb{C}[X]\) 
allowing only finite sums.

\begin{proposition}
  If \((\cdot)\) is a group action of \(G\) on \(X\), then,
  the map \((g, \sum a_x x) \mapsto \sum a_x (g \cdot x)\) is a 
  group action of \(G\) on \(\mathbb{C}[X]\).
\end{proposition}

\begin{definition}
  The left regular, right regular, adjoint representations are representations 
  \[\tilde \rho_L, \tilde \rho_R, \tilde \rho_{\text{ad}} :
    G \to GL(\mathbb{C}[G])\] 
  obtained from the left regular, right regular and adjoint actions 
  \[\rho_L, \rho_R, \rho_{\text{ad}} :
    G \to \text{Perm}(G).\] 
\end{definition}

\begin{proposition}
  If \(X\) is any set with a \(G\)-actin, then for all \(g \in G\), 
  \([\rho_{\mathbb{C}[X]}(g)]_B\) is always a permutation matrix.
\end{proposition}
\begin{proof}
  Exercise.
\end{proof}

\begin{definition}
  Given \((V, \rho_V), (W, \rho_W)\) representations of \(G\), we denote 
  \[\text{Hom}_G(V, W) := \{T : V \to W \mid G\text{-linear}\}.\]
\end{definition}

\begin{proposition}
  Let \((V, \rho_V)\) be a representation of \(G\) and \(v \in V\). Then, 
  there exists a unique homomorphism of representations 
  \(\mathbb{C}[G] \to V\) where \(\mathbb{C}[G]\) is equipped with the left 
  regular representation such that \(e_G \mapsto v\) and thus, 
  \[\text{Hom}_G(\mathbb{C}[G], V) \cong (V, \rho_V).\]
\end{proposition}
\begin{proof}
  For all \(g \in G\), \(c = \sum_{h \in G} a_h h\), we have 
  \[\begin{split}
    T(g \cdot c) = \rho_V(g)(Tc) 
    & \iff T\left(\sum a_h gh\right) = 
      \rho_V(g)\left(T\left(\sum a_h h\right)\right)\\
    & \iff \sum a_h T(gh) = \sum a_h \rho_V(g)(Th)\\
    & \iff T(gh) = \rho_V(g)(Th), \ \forall h \in G,
  \end{split}\]
  where the second if and only if follows as both \(T\) and \(\rho_V\) 
  are linear. Then choosing \(h = e_G\), we have \(T(g) = \rho_V(g)(v)\) and 
  thus \(T\) is uniquely determined on \(G\) and hence is unique as \(G\) is 
  a basis of \(\mathbb{C}[G]\).
  
  It remains to show that the map \(T\) defined by \(g \mapsto \rho_V(g)(v)\) is 
  a homomorphism of representations. This is clear since 
  \[\begin{split}
    T(g \cdot c) & = T\left(\sum a_h gh\right) = \sum a_h T(gh) \\
      & = \sum a_h \rho_V(gh)(v) = \sum a_h \rho_V(g)(\rho_V(h)(v))\\
      & = \sum a_h \rho_V(g)(Th) = \rho_V(g)\left(T\left(\sum a_h h\right)\right), 
  \end{split}\]
  where the fourth equality follows by the associativity of group actions.
\end{proof}

\subsection{Subrepresentation and Quotient Representation}

\begin{definition}[Subrepresentation]
  A subrepresentation of a representation \((V, \rho_V)\) is a subspace \(W \le V\) 
  such that \(\rho_V(g)(W) \subseteq W\) for all \(g \in G\).
\end{definition}

Clearly, both \(\{0\}\) and \(V\) are subrepresentations of \((V, \rho_V)\), 
and we say a representation is irreducible if these two subrepresentations are 
the only subrepresentations. We say a representation is reducible if it is 
not irreducible. In general, every 1-dimension representation is irreducible.

\begin{proposition}
  Irreducibility is invariant under isomorphisms.
\end{proposition}
\begin{proof}
  Exercise.
\end{proof}

\begin{proposition}
  Let \(G\) be finite and \((V, \rho_V)\) is an irreducible representation 
  of \(G\). Then \(\dim V < \infty\).
\end{proposition}
\begin{proof}
  Let \(w \in V \setminus \{0\}\) and let 
  \(W := \text{span}(\{\rho_V(g)(w) \mid g \in G\})\) which is a finite dimensional 
  subrepresentation as \(G\) is finite and for all \(h \in G\), 
  \(\rho_V(h)(\rho_V(g)(w)) = \rho_V(hg)(w)\). Thus, if \(\dim V\) is not finite, 
  we have found a proper subrepresentation which contradicts the irreducibility 
  of \((V, \rho_V)\). 
\end{proof}

\begin{definition}[Quotient Representation]
  For \(W \le V\) a subrepresentation, the quotient representation 
  is \((V / W, \rho_{V / W})\) given by 
  \[\rho_{V / W}(g)(v + W) := \rho_V(g)(v) + W.\]
  This is well-defined as \(W\) is stable under \(\rho_V\).
\end{definition}

\begin{proposition}
  For \(T : (V, \rho_V) \to (W, \rho_W)\) a \(G\)-linear map, \(\ker T\) and 
  \(\text{Im} T\) are subrepresentations.
\end{proposition}
\begin{proof}
  Let \(v \in \ker T\), then \(T(\rho_V(g)(v)) = \rho_W(g)(Tv) = 
  \rho_W(g)(0) = 0\) implying \(\rho_V(g)(v) \in \ker T\) and thus, \(\ker T\) 
  is a subrepresentation. On the other hand, for all \(w \in \text{Im}T\), there 
  exists some \(v \in V\) such that \(Tv = w\). Then \(\rho_W(g)(w) = 
  \rho_W(g)(Tv) = T\rho_W(g)(v)\) implying \(\rho_W(g)(w) \in \text{Im}T\) 
  showing \(\text{Im}T\) is also a subrepresentation.
\end{proof}

\begin{proposition}
  For \(T : (V, \rho_V) \to (W, \rho_W)\) a \(G\)-linear map, we have 
  \[\text{Im} T \cong V / \ker T.\]
\end{proposition}
\begin{proof}
  Follows from the first isomorphism for vector spaces and it remains to check 
  \(V / \ker T \to \text{Im} T\) is \(G\)-linear.
\end{proof}

\begin{proposition}
  If \(T \in \text{End}_G V\) is a \(G\)-linear projection (i.e. \(T^2 = T\)), 
  then \(V\) is a direct sum of subrepresentations \(\ker T \oplus \text{Im} T\).
\end{proposition}
\begin{proof}
  Follows from the vector space case.
\end{proof}

\subsection{Maschke's Theorem}

Recalling internal and external direct sums of vector spaces, we will in this 
section introduce and prove a powerful result in representation theory known as 
the Maschke's theorem.

\begin{definition}[Decomposable]
  The representation \((V, \rho_V)\) is decomposable if there exists a 
  decomposition \(V = U \oplus W\) where \(U, W\) are non-zero subrepresentations.
\end{definition}

\begin{definition}[Semisimple]
  The representation \((V, \rho_V)\) is semisimple if there exists a 
  irreducible subrepresentations \(W_1, \cdots, W_n\) such that 
  \[V = \bigoplus_{i = 1}^n W_i.\]
\end{definition}

\begin{theorem}[Maschke's Theorem]
  If \(G\) is finite, then for all \(W \le V\) subrepresentations of 
  \((V, \rho_V)\), there exists a complementary subrepresentation \(U\), 
  \(V = W \oplus U\). 
\end{theorem}

A direct consequence of Maschke's theorem is that every finite-dimensional 
representation of \(G\) is semisimple.

Maschke's theorem does not hold in the case that \(G\) is not finite. 
Consider \(G = \mathbb{Z}\) and let 
\[\rho : g \to GL_2(\mathbb{C}) : m \mapsto \begin{bmatrix}
  1 & m \\ 0 & 1
\end{bmatrix}.\]
Then the only non-zero proper subrepresentation is \(\text{span}\{e_1\}\) since 
\(e_1\) is the only eigenvector and as \(\rho\) is a 2-dimensional representation, 
the only non-zero proper subrepresentation is 1-dimensional, hence an eigenspace.
Thus, \(\rho\) is indecomposable but not irreducible, and hence not semisimple.

\end{document} 

