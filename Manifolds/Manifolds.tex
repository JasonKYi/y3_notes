% Options for packages loaded elsewhere
\PassOptionsToPackage{unicode}{hyperref}
\PassOptionsToPackage{hyphens}{url}
\PassOptionsToPackage{dvipsnames,svgnames*,x11names*}{xcolor}
%
\documentclass[]{article}
\usepackage{lmodern}
\usepackage{amssymb,amsmath}
\usepackage{ifxetex,ifluatex}
\ifnum 0\ifxetex 1\fi\ifluatex 1\fi=0 % if pdftex
  \usepackage[T1]{fontenc}
  \usepackage[utf8]{inputenc}
  \usepackage{textcomp} % provide euro and other symbols
\else % if luatex or xetex
  \usepackage{unicode-math}
  \defaultfontfeatures{Scale=MatchLowercase}
  \defaultfontfeatures[\rmfamily]{Ligatures=TeX,Scale=1}
\fi
% Use upquote if available, for straight quotes in verbatim environments
\IfFileExists{upquote.sty}{\usepackage{upquote}}{}
\IfFileExists{microtype.sty}{% use microtype if available
  \usepackage[]{microtype}
  \UseMicrotypeSet[protrusion]{basicmath} % disable protrusion for tt fonts
}{}
\makeatletter
\@ifundefined{KOMAClassName}{% if non-KOMA class
  \IfFileExists{parskip.sty}{%
    \usepackage{parskip}
  }{% else
    \setlength{\parindent}{0pt}
    \setlength{\parskip}{6pt plus 2pt minus 1pt}}
}{% if KOMA class
  \KOMAoptions{parskip=half}}
\makeatother
\usepackage{xcolor}\pagecolor[RGB]{28,30,38} \color[RGB]{213,216,218}
\IfFileExists{xurl.sty}{\usepackage{xurl}}{} % add URL line breaks if available
\IfFileExists{bookmark.sty}{\usepackage{bookmark}}{\usepackage{hyperref}}
\hypersetup{
  pdftitle={Manifolds},
  pdfauthor={Kexing Ying},
  colorlinks=true,
  linkcolor=Maroon,
  filecolor=Maroon,
  citecolor=Blue,
  urlcolor=red,
  pdfcreator={LaTeX via pandoc}}
\urlstyle{same} % disable monospaced font for URLs
\usepackage[margin = 1.5in]{geometry}
\usepackage{graphicx}
\makeatletter
\def\maxwidth{\ifdim\Gin@nat@width>\linewidth\linewidth\else\Gin@nat@width\fi}
\def\maxheight{\ifdim\Gin@nat@height>\textheight\textheight\else\Gin@nat@height\fi}
\makeatother
% Scale images if necessary, so that they will not overflow the page
% margins by default, and it is still possible to overwrite the defaults
% using explicit options in \includegraphics[width, height, ...]{}
\setkeys{Gin}{width=\maxwidth,height=\maxheight,keepaspectratio}
% Set default figure placement to htbp
\makeatletter
\def\fps@figure{htbp}
\makeatother
\setlength{\emergencystretch}{3em} % prevent overfull lines
\providecommand{\tightlist}{%
  \setlength{\itemsep}{0pt}\setlength{\parskip}{0pt}}
\setcounter{secnumdepth}{5}
\usepackage{tikz}
\usepackage{physics}
\usepackage{amsthm}
\usepackage{mathtools}
\usepackage[ruled,vlined]{algorithm2e}
\theoremstyle{definition}
\newtheorem{theorem}{Theorem}
\newtheorem{definition*}{Definition}
\newtheorem{prop}{Proposition}
\newtheorem{corollary}{Corollary}[theorem]
\newtheorem*{remark}{Remark}
\theoremstyle{definition}
\newtheorem{definition}{Definition}[section]
\newtheorem{lemma}{Lemma}[section]
\newtheorem{proposition}{Proposition}[section]
\newtheorem{example}{Example}[section]
\newcommand{\diag}{\mathop{\mathrm{diag}}}
\newcommand{\Arg}{\mathop{\mathrm{Arg}}}
\newcommand{\hess}{\mathop{\mathrm{Hess}}}

\title{Manifolds}
\author{Kexing Ying}
\date{July 24, 2021}

\begin{document}
\maketitle

{
\hypersetup{linkcolor=}
\setcounter{tocdepth}{2}
\tableofcontents
}
\newpage

\section{Introduction}

This module introduces the notion of manifolds and provides the infrastructure 
for generalizing theorems from calculus to manifolds. In particular, we will 
talk about 
\begin{itemize}
  \item Smooth manifolds and smooth functions;
  \item Tangent spaces and vector fields;
  \item Differential forms, integrations and Stoke's theorem.
\end{itemize}
In contrast to the curves and spaces module, instead of working on Euclidean spaces, 
we will define these notions for general manifolds. Thus, many definitions 
such as the tangent space will be defined in a more intrinsic point of view, without 
requiring our manifold to be within a Euclidean space.

Furthermore, a goal of this module is to differentiate between different manifolds, 
that is determine whether or not two manifolds are diffeomorphic with one another. 
This is achieved through introducing invariants such as the notion of differential 
forms and these notions will appear in many other places especially in geometry.

Manifolds is the subject of studying geometric shapes, and in mathematics, there 
are in general two ways of doing this. The first of which is by embedding the 
object into an ambient space such as \(\mathbb{R}^2\) or \(\mathbb{R}^3\). An 
example of this is studying the unit circle through the parametrisation 
\[\{(x, y) \mid x^2 + y^2 = 1\} \subseteq \mathbb{R}^2,\]
and is the more common method of what we have done thus far. On the other hand, 
one may study the object independently of the ambient space. This is the approach 
we shall take throughout this course. In particular, we will study spaces 
which at a local level ``looks like'' a Euclidean space directly without embedding 
the structure into \(\mathbb{R}^n\).

\newpage
\section{Topological and Smooth Manifolds}

Let us first recall some notions from topology.

\begin{definition}
  Let \(X, Y\) be topological spaces and let \(f : X \to Y\) be a function,
  then 
  \begin{itemize}
    \item \(f\) is continuous if \(f^{-1}(U)\) is open in \(X\) for all \(U\) open 
      in \(Y\).
    \item \(f\) is a homeomorphism if it is continuous and has a continuous inverse.
  \end{itemize}
\end{definition}

\begin{definition}
  A topological space \(X\) is 
  \begin{itemize}
    \item Hausdorff if for all \(x, y \in X\), \(x \neq y\), there exists open 
      sets \(U, V\) in \(X\) such that \(x \in U, y \in V\) and 
      \(U \cap V = \varnothing\).
    \item second-countable if there exists countable 
      \(\mathcal{F} \subseteq \mathcal{T}_X\) such that any open set in \(X\) 
      can be written as a union of elements of \(\mathcal{F}\), i.e. 
      \(\mathcal{F}\) is a countable basis of \(X\).
  \end{itemize}
\end{definition}

In general, in this module, we will assume our topology is Hausdorff and 
second-countable in order to avoid pathological examples in smooth and 
topological manifolds. 

\begin{definition}[Co-ordinate Chart]
  Let \(X\) be a topological space. A co-ordinate chart on \(X\) is 
  the collection of
  \begin{itemize}
    \item an open set \(U \subseteq X\),
    \item an open set \(\tilde U \subseteq \mathbb{R}^n\) for some \(n \ge 0\), 
    \item a homeomorphism \(f : U \to \tilde U\).
  \end{itemize}
  We denote a co-ordinate chart by \((U, f)\).
\end{definition}

\begin{definition}
  Let \(X\) be a (Hausdorff and second-countable) topological space. We say 
  that \(X\) is a topological manifold of dimension \(n\) if for all 
  \(x \in X\), there exists a co-ordinate chart \((U, f)\) with 
  \(\tilde U \subseteq \mathbb{R}^n\) such that \(x \in U\).
\end{definition}

The classical example of a topological manifold is the circle, in particular 
\[S^1 := \{(x, y) \mid x^2 + y^2 = 1\} \subseteq \mathbb{R}^2,\]
is a 1 dimensional topological manifold. Consider \(U_1 = S^1 \setminus \{(0, -1)\}\),
and we define the stereographic projection \(f_1 : U_1 \to \mathbb{R}\), 
\[f : (x, y) \mapsto \frac{x}{y + 1} := \tilde x.\]
It is not difficult to see that \(f_1\) is invertible with the inverse 
\[f_1^{-1} : \tilde x \mapsto \left(\frac{2\tilde x}{1 + {\tilde x}^2}, 
  \frac{1 - {\tilde x}^2}{1 + {\tilde x}^2}\right).\]
Furthermore, as \(f_1\) and \(f^{-1}\) are continuous, we have \((U_1, f_1)\) is 
a co-ordinate chart. Similarly, we define \(U_2 = S^1 \setminus \{(0, 1)\}\), 
and we my show the existence of a homeomorphism \(f_2 : U_2 \to \mathbb{R}\), 
providing the second co-ordinate chart \((U_2, f_2)\). Thus, as 
\(S^1 = U_1 \cup U_2\), we have \(S^1\) is a 1 dimensional topological manifold.

The above example can be expanded to \(n\)-dimensional sphere 
\[S^n := \{(x_0, \cdots, x_n) \mid x_0^2 + \cdots + x_n^2 = 1\} 
  \subseteq \mathbb{R}^n.\]
Similarly as before, we can construct two co-ordinate charts covering all points 
on the sphere except for the poles allowing us to conclude \(S^n\) is a 
\(n\)-dimensional topological manifold.

\begin{definition}[Transition Function]
  Let \(X\) be a topological manifold and let \((U_1, f_1)\) and \((U_2, f_2)\) 
  be two co-ordinate charts on \(X\) such that \(U_1 \cap U_2 \neq \varnothing\). 
  Then the transition function between these two co-ordinate charts is the 
  function 
  \[\phi_{21} := f_2 \circ f_1^{-1} : f_1(U_1 \cap U_2) \to f_2(U_1 \cap U_2).\]
\end{definition}

Let \(X\) be a topological manifold with co-ordinate charts \((U_i, f_i)\) for 
\(i = 1, 2, 3\) such that \(U_1 \cap U_2 \cap U_3 \neq \varnothing\). Then it 
is clear that \(\phi_{21} := f_2 \circ f_1^{-1}\) is a homeomorphism with 
the inverse \(\phi_{12} := f_1 \circ f_2^{-1}\). Furthermore, by considering 
\(\phi_{31} := f_3 \circ f_1^{-1}\) we observe 
\[\phi_{31} = (f_3 \circ f_2^{-1}) \circ (f_2 \circ f_1^{-1}) = 
  \phi_{32} \circ \phi_{21}.\]
This is known as the cocycle property and explains the subscript notation.

\begin{definition}[Atlas]
  Let \(X\) be a topological manifold. An atlas for \(X\) is the collection of
  co-ordinate charst \(\{(U_i, f_i)\}_{i \in I}\) such that 
  \[\bigcup_{i \in I} U_i = X.\]
\end{definition}

We note that we do not require the index set \(I\) to be finite. 
Although, since \(\{U_i\}_{i \in I}\) is an open cover, if \(X\) is compact, 
it is possible to obtain a finite sub-cover, and hence a finite atlas. Nonetheless, 
since we assumed \(X\) is second-countable, we can always choose \(I\) to be 
countable.

So far, we have only considered ourselves with the topological structure. As 
we would like to do calculus on our manifolds, we will now equip our manifolds 
with the property of smoothness. Recall the following definition for Euclidean 
spaces. 

\begin{definition}
  A function \(F : \mathbb{R}^n \to \mathbb{R}^n\) is smooth (or \(C^\infty\)) 
  if all the partial derivatives of \(F\) of any order exists.
\end{definition}

Of course, this is technically a property not a definition though it will suffice 
for our purposes.

\begin{definition}[Smooth Atlas]
  Let \(X\) be a topological manifold of dimension \(n\). Then an atlas 
  \(\{(U_i, f_i)\}_{i \in I}\) on \(X\) is smooth if for all \(i, j \in I\),
  the transition function 
  \[\phi_{ij} : f_j(U_i \cap U_j) \subseteq \mathbb{R}^n \to 
    f_i(U_i \cap U_j) \subseteq \mathbb{R}^n\]
  is smooth.
\end{definition}

Since \(\phi_{ij}\) is a (bijective) map between open subsets of Euclidean 
spaces, it makes sense to ask whether or not \(\phi_{ij}\) is smooth.

\begin{definition}[Diffeomorphism]
  Let \(U, V \subseteq \mathbb{R}^n\) be open sets and let \(f : U \to V\). 
  Then \(f\) is a diffeomorphism if \(f\) is smooth and has a smooth inverse.
\end{definition}

As \((\phi_{ij})^{-1} = \phi_{ji}\), and both \(\phi_{ij}\) and \(\phi_{ji}\) 
are smooth, the transition functions of any smooth manifold are diffeomorphisms.

\begin{definition}[Compatible]
  Let \(X\) be a topological manifold and let \(\mathcal{A} := \{(U_i, f_i)\}\) 
  be a smooth atlas. Let \((U, f)\) be any co-ordinate chart on \(X\), then 
  \((U, f)\) is compatible with the atlas \(\mathcal{A}\) if the transition 
  function between \((U, f)\) and any chart in \(\mathcal{A}\) is a 
  diffeomorphism.
\end{definition}

Clearly, any chart in a smooth atlas is compatible with that atlas, and if 
\((U, f)\) is compatible with the smooth atlas \(\mathcal{A}\), then 
\((U, f) \cup \mathcal{A}\) is also a smooth atlas.

\begin{definition}
  Let \(X\) be a topological manifold and \(\mathcal{A}, \mathcal{B}\) be 
  two atlases on \(X\). Then \(\mathcal{A}\) is compatible with \(\mathcal{B}\) 
  if every chart in \(\mathcal{B}\) is compatible with \(\mathcal{A}\).
\end{definition}

Similarly as before, if \(\mathcal{A}, \mathcal{B}\) are compatible, then 
\(\mathcal{A} \cup \mathcal{B}\) is a smooth atlas on \(X\).

\begin{lemma}
  Let \(X\) be a topological manifold and let 
  \[\mathcal{A} := \{(U_i, f_i)\}_{i \in I}, 
    \mathcal{B} := \{(U_j, f_j)\}_{j \in J},\]
  be two compatible smooth atlases on \(X\). Then for all \((U, f)\) co-ordinate 
  charts compatible with \(\mathcal{A}\), \((U, f)\) is compatible with 
  \(\mathcal{B}\).
\end{lemma}
\begin{proof}
  It suffices to show that for all \((U_j, f_j) \in \mathcal{B}\), 
  \(U \cap U_j \neq \varnothing\), the transition map 
  \[\phi := f_j \circ f^{-1} : f(U \cap U_j) \to f_j(U \cap U_j)\]
  and its inverse are smooth. 

  Let \(y \in f(U \cap U_j)\), then there exist some \(x \in U \cap U_j\) such 
  that \(f(x) = y\). As \(\mathcal{A}\) is an atlas, it contains a co-ordinate 
  chart \((U_i, f_i) \in \mathcal{A}\) such that \(x \in U_i\). Then, defining 
  \(W := U \cap U_i \cap U_j \neq \varnothing\), we have the homomorphisms 
  \(f : W \to f(W), f_i : W \to f_i(W)\) and \(f_j : W \to f_j(W)\). As 
  remarked before, we have 
  \[\phi = (f^{-1} \circ f_i) \circ (f_i^{-1} \circ f_j)\]
  on \(W\). Now, by compatibility, the right hand side is smooth, and so 
  we have \(\phi\) is smooth on \(W\) implying it is smooth at \(y\). Thus, 
  as \(y \in f(U \cap U_j)\) was arbitrary, \(\phi\) is smooth (by a similar 
  argument \(\phi^{-1}\) is also smooth) and \((U, f)\) is compatible with 
  \(\mathcal{B}\).
\end{proof}

With this lemma it is easy to see that compatibility defines an equivalence 
relation on the set of smooth atlases and with this we can define smooth 
manifolds.

\begin{definition}[Smooth Manifold]
  A smooth manifold is a topological manifold with an equivalence class 
  \([\mathcal{A}]\) of compatible smooth atlases on \(X\). The equivalence 
  class of atlases is called a smooth structure on \(X\).
\end{definition}

The reason for the definition considering only the equivalence class 
of compatible smooth atlases is because we do not want to distinguish between 
compatible smooth atlases. Indeed, recalling our example of a sphere, 
we would like to not consider the atlases which projects the sphere with 
respect to two other points that are not the poles as an alternative manifold.

From this point forward, we will always work with smooth manifolds and thus, 
omit the word ``smooth'' whenever it is clear from the context, i.e. a manifold 
is a smooth manifold and a atlas is a smooth atlas. 

\end{document}
