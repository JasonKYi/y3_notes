% Options for packages loaded elsewhere
\PassOptionsToPackage{unicode}{hyperref}
\PassOptionsToPackage{hyphens}{url}
\PassOptionsToPackage{dvipsnames,svgnames*,x11names*}{xcolor}
%
\documentclass[]{article}
\usepackage{lmodern}
\usepackage{amssymb,amsmath}
\usepackage{ifxetex,ifluatex}
\ifnum 0\ifxetex 1\fi\ifluatex 1\fi=0 % if pdftex
  \usepackage[T1]{fontenc}
  \usepackage[utf8]{inputenc}
  \usepackage{textcomp} % provide euro and other symbols
\else % if luatex or xetex
  \usepackage{unicode-math}
  \defaultfontfeatures{Scale=MatchLowercase}
  \defaultfontfeatures[\rmfamily]{Ligatures=TeX,Scale=1}
\fi
% Use upquote if available, for straight quotes in verbatim environments
\IfFileExists{upquote.sty}{\usepackage{upquote}}{}
\IfFileExists{microtype.sty}{% use microtype if available
  \usepackage[]{microtype}
  \UseMicrotypeSet[protrusion]{basicmath} % disable protrusion for tt fonts
}{}
\makeatletter
\@ifundefined{KOMAClassName}{% if non-KOMA class
  \IfFileExists{parskip.sty}{%
    \usepackage{parskip}
  }{% else
    \setlength{\parindent}{0pt}
    \setlength{\parskip}{6pt plus 2pt minus 1pt}}
}{% if KOMA class
  \KOMAoptions{parskip=half}}
\makeatother
\usepackage{xcolor}\pagecolor[RGB]{28,30,38} \color[RGB]{213,216,218}
\IfFileExists{xurl.sty}{\usepackage{xurl}}{} % add URL line breaks if available
\IfFileExists{bookmark.sty}{\usepackage{bookmark}}{\usepackage{hyperref}}
\hypersetup{
  pdftitle={Quantum Mechanics I},
  pdfauthor={Kexing Ying},
  colorlinks=true,
  linkcolor=Maroon,
  filecolor=Maroon,
  citecolor=Blue,
  urlcolor=red,
  pdfcreator={LaTeX via pandoc}}
\urlstyle{same} % disable monospaced font for URLs
\usepackage[margin = 1.5in]{geometry}
\usepackage{graphicx}
\makeatletter
\def\maxwidth{\ifdim\Gin@nat@width>\linewidth\linewidth\else\Gin@nat@width\fi}
\def\maxheight{\ifdim\Gin@nat@height>\textheight\textheight\else\Gin@nat@height\fi}
\makeatother
% Scale images if necessary, so that they will not overflow the page
% margins by default, and it is still possible to overwrite the defaults
% using explicit options in \includegraphics[width, height, ...]{}
\setkeys{Gin}{width=\maxwidth,height=\maxheight,keepaspectratio}
% Set default figure placement to htbp
\makeatletter
\def\fps@figure{htbp}
\makeatother
\setlength{\emergencystretch}{3em} % prevent overfull lines
\providecommand{\tightlist}{%
  \setlength{\itemsep}{0pt}\setlength{\parskip}{0pt}}
\setcounter{secnumdepth}{5}
\usepackage{tikz}
\usepackage{physics}
\usepackage{amsthm}
\usepackage{mathtools}
\usepackage{esint}
\usepackage[ruled,vlined]{algorithm2e}
\theoremstyle{definition}
\newtheorem{theorem}{Theorem}
\newtheorem{definition*}{Definition}
\newtheorem{prop}{Proposition}
\newtheorem{corollary}{Corollary}[theorem]
\newtheorem*{remark}{Remark}
\theoremstyle{definition}
\newtheorem{definition}{Definition}[section]
\newtheorem{lemma}{Lemma}[section]
\newtheorem{proposition}{Proposition}[section]
\newtheorem{example}{Example}[section]
\newcommand{\diag}{\mathop{\mathrm{diag}}}
\newcommand{\Arg}{\mathop{\mathrm{Arg}}}
\newcommand{\hess}{\mathop{\mathrm{Hess}}}
% the redefinition for the missing \setminus must be delayed
\AtBeginDocument{\renewcommand{\setminus}{\mathbin{\backslash}}}

\title{Quantum Mechanics I}
\author{Kexing Ying}
\date{July 24, 2021}

\begin{document}
\maketitle

{
\hypersetup{linkcolor=}
\setcounter{tocdepth}{2}
\tableofcontents
}
\newpage

\section{Classical Mechanics}

In order to later compare quantum mechanics, let us first introduce some 
classical mechanics. 

In classical mechanics, we study classical objects/particles 
which has a mass \(m \in \mathbb{R}\) and a state. In particular, the state 
of the particle is represented by its position, commonly \(r \in \mathbb{R}^3\), 
and its velocity \(v = \dot r \in \mathbb{R}^3\). More conveniently, we can also 
represent the velocity in terms of its momentum \(p = mv\).

We recall Newton's second law which describes how the state of a particle changes 
in time in the presence of external forces. That is, 
\[\dot p = F(r),\]
where \(F\) is the external force depending on \(r\).

As the state of a particle is represented by its position and momentum, visually 
the state of a particle can be represented by a phase-space with a trajectory 
corresponding to \((r(t), p(t))\). 

An another formulation of classical mechanics is Hamilton's formulation. While 
Hamilton's formulation is very powerful, it does not apply to every classical system. 
In particular, Hamilton's formulation requires the system to be conservative. 

\begin{definition}[Conservative]
  A classical system is said to be conservative if 
  \[F(r) = - \nabla V(r),\]
  where \(V\) is the potential given the position.
\end{definition}

\begin{definition}[Hamiltonian Function]
  The Hamiltonian function \(H\) is defined as 
  \[H(p, q) = \frac{p^2}{2m} + V(q),\]
  where \(p^2/2m\) is the kinetic energy and \(V\) the potential.
\end{definition}

Thus, with the definition of conservative in mind, we see that for a one 
dimensional system with position given by \(q \in \mathbb{R}\), we have
\[\dot p = F(q) = - \pdv{V}{q} \text{ and } \dot q = \frac{p}{m}.\]
Writing in terms of the Hamiltonian function, we obtain, 
\[\dot p = - \pdv{H}{q} \text{ and } \dot q = \pdv{H}{p}.\]
These two equations are known as Hamilton's canonical equations and describe 
the motion of a particle in a conservative system. The theory itself is more 
general in which we simply require \(p, q\) to be canonically conjugate 
variables.

\begin{example}[Free Particle]
  Consider a free particle with \(V(q) = 0\) (thus, \(H = p^2 / 2m\)), we have 
  the canonical equations \(\dot p = 0\) and \(\dot q = p / m\), and thus, 
  \(p(t) = p(0)\) and \(q(t) = q(0) + \frac{p}{m} t\).
\end{example}

\begin{example}[Harmonic Oscillator]
  A harmonic oscillator is described by \(V(q) \propto q^2\). By similar calculation 
  we find \(\ddot q = - \frac{2k}{m} q\) for some \(k\) such that \(V = k q^2\).
\end{example}

As for a particle in classical mechanics, the state is given by its position and 
momentum, any measurable quantity \(A\) is given as a function \(A(p, q)\) 
such that 
\[\dv{A}{t} = \pdv{A}{p}\dot p + \pdv{A}{q} \dot q + \pdv{A}{t}.\]
Substituting the Hamiltonian equations, we have 
\[\dv{A}{t} = - \pdv{A}{p} \pdv{H}{q} + \pdv{A}{q} \pdv{H}{p} + \pdv{A}{t}
  = \pdv{A}{q} \pdv{H}{p} - \pdv{A}{p} \pdv{H}{q} + \pdv{A}{t}.\]
As the first term of this equation is very common, we denote it as \(\{H, A\}\) 
such that 
\[\dv{A}{t} = \{H, A\} + \pdv{A}{t}.\] 
Similarly, for general variables \(F, G\), 
\[\{F, G\} := \sum_{n = 1}^N \pdv{F}{p_n}\pdv{G}{q_n} - \pdv{F}{q_n}\pdv{G}{q_n},\]
and is known as the Poisson bracket of \(F\) and \(G\).

\begin{definition}[Poisson Bracket]
  A Poisson bracket is simply any bracket of functions satisfying 
  \begin{itemize}
    \item \(\{A, A\} = 0\);
    \item \(\{c_1 A + c_2 B, C\} = c_1\{A, C\} + c_2\{B, C\}\);
    \item \(\{A, B\} = -\{B, A\}\).
    \item \(\{c, A\} = 0\) for any constant \(c\);
    \item \(\{AB, C\} = A\{B, C\} + \{A, C\}B\) (Leibniz rule);
    \item \(\{A, \{B, C\}\} + \{B, \{C, A\}\} + \{C, \{A, B\}\} = 0\) (Jacobi identity).
  \end{itemize}
\end{definition}
As an exercise, one may check that the Poisson bracket defined above is indeed 
a Poisson bracket.
\begin{proposition}
  \(\{p, q\} = 1\) and in higher dimensions. \(\{p_i, q_j\} = \delta_{ij}\).
\end{proposition}

\begin{definition}[Canonical Conjugate Variables]
  \(P(p, q), Q(p, q)\) are called canonical conjugate variables if \(\{P, Q\} = 1\). 
  Similarly, for higher dimensions, \(P, Q\) are canonical conjugates if 
  \(\{P_i, Q_j\} = \delta_{ij}\).
\end{definition}

\begin{proposition}
  For any pair of canonical conjugate variables \(P, Q\), we have 
  \[\dot P_j = - \pdv{H}{Q_j} = \{H, P_j\} \text{ and } 
    \dot Q_j = \pdv{H}{P_j} = \{H, Q_j\}.\]
\end{proposition}

\newpage
\section{Schödinger Dynamics}

The Scrhödinger equations is a function of position and time which is written 
in its compact form as 
\[i \hbar \dot \psi = \hat H \psi\]
where \(\hat H\) is known as the Hamiltonian and \(\hbar = h / 2\pi\) where 
\(h\) is Plank's constant. 

Similar to the Hamiltonian in classical mechanics, 
the Hamiltonian is an linear operator that often encodes energy and in that case, 
it is written as 
\[\hat H = - \frac{\hbar^2}{2m} \grad^2 + V(r),\]
where \(\grad^2\) is the Laplacian operator. Furthermore, the expectation 
of the Hamiltonian, defined by 
\[\langle \hat H \rangle := \frac{1}{\int_{\mathbb{R}^n} |\psi|^2 \dd x^n} 
  \int_{\mathbb{R}^n}\psi^* \hat H \psi \dd x^n\]
provides the total energy of the system.
As \(\hat H\) is a linear operator, it has eigenfunctions where \(\phi_E\) is 
said to be an eigenfunction if \(\hat H \phi_E = E \phi_E\) and this equation 
is known as the time-independent Schrödinger equation. It is possible to show that 
this eigenfunctions are orthogonal with respect to the \(L^2\) inner product 
and form an eigenbasis of all possible states of a system.

As the Schrödinger equation is a linear differential equation, the solution space 
of the equation is a linear space. In particular, if \(\psi_i\) are solutions 
to the Schrödinger equations, so is \(c_1 \psi_1 + c_2 \psi_2\) for \(c_1, c_2\) 
constants.

The function \(\psi\) is known as the wave function and it is interpreted as 
the probability of finding the particle it describes at a given time in a certain 
region. As \(\psi\) is a complex functions, it values is known as a probability 
amplitude by while \(|\psi|^2\) is the probability distribution function. Thus, 
the probability of finding a particle is the interval \([a, b]\) is 
\[\int_a^b |\psi(x, t)|^2 \dd x,\]
if \(\psi\) is normalized, i.e. \(\int_{-\infty}^\infty |\psi(x, t)| \dd x = 1\) 
for all \(t\). In particular, we note that this interpretation is meaningful 
only if \(\psi\) is square integrable, i.e. \(\psi \in L^2\).

Let us consider the following 1-dimensional example. Let 
\(\hat H\) be the Hamiltonian as described above, then we have the Schödinger 
equation
\[i\hbar \dot \psi(x, t) = - \frac{\hbar^2}{2m} \pdv[2]{x} \psi(x, t) +
   V(x) \psi(x, t).\]
Then, defining \(N(t) := \int_{-\infty}^\infty |\psi(x, t)|^2 \dd x\), we have 
\[\begin{split}
  \dv{N}{t} & = \dv{t}\int_{-\infty}^\infty \psi^*(x, t) \psi(x, t) \dd x\\
  & = \int_{-\infty}^\infty \dot \psi^* \psi + \psi^* \dot \psi \dd x\\
\end{split}\]
substituting \(\dot \psi\) and \(\dot \psi^*\) using the Schrödinger equation, 
we have 
\[\begin{split}
  \dv{N}{t} & = \frac{i}{\hbar}\int_{-\infty}^\infty 
    \left(-\frac{\hbar^2}{2m}\pdv[2]{x} \psi^* + V \psi^*\right) \psi 
    - \psi^*\left(-\frac{\hbar^2}{2m}\pdv[2]{x}\psi + V \psi\right)\dd x\\
  & = \frac{i\hbar}{2m} \int_{-\infty}^\infty \psi \pdv[2]{x} \psi^* - 
    \psi^* \pdv[2]{x} \psi \dd x\\
  & = \frac{i\hbar}{2m} \int_{-\infty}^\infty \pdv{x}\left(
    \psi \pdv{x} \psi^* - \psi^* \pdv{x} \psi\right) \dd x\\
  & = 0
\end{split}\]
where the last equality follows by the fundamental theorem of calculus and 
the fact that \(\psi \in L^2\). Thus, \(N\) is conserved over time and we may 
normalize \(\psi\) by simply divide by \(N\).

\begin{definition}[Probability Flux]
  The probability flux of a wave function \(\psi\) is defined as 
  \[j(x, t) := \frac{i\hbar}{2m} \left(\psi^* \pdv{x} \psi -
    \psi \pdv{x} \psi^*\right).\]
\end{definition}
In particular, we see that \(\pdv{|\psi|^2}{t} = -\pdv{j}{x}\), and so, we have 
the continuity equation
\[\pdv{|\psi|^2}{t} + \pdv{j}{x} = 0.\]
Unlike the total probability \(N\), the probability in a certain region does 
fluctuate over time and we see that, if \(P_{[a, b]}(t)\) is the probability that 
a particle is in the region \([a, b]\) at time \(t\), then
\[\dv{P_{[a, b]}}{t} = \dv{t}\int_a^b |\psi|^2 \dd x = 
  -\int_a^b \pdv{j}{x} \dd x = j(a, t) - j(b, t).\]
Similarly, for higher dimensions, we define the probability flux of \(\psi\) as 
\[j(r, t) = \frac{\hbar}{2mi}(\psi^* \grad \psi - \psi \grad \psi^*),\]
and we have the continuity equation 
\[\pdv{|\psi|^2}{t} + \grad \cdot j = 0.\]
Then, if \(P_V\) is the probability of finding a particle in the region \(V\), 
we have 
\[\dv{P_V}{t} = - \int_V \grad \cdot j \dd V = - \int_S j \cdot \dd S\]
by the divergence theorem. Thus, the probability change is the total flux 
through the boundary of the volume.

\subsection{Stationary Solution}

Let us consider a special family of solutions. Consider the Schrödinger equation 
\[i\hbar \dot \psi = \hat H \psi,\]
where \(\psi\) is expressible in the form \(\psi(r, t) = \phi(r)\chi(t)\). Then,
we have 
\[i\hbar \dot \chi(t) \phi(r) = 
  i\hbar \dot \psi = 
  \hat H \phi(r) \chi(t) = \chi(t) \hat H \phi(r),\]
and so,
\[i\hbar \frac{\dot \chi(t)}{\chi(t)} = \frac{\hat H \phi(r)}{\phi(r)}.\]
By observing that the right hand side and the left hand sides of the equation 
depend on different variables, we conclude that both values must be constants 
and we denote this constant by \(E\) such that 
\[\dot \chi(t) = - \frac{i}{\hbar}E \chi(t), \text{ and } 
  \hat H \phi(r) = E \phi(r).\]
Thus, solving the first differential equation, we have 
\[\chi(t) = e^{-iEt / \hbar}\chi(0),\]
while the second is the time-independent Schrödinger equation.
The time-independent Schrödinger equation is not always easy to solve though 
it is solvable analytically in special cases. In particular, for Hamiltonians 
in of form \(- \frac{\hbar^2}{2m} \pdv[2]{x} + V(x)\) where the potient \(V\) 
tends to \(\infty\) as \(x \to \infty\), normalizable solutions to the 
time-independent Schrödinger equation only exist for special discrete 
values of \(E\).

In this case, the wave function is simply 
\[\psi(r, t) = \chi_0 e^{-iEt/ \hbar} \phi_E(r),\]
and if \(\psi(r, 0) = \phi(r)\) the probability distribution of the particle is
\(|\psi(r, t)|^2 = |\phi_E(r)|^2\) which is independent of time, and thus 
we call solutions of this form stationary states. Note that the superposition 
of stationary states is not necessarily stationary (exercise).

\newpage
\section{Principles}

\subsection{Mathematics Review}

Let us recall some definitions which is required for quantum mechanics. 

\begin{definition}[Inner Product Space]
  A inner product space is a vector space \(V\) equipped with a map 
  \(\langle \cdot, \cdot \rangle : V \times V \to \mathbb{C}\) such that 
  for all \(u, v, w \in V, \lambda, \mu \in \mathbb{C}\),
  \begin{itemize}
    \item \(\langle u, v \rangle = \langle v, u \rangle^*\);
    \item \(\langle u, u \rangle \ge 0\) with equality if and only if \(u = 0\);
    \item \(\langle u, \lambda v + \mu w \rangle = 
      \lambda \langle u, v \rangle + \mu \langle u, w\rangle\).
  \end{itemize}
\end{definition}

Recall that inner product induces a norm by defining 
\(\|\cdot\| : V \to \mathbb{R} : v \mapsto \sqrt{\langle v, v \rangle}\).

\begin{definition}[Complete]
  A topological space is complete if every Cauchy sequence converge in that space.
\end{definition}

\begin{definition}[Separable]
  A topological space is separable if there exists a countable dense subset.
\end{definition}

\begin{definition}[Hilbert Space]
  A Hilbert space is a inner product space such that the induced topology 
  is complete and is separable\footnote{We note that the condition for 
  separability is normally omitted for in mathematical contexts.}.
\end{definition}

In finite dimensional spaces, an inner product space is automatically complete 
and separable. In particular, a finite dimensional inner product space over 
\(\mathbb{R}\) or \(\mathbb{C}\) is complete and separable since 
\(\mathbb{R}\) and \(\mathbb{C}\) are complete and separable.

\begin{definition}[Linear Operator]
  A linear operator is simply a function belonging to \(\text{End}(V)\).
\end{definition}

As linear operators are simply linear maps, definitions such as eigenvalues and 
eigenvectors are defined analogously.

\begin{definition}[Adjoint Operator]
  Given a linear operator \(A : V \to V\), the adjoint of \(A\) is a linear 
  operator \(A^\dagger\) such that 
  \[\langle A u, v \rangle = \langle u, A^\dagger v \rangle\] 
  for all \(u, v \in V\).
\end{definition}

\begin{definition}[Self-Adjoint Operator]
  A linear operator \(A\) is self-adjoint if \(A = A^\dagger\) and an operator 
  satisfying this property is refereed to as a Hermitian operator.
\end{definition}

Recalling the spectral theorem from second year, we have that the eigenvalues 
of a self-adjoint operator are real and furthermore, its eigenvectors form a 
basis and are orthogonal. One may also show that for \(A\) is Hermitian if and 
only if the expectation values \(\langle v, Av\rangle\) are real for all 
\(v \in V\).

\begin{definition}[Dual Space]
  The dual space of a vector space \(V\) is the vector space of linear 
  functionals of \(V\).
\end{definition}

Recall that the dual space of a vector space is (non-canonically) isomorphic 
to that space, and this is summarized by the Riesz representation theorem.

\begin{theorem}[Riesz Representation Theorem]
  Every Hilbert space \(\mathcal{H}\) is (anti-)isomorphic to its dual space. 
  In particular, there exists a bijective map between the linear functionals 
  \(F \in \mathcal{H}^*\) and vectors \(f \in \mathcal{H}\) such that 
  \(F(\phi) = \langle f, \phi \rangle\) for all \(\phi \in \mathcal{H}\).
\end{theorem}
\begin{proof}
  See last year for the finite dimensional case while the general case can 
  be found in the functional analysis course.
\end{proof}

\subsection{Dirac Notation}

Given a vector \(\phi\) in the Hilbert space \(\mathcal{H}\), we denote 
\(\phi\) by \(\ket{\phi}\) and we call it a ``ket''-vector. On the other hand, 
given \(F = \phi \mapsto \langle f, \phi\rangle \in \mathcal{H}^*\) we 
denote \(F\) by \(\bra{f}\) and we call it a ``bra''-vector. With this, we denote 
\[F(\phi) = \langle f, \phi \rangle = \bra{f} \ket{\phi}.\]

If \(A\) is a linear operator, it makes sense to denote \(A \ket{\phi}\) by 
the ket-vector \(\ket{A \phi}\). On the other hand, if 
\(\ket{\chi} := A \ket{\phi}\), we see that, for all \(\psi \in \mathcal{H}\),
\[\bra{\chi} \ket{\psi} = \langle A \phi, \psi\rangle = 
  \langle \phi, A^\dagger \psi \rangle = \bra{\phi} A^\dagger \ket{\psi}.\]
Hence, we denote the linear operator \(\bra{\chi} = \langle A\phi, \cdot\rangle\) 
by the notation \(\bra{\phi} A^\dagger\). Thus, the notation \(\bra{\phi} A \ket{\psi}\)
might denotes two equal equations, i.e. 
\(\langle A^\dagger \phi, \psi \rangle = \bra{\phi} A \ket{\psi} = 
\langle \phi, A \psi \rangle\).

Given an expression in Dirac notation, there are a set of easy rules for finding 
the adjoint of that expression (easy check by unfolding definitions). In particular,
to find the adjoint, one needs to replace
\begin{itemize}
  \item \(c \leftrightarrow c^*\) for scalars \(c\);
  \item \(\bra{\phi} \leftrightarrow \ket{\phi}\) for vectors \(\phi\);
  \item \(A \leftrightarrow A^\dagger\) for linear operators \(A\);
  \item reverse the order of factors.
\end{itemize}

By definition, we have \(\bra{\phi}\ket{\psi} = \langle \phi, \psi \rangle\) 
is the inner product of \(\phi\) and \(\psi\). On the other hand, one can define 
the outer product by \(\ket{\phi} \bra{\psi}\) which is a linear operator such 
that \(\ket{\chi} \mapsto \ket{\phi}\bra{\psi}\ket{\chi} = \bra{\psi}\ket{\chi}\ket{\phi}\). 
In the case that \(\phi = \psi\), we see that the operator \(\ket{\phi} \bra{\phi}\) 
is the projection operator on to the vector \(\phi\).

Now, if \(\{\ket{\phi_n}\}\) is countable basis of a Hilbert space \(\mathcal{H}\), 
then for all \(\ket{\chi} \in \mathcal{H}\), we have 
\[\ket{\chi} = \sum_n \chi_n \ket{\phi_n}.\]
Then, if \(\{\ket{\phi_n}\}\) is orthonormal, we have 
\[\ket{\chi} = \sum_n \bra{\phi_n}\ket{\chi} \ket{\phi_n} = 
  \sum_n \ket{\phi_n} \bra{\phi_n} \ket{\chi} = 
  (\sum_n \ket{\phi_n} \bra{\phi_n}) \ket{\chi}.\]
Thus, the operator \(\sum_n \ket{\phi_n} \bra{\phi_n}\) is the identity operator.

\subsection{Principles of Quantum Mechanics}

Quantum mechanics is built upon principles on which all results follow. They 
are like the axioms in mathematics from which all states can eventually reduce 
down to them.

The first principle states that, the \textit{state} of a quantum system is 
described by a non-zero vector in a Hilbert space.

We have already seen a application of this principle in Schrödinger dynamics 
where the Hilbert space is \(L^2\) and the state is described by the wave 
function \(\psi \in L^2\).

\begin{definition}[State Space]
  The state space is a projective Hilbert space, i.e. a Hilbert space 
  \(\mathcal{H} \setminus \{0\}\) quotiented by the equivalence relation \(\sim\) where 
  \(x \sim y \iff\) there exists some \(c \in \mathbb{C}\) such that 
  \(x = c y\).
\end{definition}

\end{document}
