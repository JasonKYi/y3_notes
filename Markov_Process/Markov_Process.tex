% Options for packages loaded elsewhere
\PassOptionsToPackage{unicode}{hyperref}
\PassOptionsToPackage{hyphens}{url}
\PassOptionsToPackage{dvipsnames,svgnames*,x11names*}{xcolor}
%
\documentclass[]{article}
\usepackage{lmodern}
\usepackage{amssymb,amsmath}
\usepackage{ifxetex,ifluatex}
\ifnum 0\ifxetex 1\fi\ifluatex 1\fi=0 % if pdftex
  \usepackage[T1]{fontenc}
  \usepackage[utf8]{inputenc}
  \usepackage{textcomp} % provide euro and other symbols
\else % if luatex or xetex
  \usepackage{unicode-math}
  \defaultfontfeatures{Scale=MatchLowercase}
  \defaultfontfeatures[\rmfamily]{Ligatures=TeX,Scale=1}
\fi
% Use upquote if available, for straight quotes in verbatim environments
\IfFileExists{upquote.sty}{\usepackage{upquote}}{}
\IfFileExists{microtype.sty}{% use microtype if available
  \usepackage[]{microtype}
  \UseMicrotypeSet[protrusion]{basicmath} % disable protrusion for tt fonts
}{}
\makeatletter
\@ifundefined{KOMAClassName}{% if non-KOMA class
  \IfFileExists{parskip.sty}{%
    \usepackage{parskip}
  }{% else
    \setlength{\parindent}{0pt}
    \setlength{\parskip}{6pt plus 2pt minus 1pt}}
}{% if KOMA class
  \KOMAoptions{parskip=half}}
\makeatother
\usepackage{xcolor}\pagecolor[RGB]{28,30,38} \color[RGB]{213,216,218}
\IfFileExists{xurl.sty}{\usepackage{xurl}}{} % add URL line breaks if available
\IfFileExists{bookmark.sty}{\usepackage{bookmark}}{\usepackage{hyperref}}
\hypersetup{
  pdftitle={Markov Process},
  pdfauthor={Kexing Ying},
  colorlinks=true,
  linkcolor=Maroon,
  filecolor=Maroon,
  citecolor=Blue,
  urlcolor=red,
  pdfcreator={LaTeX via pandoc}}
\urlstyle{same} % disable monospaced font for URLs
\usepackage[margin = 1.5in]{geometry}
\usepackage{graphicx}
\makeatletter
\def\maxwidth{\ifdim\Gin@nat@width>\linewidth\linewidth\else\Gin@nat@width\fi}
\def\maxheight{\ifdim\Gin@nat@height>\textheight\textheight\else\Gin@nat@height\fi}
\makeatother
% Scale images if necessary, so that they will not overflow the page
% margins by default, and it is still possible to overwrite the defaults
% using explicit options in \includegraphics[width, height, ...]{}
\setkeys{Gin}{width=\maxwidth,height=\maxheight,keepaspectratio}
% Set default figure placement to htbp
\makeatletter
\def\fps@figure{htbp}
\makeatother
\setlength{\emergencystretch}{3em} % prevent overfull lines
\providecommand{\tightlist}{%
  \setlength{\itemsep}{0pt}\setlength{\parskip}{0pt}}
\setcounter{secnumdepth}{5}
\usepackage{tikz}
\usepackage{physics}
\usepackage{amsthm}
\usepackage{mathtools}
\usepackage{esint}
\usepackage[ruled,vlined]{algorithm2e}
\theoremstyle{definition}
\newtheorem{theorem}{Theorem}
\newtheorem{definition*}{Definition}
\newtheorem{prop}{Proposition}
\newtheorem{corollary}{Corollary}[theorem]
\newtheorem*{remark}{Remark}
\theoremstyle{definition}
\newtheorem{definition}{Definition}[section]
\newtheorem{lemma}{Lemma}[section]
\newtheorem{proposition}{Proposition}[section]
\newtheorem{example}{Example}[section]
\newcommand{\diag}{\mathop{\mathrm{diag}}}
\newcommand{\Arg}{\mathop{\mathrm{Arg}}}
\newcommand{\hess}{\mathop{\mathrm{Hess}}}

% the redefinition for the missing \setminus must be delayed
\AtBeginDocument{\renewcommand{\setminus}{\mathbin{\backslash}}}

\title{Markov Process}
\author{Kexing Ying}

\begin{document}
\maketitle

{
\hypersetup{linkcolor=}
\setcounter{tocdepth}{2}
\tableofcontents
}
\newpage

\section{Introduction and Review}

We will in this course assume the following notation:
\begin{itemize}
  \item \((\Omega, \mathcal{F}, \mathbb{P})\) is a probability space;
  \item \(\mathcal{X}\) is a Polish space, i.e. a separable, completely 
    metrizable, topological space;
  \item \(\mathcal{B}(\mathcal{X})\) is the Borel \(\sigma\)-algebra of \(\mathcal{X}\).
\end{itemize}

\begin{definition}[Stochastic Process]
  A stochastic process \((x_n)_{n \in I}\) is a collection of random variables. 
  In the case that \(I = \mathbb{N}\) or \(\mathbb{Z}\), we say that the 
  stochastic process is discrete time. On the other hand if \(I = \mathbb{R}_{\ge 0}\) 
  or \([0, 1] \subseteq \mathbb{R}\), then we say the process is continuous time.  
\end{definition}

We recall some definitions from elementary probability theory. 

\begin{definition}[Random Variable]
  A random variable \(x : \Omega \to \mathcal{X}\) is simply a measurable function.
\end{definition}

\begin{definition}[Probability Distribution]
  Given a random variable \(x : \Omega \to \mathcal{X}\), the probability 
  distribution of \(x\), denoted by \(\mathcal{L}(x)\) is the push-forward 
  measure of \(\mathbb{P}\) along \(x\), i.e. 
  \[\mathcal{L}(x) = x_* \mathbb{P} : A \in \mathcal{F} \mapsto \mathbb{P}(x^{-1}(A)).\]
\end{definition}

\begin{proposition}
  Let \(x : \Omega \to \mathcal{X}\) be a random variable where \(\mathcal{X}\) 
  is countable, then
  \[\mathcal{L}(x) = \sum_{i \in X} \mathbb{P}(x = i) \delta_i := 
    \sum_{i \in X} x_* \mathbb{P}(\{i\}) \delta_i\]
  where \(\delta_i\) is the Dirac measure concentrated at \(i\).
\end{proposition}
\begin{proof}
  Let \(A \subseteq X\), then 
  \[\mathcal{L}(x)(A) = \sum_{i \in A} \mathcal{L}(x)(\{i\}) = 
    \sum_{i \in X} \mathcal{L}(x)(\{i\})\delta_i(A) = 
    \sum_{i \in X} x_* \mathbb{P}(\{i\}) \delta_i(A),\]
  as required.
\end{proof}

\begin{definition}[Independence]
  Given random variables \(x_1, \cdots, x_n\), we say \(x_1, \cdots, x_n\) are 
  independent if 
  \[\mathcal{L}((x_1, \cdots, x_n)) = \bigotimes_{i = 1}^n \mathcal{L}(x_i),\]
  where \(\otimes\) denotes the product measure.
\end{definition}

As the name suggests, we will in this course mostly focus on a class of stochastic 
processes known as Markov processes. These are processes in which given information 
about the process at the present time, its future is independent from its 
history. In particular, if \((x_n)\) is a Markov process, given its value at 
\(x_k\), the value of \(x_j\) is independent of the values of \(x_i\) for all 
\(i < k < j\).

\begin{definition}[Invariant Probability Measure]
  A probability measure \(\pi\) is said to be an invariant probability measure 
  or an invariant distribution of a Markov process \((x_n)_{n \in I}\) if for all 
  \(n \in I\), we have \(\pi = \mathcal{L}(x_n)\).
\end{definition}

A Markov chain started from an invariant distribution does is called a 
stationary Markov process as its distribution do not evolve and we say that the 
chain is in equilibrium.

In this course we will study the behaviour of the distribution of Markov processes. 
In particular, we ask
\begin{itemize}
  \item does there exists an invariant measure? If so, is it unique?
  \item how does the distribution evolve over time?
  \item does \(\mathcal{L}(x_n)\) converge as \(n \to \infty\) 
    (convergence in distribution)?
\end{itemize}

\newpage
\section{Markov Property}

Let us now consider the Markov property in a more formal context.

\subsection{Filtration and Simple Markov Property}

In formation and filtration is an important notion not only for Markov processes 
but for stochastic processes in general. 

In formally, the information of a random 
variable \(x\) is the collection of all possible events, i.e. the sigma algebra 
generated by \(x\), 
\[\sigma(x) = \sigma(\{x^{-1}(A) \mid A \in \mathcal{B}(\mathcal{X})\}).\]
In the case of a stochastic process \((x_n)\), the information on the process up 
to time \(n\) is the \(\sigma\)-algebra generated by \(x_0, \cdots, x_n\), 
i.e. \(\sigma(x_0, \cdots, x_n)\).

With this in mind, we see that the notion of possible events evolving in time 
is naturally described by a sequence of increasing \(\sigma\)-algebras. 
We call such a sequence a filtration.

\begin{definition}[Filtration]
  A filtration is a sequence \((\mathcal{F}_n)\) of 
  increasing sub-\(\sigma\)-algebras of \(\mathcal{F}\). 
\end{definition}

\begin{definition}[Adapted]
  A stochastic process \((x_n)\) is adapted to the filtration \((\mathcal{F}_n)\) if 
  for all \(n\), \(x_n\) is \(\mathcal{F}_n\)-measurable.
\end{definition}

\begin{definition}[Natural Filtration]
  Given a stochastic process \((x_n)\), the natural filtration \((\mathcal{F}^x_n)\) 
  for \((x_n)\) is 
  \[\mathcal{F}^x_n := \sigma(x_0 ,\cdots, x_n).\]
  We note that by definition, a stochastic process is always adapted to its natural 
  filtration.
\end{definition}

Recalling the definition of conditional expectation, we introduce the following 
notations. 

\begin{definition}[Conditional Probability]
  Given a \(\sigma\)-algebra \(\mathcal{G} \subseteq \mathcal{F}\) and a random 
  variable \(x\), we define the conditional probability of \(x\) with respect 
  to \(\mathcal{G}\) to be 
  \[\mathbb{P}(x \in A \mid \mathcal{G}) := \mathbb{E}(\mathbf{1}_A(X) \mid \mathcal{G}),\]
  for all \(A \in \mathcal{B}(\mathcal{X})\) where \(\mathbf{1}_A\) is the indicator 
  function of \(A\). 

  Furthermore, given random variables \(x_0, \cdots, x_n\), we denote 
  \[\mathbb{P}(x \in A \mid x_0, \cdots, x_n) := \mathbb{P}(x \in A \mid \sigma(x_0, \cdots, x_n)).\]
\end{definition}

\begin{definition}[Simple Markov Property]
  A stochastic process \((x_n)\) with state space \(\mathcal{X}\) is said to have the 
  simple Markov property if for any \(A \in \mathcal{B}(\mathcal{X})\) 
  and \(n \ge 0\), we have 
  \[\mathbb{P}(x_{n + 1} \in A \mid x_0, \cdots, x_n) = \mathbb{P}(x_{n + 1} \in A \mid x_n),\]
  almost surely.

  Unfolding the notation, the simple Markov property states that 
  \[\mathbb{E}(\mathbf{1}_A(x_{n + 1}) \mid \mathcal{F}^x_n) = 
    \mathbb{E}(\mathbf{1}_A(x_{n + 1}) \mid \sigma(x_n)).\]

  We call a stochastic process which has the simple Markov property a Markov 
  process and we call \(\mathcal{L}(x_0)\) the initial distribution. Furthermore, 
  if the Markov process is discrete, we call it a Markov chain. 
\end{definition}

The definition of the simple Markov property can be generalized to continuous 
stochastic processes by taking the property to be 
\(\mathbb{E}(\mathbf{1}_A(x_t) \mid \mathcal{F}^x_s) = 
\mathbb{E}(\mathbf{1}_A(x_t) \mid \sigma(x_s))\) for all \(s \le t\).

In the case that \(\mathcal{X} = \mathbb{N}\), the simple Markov property is 
equivalent to the statement that 
\[\mathbb{P}(x_{n + 1} = j \mid x_0 = i_0, \cdots, x_n = i_n) = 
  \mathbb{P}(x_{n + 1} = j \mid x_n = i_n),\]
almost surely for every \(n\) where \(i_0, \cdots, i_n \in \mathcal{X} = \mathbb{N}\)
\[\mathbb{P}(x_0 = i_0, \cdots, x_n = i_n) > 0.\]

\begin{lemma}
  Let \(\mathcal{G} \subseteq \mathcal{F}\), \(X : \Omega \to \mathcal{X}, 
  Y : \Omega \to \mathcal{Y}\) be random variables such that \(X\) is 
  \(\mathcal{G}\)-measurable, \(Y\) is independent of \(\mathcal{G}\). 
  Then, if \(\phi : \mathcal{X} \times \mathcal{Y} \to \mathbb{R}\) is measurable 
  such that \(\phi(X, Y) \in L^1\), we have 
  \[\mathbb{E}(\phi(X, Y) \mid \mathcal{G})(\omega) = \mathbb{E}_Y(\phi(X(\omega), Y))\]
  almost surely. 
\end{lemma}
\begin{proof}
  Exercise.
\end{proof}

\begin{proposition}
  Let \(\xi_1, \xi_2, \cdots\) be a sequence of independent random variables with 
  state space \(\mathcal{Y}\) and is independent with respect to 
  \(x_0 : \Omega \to \mathcal{X}\). Then, if \(F : \mathcal{X} \times \mathcal{Y} \to \mathcal{X}\) 
  is a measurable function, we may define the stochastic process 
  \[x_{n + 1} = F(x_n, \xi_{n + 1}).\]
  \((x_n)\) is a Markov process. 
\end{proposition}
\begin{proof}
  Let \(A \in \mathcal{B}(\mathcal{X})\). Then, 
  \[\begin{split}
    \mathbb{E}(\mathbf{1}_A(x_{n + 1}) \mid x_0, \cdots, x_n) 
    & = \mathbb{E}(\mathbf{1}_A(F(x_n, \xi_{n + 1}) \mid x_0, \cdots, x_n) \\
    & = \omega \mapsto \mathbb{E}(\mathbf{1}_A(F(x_n(\omega), \xi_{n + 1})),
  \end{split}\]
  where the second equality follows by the above lemma (setting \(\phi = \mathbf{1}_A \circ F\) 
  and observing that \(x_n\) is \(\sigma(x_0, \cdots, x_n)\)-measurable and 
  \(\xi_{n + 1}\) is independent of \(\sigma(x_0, \cdots, x_n)\)). Similarly, 
  \[\begin{split}
    \mathbb{E}(\mathbf{1}_A(x_{n + 1}) \mid x_n) 
    & = \mathbb{E}(\mathbf{1}_A(F(x_n, \xi_{n + 1}) \mid x_n) \\
    & = \omega \mapsto \mathbb{E}(\mathbf{1}_A(F(x_n(\omega), \xi_{n + 1})),
  \end{split}\]
  we have \(\mathbb{E}(\mathbf{1}_A(x_{n + 1}) \mid \mathcal{F}^x_n) = 
  \mathbb{E}(\mathbf{1}_A(x_{n + 1}) \mid \sigma(x_n))\) as required.
\end{proof}

\subsection{Markov Property}

So far we have looked at the simple Markov property in which we have taken the 
filtration to be the natural filtration of the process. However, in the case 
that we are looking at multiple processes, we would like to consider a larger 
filtration such that each process is adapted. This motivates the general definition 
for the Markov property.

\begin{definition}
  Let \((\mathcal{F}_t)_{t \in I}\) be a filtration indexed by the set \(I\) on 
  the measurable space \((\Omega, \mathcal{F})\). A stochastic process \((x_t)_{t \in I}\) 
  on \(\mathcal{X}\) is a Markov process with respect to \(\mathcal{F}_t\) if it 
  is adapted to \(\mathcal{F}_t\) and 
  \[\mathbb{P}(x_t \in A \mid \mathcal{F}_s) = \mathbb{P}(x_t \in A \mid x_s)\]
  almost surely for all \(s, t \in I\), \(t > s\) and \(A \in \mathcal{B}(\mathcal{X})\).

  Again, unfolding the notation, the above statement says 
  \[\mathbb{E}(\mathbf{1}_A(x_t) \mid \mathcal{F}_s) = \mathbb{E}(\mathbf{1}_A(x_t) \mid \sigma(x_s))\]
  almost surely.
\end{definition}

\begin{proposition}
  If \((x_t)\) is a Markov process with respect to the filtration \((\mathcal{F}_t)\), 
  then it is also a Markov process with respect to its natural filtration \((\mathcal{F}_t^x)\).
\end{proposition}
\begin{proof}
  Recalling that \(\mathcal{F}_t^x \subseteq \mathcal{F}_t\) for all \(t\),
  by the tower property of the conditional expectation, we have 
  \[\begin{split}
    \mathbb{P}(x_{t + s} \in A \mid \mathcal{F}_s^x) & = 
    \mathbb{E}(\mathbf{1}_A(x_{t + s}) \mid \mathcal{F}_s^x) \\
    & = \mathbb{E}(\mathbb{E}(\mathbf{1}_A(x_{t + s}) \mid \mathcal{F}_s) \mid \mathcal{F}_s^x)\\
    & = \mathbb{E}(\mathbb{E}(\mathbf{1}_A(x_{t + s}) \mid \sigma(x_s)) \mid \mathcal{F}_s^x),
  \end{split}\]  
  where the equalities denotes equal a.e. Thus, as \(\mathbb{E}(\mathbf{1}_A(x_{t + s}) \mid \sigma(x_s))\) 
  is \(\sigma(x_s)\)-measurable, and thus \(\mathcal{F}_s^x\)-measurable (since 
  \(\sigma(x_s) \subseteq \sigma(x_r \mid r \le s) = \mathcal{F}_s^x)\)), we have 
  \[\mathbb{E}(\mathbb{E}(\mathbf{1}_A(x_{t + s}) \mid \sigma(x_s)) \mid \mathcal{F}_s^x) = 
  \mathbb{E}(\mathbf{1}_A(x_{t + s}) \mid \sigma(x_s))\]
  implying that the Markov property is satisfied.
\end{proof}

\begin{theorem}
  If \((x_t)\) is a Markov process with respect to the filtration \((\mathcal{F}_t)\), 
  then 
  \[\mathbb{E}(f(x_t) \mid \mathcal{F}_s) = \mathbb{E}(f(x_t) \mid \sigma(x_s))\]
  almost surely for any \(f : \mathcal{X} \to \mathbb{R}\) bounded and measurable. 
  In particular, this property is equivalent to the Markov property by choosing 
  \(f = \mathbf{1}_A\) for all \(A \in \mathcal{B}(\mathcal{X})\).
\end{theorem}
\begin{proof}
  By linearity, the property holds for simple functions. Furthermore, by the 
  conditional monotone convergence theorem, the property holds for any non-negative 
  bounded measurable functions. Finally, for arbitrary bounded measurable functions 
  \(f\), the result follows by taking \(f = f^+ - f^-\) and applying the non-negative 
  case.
\end{proof}

\begin{proposition}
  Let \(C \in \mathcal{F}_s\) and suppose \(\mathcal{B}(\mathcal{X}) = 
  \sigma(\mathcal{D})\) where 
  \(\mathcal{D}\) is a \(\pi\)-system (i.e. non-empty and closed under finite intersections), 
  then, if 
  \[\mathbb{E}(\mathbf{1}_A(x_{t + s}) \mathbf{1}_C) = 
    \mathbb{E}(\mathbb{P}(x_{t + s} \in A \mid x_s) \mathbf{1}_C)\] 
  holds for any \(A \in \mathcal{D}\), it holds for any \(A \in \mathcal{B}(\mathcal{X})\).
\end{proposition}
\begin{proof}
  Let \(\mathcal{A}\) be the set of Borel sets which the equation holds. Then, 
  by definition \(\mathcal{D} \subseteq \mathcal{A}\) and so, it suffices to show 
  \(\mathcal{A}\) is a \(\lambda\)-system (i.e. \(\mathcal{A}\) contains \(\mathcal{X}\), 
  closed under complements and closed under countable unions of increasing sets). 
  Indeed, Dynkin's \(\pi - \lambda\) theorem states that if \(\mathcal{D}\) is 
  a \(\pi\)-system, \(\mathcal{A}\) is a \(\lambda\)-system and 
  \(\mathcal{D} \subseteq \mathcal{A}\), then \(\sigma(\mathcal{D}) \subseteq \mathcal{A}\).

  Clearly \(\mathcal{X} \in \mathcal{A}\) since 
  \[\mathbb{E}(\mathbf{1}_{\mathcal{X}}(x_{t + s}) \mathbf{1}_C) = 
    \mathbb{E}(\mathbf{1}_C) = 
    \mathbb{E}(\mathbb{P}(x_{t + s} \in \mathcal{X} \mid x_s) \mathbf{1}_C).\]
  Suppose now \(A \in \mathcal{A}\). Then, the property holds as 
  \(\mathbf{1}_{A^c} = 1 - \mathbf{1}_A\) and so, the result follows by linearity.
  Finally, if \((A_n) \subseteq \mathcal{A}\) is increasing. Then by the 
  monotone convergence theorem for conditional expectations, it follows that 
  \(\bigcup A_n \in \mathcal{A}\) and hence, \(\mathcal{A}\) is a \(\lambda\)-system 
  as required.
\end{proof}

\begin{proposition}
  Suppose \(\mathbb{E}(f(x_{n + 1}) \mid x_0, \cdots, x_n) = \mathbb{E}(f(x_{n + 1}) \mid x_n)\)
  for any bounded measurable \(f\). Then, if we have a sequence 
  \[0 \le t_1 < t_2 < \cdots < t_{m - 1} < t_m = n - 1,\]
  where \(n > 1, t_i \in \mathbb{N}\), for any bounded measurable functions \(f, h\), 
  we have 
  \[\mathbb{E}(f(x_{n + 1})h(x_n) \mid x_{t_1}, \cdots, x_{t_m}) = 
    \mathbb{E}(f(x_{n + 1})h(x_n) \mid x_{n - 1}).\]
\end{proposition}
\begin{proof}
  Exercise.
\end{proof}

As we will often use the bounded measurable functions, let us denote the set 
of bounded measurable functions \(f : \mathcal{X} \to \mathbb{R}\) by 
\(\mathcal{B}_b(\mathcal{X})\).

\begin{lemma}
  Let \(X, Y \in L_1\) and \(\mathcal{G} \subseteq \mathcal{F}\). Then, if \(X\) 
  is \(\mathcal{G}\)-measurable and \(XY \in L_1\), we have
  \[\mathbb{E}(XY \mid \mathcal{G}) = X\mathbb{E}(Y \mid \mathcal{G}).\]
  We call this property ``taking out what is known''.
\end{lemma}
\begin{proof}
  See problem sheet 1.
\end{proof}

\begin{theorem}
  Given a stochastic process \((x_n)\) and indices \(l < m < n\), TFAE.
  \begin{itemize}
    \item For any \(f \in \mathcal{B}_b(\mathcal{X})\), 
      \[\mathbb{E}(f(x_n) \mid x_l, x_m) = \mathbb{E}(f(x_n) \mid x_m).\]
    \item For any \(g \in \mathcal{B}_b(\mathcal{X})\), 
      \[\mathbb{E}(g(x_l) \mid x_m, x_n) = \mathbb{E}(g(x_l) \mid x_m).\]
    \item For any \(f, g \in \mathcal{B}_b(\mathcal{X})\), 
      \[\mathbb{E}(f(x_n)g(x_l) \mid x_m) = \mathbb{E}(f(x_n) \mid x_m) \mathbb{E}(g(x_l) \mid x_m).\]
      That is to say, given now, the past is independent of the future.
  \end{itemize}
\end{theorem}
\begin{proof}
  Suppose the first statement holds, we will prove the third property. 
  Let \(f, g \in \mathcal{B}_b(\mathcal{X})\), then by the tower law and the above lemma, 
  we have 
  \[\begin{split}
    \mathbb{E}(f(x_n)g(x_l) \mid x_m) & = \mathbb{E}(\mathbb{E}(f(x_n)g(x_l) \mid x_m, x_l) \mid x_m)\\
    & = \mathbb{E}(g(x_l) \mathbb{E}(f(x_n) \mid x_m, x_l) \mid x_m)\\
    & = \mathbb{E}(g(x_l) \mathbb{E}(f(x_n) \mid x_m) \mid x_m)\\
    & = \mathbb{E}(f(x_n) \mid x_m)\mathbb{E}(g(x_l) \mid x_m)
  \end{split}\]
  which is exactly the third property.

  On the other hand, if the third property holds, for any \(g, h \in \mathcal{B}_b(\mathcal{X})\), 
  we have 
  \[\begin{split}
    \mathbb{E}(f(x_n)h(x_m)g(x_l)) & = \mathbb{E}(\mathbb{E}(f(x_n)g(x_l) \mid x_m)) h(x_m))\\
    & = \mathbb{E}(\mathbb{E}(f(x_n) \mid x_m)) \mathbb{E}(g(x_l) \mid x_m) h(x_m))\\
    & = \mathbb{E}(\mathbb{E}(\mathbb{E}(f(x_n) \mid x_m)g(x_l)h(x_m) \mid x_m))\\
    & = \mathbb{E}(\mathbb{E}(f(x_n) \mid x_m)g(x_l)h(x_m))
  \end{split}\]
  where the last equality is due to the law of total expectation. Now, by considering 
  this equality implies that, for all \(A = A_1 \cap A_2\) where 
  \(A_1 \in \sigma(x_l), A_2 \in \sigma(x_m)\), by choosing \(g = \mathbf{1}_{x_l^{-1}(A_1)}\) 
  and \(h = \mathbf{1}_{x_m^{-1}(A_2)}\), we have
  \[\int_A f(x_n) \dd \mathbb{P} = \int_A \mathbb{E}(f(x_n) \mid x_m) \dd \mathbb{P},\]
  and so, \(\mathbb{E}(f(x_n) \mid x_m) = \mathbb{E}(f(x_n) \mid x_l, x_m)\) almost 
  surely. Hence, the first and third property are equivalent. Similarly, one 
  can should that the second property is equivalent to the third property and 
  hence the equivalence.
\end{proof}

\begin{proposition}
  A stochastic process \((x_n)\) is a Markov process if and only if one of the following 
  conditions holds:
  \begin{itemize}
    \item for any \(f_i \in \mathcal{B}_b(\mathcal{X})\),
    \[\mathbb{E}\left(\prod_{i = 1}^n f_i(x_i)\right) = \mathbb{E}\left(\prod_{i = 1}^{n - 1} f_i(x_i) 
      \mathbb{E}(f_n(x_n) \mid x_{n - 1})\right).\]
    \item for any \(A_i \in \mathcal{B}(\mathcal{X})\), 
    \[\mathbb{P}(x_0 \in A_0, \cdots, x_n \in A_n) = 
      \int_{\bigcap_{i = 0}^{n - 1}\{x_i \in A_i\}} \mathbb{P}(x_n \in A_n \mid x_{n - 1}) \dd \mathbb{P}.\]
  \end{itemize}
\end{proposition}
\begin{proof}
  We note that by choosing \(f_i = \mathbf{1}_{A_i}\), the first condition implies 
  the second. On the other hand, the reverse implication follows by the standard routine 
  of proving it for simple function and using monotone convergence. Thus, it 
  suffices to establish an equivalence between the first condition 
  and the Markov property. This is left as an exercise.
\end{proof}

\subsection{Gaussian Measure and Gaussian Process}

As one of the most important distributions in probability theory, let us in this 
short section introduce the Gaussian measure which we will again encounter later 
on with this course.

\begin{definition}[Gaussian Measure]
  A measure \(\mu\) on \(\mathbb{R}^n\) is Gaussian if there exists a non-negative 
  definite symmetric matrix \(K\) and \(m \in \mathbb{R}^n\) such that the Fourier 
  transform of \(\mu\) is
  \[\int_{\mathbb{R}^n} e^{i\langle\lambda, x\rangle} \mu(\dd x) = 
    e^{i \langle \lambda, m \rangle - \frac{1}{2} \langle K \lambda, \lambda \rangle}\]
  for any \(\lambda \in \mathbb{R}^n\). We call the matrix \(K\) the covariance 
  of \(\mu\) and \(m\) its mean.

  We remark that if \(X\) is a random variable with distribution \(\mu\), then 
  the Fourier transform of \(\mu\) is simply the characteristic function of 
  \(X\), \(\mathbb{E}(e^{i\langle\lambda, X\rangle})\).
\end{definition}

\begin{proposition}
  If \(\mu\) is a Gaussian measure with covariance \(K\) and mean \(m\) 
  is absolutely continuous with respect to the Lebesgue measure if and only if 
  \(K\) is non-degenerate. In this case, for all \(A \in \mathcal{B}(\mathbb{R}^n)\),
  \[\mu(A) = \int_A \frac{1}{\sqrt{(2\pi)^n \det K}} 
    e^{-\frac{1}{2}\langle K^{-1}(x - m), x - m \rangle} \lambda(\dd x).\]
\end{proposition}

We observe that if \(X\) is a random variable with Gaussian distribution \(\mu\), 
then as one might expect, \(\mathbb{E}(X) = m\) and 
\[\text{Cov}(X_i, X_j) := \mathbb{E}(X_i - m_i)(X_j - m_j) = K_{ij}.\]
For this reason, we call \(K\) the covariance operator.

\begin{theorem}
  If \(X\) is a Gaussian random variable (i.e. its distribution is Gaussian) 
  on \(\mathbb{R}^d\) with covariance \(K\). Then, if \(A : \mathbb{R}^d \to \mathbb{R}^m\) 
  is a linear transformation, then \(AX\) is Gaussian with covariance \(AKA^T\) 
  and mean \(Am\).
\end{theorem}
\begin{proof}
  Follows since,
  \[\mathbb{E}e^{i \langle \lambda, AX \rangle} = \mathbb{E}e^{i \langle A^T\lambda, X\rangle}
    = e^{i \langle A^T \lambda, m \rangle - \frac{1}{2}\langle KA^T \lambda, A^T \lambda\rangle}
    = e^{i \langle \lambda, Am \rangle - \frac{1}{2}\langle AKA^T \lambda, \lambda\rangle}.\]
\end{proof}

\begin{proposition}
  Linear combinations of independent Gaussian random variables are also Gaussian.
\end{proposition}
\begin{proof}
  Exercise.
\end{proof}

\begin{definition}[Gaussian Process]
  A stochastic process is Gaussian if its finite dimensional distributions 
  are Gaussian.
\end{definition}

\subsection{Kolmogorov's Extension Theorem}

Let \((x_n)\) be a stochastic process with state space \(\mathcal{X}\). 
Denote 
\[\mathcal{X}^{\mathbb{N}_0} := \prod_{i = 0}^\infty \mathcal{X} = 
  \{(a_0, a_1, \cdots) \mid a_i \in \mathcal{X}\}.\]
We may consider \((x_n)\) as a map from \(\Omega \to \mathcal{X}^{\mathbb{N}_0}\) by 
defining 
\[(x_n) : \Omega \to \mathcal{X}^{\mathbb{N}_0} : \omega \mapsto (x_n(\omega))_{n = 0}^\infty.\]
We would like \((x_n)\) to be measurable as a map from \(\Omega \to \mathcal{X}^{\mathbb{N}_0}\) 
and so, let us first equip \(\mathcal{X}^{\mathbb{N}_0}\) with a \(\sigma\)-algebra.

\begin{definition}
  Given \(\mathcal{X}_i\) complete separable metric spaces for \(i \in \Lambda\), 
  define the projection maps 
  \[\pi_m : \prod_{i \in \Lambda} \mathcal{X}_i \to \mathcal{X}_m : (a_i)_{i \in \Lambda} \mapsto a_m.\]
  Then, we define 
  \(\bigotimes_{i \in \Lambda} \mathcal{B}(\mathcal{X}_i) = 
    \sigma(\pi_i \mid i \in \Lambda).\)

  We note that this definition can be easily extended to arbitrary measurable 
  spaces.
\end{definition}

\begin{proposition}
  If \((x_n)\) is a stochastic process, then as a map from \(\Omega \to \mathcal{X}^{\mathbb{N}_0}\),
  \((x_n)\) is \(\bigotimes_{n} \mathcal{B}(\mathcal{X})\)-measurable.
\end{proposition}

With this in mind, we can push forward the probability measure along a stochastic 
process, inducing a measure on \(\mathcal{X}^{\mathbb{N}_0}\). In particular, 
we have the measure space \((\mathcal{X}^{\mathbb{N}_0}, 
\bigoplus_n \mathcal{B}(\mathcal{X}), (x_n)_* \mathbb{P})\).

On the other hand, if we only consider the first \(n\)-components of the 
process \((x_i)\), by the same argument, \((x_i)_{i = 1}^n\) forms a measurable 
map from \(\Omega \to \mathcal{X}^{n}\). Then, in this case, we call the 
push-forward measure \(\mathcal{L}((x_i)_{i = 1}^n) = {(x_i)_{i = 1}^n}_* \mathbb{P}\) 
the joint distribution. These are known as the finite dimensional 
distributions of \((x_n)\).

\begin{definition}
  Let \((\mu_n)_{n = 1}^\infty\) be a sequence of probability measures on 
  \(\mathcal{X}^n\) (i.e. \(\mu_n\) is aprobability measures on \(\mathcal{X}^n\)). 
  Then \((\mu_n)_{n = 0}^\infty\) is said to satisfy Kolmogorov's consistency 
  condition of 
  \[\mu_{n + 1}(A_1 \times \cdots A_n \times \mathcal{X}) = 
    \mu_n(A_1 \times \cdots \times A_n)\]
  for all \(n \ge 0\), \(A_i \in \mathcal{B}(\mathcal{X})\).
\end{definition}

\begin{theorem}[Kolmogorov's Extension Theorem]
  Let \((\mu_n)_{n = 1}^\infty\) be a sequence of probability measures which are 
  consistent. Then, there exists a unique probability measure \(\mu\) on 
  \(\mathcal{X}^{\mathbb{N}_0}\) such that for any \(n\), 
  \(A \in \bigotimes_{i = 1}^n \mathcal{B}(\mathcal{X})\), 
  \[\mu(A \times \mathcal{X}^{\mathbb{N}_0}) = \mu_n(A).\]
  In other words, if we denote \(\text{pr}_n\) the map 
  \[\text{pr}_n : \mathcal{X}^{\mathbb{N}_0} \to \mathcal{X}^n : 
    (a_i)_{i = 1}^\infty \mapsto (a_i)_{i = 1}^n,\]
  \(\mu\) is the unique measure which satisfies 
  \[(\text{pr}_n)_* \mu = \mu_n.\]
\end{theorem}

\begin{corollary}
  The finite dimensional distributions of a stochastic process \((x_n)\) determines 
  uniquely the probability distribution of the process on \(\mathcal{X}^{\mathbb{N}_0}\).
\end{corollary}

\begin{corollary}
  Given any consistent family of probability measures \((\mu_n)\), there exists a 
  stochastic process with \((\mu_n)\) as its finite dimensional distributions.
\end{corollary}
\begin{proof}
  Kolmogorov's extension theorem implies that there exists a compatible measure 
  \(\mu\) on \(\mathcal{X}^{\mathbb{N}_0}\). Thus, it suffices to find 
  a \(\mathcal{X}^{\mathbb{N}_0}\)-valued random variable with distribution 
  \(\mu\). We will describe a trivial method for this purpose below.
\end{proof}

Let \(\mu\) be a probability measure on \(\mathcal{X}\). Then, setting 
\(\Omega = \mathcal{X}, \mathcal{F} = \mathcal{B}(\mathcal{X})\) and \(\mathbb{P} = \mu\),
it is clear that the push-forward of \(\mathbb{P}\) along the identity map 
provides \(\mu\). Thus, the identity is a random variable with the distribution 
\(\mu\).

Thus, in the case of the corollary above, we set \(\Omega = \mathcal{X}^{\mathbb{N}_0}\),
\(\mathcal{F} = \bigotimes\mathcal{B}(\mathcal{X})\), and \(\mathbb{P} = \mu\). 
We call this probability space the canonical probability space and call the 
resulting process \((\pi_n)\) the canonical process.

\begin{definition}[Shift Operator]
  For \(n \in \mathbb{N}\), we define the \(n\)-th shift operator by 
  \[\theta_n : \mathcal{X}^{\mathbb{N}_0} \to \mathcal{X}^{\mathbb{N}_0} : 
    (a_0, a_1, \cdots) \mapsto (a_n, a_{n + 1}, \cdots).\]
  For connivance, we will denote the above property by \(\theta_n(a.) = (a_{n +}.)\). 
\end{definition}

\begin{definition}[Stationary Process]
  A stochastic process \((x.)\) is stationary if for any \(n\), 
  \[\mathcal{L}(\theta_n x.) = \mathcal{L}(x.).\]
\end{definition}

Straight away, by Kolmogorov's extension, we see that an equivalent definition 
for the stationary process is that the finite dimensional distributions 
of \((\theta_n x.)\) are the same as the finite dimensional distributions of \((x.)\).

In general, a process \((x_n)\) is stationary if 
\[\mathcal{L}(x_n, \cdots, x_{n + m}) = \mathcal{L}(x_0, \cdots, x_m)\]
for all \(n, m \ge 0\). In the case that \((x_n)\) is a Gaussian process, 
as \(\mathcal{L}(x_{i_1} \cdots x_{i_n})\) is determined by 
\((\mathbb{E}x_{i_1}, \cdots, \mathbb{E}x_{i_n})\) and \(\text{Cov}(x_{i_k}, x_{i_l})\),
it is stationary if 
\[\mathbb{E}x_n = \mathbb{E}x_0\]
for all \(n\) and the covariances are shift invariant.

\newpage 

\section{Strong Markov Property}

We will in the section continue to let \((\Omega, \mathcal{F}, \mathbb{P})\) be a 
probability space and let \((\mathcal{F}_n)\) be a filtration on this probability 
space.

\subsection{Transition Probability}

Given a Markov process \((x_n)\) on the state space \(\mathcal{X}\), for 
\(A \in \mathcal{B}(\mathcal{X})\), the function \(\mathbb{P}(x_{n + 1} \in A \mid x_n)\) 
is a Borel function of \(x_n\). This function might depend on \(A\), 
\(n\) and \(n - 1\). Suppose the case that this function does not depend on time 
(i.e. time homogeneous), that is there exists some function \(\phi(x, A)\) such that 
\[\mathbb{P}(x_{n + 1} \in A \mid x_n) = \phi(x_n, A)\]
almost surely. Fixing \(x\), under some regularities, its not difficult to show that 
\(\phi(x, \cdot)\) form a probability measure. We shall assume this.

\begin{definition}
  The set \(P := \{P(x, A) \mid x \in \mathcal{X}, A \in \mathcal{B}(\mathcal{X})\}\)
  is said to be a family of transition probabilities if
  \begin{itemize}
    \item for all \(x \in \mathcal{X}\), \(P(x, \cdot)\) is a probability 
      measure;
    \item for any \(A \in \mathcal{B}(\mathcal{X})\), \(P(\cdot, A)\) is Borel 
      measurable.
  \end{itemize} 
\end{definition}

As an example, consider the Markov chain \((x_n)\) on \(\mathcal{X}\) where 
\(x_{n + 1} := F(x_n, \xi_{n + 1})\) for \(x_0, \xi_1, \xi_2, \cdots\)
independent with \((\xi_i : \Omega \to \mathcal{Y}) \sim \mu\) for all \(i\). 
Then, for all \(x \in \mathcal{X}\), we have (recall that we denote the event 
\(x_n^{-1}(\{x\})\) by \(x_n = x\)
\[\begin{split}
  \mathbb{P}(x_{n + 1} \in A \mid x_n = x) & = \mathbb{P}(F(x_n(x, \xi_{n + 1}) \in A)\\
  & = \int_{\mathcal{Y}} \mathbf{1}_A(F(x, \xi_{n + 1})) \dd \mathbb{P}\\
  & = \int_{\mathcal{Y}} \mathbf{1}_A(F(x, y)) \mu(\dd y).
\end{split}\]
Hence, defining 
\[P(x, A) := \int_{\mathcal{Y}} \mathbf{1}_A(F(x, y)) \mu(\dd y),\]
\(\{P(x, A)\}\) are the transition probabilities and 
\[\mathbb{P}(x_{n + 1} \in A \mid x_n = x) = P(x, A).\]

We remark that, in the case that the state space is countable, by \(\sigma\)-additivity,
it is sufficient to work with transitional probabilities of singletons. In particular, 
the transitional probability is simply determined by 
\[P(i, j) = P(i, \{j\}), i, j \in \mathcal{X}.\]

\begin{proposition}
  Let \((x_n)\) be a Markov chain such that there exists a transition 
  probability \(P\) for which 
  \[\mathbb{P}(x_{n + 1} \in A \mid x_n) = P(x_n, A)\]
  for all \(A \in \mathcal{B}(\mathcal{X})\). Then, for all \(f \in \mathcal{B}_b(\mathcal{X})\),
  \[\mathbb{E}(f(x_{n + 1}) \mid x_n) = \int_{\mathcal{X}} f(y) P(x_n, \dd y).\]
\end{proposition}
\begin{proof}
  Choosing \(f = \mathbf{1}_A\), we see that the property is true for simple functions 
  and so, the results can be extended to all functions by the monotone convergence theorem.
\end{proof}

\begin{proposition}
  Let \((x_n)\) as defined above, for all \(f \in \mathcal{B}_b(\mathcal{X})\), 
  we have 
  \[\mathbb{P}(x_{n + 2} \in A \mid x_n) = \int_{\mathcal{X}}P(y, A)P(x_n, \dd y).\]
  Thus, we define the two-step transition probability by 
  \[P^2(x, A) = \int_{\mathcal{X}} P(y, A) P(x, \dd y),\]
  such that \(\mathbb{P}(x_{n + 2} \in A \mid x_n) = P^2(x_n, A)\) almost surely.
\end{proposition}
\begin{proof}
  By the tower law, we have 
  \[\begin{split}
    \mathbb{P}(x_{n + 2} \in A \mid x_n) & = \mathbb{E}(\mathbb{E}(\mathbf{1}_A(x_{n + 2}) \mid x_{n + 1}, x_n) \mid x_n)\\
    & = \mathbb{E}(\mathbb{E}(\mathbf{1}_A(x_{n + 2}) \mid x_{n + 1}) \mid x_n)\\
    & = \mathbb{E}(P(x_{n + 1}, A) \mid x_n)\\
    & = \int_{\mathcal{X}}P(y, A)P(x_n, \dd y),
  \end{split}\]
  where the last equality follows by the above proposition.
\end{proof}

The above process can be extended to \(k\)-steps by induction. In particular, 
for all \(k \in \mathbb{N}\), we have 
\[\mathbb{P}(x_{n + (k + 1)} \in A) \mid x_n) = 
  \int_{\mathcal{X}} P^i(y, A) P^j(x_n, \dd y)\]
for any \(i, j \in \mathbb{N}, i + j = k\). This is known as the Chapman-Kolmogorov 
equation.

\begin{definition}
  A family 
  \[\{P^n(x, \cdot) \mid x\in \mathcal{X}, n \in \mathbb{N}_0\}\]
  is said to be a transition function if 
  \begin{itemize}
    \item \(P^n(x, \cdot)\) is a transition probability for any \(x, n\);
    \item \(P^0(x, \cdot) = \delta_x\) for any \(x\);
    \item (Chapman-Kolmogorov) for all \(n, m \ge 0, x \in \mathcal{X}, A \in \mathcal{B}(\mathcal{B})\),
    \[P^{n + m}(x, A) = \int_{\mathcal{X}} P^n(y, A) P^m(x, \dd y).\]
  \end{itemize}
\end{definition}

\begin{proposition}
  The Chapman-Kolmogorov equation is satisfied if and only if for all 
  \(f \in \mathcal{B}_b(\mathcal{X}), n, m \ge 0, x \in \mathcal{X}, A \in \mathcal{B}(\mathcal{B})\), 
  \[\int_{\mathcal{X}}f(y)P^{n + m}(x, \dd y) = 
  \int_{\mathcal{X}}\left(\int_{\mathcal{X}} f(z) P^n(y, \dd z)\right) P^m(x, \dd y).\]
\end{proposition}
\begin{proof}
  Exercise.
\end{proof}

As alluded to above, the \(k\)-step transitional probabilities can be constructed from a 
1-step transitional probabilities. In particular, given the 1-step transitional 
probability \(P\),
\begin{enumerate}
  \item set \(P^0(x, \cdot) = \delta_x\);
  \item set \(P^1(x, \cdot) = P(x, \cdot)\);
  \item for all \(n > 1\), \(x \in \mathcal{X}\), for all \(A \in \mathcal{B}(\mathcal{X})\), set 
    \[P^{n + 1}(x, A) := \int_{\mathcal{X}} P(y, A) P^n(x, \dd y).\]
\end{enumerate}
It remains to show that this construction satisfies the Chapman-Kolmogorov equation.
Indeed, by induction, suppose that the Chapman-Kolmogorov equation is satisfied for 
all \(k \le n + m\), then 
\[\begin{split}
  P^{n + m + 1}(x, A) & = \int_{\mathcal{X}}P(z, A)P^{n + m}(x, \dd z) \\
  & = \int_{\mathcal{X}}\left(\int_{\mathcal{X}} P(z, A) P^j(y, \dd y)\right) P^{n + m - j}(x, \dd y)\\
  & = \int_{\mathcal{X}}P^{j + 1}(y, A)P^{n + m - j}(x, \dd y)
\end{split}\]
for all \(j = 0, 1, \cdots\), where the second and third equality follows by the 
inductive hypothesis and Fubini's theorem.

\begin{definition}
  A transition probability \(P\) is said to  be the transitional probability of 
  the Markov chain \((x_n)\) if 
  \[\mathbb{P}(x_{n + 1} \in A \mid x_n) = P(x_n, A)\]
  almost surely for all \(A \in \mathcal{B}(\mathcal{X})\) and any \(n \ge 0\).

  If a Markov chain has a transitional probability \(P\), then we say the Markov 
  chain is time homogeneous.
\end{definition}

From this point forward, unless otherwise stated, we will assume our Markov chains 
to be time homogeneous.

\begin{theorem}
  Let \((x_n)\) be a Markov process with transition probability \(P\). Then 
  \begin{itemize}
    \item \(\mathbb{P}(x_{n + m} \in A \mid x_m) = P^n(x_m, A)\) almost surely for any 
      \(n, m \ge 0, A \in \mathcal{B}(\mathcal{X})\);
    \item if \(\mathcal{L}(x_0) = \mu\), then 
      \[\mathbb{P}(x_n \in A) = \int_{\mathcal{X}} P^n(x, A) \mu(\dd x).\] 
  \end{itemize}
\end{theorem}
\begin{proof}
  The first property follows by induction. Indeed, if
  for some \(k \in \mathbb{N}\), \(\mathbb{P}(x_{k + m} \in A \mid x_m) = P^k(x_m, A)\), 
  for any \(m\), then 
  \[\begin{split}
    \mathbb{P}(x_{k + 1 + m} \in A \mid x_m) 
    & = \mathbb{E}(\mathbb{P}(x_{k + 1 + m} \in A \mid \mathcal{F}_{k + m}) \mid x_m) \\
    & = \mathbb{E}(P(x_{m + k}, A) \mid x_m) \\
    & = \int_{\mathcal{X}}P(z, A) P^k(x_m \dd z) = P^{k + 1}(x_m, A)
  \end{split}\]
  where the second to last equality follows as \(\mathbb{E}(f(x_{k + m}) \mid x_m) = 
  \int f(z) P^k(x_m, \dd z)\).

  The second property follows as 
  \[\begin{split}
    \mathbb{P}(x_n \in A) = \mathbb{E}(\mathbb{P}(x_n \in A \mid x_0)) 
    = \mathbb{E}(P^n(x_0, A)) = \int_{\mathcal{X}}P^n(y, A) \mu(\dd y)
  \end{split}\]
  as required.
\end{proof}

\begin{theorem}[Einstein's Relation]
  If \((x_n)\) is a Markov chain with transition probability \(P\) and initial 
  distribution \(\mu\), then, its finite dimensional distributions are 
  \[\mathbb{P}(x_0 \in A_0, \cdots, x_n \in A_n) = 
    \int_{A_0} \cdots \int_{A_{n - 1}} P(y_{n - 1}, A_n) P(y_{n - 2}, \dd y_{n - 1}) \cdots P(y_0, \dd y_1) \mu(\dd y_0).\]
\end{theorem}

We remark that the above definition provides a sequence of consistent measures.
Namely, if we define 
\[\mu(A_0 \times \cdots \times A_n) := 
  \int_{A_0} \cdots \int_{A_{n - 1}} P(y_{n - 1}, A_n) P(y_{n - 2}, \dd y_{n - 1}) \cdots P(y_0, \dd y_1) \mu(\dd y_0),\]
the sequence of measures \((\mu_n)\) is consistent.
\begin{proof}
  For \(A_0, A_1 \in \mathcal{B}(\mathcal{X})\), we observe 
  \[\begin{split}
    \mathbb{P}(x_0 \in A_0, x_1 \in A_1) 
    & = \mathbb{E}(\mathbb{E}(\mathbf{1}_{A_0}(x_0) \mathbf{1}_{A_1}(x_1) \mid x_0)) \\
    & = \mathbb{E}(\mathbf{1}_{A_0}(x_0)\mathbb{E}(\mathbf{1}_{A_1}(x_1) \mid x_0)) \\
    & = \mathbb{E}(\mathbf{1}_{A_0}(x_0) P(x_0, A_1))\\
    & = \int_{A_0} P(y_0, A_1) \mu(\dd y_0).
  \end{split}\]
  Hence, by induction, the relation follows.
\end{proof}

\begin{theorem}
  If \((x_n)\) is a process satisfying Einstein's relation, then it is a 
  Markov chain with transition probability \(P\).
\end{theorem}
\begin{proof}
  Einstein's relation can be extended to all bounded measurable functions through 
  the usual process with the monotone convergence theorem and thus, we have 
  for any \(f_i \in \mathcal{B}_b(\mathcal{X})\), 
  \[\begin{split}
    \mathbb{E}\left(\prod_{i = 0}^n f_i(x_i)\right) 
    & = \int \cdots \int \prod_{i = 0}^n f_i(y_i) P(y_{n - 1}, \dd y_{n}) \cdots P(y_0, \dd y_1) \mu(\dd y_0).
  \end{split}\]
  By Fubini's theorem, the above becomes
  \[\begin{split}
    \cdots & = \int f_n(y_n)P(y_{n - 1}, \dd y_n)
      \int \cdots \int \prod_{i = 0}^{n - 1} f_i(y_i) \prod_{i = 1}^{n - 1} P(y_i, \dd y_{i + 1}) \mu(\dd y_0)\\
    & = \mathbb{E}(f_n(x_n) \mid x_{n - 1} = y_{n - 1}) 
      \int \cdots \int \prod_{i = 0}^{n - 1}f_i(y_i) \prod_{i = 1}^{n - 1} P(y_i \dd y_{i + 1}) \mu(\dd y_0)\\
    & = \mathbb{E}\left(\mathbb{E}(f_n(x_n) \mid x_{n - 1}) \prod_{i = 0}^{n - 1}f_i(x_i)\right)
  \end{split}\]
  which is equivalent to the Markov property.
\end{proof}

\begin{theorem}[Existence of Markov Chain]
  Given a family of transition probabilities on \(P\) on \(\mathcal{X}\) and any 
  probability measure \(\mu_0\) on \(\mathcal{X}\), there exists a unique (in distribution) 
  Markov process \(x\) with transition probability \(P\) and initial distribution 
  \(\mu_0\). 
\end{theorem}
\begin{proof}
  Define \(\mu_n\) on \(\mathcal{X}^{n + 1}\) such that 
  \[\mu_n(A_0 \times A_n) := \int_{A_0} \cdots \int_{A_{n - 1}} 
  P(y_{n - 1}, A_n) P(y_{n - 2}, \dd y_{n - 1}) \cdots P(y_1, \dd y_0) \mu_0(\dd y_0).\]
  It is routine to check that this sequence of measures is well-defined and consistent, and 
  thus, by the Kolmogorov extension theorem, there exists a unique \(\mathbb{P}_\mu\) on \(\mathcal{X}^\infty\) 
  such that \(\mathbb{P}_\mu\) projects to \(\mu_n\) on \(\mathcal{X}^{n + 1}\). Thus, 
  taking \((\pi_n)\) to be the canonical process on the canonical space 
  \((\mathcal{X}^\infty, \otimes \mathcal{B}(\mathcal{X}), \mathbb{P}_\mu)\), we have 
  found a Markov process which satisfies the condition.
\end{proof}

Consider again the case where the state space is countable \(\mathcal{X} = \mathbb{N}\). 
As mentioned previously, the transition probability on \(\mathcal{X}\) is then 
determined by \(p_{ij} = P(i, \{j\})\). As \(P(i, \cdot)\) is a probability 
measure by definition, 
\[1 = P(i, \mathcal{X}) = \sum_{j \in \mathcal{X}} p_{ij}.\]
In the case that \(\mathcal{X}\) is finite, these \(p_{ij}\) can be represented 
as a matrix, motivating the definition of a stochastic matrix.

\begin{definition}[Stochastic Matrix]
  A matrix \(p = (p_{ij})\) with \(p_{ij} \ge 0\) is said to be a stochastic 
  matrix if \(\sum_{j \in \mathcal{X}} p_{ij} = 1\).
\end{definition}

In the discrete case, our construction of the transition probability from 
the 1-step transition probability is straightforward. In particular, we obtain
\[P^{n + 1}(i, A) = \int_{\mathcal{X}} P(y, A) P^n(i, \dd y) = \sum_{k \in \mathcal{X}} P(k, A) P^n(i, k).\]
Thus, if we write \(P(i, \{j\}) = p_{ij}\), then
\[P^2(i, \{j\}) = \sum_{k \in \mathcal{X}} p_{ik}p_{kj} = ((p_{kl})_{k,l \in \mathcal{X}}^2)_{ij},\]
where the last term denotes matrix multiplication. Thus, by induction, we obtain that 
\[P^n(i, \{j\}) = \sum_{k_1 \in \mathcal{X}} \cdots \sum_{k_{n - 1} \in \mathcal{X}}
  p_{ik_1} p_{k_1k_2} \cdots p_{k_{n-1}j} = ((p_{kl})_{k,l \in \mathcal{X}}^n)_{ij}.\]

\subsection{Transition Operator}

In the case of the transition probability \(P\) of a Markov chain \((x_n)\), we have the 
relation 
\[\mathbb{P}(x_{n + 1} \in A) = \int_{\mathcal{X}} P(y, A) \mu_n(\dd y)\]
where \(\mu_n = \mathcal{L}(x_n)\). Thus, in some sense, the transitional probability 
the an operator on measures changing the distribution to the next time step. 
This motivates the following definition.

\begin{definition}[Transition Operator]
  The transition operator \(T^*\) given the transition probability \(P\) on the 
  set of probability measures on \(\mathcal{X}\) is defined to be 
  \[T^* \mu(A) := \int_{\mathcal{X}} P(y, A) \mu(\dd y)\]
  for all \(A \in \mathcal{B}(\mathcal{X})\).
\end{definition}

With this definition, we obtain that, given a Markov process \((x_n)\) with the 
transition probability \(P\), we have \((T^*)^n(\mathcal{L}(x_m)) = \mathcal{L}(x_{n + m})\).  

\begin{definition}[Dual Transition Operator]
  The dual transition operator \(T_*\) given the transition probability \(P\) is 
  defined to be an operator acting on \(\mathcal{B}_b(\mathcal{X})\) such that 
  \[T_* f(x) = \int_{\mathcal{X}} f(y) P(x, \dd y)\]
  for all \(f \in \mathcal{B}_b(\mathcal{X})\).
\end{definition}

Equivalently, the dual transition operator acting on \(f\) is 
\[T_* f(x) = \mathbb{E}(f(x_1) \mid x_0 = x)\]
where \((x_n)\) is the Markov process associated with \(P\).

\begin{proposition}
  The above operators are dual in the sense that, for all \(f \in \mathcal{B}_b(\mathcal{X})\),
  \[\int_{\mathcal{X}} T_* f \dd \mu = \int_{\mathcal{X}} f \dd(T^* \mu).\]
\end{proposition}
\begin{proof}
  Hint: first prove for \(f = \mathbf{1}_A\).
\end{proof}

For simplicity, we will denote both \(T^*\) and \(T_*\) with \(T\) when there is no 
confusion.

We note that \(T^*\) extends to signed measures allowing us to show linearity 
(recall that the set of signed measures form a vector space over \(\mathbb{R}\)).

In the case that \(\mathcal{X} = \{1, \cdots, N\}\) is finite, a probability 
measure \(\nu\) is uniquely determined by its values on singletons 
\(\{\nu(\{1\}), \cdots, \nu(\{N\})\}\). Then, if \(P = (p_{ij})\) is a stochastic matrix,
we have \(T\nu = (\nu(\{i\}))_{i = 1}^N P\).

\subsection{Stopping Times}

\begin{definition}[Stopping Time]
  A function \(T : \Omega \to \overline{\mathbb{N}} := \{0, 1, \cdots \} \cup \{\infty\}\) 
  is said to be a stopping time with respect to the filtration \((\mathcal{F}_n)\) if 
  \[\{\omega \mid T(\omega) = n\} \in \mathcal{F}_n\]
  for all \(n \ge 0\).
\end{definition}

This definition can be easily generalized to continuous time with the codomain 
being \(\overline{\mathbb{R}}_+\) and taking 
\[\{\omega \mid T(\omega) \le t\} \in \mathcal{F}_t\]
for all \(t \ge 0\) instead. The generalized definition is consistent with the 
discrete version since \(\{T \le n\} = \bigcup_{k = 1}^n {T = k}\) and 
\(\{T = n\} = \{T \le n + 1\} \setminus \{T \le n\}\) and thus, 
\(T\) is an \((\mathcal{F}_n)\) stopping time if and only if \(\{T \le n\} \in \mathcal{F}_n\) 
for all \(n \ge 0\).

\begin{definition}[Stopped Process]
  Let \(T\) be a stopping time and let \((x_n)\) be a stochastic process. We define 
  the stopped process \((x^T_n)\) by 
  \[x^T_n(\omega) := x_{\min \{n, T(\omega)\}}(\omega),\]
  for all \(\omega \in \Omega\), \(n \in \overline{\mathbb{N}}\).
\end{definition}

In some sense, the stopped process as the name suggests, stop the process once some 
condition has been achieved. Consider the random walk on the integers with the 
stopping time being the walk reaches 4. Then, for each \(\omega \in \Omega\), 
the stopped process is the same as the process as long as \(x_n(\omega) \le 4\) 
while after \(x_k(\omega) = 4\), the process stops in the sense that 
\(x^T_n(\omega)\) is constant for all \(n \ge k\).

The following three propositions are exercises.

\begin{proposition}
  If \(S, T\) are stopping times, then so are 
  \[S \wedge T := \min\{S, T\} \text{ and } S \vee T := \max\{S, T\}.\]
\end{proposition}

\begin{proposition}
  If \((S_n)\) is a sequence of stopping times, then 
  \[\limsup_{n \to \infty} S_n \text{ and } \liminf_{n \to \infty} S_n\]
  are stopping times.
\end{proposition}

\begin{proposition}
  Constant functions are stopping times.
\end{proposition}

\begin{definition}[Stopped \(\sigma\)-algebra]
  Given a stopping time \(T\), let \(\mathcal{F}_\infty := \bigvee_{n = 0}^\infty \mathcal{F}_n\) 
  and define 
  \[\mathcal{F}_T := \{A \in \mathcal{F}_\infty \mid A \cap \{T = n\} \in \mathcal{F}_n, \forall n \ge 0\}.\]
  For the continuous case, the set \(\{T = n\}\) is replaced by \(\{T \le t\}\).
\end{definition}

We note that in the case that \(T\) is a constant, \(\mathcal{F}_T = \mathcal{F}_m\).
Furthermore, if \(S \le T\) a.e. then \(\mathcal{F}_S \subseteq \mathcal{F}_T\) 
(we assume the probability space is complete, i.e. null-sets are measurable).

\begin{proposition}
  For a stopping time \(T < \infty\), the stopped \(\sigma\)-algebra can be 
  equivalently defined as
  \[\mathcal{F}_T = \{A \in \mathcal{F} \mid A \cap \{T \le t\} \in \mathcal{F}_t, \forall t \ge 0\}.\]
\end{proposition}
\begin{proof}
  Clearly the right hand side is larger and so, it suffices to show that, for all 
  \(A \in \mathcal{F}\) such that \(A \cap \{T \le t\} \in \mathcal{F}_t, \forall t \ge 0\),
  \(A \in \mathcal{F}_T\). Now, by considering 
  \[A = \bigcup_{n \in \mathbb{N}} A \cap \{T \le n\},\]
  where \(A \cap \{T \le n\} \in \mathcal{F}_n\), we have 
  \[A = \bigcup_{n \in \mathbb{N}} A \cap \{T \le n\} \in \bigvee_{n = 0}^\infty \mathcal{F}_n,\]
  and hence, \(A \in \mathcal{F}_T\) as required.
\end{proof}

\begin{proposition}
  \(T\) is \(\mathcal{F}_T\)-measurable.
\end{proposition}
\begin{proof}
  Let \(A = \{T = m\}\) and by construction, for all \(n \ge 0\), we have 
  \(A \cap {T = n}\) is \(\varnothing\) if \(n \neq m\) and \(\{T = n\}\) if 
  \(n = m\) both contained in \(\mathcal{F}_n\). Thus, \(\{T = m\} \in \mathcal{F}_T\) 
  and hence \(T\) is \(\mathcal{F}_T\)-measurable.
\end{proof}

\begin{proposition}
  For a stopping time \(T < \infty\), and an adapted process \((x_n)\), 
  \begin{itemize}
    \item \(\omega \mapsto x_{T(\omega)}(\omega)\) (denoted by \(x_T\)) is \(\mathcal{F}_T\)-measurable; 
    \item for all \(n\), \(x_n^T\) is \(\mathcal{F}_T\)-measurable. 
  \end{itemize}
\end{proposition}
\begin{proof}
  We have \(\{x_T \in A\} \cap \{T = n\} = \{x_n \in A\} \cap \{T = n\} \in \mathcal{F}_n\) 
  as \((x_n)\) is adapted. Thus, \(x_T\) is \(\mathcal{F}_T\)-measurable.

  The second property follows as \(x^T_n = x_{T \wedge n}\) where \(T \wedge n\) 
  is a stopping time. Thus, \(x_{T \wedge n}\) is \(\mathcal{F}_{T \wedge n}\)-measurable. 
  On the other hand, as \(T \wedge n \le T\), we have \(\mathcal{F}_{T \wedge n} \subseteq \mathcal{F}_T\)
  and thus, \(x^T_n\) is \(\mathcal{F}_T\)-measurable as required.
\end{proof}

\begin{proposition}
  If \(T < \infty\) is a \(\mathcal{F}_T^x\)-stopping time, then for all \(n \ge 0\), 
  \[\{T = n\} \in \sigma(x_{T \wedge 0}, \cdots, x_{T \wedge n}).\]
  That is to say \(T\) is a stopping time with respect to the natural filtration of 
  stopped process \(x_n^T\).
\end{proposition}
\begin{proof}
  It suffices to show \(\mathbf{1}_{T = n}\) is of the form \(\phi(x_{T \wedge 0}, \cdots, x_{T \wedge n})\)
  for some \(\phi \in \mathcal{B}_b(\mathcal{X}^{n + 1})\). We will prove this by induction.

  Suppose there exists some \(\phi_l \in \mathcal{B}_b(\mathcal{X}^{l + 1})\) for all 
  \(l \le k - 1\) such that 
  \[\mathbf{1}_{\{T = l\}} = \phi(x_{T \wedge 0}, \cdots, x_{T \wedge l}).\]
  We observe (factorisation lemma implies the existence of \(\psi\)),
  \[\begin{split}
    \mathbf{1}_{\{T = k\}} & = \mathbf{1}_{\{T = k\}}\mathbf{1}_{\{T \ge k\}} = \psi(x_0, \cdots, x_k) \mathbf{1}_{\{T \ge k\}}\\
    & = \psi(x_{T \wedge 0}, \cdots, x_{T \wedge k}) \mathbf{1}_{\{T \ge k\}} \\ 
    & = \psi(x_{T \wedge 0}, \cdots, x_{T \wedge k}) (1 - \mathbf{1}_{\{T \le k - 1\}}).\\
  \end{split}\]
  Now, since \(1 - \mathbf{1}_{\{T \le k - 1\}} = 1 - \sum_{l \le k - 1} \phi_l\), the result follows.
\end{proof}

\begin{proposition}
  Let us denote \(\sigma(x^T) := \sigma(x_{T \wedge n} \mid n)\), then if 
  \(T < \infty\) is a stopping time with respect to \(\mathcal{F}_n^x\), 
  \[\mathcal{F}_T = \sigma(x^T).\] 
\end{proposition}
\begin{proof}
  Clearly \(\mathcal{F}_T \supseteq \sigma(x^T)\) so we will prove the reverse. 
  Let \(A \in \mathcal{F}_T\), then, for all \(n \ge 0\), as \(A \cap \{T = n\}\) 
  is \(\mathcal{F}_n\) measurable, by the factorisation lemma, there exists some 
  \(\psi \in \mathcal{B}_n(\mathcal{X}^{n + 1})\) such that
  \[\psi(x_0, \cdots, x_n) = \mathbf{1}_{A \cap \{T = n\}} = \mathbf{1}_A\mathbf{1}_{\{T = n\}}.\]
  So,
  \[\mathbf{1}_A\mathbf{1}_{\{T = n\}} =  \mathbf{1}_A \mathbf{1}_{\{T = n\}}^2 = 
    \mathbf{1}_{\{T = n\}} \psi(x_0, \cdots, x_n) = 
    \mathbf{1}_{\{T = n\}} \psi(x_{T \wedge 0}, \cdots, x_{T \wedge n}).\]
  Thus, as \(\{T = n\} \in \sigma(x_{T \wedge 0}, \cdots, x_{T \wedge n}) \subseteq \sigma(x^T)\) by the 
  above lemma, \(\mathbf{1}_A \mathbf{1}_{\{T = n\}}\) is \(\sigma(x^T)\)-measurable. Hence,
  \[\mathbf{1}_A = \sum_{n = 0}^\infty \mathbf{1}_A \mathbf{1}_{\{T = n\}} 
    \in \sigma(x^T)\]
\end{proof}

\subsection{Strong Markov Property}

Recall the shift operator, we define the following operator.

\begin{definition}
  Given a stopping time \(T < \infty\) a.e. we define \(\theta_T\) 
  such that for all stochastic process \(x. = (x_n)\) 
  \[(\theta_T x.)_n(\omega) := x_{T(\omega) + n}(\omega).\]
\end{definition}

\begin{definition}[Strong Markov Property]
  A stochastic process \(x.\) is said to have the strong Markov property if 
  for every stopping time \(T < \infty\) a.e. and every 
  \(\Phi \in \mathcal{B}_b(\mathcal{X}^\infty)\), we have 
  \[\mathbb{E}(\Phi(\theta_T x.) \mid \mathcal{F}_T) = \mathbb{E}(\Phi(\theta_T x.) \mid x_T),\]
  almost everywhere.
\end{definition}

We may assume that \(\Phi\) is independent in its components. Namely, the strong Markov 
property is equivalent to 
\[\mathbb{E}\left(\prod_{i = 1}^m f_i(\theta_T x.)_{n_i} \mid \mathcal{F}_T\right) = 
  \mathbb{E}\left(\prod_{i = 1}^m f_i(\theta_T x.)_{n_i} \mid x_T\right),\]
for some \(f_i \in \mathcal{B}_b(\mathcal{X})\) and \(n_1 < n_2 < \cdots < n_m\).

Our goal is to show that if \((x_n)\) is a time homogeneous Markov process with 
transition probability \(P\), then it has the strong Markov property.

\begin{proposition}
  Let \(T < \infty\) a.e. be a stopping time. Then 
  \[\mathbb{P}(x_{n + T} \in A \mid \mathcal{F}_T) = P^n(x_T, A)\]
  almost everywhere. In particular, \((x_{n + T})= \theta_T x.\) is a Markov process 
  with transition probability \(P\).
\end{proposition}
\begin{proof}
  Let \(f \in \mathcal{B}_b(\mathcal{X})\), then, as \(\{T = \infty\}\) has measure 0, 
  for all \(B \in \mathcal{F}_T\),
  \[\begin{split}
    \int_B f(x_{n + T}) \dd \mathbb{P} & = \sum_{m = 0}^\infty \int_{B \cap \{T = m\}} f(x_{n + m}) \dd \mathbb{P} \\
    & = \sum_{m = 0}^\infty \int_{B \cap \{T = m\}} \mathbb{E}(f(x_{n + m}) \mid \mathcal{F}_m) \dd \mathbb{P}\\
    & = \sum_{m = 0}^\infty \int_{B \cap \{T = m\}}  \int f(y)P^n(x_m, \dd y) \dd \mathbb{P}\\
    & = \sum_{m = 0}^\infty \int_{B \cap \{T = m\}}  \int f(y)P^n(x_T, \dd y) \dd \mathbb{P}\\
    & = \int_B \int f(y) P^n(x_T, \dd y) \dd \mathbb{P},
  \end{split}\]
  where the second equality follows as \(B \cap \{T = m\}\) is \(\mathcal{F}_m\)-measurable 
  by the definition of stopping time while the third equality follows the the property 
  of the transition probability. Thus, 
  \[\mathbb{E}(f(x_{n + T}) \mid \mathcal{F}_T) = \int f(y) P^n(x_T, \dd y).\]
  Hence, choosing \(f = \mathbf{1}_A\) completes the proof.
\end{proof}

Recall that given any measure \(\mu\) and transition probability \(P\), there exists 
a unique probability measure \(\mathbb{P}_\mu\) on \(\mathcal{X}^\infty\) 
(distribution of the canonical process) 
which is the distribution of a Markov process (denoted by \(X^x\) with transition 
probabilities \(P\) and initial distribution \(\mathbb{P}_\mu\). If \(\mu = \delta_x\) 
the Dirac measure, then we denote \(\mathbb{P}_{\delta_x}\) by \(\mathbb{P}_x\).

We introduce the following notation. If \(\Phi \in \mathcal{B}_b(\mathcal{X}^\infty)\),
we denote 
\[\mathbb{E}_x[\Phi] := \mathbb{E}(\Phi(X^x.)) = \int_{\mathcal{X}^\infty} \Phi \dd \mathbb{P}_x,\]
where the equality follows by the change of variable formula. 

\begin{theorem}[Strong Markov Property for Finite Stopping Time]
  Let \((x_n)\) be a time homogeneous Markov process with transition probability \(P\) 
  and let \(T\) be a finite stopping time. Then,
  \begin{itemize}
    \item \(\theta_T x.\) is also a time homogeneous Markov process with transition probability 
      \(P\) and initial value \(x_T\);
    \item if \(\Phi \in \mathcal{B}_b(\mathcal{X}^\infty)\), then 
    \[\mathbb{E}(\Phi(\theta_T x.) \mid \mathcal{F}_T) = \mathbb{E}_{x_T}[\Phi].\]
    That is to say, \(\theta_T x.\) also has the strong Markov property.
  \end{itemize}
\end{theorem}

We have already proved the first property. We shall provide a proof for a simpler 
case where \(T = n\) though the proof also works for the strong case (see official notes).

Again, since the product \(\sigma\)-algebra on \(\mathcal{X}^\infty\) is determined 
by the \(\pi\)-system of cylindrical sets, it is sufficient to check the property 
for functions of the form \(x \mapsto \prod_{i = 1}^m f_i(x_{n_i})\) for 
\(f_i \in \mathcal{B}_b(\mathcal{X})\). Thus, the property is equivalent to 
\[\mathbb{E}\left(\prod_{i = 1}^k f_i(x_{n_i + T}) \mid \mathcal{F}_T\right) = 
  \mathbb{E}_{x_T}\left[\prod_{i = 1}^k f_i \circ \pi_{n_i}\right],\]
where \(\pi_{n_i}\) is the \(n_i\)-th projection map from \(\mathcal{X}^\infty\).

Let us first consider the case where \(T = \text{id}\). Then, 
\[\mathbb{E}_{x}\left[\prod_{i = 1}^k f_i \circ \pi_{n_i}\right] = 
  \int \cdots \int \prod_{i = 1}^m f_i(y_i) \prod_{j = 1}^m
  P^{n_j - n_{j - 1}}(y_{j - 1}, \dd y_j).\]
In fact, \(\prod_{j = 1}^m P^{n_j - n_{j - 1}}(y_{j - 1}, \dd y_j)\) is the distribution 
of \((\pi_{n_1}, \cdots, \pi_{n_m})\) on 
\((\mathcal{X}^\infty, \bigotimes \mathcal{B}(\mathcal{X}), \mathbb{P}_x)\).

\begin{proposition}
  Let \((x_n)\) be a time homogeneous Markov chain with transition probability \(P\). 
  Then, for any \(\Phi \in \mathcal{B}_b(\mathcal{X}^\infty)\), 
  \[\mathbb{E}(\Phi(\theta_n x.) \mid \mathcal{F}_n)(\omega) = \mathbb{E}_{x_n(\omega)}[\Phi]\]
  almost everywhere.
\end{proposition}
\begin{proof}
  We may assume \(\Phi = \prod_{i = 1}^k f_i\), \(f_i \in \mathcal{B}_b(\mathcal{X})\)
  and it suffices to show that, for \(m_1 < \cdots < m_k\), 
  \[\mathbb{E}\left(\prod_{i = 1}^k f_i(x_{n + m_i}) \mid \mathcal{F}_n\right) 
    = \mathbb{E}_{x_n}\left[\prod_{i = 1}^k f_i(x_{m_i})\right].\]
  We will induct on \(k\). For \(k = 1\), 
  \[\mathbb{E}(f(x_{n + m}) \mid \mathcal{F}_n)(\omega) = \int f(y) P^m(x_n, \dd y)\]
  almost surely by the Markov property. Suppose now, the property holds for 
  \(k - 1\). Then,
  \[\begin{split}
    & \mathbb{E}\left(\prod_{i = 1}^k f_i(x_{n + m_i}) \mid \mathcal{F}_n\right)\\
    & = \mathbb{E}\left(\prod_{i = 1}^{k - 1} f_i(x_{n + m_i}) \mathbb{E}(f_k(x_{n + m_k}) \mid \mathcal{F}_{n + m_{k - 1}}) \mid \mathcal{F}_n\right)\\
    & = \mathbb{E}\left(\prod_{i = 1}^{k - 1} f_i(x_{n + m_i}) \int f_k(y_k) P^{m_k - m_{k - 1}}(x_{n + m_{k - 1}}, \dd y_k) \mid \mathcal{F}_n\right)\\
    & = \mathbb{E}_{x_n}\left[
      \prod_{i = 1}^{k - 1} f_i(x_{n + m_i})\int f_k(y_k) P^{m_k - m_{k - 1}}(y_{k - 1}, \dd y_k)
    \right]\\
    & = \int \cdots \int \prod_{i = 1}^{k - 1} f_i(y_i) \int f_k(y_k) P^{m_k - m_{k - 1}}(y_{k - 1}, \dd y_k)
      \prod_{i = 1}^{k - 1}P^{m_i - m_{i - 1}}(y_{i - 1}, \dd y_i)\\
    & = \int \cdots \int \prod_{i = 1}^k f_i(y_i) \prod_{j = 1}^k P^{m_i - m_{i - 1}}(y_{i - 1}, \dd y_i)
      = \mathbb{E}_{x_n}\left[\prod_{i = 1}^k f_i(x_{m_i})\right]
  \end{split}\]
  as required.
\end{proof}

\begin{theorem}[Strong Markov Property for non-finite Stopping Times]
  Let \((x_n)\) be a time homogeneous Markov process with transition probability \(P\) 
  and let \(T\) be a stopping time. Then for any \(\Phi : \mathcal{B}_b(\mathcal{X}^\infty)\),
  \[\mathbb{E}(\Phi(\theta_T x.)\mathbf{1}_{\{T < \infty\}} \mid \mathcal{F}_T)(\omega) = 
    \mathbb{E}_{x_T(\omega)}[\Phi]\]
  on \(\{T < \infty\}\) almost everywhere.
\end{theorem}

More or less the same as the finite case since we are working on \(\{T < \infty\}\).

Recall that the transition operator \(T^*\) is an operator acting on the space of 
measures such that, for \(\mu\) a measure on \(\mathcal{X}\), 
\[T^* \mu(A) = \int P(y, A) \mu(\dd y).\]
We say \(\mu\) is invariant if \(\mu = T^* \mu\). If \(\mathcal{L}(x_0) = \pi\) 
is an invariant probability measure, then \(\mathcal{L}(x_n) = \pi\) for all \(n \ge 0\).
Thus, the distribution of the process does not change with time. Indeed, 
by definition 
\[\mathcal{L}(x_{n + 1})(A) = \mathbb{P}(x_{n + 1} \in A) = \int P(y, A) \mathcal{L}(x_n)(\dd y),\]
and so follows by induction.

\begin{definition}[Invariant]
  A measure \(\mu\) is said to be invariant if \(\mu = T^* \mu\) where \(T^*\) is 
  the transition operator with respect to some transition probability.
\end{definition}

\begin{definition}[Stationary]
  A process \((x_n)\) is said to be stationary such that \(\theta_n x.\) has invariant 
  distributions for all \(n \ge 0\).
\end{definition}

\begin{proposition}
  A time homogeneous Markov process with invariant initial distribution is stationary.
\end{proposition}
\begin{proof}
  Suppose \((x_n)\) is a Markov process with initial distribution \(\pi\) such that 
  \(\pi\) is invariant. Then, we denote \(\mathbb{P}_\pi\) the distribution of \(x.\) 
  on \(\mathcal{X}^\infty\), namely \(\mathbb{P}_\pi = (x.)_*\mathbb{P}\). By the 
  Markov property, \(\theta_n x.\) is a Markov process with transition probability \(P\) 
  and initial distribution \(\mathcal{L}((\theta_n x.)_0) = \mathcal{L}(x_n) = \pi\). 
  Hence, \(\mathcal{L}(\theta_n x.) = \mathbb{P}_\pi\) and so, \((x_n)\) is a stationary 
  process.
\end{proof}

\newpage
\section{THMC With Discrete Time}

We will in this section consdier time homogeneous Markov chains (THMC) on discrete time.

We will in this section denote \(\mathcal{X} = \{1, 2, \cdots, N\}\) if \(|\mathcal{X}| < \infty\)
and \(\mathcal{X} = \{1, 2, \cdots\}\) otherwise. Given a THMC on \(\mathcal{X}\) with 
transition probability \(P\) with initial distribution \(\nu\), then, we write
\(\mathcal{L}(x_n) =: \nu P^n\). As \(\nu\) is discrete, it is represented by a vector 
\(v \in [0, 1]^{\mathcal{X}}\) such that for all \(i \in \mathcal{X}\), \(\nu(\{i\}) = v_i\).
For short hand we write \(\nu(i) := \nu(\{i\})\). Then, 
\[\nu P(i) = \sum_{k \in \mathcal{X}} \nu(k)P_{ki}.\]

\begin{definition}[Accessible]
  Let \(i, j \in \mathcal{X}\). Then we say \(j\) is accessible from \(i\) if there 
  exists some \(n\) such that \(P^n_{ij} = \mathbb{P}(x_n = j \mid x_0 = i) > 0\). 
  We denote this by \(i \longrightarrow j\).
\end{definition}

\begin{definition}[Communicating]
  For \(i, j \in \mathcal{X}\), \(i, j\) are communicating if \(i \longrightarrow j\) 
  and \(j \longrightarrow i\). We denoted this by \(i \sim j\).
\end{definition}

We note the communicating is not necessarily an equivalence relation as a state 
might not communicate with itself.

\begin{definition}[Communicating Class]
  A communicating class for some \(i \in \mathcal{X}\) is the set 
  \([i] = \{j \in \mathcal{X} \mid i \sim j\}\).
\end{definition}

\begin{definition}[Irreducible]
  A chain is irreducible if there exists only one communicating class, otherwise 
  it is reducible.
\end{definition}

\begin{lemma}
  Communicating is transitive.
\end{lemma}
\begin{proof}
  Suppose \(i, j, k \in \mathcal{X}\) and \(i \longrightarrow j, j \longrightarrow k\),
  then, there exists some \(n_1, n_2\) such that \(P^{n_1}_{ij}, P^{n_2}_{jk} > 0\). 
  Then, by the C-K,
  \[\begin{split}
    P^{n_2 + n_1}_{ik} & = P^{n_2 + n_1}(i, \{k\}) = \int P^{n_2}(y, \{k\}) P^{n_1}(i, \dd y) 
      = \sum_{l \in \mathcal{X}} P^{n_2}(l, \{k\}) P^{n_1}(i, \{l\})\\
      & \ge P^{n_2}(j, \{k\}) P^{n_1}(i, \{j\}) = P^{n_2}_{jk}P^{n_1}_{ij} > 0,
  \end{split}\]
  implying \(i \longrightarrow k\) as required.
\end{proof}

\begin{lemma}
  Let \(i, j \in \mathcal{X}\) and suppose \(i \longrightarrow j\). Then, 
  any element in \([j]\) is accessible from any element in \([i]\).
\end{lemma}
\begin{proof}
  Let \(i' \in [i]\) and \(j' \in [j]\). Then, by definition
  \[i' \longrightarrow i \longrightarrow j \longrightarrow j'\]
  implying \(i' \longrightarrow j'\) by transitivity.
\end{proof}

With this lemma in hand, we may define a partial order (antisymmetric) on the 
communicating classes. In particular, we say \([i] \le [j]\) if every element of 
\(i\) can be assessed from any element of \(j\). Equivalently, \(j \longrightarrow i\). 

\begin{definition}[Minimal]
  Am communicating class \([i]\) is said to be minimal (or closed) if there does \textbf{not} exist 
  a communicating class \([j] \neq [i]\) such that \([j] \le [i]\).
\end{definition}

\subsection{Recurrence and Transience}

\begin{definition}[Recurrent \& Transient]
  Let \(\mathcal{X}\) be countable. A state \(i \in \mathcal{X}\) is recurrent 
  if \(\mathbb{P}(T_i < \infty \mid x_0 = i) = 1\) where \(T_i := \inf\{n \ge 1 \mid x_n = i\}\)
  is the first hitting time of \(i\) by the process. If a process is not recurrent 
  at \(i\), then \(i\) is called a transient state.
  
  If every element of \(\mathcal{X}\) is recurrent, then the process is called 
  recurrent. Similarly, the process is called transient if every state is transient.
\end{definition}

Let us introduce the following notation:
\begin{itemize}
  \item \(\mathbb{P}_i(A) := \mathbb{P}(A \mid x_0 = i)\),
  \item \(\mathbb{P}_i(T_i < \infty) := \mathbb{P}(T_i < \infty \mid x_0 = i)\),
  \item \(\mathbb{E}_i(Y) := \mathbb{E}(Y \mid x_0 = i)\).
\end{itemize}

We shall see later that this existence of a recurrent state implies the existence of 
and invariant measure. Also, recurrent and transient are properties invariant 
on communicating classes.

\begin{lemma}
  Given two states \(i, j \in \mathcal{X}\), \(i \longrightarrow j\) if and only 
  if \(\mathbb{P}_i(T_j < \infty) > 0\). Also, 
  \[\mathbb{P}_i(T_j < \infty) \le \sum_{n = 1}^\infty P_{ij}^n.\]
\end{lemma}
\begin{proof}
  We note that \(\{T_j < \infty\} = \bigsqcup_{n = 1}^\infty \{T_j = n\}\) and so, 
  if \(P^n_{ij} > 0\), \(0 < P^n_{ij} = \mathbb{P}_i(x_j = n) \le \mathbb{P}(\{T_j < \infty\})\).
  On the other hand, 
  \[\mathbb{P}_i(T_j < \infty) = \sum_{n = 1}^\infty \mathbb{P}_i(T_j = n) \le 
    \sum_{n = 1}^\infty \mathbb{P}_i(x_n = j) = \sum_{n = 1}^\infty P^n_{ij}\] 
  and so, if \(\mathbb{P}_i(T_j < \infty) > 0\), then, \(P_{ij}^n > 0\) for some \(n\).
\end{proof}

Let us define the following sequence of stopping times. Let \(T_j^0 = 0\), 
\(T_j^1 = T_j\) and 
\[T_j^{n + 1} := \inf \{k \ge T_j^n \mid x_k = j.\]
That is the next returning time after \(T_j^n\). In the case that there is no confusion, 
we simply denote \(T^n = T^n_j\).

\begin{lemma}
  If \(j \in \mathcal{X}\) is recurrent, then \(\{T_j^{n + 1} - T_j^n, n \ge 0\}\) 
  are independent. Furthermore, \(T_j^{n + 1} - T_j^n\) are identically distributed 
  and 
  \[\mathbb{P}(T_j^{n + 1} - T_j^n = m) = \mathbb{P}_j(T_j = m)\]
  for all \(n, m = 1, 2, \cdots\).
\end{lemma}
\begin{proof}
  It is sufficient to show that for every \(n\), \(T^{n + 1} - T^n\) is independent 
  of \(\mathcal{F}_{T^n} = \sigma(\theta_{T^n} x.)\) (since \(T^{k + 1} - T^k\) 
  is \(\mathcal{F}_{T^n}\)-measurable for all \(k \le n\), so \(\sigma(T^{k + 1} - T^k) 
  \subseteq \mathcal{F}_{T^n}\)). Now, since \(j\) is recurrent, 
  \(\mathbb{P}_j(T_j < \infty) = 1\) and so, for any \(n \ge 1\), by the strong 
  Markov property,
  \[\begin{split}
    \mathbb{P}(T^{n + 1} - T^n = m \mid \mathcal{F}_{T^n})(\omega) & = 
    \mathbb{P}_{x_{T^n(\omega)}}(T = m) = \mathbb{P}_j(T = m).
  \end{split}\]
  Hence, taking expectation on both sides, we obtain
  \[\mathbb{P}(T_j^{n + 1} - T_j^n = m) = \mathbb{P}_j(T_j = m).\]
  Now, taking \(A \in \mathcal{F}_{T^n}\), we have 
  \[\mathbb{P}(A \cap \{T^{n + 1} - T^n = m\}) = \mathbb{E}(\mathbf{1}_A \mathbb{P}_j(T = m))
    = \mathbb{P}(A) \mathbb{P}(T^{n + 1} - T^n = m)\]
  and thus, is independent.
\end{proof}

\begin{lemma}
  Let \(i, j \in \mathcal{X}\) and \(k \ge 1\). Then 
  \[\mathbb{P}_i(T_j^{k + 1} < \infty) = \mathbb{P}_i(T_j < \infty) \mathbb{P}_j(T_j^k < \infty).\]
\end{lemma}
\begin{proof}
  Define 
  \[\Phi : \mathcal{X}^\infty \to \mathbb{R} : (a_n) \mapsto 
  \begin{cases}
    1, \ & a_n = j \text{ for some } n,\\
    0, \ & \text{ otherwise}.
  \end{cases}\]
  Then, \(T_j^{k + 1} < \infty\) if and only if \(\Phi(\theta_{T^k} x.) = 1\).
  Again, by the strong Markov property, we have 
  \[\begin{split}
    \mathbb{E}(\mathbf{1}_{\{T_j^{k + 1} < \infty\}}\mathbf{1}_{T_j < \infty} \mid \mathcal{F}_{T_j}) & = 
    \mathbb{E}(\Phi(\theta_{T^k} x.)\mathbf{1}_{T_j < \infty} \mid \mathcal{F}_{T_j})\\ 
    & = \mathbb{E}_j(\Phi(x.))\mathbf{1}_{T_j < \infty}
      = \mathbb{E}_j(\mathbf{1}_{T_j^{k} < \infty})\mathbf{1}_{T_j < \infty}.
  \end{split}\]
  Hence, the result follows by taking expectation on both sides.
\end{proof}

\begin{definition}
  Let \(\eta_j := \sum_{n = 1}^\infty \mathbf{1}_{s_n = j}\). This is known as the occupation 
  time of \(j\) and counts the number of visits to \(j\).
\end{definition}

We see that \(\mathbb{E}_j\eta_j = \sum_{n = 1}^\infty \mathbb{P}_j(x_n = j) = 
\sum_{n = 1}^\infty P_{jj}^n\). 

\begin{theorem}[Recurrence Criterion]
  A state \(j \in \mathcal{X}\) is transient if and only if \(\sum_{n = 1}^\infty P_{jj}^n < \infty\) 
  and recurrent if and only if \(\sum_{n = 1}^\infty P_{jj}^n = \infty\).
\end{theorem}
\begin{proof}
  We have \(\mathbb{E}_j\eta_j = \sum_{n = 1}^\infty P_{jj}^n\). On the other hand, 
  by the tail probability formula, 
  \[\mathbb{E}_j \eta_j = \sum_{n = 1}^\infty \mathbb{P}(\eta_j \ge n) = 
    \sum_{n = 1}^\infty \mathbb{P}_j(T^n_j < \infty) = \sum_{n = 1}^\infty(\mathbb{P}_j(T_j < \infty))^n\]
  where \(\mathbb{P}_j(T^n_j < \infty) = (\mathbb{P}_j(T_j < \infty))^n\) by 
  induction using the above lemma. As the right hand side is a geometric series, 
  the sum is convergent if and only if \(\mathbb{P}_j(T_j < \infty) < 1\), i.e. 
  \(j\) is transient. Hence, \(\sum_{n = 1}^\infty P_{jj}^n < \infty\) if and only 
  if \(j\) is transient as required.
\end{proof}

With the above criterion, we can show whether or not a state is recurrent or transient 
by considering the transition probabilities. As an example, consider again the symmetric 
walk on \(\mathbb{Z}\). For all \(i \in \mathbb{Z}\), we have 
\[P_{ii}^{2n} = {2n \choose n} \left(\frac{1}{2}\right)^{2n}\]
as a path starting from \(i\) and ending at \(i\) in \(2n\) steps results in \(n\) 
steps upwards and \(n\) steps down. On the other hand \(P_{ii}^{2n + 1} = 0\). 
Hence, we have 
\[\sum_{n = 1}^\infty P_{ii}^n = \sum_{n = 1}^\infty P_{ii}^{2n} = 
  \sum_{n = 1}^\infty {2n \choose n} \left(\frac{1}{2}\right)^{2n}.\]
By Stirling's formula, \(n! \sim (n / e)^n \sqrt{2\pi n}\), and so, 
\(\sum_{n = 1}^\infty {2n \choose n} \left(\frac{1}{2}\right)^{2n} \sim 
  c \sum_{n = 1}^\infty \frac{1}{\sqrt{n}} = \infty\)
for some constant \(c\). Thus, every state in the symmetric walk is recurrent. 

\begin{corollary}
  If \(j \in [i]\) where \(i, j \in \mathcal{X}\). Then both \(i, j\) are recurrent 
  or transient.
\end{corollary}
\begin{proof}
  By symmetry, it suffices to show that \(i\) is recurrent implies \(j\) is. 
  As \(i \sim j\), there exists some \(m_1, m_2\) such that \(P_{ij}^{m_1}, P_{ji}^{m_2} > 0\). 
  Then, 
  \[\sum_{n = 1}^\infty P_{jj}^n \ge P_{jj} + \cdots + P_{jj}^{m_1 + m_2} + 
    \sum_{n = 1}^\infty P_{ji}^{m_2} P_{ii}^n P_{ij}^{m_1} 
    \ge P_{ij}^{m_1} P_{ji}^{m_2} \sum_{n = 1}^\infty P_{ii}^n = \infty.\]
  Hence \(j\) is recurrent as required.
\end{proof}

\begin{corollary}
  Let \(j \in \mathcal{X}\). Then \(j\) is recurrent if and only if 
  \(\mathbb{P}_j(\{x_n = j \text{ i.o.}\}) = 1\), and if transient if and only 
  if \(\mathbb{P}_j(\{x_n = j \text{ i.o.}\}) = 0\).
\end{corollary}
\begin{proof}
  We see that \(\{x_n = j \text{ i.o.}\} = \{\eta_j = \infty\}\) and 
  \(\{\eta_j > m\} = \{T_j^{m + 1} < \infty\}\). Then, 
  \[\mathbb{P}_j(\eta_j > m) = \mathbb{P}_j(T_j^{m + 1} < \infty) = 
    (P_j(T_j < \infty))^{m + 1}.\]
  Then, by continuity from above, 
  \[\begin{split}
    \mathbb{P}(\{x_n = j \text{ i.o.}\}) & = 
    \mathbb{P}(\eta_j = \infty) = \mathbb{P}\left(\bigcup_{m = 1}^\infty\{\eta_j > m\}\right)\\
    & =
    \lim_{m \to \infty} \mathbb{P}(\eta_j > m) = \lim_{m \to \infty}(P_j(T_j < \infty))^{m + 1}.
    \end{split}\]
  So, \(j\) is recurrent (\(P_j(T_j < \infty) = 1\)) if and only if 
  \(\mathbb{P}(\{x_n = j \text{ i.o.}\}) = 1\) and transient if and only if 
  \(\mathbb{P}(\{x_n = j \text{ i.o.}\}) = 0\) as required.
\end{proof}

\begin{lemma}
  Let \(\mathcal{X}\) be finite. A state is recurrent if and only if it is in a closed 
  class.
\end{lemma}
\begin{proof}
  The reverse direction is left as an exercise. Suppose \([i]\) is not closed, 
  then there exists some \(j \in [i]\) and \(k \not\in [i]\) such that \(P_{jk} > 0\).
  Then, by definition, a path starting from \(j\) arriving at \(k\) cannot return 
  to \([i]\) and thus, \(\mathbb{P}_j(T_j < \infty) < 1\) implying \(j\) is transient.
\end{proof}

\subsection{Existence and Uniqueness of Invariant Measures}

\begin{lemma}
  \[\sum_{n = 1}^\infty P^n_{ij} = \frac{\mathbb{P}_i(T_j < \infty)}{1 - \mathbb{P}_j(T_j < \infty)}.\]
\end{lemma}
\begin{proof}
  We may assume \(i \sim j\) (since otherwise, both sides are 0). Denote 
  \(\eta_j = \sum_{n = 1}^\infty \mathbf{1}_{x_n = j}\). Then (by the tail probability)
  \[\begin{split}
    \sum_{n = 1}^\infty P^n_{ij} & = \mathbb{E}_i(\eta_j) = \sum_{n = 1}^\infty \mathbb{P}_i(\eta_j \ge n)
      = \sum \mathbb{P}_i(T_j^k < \infty) \\
    & = \mathbb{P}_i(T_j < \infty) \sum_{n = 1}^\infty (\mathbb{P}_j(T_j < \infty))^{k - 1}
    = \frac{\mathbb{P}_i(T_j < \infty)}{1 - \mathbb{P}_j(T_j < \infty)}
  \end{split}\]
  where the last equality holds for both cases where \(\mathbb{P}_j(T_j < \infty) = 1\) 
  or \(\mathbb{P}_j(T_j < \infty) < 1\).
\end{proof}

\begin{proposition}
  If a state \(j \in \mathcal{X}\) is transient, then \(\sum_{n = 1}^\infty P^n_{ij} < \infty\)
  and so, \(\lim_{n \to \infty} P_{ij}^n = 0\).
\end{proposition}
\begin{proof}
  From the above lemma, we have 
  \[\sum_{n = 1}^\infty P^n_{ij} = \frac{\mathbb{P}_i(T_j < \infty)}{1 - \mathbb{P}_j(T_j < \infty)}\]
  where the right hand side is finite for \(\mathbb{P}_j(T_j < \infty) < 1\).
\end{proof}

Recalling that an invariant measure is invariant with respect to the transition operator, 
we have the following theorem.

\begin{theorem}
  If \(\pi\) is an invariant probability measure (i.e. \(\pi(A) = \int P(x, A) \pi(\dd x)\)) 
  and \(\pi(j) > 0\) for some \(j \in \mathcal{X}\), 
  then \(j\) is recurrent.
\end{theorem}
\begin{proof}
  Since \(\pi\) is invariant, \(\pi(j) = \int P^n_{ij} \pi(\dd i) = 
  \sum_{i \in \mathcal{X}} P^n_{ij}\pi(i)\) for any \(n\). Then, we have 
  \[\begin{split}
    \infty & = \sum_{n = 1}^\infty \pi(j) = \sum_{n = 1}^\infty \sum_{i \in \mathcal{X}} P^n_{ij}\pi(i) 
      = \sum_{i \in \mathcal{X}} \pi(i) \sum_{n = 1}^\infty P^n_{ij}\\
      & = \sum_{i \in \mathcal{X}} \pi(i) \frac{\mathbb{P}_i(T_j < \infty)}{\mathbb{P_j}(T_j = \infty)}
      = \frac{\mathbb{P}_{\pi}(T_j < \infty)}{\mathbb{P}_j(T_j = \infty)}
  \end{split}\]
  where numerator of the last fraction is positive as \(\pi(j) > 0\) and \(\pi\) 
  is invariant. Thus, in order for the equation to hold, \(\mathbb{P}_j(T_j = \infty) = 0\) 
  and thus, \(j\) is recurrent.
\end{proof}

\begin{corollary}
  A transient Markov chain (every state is transient) does not have an invariant probability 
  measure.
\end{corollary}
\begin{proof}
  If \(\pi\) is invariant, then \(\pi(j) = 0\) for all \(j \in \mathcal{X}\) since the process 
  is transient. Thus, \(\pi(\mathcal{X}) = \sum_{i \in \mathcal{X}} \pi(i) = 0\) implying \(\pi\) is 
  not a probability measure.
\end{proof}

\begin{definition}[Positive Recurrent]
  A state \(i \in \mathcal{X}\) is said to be positive recurrent if \(\mathbb{E}_i T_i < \infty\).
  On the other hand, we say \(i\) is null-recurrent if \(\mathbb{E}_i T_i = \infty\).

  One may show positive recurrence and null-recurrence are class properties.
\end{definition}

Let us now attempt to construct invariant measures from recurrent states. 
Let \(i \in \mathcal{X}\) be recurrent. Then, for all \(j \in \mathcal{X}\), let 
us define 
\[\mu(j) = \mathbb{E}_i\left(\sum_{n = 0}^{T_i - 1} \mathbf{1}_{x_n = j}\right).\]
This is the expected number of visits to \(j\) starting from \(i\) till the 
first time the chain returns to \(i\). We note that \(\mu(i) = 1\) since 
\(x_n \neq i\) for all \(n < T_i\) and \(x_{T_i} = i\) be definition.
Furthermore, as \(i\) is recurrent, \(\mathbb{P}_i(T_i < \infty) = 1\),
\[\mu(j) = \mathbb{E}_i\left(\sum_{n = 0}^\infty \mathbf{1}_{n < T_i} \mathbf{1}_{x_n = j}\right)
  = \sum_n \mathbb{P}_i(x_n = j, n < T_i).\]
Then, summing over all states, we have 
\[\sum_{j \in \mathcal{X}} \mu(j) = \sum_{j \in \mathcal{X}} \sum_{n = 0}^\infty \mathbb{P}_i(x_n = j, T_i > n) = 
  \sum_{n = 0}^\infty \mathbb{P}(T_j > n) = \mathbb{E}_i T_i.\]

\begin{theorem}
  Let \(i \in \mathcal{X}\) be recurrent. Then, the measure \(\mu\) defined by  
  \[\mu(j) := \sum_{n = 0}^\infty \mathbb{P}_i(x_n = j, T_i > n)\]
  is invariant.
\end{theorem}
\begin{proof}
  Suppose first \(j \neq i\). Then, \(\mu(j) = \sum_{n = 1}^\infty \mathbb{P}_i(x_n = j, T_i > n)\)
  and so, 
  \[\begin{split}
    \mu(j) & = \sum_{k \in \mathcal{X}}\sum_{n = 1}^\infty \mathbb{P}_i(x_n = j, T_i > n, x_{n - 1} = k, T_i > n - 1)\\
    & = \sum_{k \in \mathcal{X}}\sum_{n = 1}^\infty \mathbb{P}_i(x_n = j, T_i > n | x_{n - 1} = k, T_i > n - 1)\mathbb{P}_i(x_{n - 1} = k, T_i > n - 1)\\
    & = \sum_{k \in \mathcal{X}}\mu(k)\sum_{n = 1}^\infty \mathbb{P}_i(x_n = j | x_{n - 1} = k, T_i > n - 1)\\
    & = \sum_{k \in \mathcal{X}}\mu(k)P_{kj} = T\mu(j)
  \end{split}\]
  where \(T\) is the transition operator. Now considering the \(i\) case. 
  Consider 
  \[\begin{split}
    \mathbb{P}_i(T_i = n + 1) & = \sum_{k \neq i}\mathbb{P}_i(T_i > n, x_{n + 1} = i, x_n = k)\\
    & = \sum_{k \neq i} \mathbb{P}_i(x_{n + 1} \mid x_n = k, T_i > n) \mathbb{P}_i(T_i > n, x_n = k)\\
    & = \sum_{k \neq i} P_{ki} \mathbb{P}_i(T_i > n, x_n = k) = \sum_{k \in \mathcal{X}} P_{ki}\mathbb{P}_i(T_i > n, x_n = k).
  \end{split}\]
  Thus, 
  \[T\mu(i) = \sum_{k \in \mathcal{X}} \sum_{n = 0}^\infty \mathbb{P}_i(T_i > n, x_n = k) P_{ki} 
    = \sum_{n = 0}^\infty) \mathbb{P}(T_i = n + 1) = \mathbb{P}(T_i < \infty) = 1 = \mu(i).\]
  Hence, \(T\mu = \mu\) as required.
\end{proof}

We now ask whether or not the invariant measure is unique. Obviously, a scalar multiple 
of a invariant measure if invariant, so, we ask for uniqueness up to an scalar multiple.

\begin{lemma}
  Let \(i \in \mathcal{X}\) be recurrent and let \(\mu\) be the invariant measure 
  constructed from \(i\). Then, if \(\nu\) is an invariant measure, then 
  \[\nu(k) \ge \mu(k) \nu(i)\]
  for all \(k \in \mathcal{X}\). Note that since \(\mu(i) = 1\), we may reformulate 
  this as 
  \[\frac{\nu(k)}{\nu(i)} = \frac{\mu(k)}{\mu(i)}\]
\end{lemma}
\begin{proof}
  Clearly, equality holds for \(k = i\) and so assume \(k \neq i\). Let \(L^n\) be 
  the last visit the chain visits \(i\) before \(n\). We note that 
  \[\Omega = A_0 \cup \bigcup_{m = 0}^{n - 1} \{L^n = m\}\]
  where \(A_0\) is the set of \(\omega \in \Omega\) for which \((x_n(\omega))\) 
  does not visit \(i\) before \(n\). Then, 
  \[\begin{split}
    P^n_{jk} & = \mathbb{P}_j(x_n = k) \ge \sum_{m = 0}^{n - 1}\mathbb{P}_j(x_n = k, L^n = m)\\
    & = \sum_{m = 0}^{n - 1} \mathbb{P}(x_n = k, x_{n - 1} \neq i, \cdots, x_{m + 1} \neq i, x_m = i)\\
    & = \sum_{m = 0}^{n - 1} \mathbb{P}(x_n = k, x_{n - 1} \neq i, \cdots, x_{m + 1} \neq i | x_m = i) P_{ji}^m\\
    & = \sum_{m = 0}^{n - 1} \mathbb{P}(x_{n - m} = k, x_{n - m - 1} \neq i, \cdots, x_1 \neq i | x_m = i) P_{ji}^m\\
    & = \sum_{m = 0}^{n - 1} \mathbb{P}_i(x_{n - m} = k, T_i > n - m) P_{ji}^m
  \end{split}\]
  where the last equality holds as \(x_{n - m} = k \neq i\). Thus, 
  \[\begin{split}
    \nu(k) & = T^n\nu(k) = \sum_{j \in \mathcal{X}} \nu(j) P_{jk}^m
    \ge \sum_{j \in \mathcal{X}} \nu(j) \sum_{m = 0}^{n - 1} \mathbb{P}_j(x_{n - m} = k, T_i > n - m) P_{ji}^m\\
    & = \sum_{m = 0}^{n - 1} \mathbb{P}_j(x_{n - m} = k, T_i > n - m) \sum_{j \in \mathcal{X}} \nu(j) P_{ji}^m
     = \sum_{m = 0}^{n - 1}\mathbb{P}_i(x_{n - m} = k, T_i > n + m)\nu(i).
  \end{split}\]
  Hence, change of index and taking the sum to \(\infty\), we have the inequality 
  as required.
\end{proof}

\begin{theorem}
  A irreducible, recurrent Markov process has invariant measure unique up to a scalar multiple.
\end{theorem}
\begin{proof}
  If \(\nu\) be any invariant measure and let \(\mu\) be the invariant measure 
  constructed from \(i\). Then, as \(\mu(i) = 1\),
  \[\begin{split}
    0 & = \nu(i) - \nu(i)\mu(i) = T^n \nu(i) - \nu(i)T^n\mu(i)
       = \sum_{k \in \mathcal{X}} (\nu(k) - \nu(i) \mu(k)) P^n_{ki}
  \end{split}\]
  where \(\nu(k) - \nu(i) \mu(k) \ge 0\) for all \(k\) by the above lemma.
  Thus, \((\nu(k) - \nu(i) \mu(k)) P^n_{ki} = 0\) for all \(n, k\). Then, 
  as the process is irreducible, for any \(k, i\), \(k \sim i\) and so, there 
  exists some \(n\) such that \(P^n_{ki} > 0\). Thus, \(\nu(k) = \nu(i)\mu(k)\) 
  for all \(k \in \mathcal{X}\) implying \(\nu = \nu(i)\mu\) as required. 
\end{proof}

\begin{theorem}
  If a Markov process is irreducible and has an invariant probability measure \(\pi\). 
  Then, \(\mathbb{E}_iT_i < \infty\) (positively recurrent) for all \(i \in \mathcal{X}\) and 
  \[\pi(i) = \frac{1}{\mathbb{E}_i T_i} (> 0).\]
\end{theorem}
\begin{proof}
  Since \(\pi\) is a probability measure, there exists some \(i \in \mathcal{X}\) 
  such that \(\pi(i) > 0\). Thus, \(i\) is recurrent and thus, the chain is recurrent 
  by irreducibility. Thus, taking \(i\) arbitrarily, by uniqueness, 
  \(\pi = \pi(i) \mu\) where \(\mu\) is 
  the invariant measure associated with \(i\), namely, 
  \[\mu(j) = \sum_{n = 0}^\infty \mathbb{P}_i(x_n = j, T_i > n).\]
  Hence, since \(\sum_{j \in \mathcal{X}} \mu(j) = \mathbb{E}_i T_i\), we have 
  \[\pi(i) = \frac{\mu(i)}{\mathbb{E}_i T_i} = \frac{1}{\mathbb{E}_i T_i}\]
  as required.
\end{proof}

\begin{corollary}
  An irreducible Markov process has an invariant probability measure if and only if 
  it is \textbf{positively recurrent}.
\end{corollary}

\subsection{Long Run Probabilities}

We have seen that if \(\lim_{n \to \infty} P^n_{ij}\) exists for any \(j\) and 
we define \(\nu(j) := \lim_{n \to \infty} P^n_{ij}\) then, \(\nu\) is an invariant 
measure (Why?). We now ask whether or not the converse is true. 

Suppose 
the chain is irreducible and \(\pi\) is invariant, then, for all transient states 
\(j \in \mathcal{X}\), \(\lim_{n \to \infty} P^n_{ij} = 0\) and \(\pi(j) = 0\) 
(as invariant measures do not charge transient states). Hence, 
\(\lim_{n \to \infty} P^n_{ij} = \pi(j)\) for transient states.
On the other hand, for situation is more difficult for recurrent states. 
Indeed, if we take the transition probability 
\[P = \begin{pmatrix}
  0 & 1 \\ 1 & 0
\end{pmatrix},\]
then, \(P^{2n + 1} = P\) and \(P^{2n} = \text{Id}\). Thus, 
\(\lim_{n \to \infty} P^n_{ij}\) does not exist. On the other hand, this is an 
irreducible recurrent chain, and thus, processes an invariant measure.
This motivates the study of periodic and non-periodic chains.

Denote \(R(i) := \{n > 0 \mid P^n_{ii} > 0\}\) such that \(n \in R(i)\) if 
there exists a path from \(i\) to \(i\) of length \(n\). Clearly, 
if \(n, m \in R(i)\), then \(n + m \in R(i)\) (C-K) and if \(R(i) \neq \varnothing\), 
then \(|R(i)| = \infty\) (since if \(n \in R(i)\) then \(kn \in R(i)\) by the 
previous statement for all \(k \in \mathbb{N}^*\).

\begin{definition}[Period]
  Given a state \(i \in \mathcal{X}\), we define its period to be 
  \[d(i) := \begin{cases}
    \gcd(R(i)), & \ R(i) \neq \varnothing,\\
    \infty, & \ \text{ otherwise}.
  \end{cases}\]
\end{definition}

\begin{definition}
  A state \(i \in \mathcal{X}\) is aperiodic if \(d(i) = 1\) and periodic if 
  \(d(i) > 1\).
\end{definition}

\begin{proposition}
  Given \(i, j \in \mathcal{X}\) such that \(i \sim j\). Then \(d(i) = d(j)\).
  Namely, the period is a class property.
\end{proposition}
\begin{proof}
  Since \(i \sim j\), there exists some \(n, m\) such that \(P^n_{ij}, P^m_{ji} > 0\).
  Then, by C-K, \(P^{n + m}_{ii} \ge P^{n}_{ij} + P^{m}_{ji} > 0\) and 
  similarly for \(P^{n + m}_{jj} > 0\), and hence, \(n + m \in R(i) \cap R(j)\).
  Now, taking \(k \in R(i)\), we have \(k + n + m \in R(j)\) and so, 
  \(d(j) \mid k + n + m\) and \(d(j) \mid n + m\) implying \(d(j) \mid k\), 
  and thus, \(d(j) \le d(i)\). By symmetry, \(d(i) \le d(j)\) and thus, 
  \(d(i) = d(j)\) as required.
\end{proof}

\begin{definition}
  A THMC (or \(P\)) is aperiodic if \(d(i) = 1\) for all \(i \in \mathcal{X}\) and 
  is periodic with period \(d\) if \(d(i) = d > 1\) for all \(i \in \mathcal{X}\).
\end{definition}

From the definition, we have the following proposition.

\begin{corollary}
  An irreducible chain is either aperiodic or periodic.
\end{corollary}

Let \((x_n), (x_n')\) be independent time homogeneous Markov processes
on \(\mathcal{X} = \mathbb{N}\) with the same transition probability \(P\). 
Furthermore, assume \((x_n), (x'_n)\) has initial distribution \(\mu\) and \(\nu\) 
and write 
\[P^n_\mu := \mathcal{L}(x_n), P^n_\nu := \mathcal{L}(x'_n).\]
\begin{lemma}[Doeblin Coupling]
  \(z_n := (x_n, x'_n)\) is a THMC on \(\mathcal{X}^2\) 
  with the initial distribution \(\mu \otimes \nu\) and transition probability 
  \(Q\), where 
  \[Q_{(ii'), (jj')} := P_{ij}P_{i'j'}\]
  for all \(i, i', j, j' \in \mathcal{X}\).
\end{lemma}

Denoting \(T := \inf_{n \ge 0} \{x_n = x'_n\}\) (the coalessing time), we have 
the following lemmas.

\begin{lemma}[Successful Coupling]
  \(\mathbb{P}(T < \infty) = 1\).
\end{lemma}

\begin{lemma}[Coupling Inequality]
  \(\sum_{i \in \mathcal{X}} |\mathbb{P}(x_n = i) - \mathbb{P}(x'_n = i)| \le 
  2\mathbb{P}(T > n)\). Thus, by the above lemma, the right hand side tends to 
  0 as \(n \to \infty\).
\end{lemma}

\begin{theorem}
  Let \(P\) be irreducible, aperiodic and positively recurrent (recall that this 
  means \(\mathbb{E}_i T_i < \infty\) for all \(i \in \mathcal{X}\)), and 
  let \(\pi\) be its unique invariant probability measure. Then, for all 
  \(i \in \mathcal{X}\),
  \[\lim_{n \to \infty} \sum_{j \in \mathcal{X}} |P^n_{ij} - \pi(j)| = 0\]
\end{theorem}
\begin{proof}
  With the above constructions in mind, this theorem follows by taking 
  \(x_0 := i, x'_0 := \pi\) and \(\mathbb{P}(x_n = j) = P^n_{ij}\), 
  \(\mathbb{P}(x'_n = j) = \pi(j)\).
\end{proof}

In order to prove the above lemmas, we need more machinery, namely coupling.

\begin{definition}[Coupling]
  Given two random variables \(x, y : \Omega \to \mathcal{X}\), the coupling 
  of the two is a random variable \(z = (x', y')\) with state space \(\mathcal{X}^2\) 
  such that 
  \[\mathcal{L}(x) = \mathcal{L}(x') \text{ and } \mathcal{L}(y) = \mathcal{L}(y').\]
\end{definition}

\begin{definition}[Doeblin Coupling]
  Let\((x_n), (x_n')\) be independent time homogeneous Markov processes
  on \(\mathcal{X} = \mathbb{N}\) with the same transition probability \(P\). 
  Furthermore, assume \((x_n), (x'_n)\) has initial distribution \(\mu\) and \(\nu\).
  Then, the Doeblin coupling is the stochastic process defined by 
  \[z_n := (x_n, x'_n).\]
\end{definition}

\begin{lemma}[Coupling lemma]
  Let \((x_n), (x'_n)\) be independent Markov processes with the same 
  transition probability \(P\). Then, defining 
  \[y_n := \begin{cases}
    x_n, & \ n < T,\\
    x'_n, & \ n \ge T,
  \end{cases}\]
  where \(T\) is the coalessing time, \((y_n)\) is a Markov process with 
  the transition probability \(P\) and initial distribution \(\mathcal{L}(x_0)\).
\end{lemma}
\begin{proof}
  Let \(f \in \mathcal{B}_b(\mathcal{X})\) and let 
  \[\mathcal{F}_n = \sigma(x_k \mid k \le n) \vee \sigma(x'_k \mid k \le n).\]
  Then, \((x_n), (x'_n)\) remains to be Markov processes with respect to the 
  filtration \((\mathcal{F}_n)\) (by independence). Then, 
  \[\begin{split}
    \mathbb{E}(f(y_{n + 1}) \mid \mathcal{F}_n) & = 
    \mathbb{E}(f(y_{n + 1})\mathbf{1}_{T \le n} \mid \mathcal{F}_n) + 
    \mathbb{E}(f(y_{n + 1})\mathbf{1}_{T > n} \mid \mathcal{F}_n)\\
    & = 
    \mathbf{1}_{T \le n}\mathbb{E}(f(x'_{n + 1} \mid \mathcal{F}_n)) + 
    \mathbf{1}_{T > n}\mathbb{E}(f(x_{n + 1} \mid \mathcal{F}_n))\\
    & = \mathbf{1}_{T \le n} Tf(x'_n) + \mathbf{1}_{T > n} Tf(x_n)\\
    & = \mathbf{1}_{T \le n} Tf(y_n) + \mathbf{1}_{T > n} Tf(y_n) 
    = Tf(y)
  \end{split}\]
  where \(T\) is the transition operator of the transition probability 
  \(P\). This implies \((y_n)\) is a Markov 
  process with respect to \((\mathcal{F}_n)\) (and hence to its natural 
  filtration) with the same transition probability \(P\).
\end{proof}

Let us now prove the previously claimed lemmas. 

\begin{proof}[Coupling Inequality]
  We are asked to prove 
  \[\sum_{j} |\mathbb{P}(x_n = j) - \mathbb{P}(x'_n = j)| \le 2\mathbb{P}(T > n).\]
  Defining \((y_n)\) as above, as \((x_n)\) and \((y_n)\) have the same 
  distribution, 
  \[\begin{split}
    |\mathbb{P}(x_n = j) - \mathbb{P}(x'_n = j)| & = 
    |\mathbb{P}(y_n = j) - \mathbb{P}(x'_n = j)|\\ 
    & = |\mathbb{P}(y_n = j) - \mathbb{P}(x'_n = j, n < T) - 
      \mathbb{P}(y_n = j, n \ge T)|\\
    & = |\mathbb{P}(y_n = j, n < T) - \mathbb{P}(x'_n = j, n < T)|.
  \end{split}\]
  Hence, summing over \(j\) we have 
  \[\sum_j|\mathbb{P}(x_n = j) - \mathbb{P}(x'_n = j)| \le 
    \sum_j \mathbb{P}(y_n = j, n < T) + \sum_j \mathbb{P}(x'_n = j, n < T) 
    \le 2 \mathbb{P}(n < T).\]
\end{proof}

\begin{proof}[Doeblin Coupling]
  The Doeblin Coupling asks us to prove the coupling \(z_n := (x_n, x'_n)\) 
  is a THMC on \(\mathcal{X}^2\) with initial distribution 
  \(\mathcal{L}(x_0) \otimes \mathcal{L}(x'_0)\) and transition probability \(Q\) 
  where 
  \[Q_{(ii'),(jj')} = P_{ij} P_{i'j'}.\]
  The initial distribution follows from independence. On the other hand, 
  \[\begin{split}
    \mathbb{P}(z_{n + 1} = (j, j') \mid \mathcal{F}_n) & = 
    \mathbb{P}(x_{n + 1} = j, x'_{n + 1} = j' \mid \mathcal{F})\\
    & = \mathbb{P}(x_{n + 1} = j \mid x_n)\mathbb{P}(x'_{n + 1} = j' \mid x'_n)\\
    & = P(x_n, j) P(x'_n, j') = Q_{(x_n x'_n), (jj')}
  \end{split}\]
  where the second equality holds by independence.
\end{proof}

Before we can prove \(T < \infty\) almost everywhere, let us construct a larger 
stopping time for which we know it is finite almost everywhere, hence proving the 
claim.

\begin{lemma}
  If \(i \in \mathcal{X}\) is aperiodic and recurrent, then there exists some  
  \(N\) such that \(P^n_{ii} > 0\) for all \(n \ge N\).
\end{lemma}
\begin{proof}
  Since \(i\) is recurrent, \(\sum_{n = 1}^\infty P^n_{ii} = \infty\) and so, 
  \(R(i) \neq \varnothing\). Since, by aperiodicity, \(\gcd(R(i)) = 1\), there 
  exists a finite subset \(R_0 \subseteq R(i)\) such that \(\gcd(R_0) = 1\).
  Then, by the Chinese remainder theorem, there exists some \(N\) such that 
  for all \(n > N\), \(n \in R(i)\).
\end{proof}

\begin{corollary}
  If \(P\) is irreducible, aperiodic and recurrent, then, for all 
  \(i, j \in \mathcal{X}\), there exists some \(N\) such that for all 
  \(n \ge N\), \(P^n_{ij} > 0\).
\end{corollary}
\begin{proof}
  Since \(i \to j\) as \(P\) is irreducible, there exists some \(m\) such that 
  \(P^m_{ij} > 0\). Then, by the above lemma, there exists some \(N\) such that 
  for all \(n \ge N\), \(P^n_{ii} > 0\). So, for all \(k \ge m + N\), 
  \[P^k_{ij} = P^{(k - m) + m}_{ij} \ge P^(k - m)_{ii} P^m_{ij} > 0\]
  since \(k - m \ge N\). 
\end{proof}

\begin{lemma}
  Let \(P\) be irreducible, aperiodic and positively recurrent. Then, \(Q\) 
  (as defined above from Doeblin coupling) is also irreducible and positively 
  recurrent.
\end{lemma}
\begin{proof}
  Firstly, \(\pi \otimes \pi\) is an invariant probability measure for \(Q\) 
  if \(\pi\) is invariant for \(P\). Thus, it suffices to show \(Q\) is irreducible. 

  By induction, we can show that \(Q^n_{(ii'),(jj')} = P^n_{ij} P^n_{i'j'}\). 
  On the other hand, as \(P\) is irreducible, for any \(i, i', j, j'\) there exists 
  some \(n, n'\) such that \(P^n_{ij}, P^{n'}_{i'j'} > 0\). By the above corollary,
  we may choose sufficiently large \(n\) such that \(P^n_{ij}, P^n_{i'j'} > 0\)
  and so, \(Q^n_{(ii'),(jj')} = P^n_{ij} P^n_{i'j'} > 0\).
\end{proof}

Finally, we can prove the successful coupling lemma.

\begin{proof}[Successful Coupling]
  We are required to prove \(\mathbb{P}(T < \infty) = 1\). To achieve this, 
  define 
  \[T_{(i,i)} := \inf \{z_n = (i, i)\} = \inf \{x_n = x'_n = i\}.\]
  It is not difficult to see that \(T \le T_{(i, i)}\) and so, it suffices to 
  show \(\mathbb{P}(T_{(i, i)} < \infty) = 1\). Now, since \(Q\) is irreducible 
  and recurrent, \(\mathbb{P}_z(T_{(i, i)} < \infty) = 1\) for any \(z \in \mathcal{X}^2\) 
  and so, 
  \[\mathbb{P}(T < \infty) \ge \mathbb{P}(T_{(i, i)} < \infty) = 1.\]
\end{proof}

Before moving on, let us first consider the following well-known lemma from 
elementary number theory.

\begin{lemma}
  Let \(S \subseteq \mathbb{N}\) be a non-empty additive set and let \(d = \gcd(S)\).
  Then, there exists some \(\kappa > 0\) such that \(kd \in S\) for all \(k > \kappa\).
\end{lemma}
\begin{proof}
  We may assume \(d = 1\) by dividing every element of \(S\) by \(d\). In this 
  case, by Bezout's lemma, there exists \(a_1, \cdots, a_n \in \mathbb{Z}\) 
  and \(d_1, \cdots, d_n \in S\) such that \(1 = \sum_{i = 1}^n a_i d_i\). 
  Then, defining \(M = \sum_{i = 1}^n d_i\), for all \(k < M\), we observe 
  that 
  \[NM + k = \sum_{i = 1}^n Nd_i + k \sum_{i = 1}^n a_i d_i = \sum_{i = 1}^n(N + ka_i)d_i.\]
  Hence, for \(N\) sufficiently large, i.e. \(N \ge N_0\) such that 
  \(N_0 + ka_i \ge 0\) for all \(i\) and \(0 \le k < M\), we have 
  \(N + ka_i \ge 0\) implying \(NM + k \in S\) as \(S\) is additive. 
  Thus, as all natural numbers larger than \(N_0M\) can be written in the form 
  \(NM + k\) where \(N \ge N_0\) and \(0 \le k < M\), the claim follows.
\end{proof}

\subsection{Total Variation Distance}

Recall that in measure theory, we had considered the total variation of signed 
and complex measures. With regards to probability measures directly, the notion 
of total variation is less useful as the total variation of a probability measure 
is always 1. Instead we consider the total variation distance between two 
probability measures.

\begin{definition}[Total Variation Distance]
  The total variation distance between two probability measure \(\mu, \nu\) is 
  \[\|\mu - \nu\|_{TV} := 2\sup_{A \in \mathcal{F}} |\mu(A) - \nu(A)|.\]
  We observe that the total variation distance is simply the total variation of 
  the signed measure obtained from \(\mu - \nu\) (multiplied by 2 by convention).
\end{definition}

It is easy to check that 
\begin{itemize}
  \item \(\|\mu - \nu\|_{TV} = 0 \iff \mu = \nu\);
  \item \(\|\mu - \nu\|_{TV} \le 2\);
  \item \(\|\mu - \nu\|_{TV} = 2\) if \(\mu \perp \nu\). Indeed, if \(\mu \perp \nu\), 
    then there exists \(A, B\) such that \(A \cup B = \Omega\) and 
    \(\mu(B) = \nu(A) = 0\). So, \(\mu(A) - \nu(A) = \mu(A) + \mu(B) \ge 
      \mu(A \cup B) = \mu(\Omega) = 1\).
\end{itemize}

\begin{lemma}
  If \(\mathcal{X}\) is countable and \(\mu, \nu\) are probability measures on 
  \(\mathcal{X}\), then 
  \[\|\mu - \nu\|_{TV} = \sum_{j \in \mathcal{X}} |\mu(j) - \nu(j)|.\]
  In particular, if you consider \(\mu, \nu\) as functions on \(\mathcal{X}\), 
  the right hand side is simply the \(L^1\) norm of \(\mu - \nu\).
\end{lemma}
\begin{proof}
  Define \(B = \{j \mid \mu(j) \ge \nu(j)\}\). Then, as \(\mu, \nu\) are probability 
  measures, \(\mu(B) - \nu(B) = \nu(B^c) - \mu(B^c)\). Thus, 
  \[\begin{split}
    \sum_{j \in \mathcal{X}} |\mu(j) - \nu(j)| & = 
    \sum_{j \in B} (\mu(j) - \nu(j)) + \sum_{j \in B^c} (\nu(j) - \mu(j))\\
    & = \mu(B) - \nu(B) + \nu(B^c) - \mu(B^c) = 2(\mu(B) - \nu(B))\\ 
    & \le \|\mu - \nu\|_{TV}.
  \end{split}\]
  On the other hand, for all measurable set \(A \subseteq \mathcal{X}\),
  \[\begin{split}
    |\mu(A) - \nu(A)| & \le
    |(\mu - \nu)(A \cap B) + (\mu - \nu)(A \cap B^c)|\\
    & \le (\mu - \nu)(A \cap B) \vee (\mu - \nu)(A \cap B^c)\\
    & \le (\mu - \nu)(B) \vee (\mu - \nu)(B^c) = \mu(B) - \nu(B)\\
    & = \frac{1}{2}\sum_{j \in \mathcal{X}} |\mu(j) - \nu(j)|.
  \end{split}\]
  Hence, the required inequality follows by multiplying both sides by 2.
\end{proof}

\begin{proposition}
  If \(\mu = f \lambda\) and \(\nu = g\lambda\). Then, 
  \[\|\mu - \nu\|_{TV} = \int |f - g| \dd \lambda = \|f - g\|_{L^1}.\]
\end{proposition}
\begin{proof}
  Exercise. By the same idea as above.
\end{proof}

\begin{definition}
  A sequence of measures \((\mu_n)\) is said to converge in total variation 
  to the measure \(\mu\) if 
  \[\lim_{n \to \infty}\|\mu_n - \mu\|_{TV} = 0.\]
\end{definition}

We note that convergence in total variation is different to convergence weakly. 
Indeed, consider the sequence of Dirac measures \((\delta_{1 / n})\). 
For all bounded continuous functions \(f\), we have 
\[\int f \dd \delta_{1 / n} = f(1 / n) \to f(0) = \int f \dd \delta_0.\]
Thus \(\delta_{1 / n} \to \delta_0\) weakly. On the other hand, 
for all \(n\), we have 
\[\|\delta_{1 / n} - \delta_0\|_{TV} = 2\]
and so, \(\delta_{1 / n}\) does not converge in total variation to \(\delta_0\).

\begin{proposition}
  Given probability measures \(\mu, \nu\), 
  \[\|\mu - \nu\|_{TV} = 
  \sup_{\substack{f \in \mathcal{B}_b(\mathcal{X})\\\|f\|_\infty = 1}}
  \left|\int f\dd\mu - \int f \dd \nu\right|.\]
\end{proposition}
\begin{proof}
  See Sheffe's lemma.
\end{proof}

\begin{corollary}
  Convergence in total variation implies convergence weakly.
\end{corollary}

Before moving further, let us remark that the space of all probability 
measures \(\mathbb{P}(\Omega)\) on \(\Omega\) forms a complete metric space 
with the distance \(\|\cdot\|_{TV}\). We will return to this later.

With the notion the total variation distance in mind, we may reformulate our 
convergence theorem regarding the transition probability. Namely, we have 
\[\sum_{j = 1}^\infty |P^n_{ij} - \pi(j)| = \|P^n(i, \cdot) - \pi\|_{TV}.\]
Recall that, if \(\mathcal{L}(x_0) = \mu\), then 
\[\mu P^n(j) := \mathbb{P}(x_n = j) = \sum_{i = 1}^\infty \mu(i)P^n_{ij}.\]

\begin{lemma}
  If \(\lim_{n \to \infty} \|P^n(i, \cdot) - \pi\|_{TV} = 0\) for all 
  \(i \in \mathcal{X}\), then \(\lim_{n \to \infty}\|\mu P^n - \pi\|_{TV} = 0\).
\end{lemma}\
\begin{proof}
  We observe 
  \[\|\mu P^n - \pi\|_{TV} = 
    \sum_{j = 1}^\infty \left|\sum_{i = 1}^\infty 
    \mu(i)P^n_{ij} - \sum_{i = 1}^\infty \mu(i)\pi(j)\right|.\]
  As \(\sum \mu(i) = 1\), for all \(\epsilon > 0\), there exists some 
  \(N\) such that \(\sum_{i = N + 1}^\infty \mu(i) < \epsilon / 4\). Then, 
  \[\begin{split}
    \sum_{j = 1}^\infty \left|\sum_{i = N + 1}^\infty 
    \mu(i)P^n_{ij} - \sum_{i = 1}^\infty \mu(i)\pi(j)\right|
    & \le \sum_{i = N + 1}^\infty \mu(i) \sum_{j = 1}^\infty |P^n_{ij} - \pi(j)|\\
    & \le \sum_{i = N + 1}^\infty \mu(i) \sum_{j = 1}^\infty (|P^n_{ij}| + |\pi(j)|)\\
    & < 2 \frac{\epsilon}{4} = \frac{\epsilon}{2}.
  \end{split}\]
  On the other hand, as \(\sum_{j = 1}^\infty |P^n_{ij} - \pi(j)| \to 0\) as 
  \(n \to \infty\), there exists some \(M\) such that for all \(n \ge M\),
  \(\sum_{j = 1}^\infty |P^n_{ij} - \pi(j)| < \epsilon / 2\) and so, for \(n \ge M\), 
  \[\sum_{j = 1}^\infty \left|\sum_{i = 1}^N 
  \mu(i)P^n_{ij} - \sum_{i = 1}^\infty \mu(i)\pi(j)\right| 
  \le \sum_{j = 1}^\infty \sum_{i = 1}^N \mu(i)|P^n_{ij} - \pi(j)| < \sum_{i = 1}^N \mu(i)\frac{\epsilon}{2} 
  \le \frac{\epsilon}{2}.\]
  Hence, adding the two sums together, we have \(\|\mu P^n - \pi\|_{TV} < \epsilon\) 
  for \(n \ge M\) implying \(\lim_{n \to \infty}\|\mu P^n - \pi\|_{TV} = 0\) as 
  required.
\end{proof}

\begin{theorem}
  If \((x_n)\) is an irreducible, aperiodic, positively recurrent THMC with 
  \(\mathcal{L}(x_0) = \mu\), then
  \[\lim_{n \to \infty}\|\mu P^n - \pi\|_{TV} = 0.\]
\end{theorem}
\begin{proof}
  Follows by the above lemma and the reformulation of the convergence theorem.
\end{proof}

We have so far considered aperiodic chains. Let us now consider what happens 
with periodic ones.

\begin{lemma}
  Let \((x_n)\) be irreducible and let \(d > 1\). Then \((x_n)\) has period 
  \(d\) if and only if \(\mathcal{X}\) is a disjoint union of sets 
  \(A_0, \cdots, A_{d - 1}\) such that if \(i \in A_n\) for some \(n\) and 
  \(j \in \mathcal{X}\) such that \(P_{ij} > 0\), then \(j \in A_{n + 1}\) 
  (we define \(A_d := A_0\)).
\end{lemma}
\begin{proof}
  Suppose first that \(P\) has period \(d\). Then, we define 
  \[A_n := \{j \in \mathcal{X} \mid 
    P_{1j}^{kd + n} > 0 \text{ for some } k = 0, 1, \cdots\}\]
  for \(n = 0, \cdots, d - 1\). Clearly, \(\bigcup_{n = 0}^{d - 1} A_n = \mathcal{X}\)
  since \(P\) is irreducible and every natural number can be written as 
  \(kd + n\) for some \(k \in \mathbb{N}\), \(0, \le n \le d - 1\). 
  
  Furthermore, \((A_n)_{n = 0}^{d - 1}\) is pair-wise disjoint since if 
  \(j \in A_{n_1} \cap A_{n_2}\), then \(P_{1j}^{n_1 + k_1d}, P_{1j}^{n_2 + k_2d} > 0\)
  for some \(k_1, k_2\). By irreducibility, there exists some \(m\) such that 
  \(P_{j1}^m > 0\) and so, \(P_{11}^{n_1 + k_1d + m}, P_{11}^{n_2 + k_2d + m} > 0\), 
  namely, \(n_1 + k_1d + m, n_2 + k_2d + m \in R(i)\). Thus, \(d \mid |n_1 - n_2|\) 
  implying \(n_1 = n_2\) as \(0 \le n_1, n_2 \le d - 1\).

  On the other hand, if such a sequence of sets \((A_n)_{n = 0}^{d - 1}\) exists. 
  By C-K, \(P_{ii}^m = 0\) for all \(d \not \mid m\). Since the chain is irreducible, 
  it follows \(P^{kd}_{ii} > 0\) for some \(k\), and hence, \(d = d(i)\) as 
  required.
\end{proof}

\begin{proposition}
  Let \(T\) denote the transition operator. If for some \(n\), \(T^n \mu = \mu\), 
  then 
  \[\hat \mu = \frac{1}{n}\sum_{k = 1}^n T^k \mu\]
  is an invariant measure, i.e. \(T\hat\mu = \hat\mu\).
\end{proposition}
\begin{proof}
  Follows straight away by computation,
  \[\begin{split}
    T \hat\mu & = \frac{1}{n}\sum_{k = 1}^n T^{k + 1}\mu 
      = \frac{1}{n} \left(\sum_{k = 2}^n T^k \mu + T^{n + 1}\mu\right)\\
    & = \frac{1}{n} \left(\sum_{k = 2}^n T^k \mu + T\mu\right) 
      = \frac{1}{n}\sum_{k = 1}^n T^k \mu = \hat \mu.
  \end{split}\]
\end{proof}

With this in mind, if \(P\) has period \(d\) and \(\mu\) is an invariant 
measure for the chain on \(A_0\) (where \(\mathcal{X} = \bigsqcup A_n\)) as 
constructed above, then \(\frac{1}{d}\sum_{k = 1}^d P^k \mu\) is an 
invariant measure for \(P\).

\subsection{Law of Large Numbers for THMC}

We recall the strong law of large numbers.

\begin{theorem}[Kolmogorov's Strong Law of Large Numbers]
  Let \((\xi_n)\) be a sequence of real-valued, independent and identically 
  distributed random variables. Then, if \(\mathbb{E}|\xi_n| < \infty\) and 
  \(\mathbb{E}\xi_n = a\) for all \(n\), 
  \[\frac{1}{n} \sum_{k = 1}^n \xi_k \to a\]
  almost everywhere and in \(L^1\).
\end{theorem}

We may not apply this LLN directly to Markov chains as the sequence is not 
necessarily independent. Nonetheless, there is a version of the law of large 
numbers for THMCs.

\begin{theorem}[Law of Large Numbers]
  Let \((x_n)\) be an irreducible positively recurrent THMC on 
  \(\mathcal{X} = \mathbb{N}\) and let \(\pi\) denote its invariant probability 
  measure. Then, for all \(f \in L^1\), 
  \[\lim_{n \to \infty} \frac{1}{n}\sum_{k = 1}^n f(x_k) = 
    \int_{\mathcal{X}} f \dd \pi\]
  almost everywhere.
\end{theorem}

We observe that Kolmogorov's SLLN is a special case of this by taking 
\(f = \text{id}\). Furthermore, by taking 
\(f = \mathbf{1}_{\{i\}}\), \(\mu \sim \mathcal{L}(x_0)\), we have 
\[\frac{1}{n}\sum_{k = 1}^n \mathbf{1}_{\{i\}}(x_k) \to \pi(i)\]
almost everywhere. As we have already show, if \((x_n)\) is aperiodic, 
\(\lim_{n \to \infty} \mathbb{E}_\mu(\mathbf{1}_{\{i\}}(x_k)) = 
\lim_{n \to \infty}\mathbb{P}_\mu(x_n = i) = \pi(i)\). Thus, by 
dominated convergence, 
\[\lim_{n \to \infty}\mathbb{E}_\mu \frac{1}{n}\sum_{k = 1}^n \mathbf{1}_{\{i\}}(x_k)
  = \pi(i).\]

\begin{proof}
  Let \(i \in \mathcal{X}\), \(T = T_i\), \(T^k\) the successive return times to 
  \(i\), \(x_0 = i\). Then it is clear that 
  \[\left\{\sum_{l = T^k}^{T^{k + 1}}f(x_l), k = 0, 1, \cdots\right\}\]
  are independent, identically distributed for all \(f \in L^1\). Suppose for 
  now \(f \ge 0\). Then, 
  \[\begin{split}
    \mathbb{E} \sum_0^T f(x_l) 
    & = \mathbb{E} \sum_0^T \sum_{j \in \mathcal{X}} f(j)\mathbf{1}_{x_l = j}
      = \sum_{j \in \mathcal{X}} f(j)\mathbb{E}\sum_0^T \mathbf{1}_{x_l = j}.
  \end{split}\]
  Then, by recalling that \(\mathbb{E}_i \sum_{l = 0}^T \mathbf{1}_{x_l = j} = 
  \mu(j)\) which is a invariant measure, so proportional to \(\pi\) by the factor 
  \(\mathbb{E}_i T\), we have 
  \[\mathbb{E} \sum_{l = 0}^T f(x_l)  = \sum_{j \in \mathcal{X}}f(j) \mu(j) 
    = \sum_{j} f(j)\pi(j)\mathbb{E}_i T = \mathbb{E}_i[T]\int f \dd \pi < \infty.\]
  Hence, our sequence of i.i.d. random variables is integrable, and 
  we may applying SLLN to our constructed sequence, i.e. 
  \[\frac{1}{n}\sum_{l = 0}^{T^n}f(x_l) \to \mathbb{E}_i[T] \int f \dd \pi\]
  almost everywhere.

  Now, by recalling \((T^{k + 1} - T^k)_k\) is i.i.d., by applying SLLN, we have 
  \[\lim_{n \to \infty} \frac{T^n}{n} = \mathbb{E}_i T\]
  almost everywhere. So, defining \(\eta(n) := \sum_{k = 1}^n \mathbf{1}_{x_k = i}\), 
  we have \(T^{\eta(n)} \le n < T^{\eta(n) + 1}\). Furthermore, as \(i\) is recurrent,
  \(\eta(n) \to \infty\) and so, \(T^{\eta(n)} / n \sim 1\). With this, we obtain 
  \[\frac{\eta(n)}{n}\frac{1}{\eta(n)} \sum_{l = 0}^{T^{\eta(n)}}f(x_l) 
    \le \frac{1}{n}\sum_{l = 0}^n f(x_l) 
    \le \frac{\eta(n) + 1}{n}\frac{1}{\eta(n) + 1}\sum_{l = 0}^{T^{\eta(n) + 1}}f(x_l).\]
  Thus, as 
  \[\lim_{n \to \infty}\frac{\eta(n)}{n} = 
    \lim_{n \to \infty}\frac{T^{\eta(n)}}{n}\frac{\eta(n)}{T^{\eta(n)}} = 
    \frac{1}{\mathbb{E}_iT},\]
  by sandwich, we have 
  \[\lim_{n \to \infty} \frac{1}{n}\sum_{l = 0}^n f(x_l) = 
    \lim_{n \to \infty} \frac{\eta(n)}{n}\frac{1}{\eta(n)} \sum_{l = 0}^{T^{\eta(n)}}f(x_l) 
    = \frac{1}{\mathbb{E}_i T} \mathbb{E_i}T \int f \dd \pi = \int f \dd \pi,\]
  almost everywhere as required. Now, applying the theorem to \(f^+, f^-\), we 
  obtain the theorem for the general case.
\end{proof}

We note that in the above proof, we had proved the limit 
\[\frac{1}{n}\sum_{l = 0}^{T^n} f(x_l) \to \mathbb{E}_i T \int f \dd \pi.\]
Then, taking \(f = \mathbf{1}_{\{j\}})\), we have 
\[\frac{1}{n}\sum_{l = 0}^{T^n}\mathbf{1}_{\{j\}}(x_l) \to \frac{\pi(j)}{\pi(i)}.\]
In words, this is saying the average time of a chain at \(j\) during a excursion 
to \(i\) is the proportion \(\pi(j) / \pi(i)\).

\textbf{Application}: 
Empirical average, Monte Carlo Markov chain (see official notes).

\subsection{Reversible Markov Chain}

Let \((x_n)\) be an irreducible THMC with transition probability \(P\) with 
invariant probability measure \(\pi > 0\) and define \(\hat P\) such that
\[\hat P_{ji} = \frac{\pi(i)}{\pi(j)} P_{ij}.\]
Then, as \(\pi\) is invariant,
\[\sum_{i \in \mathcal{X}} \hat P_{ji} = 
  \sum_{i \in \mathcal{X}} \frac{\pi(i)}{\pi(j)} P_{ij} = 1,\]
so \(\hat P\) is also a transition probability. Now, noting by elementary 
properties, we have 
\[\mathbb{P}(x_n = j \mid x_{n + 1} = i) = 
  \frac{\mathbb{P}(x_{n + 1} = i \mid x_n = j)\mathbb{P}(x_n = j)}{\mathbb{P}(x_{n + 1} = i)}
  = \frac{\pi(j)}{\pi(i)}P_{ji} = \hat P_{ij}\]
if \(\mathcal{L}(x_0) = \pi\). Thus, in some sense, if we take a Markov 
chain and run it backwards in time, we obtain another Markov chain with transition 
probability \(\hat P\). This notion made formal with the following theorem.

\begin{theorem}
  Let \((x_n)\) be an irreducible, positively recurrent THMC with initial 
  distribution \(\pi\) which is invariant. Then, for any \(M \in \mathbb{N}\), 
  defining \(\hat x_n := x_{M - n}\), \((\hat x_n)\) is a THMC with transition 
  probability \(\hat P\) and initial distribution \(\pi\) where 
  \[\hat P_{ji} = \frac{\pi(i)}{\pi(j)}P_{ij}.\]
\end{theorem}
\begin{proof}
  The initial distribution of \(\hat x_n\) is \(\pi\) by construction and we 
  observe 
  \[\begin{split}
    \mathbb{P}(\hat x_0 = i_0, \cdots, \hat x_n = i_n) & = 
    \mathbb{P}(x_M = i_0, \cdots, x_{M - n} = i_n)\\
    & = \pi(i_n) P_{i_ni_{n - 1}} \cdots P_{i_1i_0}\\
    & = \frac{\pi(i_n)}{\pi(i_{n - 1})}P_{i_ni_{n - 1}} \cdots 
      \frac{\pi(i_1)}{\pi(i_0)}P_{i_1i_0} \pi(i_0)\\
    & = \hat P_{i_{n - 1} i_n} \cdots \hat P_{i_1 i_0} \pi(i_0).
  \end{split}\]
\end{proof}

\begin{corollary}
  If \(\pi(i)P_{ij} = \pi(j)P_{ji}\), then \((\hat x_n)\) is a THMC with 
  transition probability \(P\) and initial distribution \(\pi\).
\end{corollary}

We note that if \(\pi(i)P_{ij} = \pi(j)P_{ji}\), then 
\[\sum_{i \in \mathcal{X}} \pi(i) P_{ij} = \pi(j)\sum_{i} P_{ji} = \pi(j)\]
implying \(\pi\) is invariant.

\begin{definition}[Reversible]
  A THMC \((x_n)\) is said to be reversible with respect to \(\pi\) if 
  \((\hat x_m)\) is a Markov chain with the same transition probability.
\end{definition}

\subsection{Finite State Markov Chain}

In this section, we will consider specifically Markov chains on finite 
state spaces. 

\begin{proposition}
  Let \(\mathcal{X} = \{1, \cdots, N\}\), the following are equivalent
  \begin{itemize}
    \item \(P\) is irreducible and aperiodic;
    \item \(P^n\) is irreducible for every \(n \ge 1\);
    \item Let \(\delta_n := \min_{i,j} (P^n)_{ij}\), then there exists some 
      \(n_0 \ge 1\) such that \(\delta_{n_0} > 0\).
  \end{itemize}
\end{proposition}
\begin{proof}
  Clearly the second statement implies the first and in particular, if \(P\) 
  has period \(d > 1\), then \(P^d\) is reducible to each cycle.

  Suppose the third statement is true, then \(P\) is irreducible since 
  \(P_ij^{n_0} > 0\) for all \(i, j\). Furthermore, for any \(m \ge 0\), we have 
  \[P_{jk}^{n_0 + m} = \sum_{l}P^m_{jl}P^{n_0}_{lk} \ge \delta_0 \sum_l P^m_{jl} = 
    \delta_0,\]
  in particular, \(P^n\) has all positive entries for all \(n \ge n_0\). Thus, 
  \(n_0, n_0 + 1, \cdots \in R(i)\) and so the chain is aperiodic. Furthermore, 
  as for all \(n \le n_0\), we may find \(k\) sufficiently large such that 
  \(kn \ge n_0\) and so, \(P^{kn} >0\) implying \(P^n\) is irreducible.

  Finally, assuming \(P\) is irreducible and aperiodic, for all \(i\), 
  there exists some \(N_i\) such that \(P^n_{ii} > 0\) for all \(n \ge N_i\). 
  Thus, for any \(i, j \in \mathcal{X}\), as \(P\) is irreducible, there 
  exists some \(m_{ij}\) such that \(P_{ij}^{m_{ij}} > 0\) and so, for all 
  \(n \ge N_i + m_{ij}\),
  \[P^n_{ij} \ge P^{n - m_{ij}}_{ii} P^{m_{ij}}_{ij} > 0.\]
  Thus, taking \(n_0 = \max_{i, j}N_i + m_{ij}\), we have the required \(n_0\).
\end{proof}

\begin{lemma}
  Let \((x_n)\) be a aperiodic, irreducible THMC on a finite state space with 
  transition probability \(P\). Then for any two states \(i, j\), 
  \[\mathbb{E}_i(T_j^\alpha) < \infty\]
  for any \(\alpha \ge 1\).
\end{lemma}
\begin{proof}
  As \(P\) is irreducible, we have \(\mathbb{P}_j(T_i < \infty) = 1\). 
  Furthermore, we note
  \[\begin{split}
    \mathbb{E}_j(T_i^\alpha) & = \sum_{n \ge 0}n^\alpha\mathbb{P}_j(T_i = n) 
      \le \sum_{n \ge 0} n^\alpha \mathbb{P}_j(T_i > n - 1).
  \end{split}\]
  Now, let \(n_0\) such that \(\delta_{n_0} > 0\), we have 
  \[\begin{split}
    \mathbb{P}_j(T_i > n_0(k + 1))
    & \le \mathbb{P}_j(x_{n_0(k + 1)} \neq i, x_{n_0k} \neq i, \cdots, x_{n_0} \neq i)\\
    & = \mathbb{P}_j(x_{n_0(k + 1)} \neq i \mid x_{n_0k} \neq i, \cdots, x_{n_0} \neq i)
      \mathbb{P}_j((x_{n_0k} \neq i, \cdots, x_{n_0} \neq i)\\
    & = \mathbb{P}_j(x_{n_0(k + 1)} \neq i \mid x_{n_0k} \neq i)
      \mathbb{P}_j((x_{n_0k} \neq i, \cdots, x_{n_0} \neq i)\\
    & \le (1 - \delta_{n_0})\mathbb{P}_j(T_j > n_0k) \le \cdots \le (1 - \delta_{n_0})^{k - 2},
  \end{split}\]
  where we used the inequality 
  \(\mathbb{P}_j(x_{n_0(k + 1)} \neq i \mid x_{n_0k} \neq i) \le 1 - \delta_{n_0}\).

  Now, by considering \(\mathbb{P}_j(T_j > n - 1) \le \mathbb{P}_j(T_j > n_0 k - 1)
  \le (1 - \delta)^k\) for \(n_0 k \le n \le n_0(k + 1)\), we have 
  \[\mathbb{E}_j(T_i) \le \sum_{n \ge 0}(1 - \delta_{n_0})^{k(n)}
    \le \sum_{k = 0}^\infty n_0^\alpha(k + 1)^\alpha (1 + \delta)^k 
    = n_0^\alpha \sum_{k = 1}^\infty k^\alpha (1 - \delta_{n_0})^{k - 1}
    < \infty\]
\end{proof}

\begin{lemma}
  Let \(P\) be irreducible and aperiodic, then there exists some \(n > 0\), 
  \(\delta > 0\) such that for every vector \(\eta \in \mathbb{R}_+^N\), 
  \[\eta P^n \ge \delta \|\eta\|_1 \mathbf{1}\]
  where \(\mathbf{1}\) the the row vector of 1s and \(\|\eta\|_1 = \sum_i |\eta_i|\).
\end{lemma}
\begin{proof}
  We observe
  \(\eta P^n(j) = \sum_{i \in \mathcal{X}} \eta(i)P^n_{ij} \ge 
  \min_{i, j}P^n_{ij} \sum_{i \in \mathcal{X}} \eta(i)\). Thus, choosing 
  \(n_0, and \delta_{n_0}\) as above, the result follows.
\end{proof}

\begin{lemma}\label{exist_T}
  Let \(P\) be a irreducible \(N \times N\) stochastic matrix and define 
  \[T_n = \frac{1}{n}(P + P^2 + \cdots + P^n).\]
  Then, there exists a number \(n_0\) such that \(T_{n_0}\) has only positive 
  entries. Furthermore, for all \(\eta \in \mathbb{R}_+^N\), 
  \[\eta T_{n_0} \ge \delta \|\eta\|_1 \mathbf{1}\]
  where \(\delta > 0\) is chosen such that \(\min_{i, j} (T_{n_0})_{ij} \ge \delta\).
\end{lemma}
\begin{proof}
  The result follows from above in the case that \(P\) is periodic so suppose 
  \(P\) has period \(d \ge 1\). Then, \(\mathcal{X} = A_0 \sqcup \cdots \sqcup A_{d - 1}\) 
  where \(P^d\) is aperiodic and irreducible on \(A_k\). Thus, there exists some 
  \(n_0\) such that \(P^{n_0 d} > 0\) in \(A_k\). Then, if \(j \in A_{k + 1}\), 
  as \(P\) is irreducible, there exists some \(i_0 \in A_k\) such htat 
  \(P_{i_0j} > 0\). Then, for all \(i \in A_k\), 
  \[P_{ij}^{n_0d + 1} \ge P_{ii_0}^{n_0d} P_{i_0j} > 0.\]
  Thus, \(P_{ij}^{n_0 d} + P_{ij}^{n_0 d + 1} > 0\) for all \(i, j \in A_k \cup A_{k + 1}\). 
  Hence, by induction, 
  \[P_{ij}^{n_0d} + \cdots, P_{ij}^{n_0d + d} > 0,\]
  for all \(i, j \in \mathcal{X}\) implying \(T^n > 0\) for some \(n\).
\end{proof}

\begin{theorem}[Perron-Frobenius Theorem]
  Let \(P\) be an irreducible \(N \times N\) stochastic matrix. Then, the following 
  holds,
  \begin{enumerate}
    \item 1 is a left eigenvalue of \(P\) and there exists exactly 
      one left eigenvector for 1 up to multiplication by a constant.

      Furthermore, \(\pi\) can be chosen to have strictly positive entries 
      and normalised such that \(\sum_{i = 1}^N \pi(i) = 1\). This vector 
      is called the Perron-Frobenius vector of \(P\).
    \item Every eigenvalue \(\lambda\) of \(P\) must satisfy \(|\lambda| \le 1\). If \(P\) is 
      furthermore aperiodic, all eigenvalues other than 1 satisfy \(|\lambda| < 1\).
    \item If \(P\) is periodic with period \(d\) it has eigenvalues
      \[\lambda_j = e^{-\frac{2\pi i j}{d}}\]
      for \(j = 1, \cdots, d\) with associated eigenvectors \(\mu_j\) such that 
      \[\mu_j(n) = (\lambda_j)^{-k} \pi(n)\]
      if \(n \in A_k\) where \(\mathcal{X} = A_0 \sqcup \cdots \sqcup A_{d - 1}\).
  \end{enumerate}
\end{theorem}
\begin{proof}
  We will not by using the 3rd. claim for this course and the proof is left in the 
  official notes.

  If \(\mu P = \lambda \mu\), then \(\|\mu P\|_1 = |\lambda| \|\mu\|_1\). But 
  \(\|\mu P\|_1 = \sum_{i, j} |\mu(i)P_{ij}| \le \sum_{i}|\mu(i)| = \|\mu\|_1\).
  Thus, \(|\lambda| \le 1\) as claimed.

  Consider that \((P\mathbf{1})_k = \sum_{j = 1}^N P_{kj} = 1\) for all 
  \(k\), we have \(P\mathbf{1} = \mathbf{1}\) implying 1 is an 
  (right)eigenvalue of \(P\) with \textbf{right}-eigenvector \(\mathbf{1}\). 
  Now, since \(P^T\) has the same (right)eigenvalues as \(P\), 1 is an eigenvalue 
  of \(P^T\) implying 1 is an left-eigenvalue of \(P\). Let \(\pi \in \mathbb{R}^N\)
  such that \(\pi P = \pi\) (where we can choose \(\pi\) to be a real vector as \(P\) 
  is real). Then, we observe 
  \[\pi T_n \equiv \frac{1}{n}\pi(P + P^2 + \cdots + P^n) = \frac{1}{n} n\pi = \pi.\]
  If \(\pi > 0\) or \(\pi < 0\) for all its entries, we may simply normalise as 
  claimed, so suppose otherwise, \(\alpha := \|\pi^+\|_1 \wedge \|\pi^-\|_1 > 0\).
  By previous lemmas, as \(P\) is irreducible, there exists some \(n\) such that 
  \(T_n > \delta > 0\) for all of its entries. Then, 
  \[\pi^+ T_n \ge \delta\alpha \mathbf{1} \text{ and, } 
    \pi^- T_n \ge \delta \alpha \mathbf{1}.\]
  So, 
  \[\begin{split}
    \|\pi T_n\|_1 & = 
    \|\pi^+ T_n - \delta \alpha \mathbf{1} + \delta \alpha \mathbf{1} - \pi^- T_n\|_1 \\
    & \le \|\pi^+ T_n - \delta \alpha \mathbf{1}\| + 
      \|\pi^- T_n - \delta \alpha \mathbf{1}\|\\
    & = \|\pi^+ T_n\| - \|\delta \alpha \mathbf{1}\|_1 + \|\pi^- T_n\| - \|\delta \alpha \mathbf{1}\|_1\\
    & = \|\pi T_n\|_1 - 2 \delta \alpha n,
  \end{split}\]
  which is a contradiction for all \(\alpha > 0\) and so, either \(\pi^+ = 0\) or 
  \(\pi^- = 0\).

  With this in mind, we may assume \(\pi \ge 0\) and \(\sum_{i = 1}^N \pi(i) = 1\) 
  and so, 
  \[\pi(i) = \pi T_n(i) = \sum_{j} \pi(j) T_{ji}^n \ge \delta \sum_{j} \pi(j) = \delta > 0,\]
  and so, \(\pi\) has only positive entries. 

  Finally, we will prove the uniqueness of the eigenvector. Suppose \(\pi, \tilde \pi\) 
  are both normalised left-eigenvectors of \(P\) of eigenvalue 1. Then, 
  \[\pi - \tilde \pi = (\pi - \tilde \pi)P,\]
  and we can choose \(\pi - \tilde \pi \ge 0\). But, 
  \[0 = \sum_i \pi(i) - \sum_i \tilde \pi(i) = \sum_i (\pi - \tilde \pi)(i),\]
  we have \(\pi(i) = \tilde \pi(i)\) for all \(i\) as required.
\end{proof}

\begin{corollary}
  An irreducible Markov chain on a finite state space is positively recurrent.
\end{corollary}
\begin{proof}
  P-F tells us its unique invariant measure has positive charge at every state. 
  Thus, recalling \(\pi(i) = 1 / \mathbb{E}_i T_i\), the chain is positively 
  recurrent.
\end{proof}

If \(\mu_1, \cdots, \mu_k\) are invariant probability measures for \(P\). 
Then, so are the convex combinations of these measures a invariant probability 
measure for \(P\). That is to say, if \(a_i \in [0, 1]\) and \(\sum_{i = 1}^k a_i = 1\), 
then \(\sum_{i = 1}^k a_i \mu_i\) is also a invariant probability measure for \(P\)
(see exercise sheet).

Let us denote \(I = \{\pi \in P(\mathcal{X}) \mid \pi P = \pi\}\) and we say 
\(\pi \in I\) is the extremal of \(I\) if it cannot be written as a convex 
combination of other measures in \(I\).

\begin{theorem}
  Let \(P\) be a stochastic matrix. Then the set of its invariant probability 
  measures are precisely all convex combinations of the Perron-Frobenius 
  vectors of the restriction of \(P\) to its recurrent comminication classes. 
\end{theorem}
\begin{proof}
  See exercise sheet.
\end{proof}

We observe that the Perron-Frobenius vectors of the restriction of \(P\) to its 
recurrent comminication classes are precisely the extremal of \(I\).

\begin{definition}[Sub-Stochastic Matrix]
  An \(N \times N\) matrix with non-negative entries \(P\) is a sub-stochastic 
  matrix if 
  \[\sum_{j \in \mathcal{X}} P_{ij} \le 1.\]
\end{definition}

Sub-stochastic matrices are in general obtained by restricting to transient 
communication classes. 

\begin{lemma}
  Let \(P\) be an irreducible sub-stochastic matrix which is not a stochastic 
  matrix. Then, for any \(\mu \in \mathbb{N}\), \(\mu P^n \to 0\). Furthermore
  there exists some \(a \in (0, 1)\) such that \(\|\mu P^n\|_1 \le a^n\). 
  In particular, the eigenvalues of \(P^n\) has modulus less than 1 and 
  \(\text{id} - P\) is invertible. 
\end{lemma}
\begin{proof}
  Clearly, if \(\mu = 0\) then we are done, so suppose otherwise. Then, 
  we may decompose \(\mu = \mu^+ - \mu^-\) and so, we may take \(\mu > 0\) 
  and hence, normalise it to be a probability measure. 
  Define 
  \[T^n = \frac{1}{n}(P + \cdots + P^n).\]
  Then, as \(\|\mu P\|_1 \le \|\mu\|_1\), we have 
  \[\|\mu P^{n + 1}\|_1 \le \frac{1}{n} (\|\mu P^{n + 1}\|_1 + \cdots + \|\mu P^{n + 1}\|_1) 
  \le \frac{1}{n}(\|\mu P^{n + 1}\|_1 + \cdots + \|\mu P\|_1)
  = \frac{1}{n} \|\mu P T^n\|_1,\]
  where the last equality holds as all entries are non-negative.

  Then, using the same argument as lemma~\ref{exist_T}, one may show there exists some 
  \(n_0\) such that \(T^{n_0}\) has only positive entries. Namely, for all \(i\),
  \[\mu T^{n_0}(i) = \sum_{k \in \mathcal{X}} \mu(k) T^{n_0}_{ki} \ge \min_{k} T_{ki}^{n_0} =: \delta > 0.\]
  Now, as \(P\) is not a stochastic matrix, there exists some row \(i_0\) such that 
  \(\sum_{j \in \mathcal{X}} P_{i_0 j} = 1 - \alpha\) for some \(\alpha \in (0, 1]\).
  Thus, denoting \(e_{i_0}\) the vector which is 1 at the \(i_0\)-th position and 
  0 otherwise, \(\|e_{i_0} P\| = 1 - \alpha\) and so, 
  \[\begin{split}
    \|\mu P^{n_0 + 1}\|_1 & \le \|\mu P T^{n_0}\|_1 = 
    \|\mu T^{n_0} P - \delta e_{i_0} P + \delta e_{i_0} P\|_1\\
    & \le \|(\mu T^{n_0} - \delta e_{i_0}) P\|_1 + \|\delta e_{i_0} P\|_1
    \le \|\mu T^{n_0} - \delta e_{i_0}\| + \delta(1 - \alpha)\\
    & = \|\mu T^{n_0}\|_1 - \delta\|e_i\|_1 + \delta(1 - \alpha)
    = \|\mu T^{n_0}\|_1 - \delta \alpha \le 1 - \delta \alpha.
  \end{split}\]
  Thus, 
  \[\|\mu P^{k(n_0 + 1)}\|_1 \le (1 - \delta \alpha)^k,\]
  and so, defining \(a = (1 - \delta \alpha)^{\frac{1}{n_0 + 1}}\), we have 
  \[\|\mu P^n\|_1 \le a^n\]
  as required.
\end{proof}

\begin{theorem}[Minorisation]
  If there exist some \(j_0 \in \mathcal{X}\), \(\delta > 0\) such that 
  \(P_{ij_0} \ge \delta\) for all \(i \in \mathcal{X}\). Then, \((\mu P^n)\) 
  is a Cauchy sequence for any \(\mu \in P(\mathcal{X})\). Furthermore, 
  as \((P(\mathcal{X}), \|\cdot\|_{TV})\) is complete, \(\mu P^n \to \pi\) 
  for some \(\pi \in P(\mathcal{X})\), \(\pi(j_0) \ge \delta\) and 
  \[\|\mu P^n - \pi\|_1 \le 2(1 - \delta)^n.\]
\end{theorem}
\begin{proof}
  Similar proof to above (Hint: \(P^n_{ij_0} > 0\)).
\end{proof}

As the final topic for finite chains, we will discuss the long run behaviour 
of a finite Markov chain. As we have seen, in the case a chain is irreducible, 
aperiodic, positively recurrent (which is automatic for finite chains), 
then \(\lim_{n \to \infty} P_{ij}^n = \pi(j)\) where \(\pi\) is an invariant 
probability measure. We will now consider the case where the chain is 
reducible.

\begin{definition}[Sink]
  A state \(i \in \mathcal{X}\) is said to be a sink if it is recurrent and 
  \([i] = \{i\}\).
\end{definition}

Let us denote 
\[B_i := \{\omega \in \Omega \mid \exists n_0, x_n(\omega) = i, 
  \ \forall n \ge n_0\}.\]

\begin{proposition}
  If \(i \in \mathcal{X}\) is a sink, then 
  \[\lim_{n \to \infty} P^n_{ji} = P_j(B_i).\]
\end{proposition} 
\begin{proof}
  Let \(B^n_i := \{x_n = i\}\). We note \(B^n_i\) is increasing and 
  \(B_i = \bigcup_{n} B^n_i\). Thus, by the continuity of measures, we have 
  \[\lim_{n \to \infty}P^n_{ij} = \lim_{n \to \infty}\mathbb{P}_j(B^n_i)
    = \mathbb{P}_j\left(\bigcup_{n} B^n_i\right) = \mathbb{P}_j(B_i).\]
\end{proof}

Let us now denote \(f(j) := \mathbb{P}_j(B_i)\). Then, we find 
\[\begin{split}
  f(j) & = \mathbb{E}(\mathbf{1}_{B_i} \mid x_0 = j) = 
    \mathbb{E}(\mathbb{E}(\mathbf{1}_{B_i} \mid \sigma(x_0) \vee \sigma(x_1)) \mid x_0 = j)\\
    & = \mathbb{E}(\mathbb{E}(\mathbf{1}_{B_i} \mid x_1) \mid x_0 = j)
      = \mathbb{E}(f(x_1) \mid x_0 = j)\\
    & = \sum_{k \in \mathcal{X}} f(k)P_{jk} = (Pf)(j).
\end{split}\]
Thus, \(f\) is a right eigenvector of \(P\) with eigenvalue 1 
(and is known as ``harmonic''), namely \(f = Pf\).

In the case that the minimal class is not a singleton, we may consider the 
minimal class as a single node and use the above method. From which, we 
may then work with the minimal class redistributing the probabilities 
to the ratio of the whole chain.

\begin{proposition}
  Let \(P\) be a stochastic matrix. Decomposing 
  \(\mathcal{X} = T \cup \bigcup_{i = 1}^k A_i\) where \(T\) are the transient 
  states and \(A_i\) are the recurrent communicating classes, then by reordering 
  we can write
  \[P = \begin{pmatrix}
    T & S\\ 
    0 & \tilde P
  \end{pmatrix},\]
  where \(\tilde P = \bigoplus P_i\) for which \(P_i\) are stochastic matrices 
  corresponding to \(A_i\) and \(T\) is a sub-stochastic matrix. Furthermore,
  denoting \(A_{ij} = \mathbb{P}_i(x_n \text{ eventually ends in } j)\), we 
  have 
  \[A := (A_{ij}) = (\text{id} - T)^{-1}S.\]
\end{proposition}
\begin{proof}
  Exercise (Hint: show \(A_{ij} = (TA)_{ij} + S_{ij}\)).
\end{proof}

\newpage
\section{Invariant Measures in General State Space}

\subsection{Weak Convergence and Feller}

We recall the transition operator \(T^* : \mu \mapsto (A \mapsto \int P(x, A) \mu(\dd x))\) 
and the dual transition operator \(T_* : f \mapsto (x \mapsto \int f(y) P(x, \dd y))\),
and the relation 
\[\int f \dd T^* \mu = \int T_* f \dd \mu.\]
We note that one may deduce \(P, T^*\) and \(T_*\) from one another and 
in general, we will denote \(T\) for both \(T^*\) and \(T_*\).

\begin{definition}[Weak Convergence of Measures]
  A sequence of measures \((\mu_n)\) is said to converge weakly to \(\mu\) 
  if for any bounded continuous real-valued function \(\phi\), 
  \[\lim_{n \to \infty} \int \phi \dd \mu_n = \int \phi \dd \mu.\]
\end{definition}

The definition of weak convergence is inspired by the following lemma.

\begin{lemma}
  Let \(\mu, \nu\) be measures on a separable complete metric space \(\mathcal{X}\). 
  Then, \(\mu = \nu\) if for every bounded real-value uniformly continuous function
  \(f\), we have 
  \[\int f \dd \mu = \int f \dd \nu.\]
  Furthermore, the space of measures \(P(\mathcal{X})\) can be equipped with 
  a topology known as the weak topology which is metrizable in which 
  \(\mu_n \to \mu\) weakly if and only if \(d(\mu_n, \mu) \to 0\).
\end{lemma}

\begin{proposition}
  If \(\mathcal{X}\) is discrete, then any function is continuous. So, 
  \(\mu_n \to \mu\) weakly if and only if \(\mu_n(A) \to \mu(A\) for all 
  measurable \(A\) (choosing \(\phi = \mathbf{1}_A\)). 
\end{proposition}

\begin{proposition}
  If \(x_n \to x\) in \(\mathcal{X}\), then \(\delta_{x_n} \to \delta_x\) weakly.
\end{proposition}

\begin{proposition}
  If \(\mathcal{X} = \mathbb{R}\), defining \(F_n(x) = \mu((-\infty, x])\) 
  and \(F(x) = \mu((-\infty, x])\), \(\mu_n \to \mu\) weakly if and only if 
  \(F_n(x) \to F(x)\) at all points of continuity of \(F\).
\end{proposition}

It is easy to check that the above holds, except perhaps the last proposition 
for which a more general proof is presented in the probability theory notes.

\begin{definition}[Feller]
  A time homogeneous Markov process with transition operator \(T\) is Feller 
  if \(Tf\) is continuous whenever \(f\) is bounded continuous.
\end{definition}

We note that \(T \phi(x) = \int \phi(y) P(x, \dd y)\) and so, \(T\) is Feller 
if and only if \(x \mapsto P(x, \cdot)\) is continuous in the weak topology of 
\(P(\mathcal{X})\).

\begin{definition}[Strong-Feller]
  A time homogeneous Markov process with transition operator \(T\) is Strong-Feller 
  if \(Tf\) is continuous whenever \(f\) is bounded measurable.
\end{definition}

\begin{lemma}
  Let \(\mu\) be a probability measure on a complete separable metric space. Then 
  for every \(\epsilon > 0\), there exists a compact set \(K\) such that 
  \(\mu(K) \ge 1 - \epsilon\).
\end{lemma}
\begin{proof}
  Recall that totally bounded + complete implies compact. So, as \(\mathcal{X}\) 
  is complete, it suffices to find a totally bounded \(K\) satisfying 
  \(\mu(K) \ge 1 - \epsilon\).

  As \(\mathcal{X}\) is separable, there exists some 
  \(\{x_i\}_{i = 1}^\infty \subseteq \mathcal{X}\) dense. So, for all \(n \in \mathcal{N}\), 
  \(\mathcal{X} = \bigcup_{i = 1}^\infty B_{1 / n}(x_i)\). Then, by the continuity 
  of measures, there exists some \(N_n\) such that 
  \[\mu\left(\bigcup_{i = 1}^{N_n} B_{1 / n}(x_i)\right) \ge 1 - \frac{\epsilon}{2^n}.\]
  Thus, defining \(K := \bigcap_{n = 1}^\infty \bigcup_{i = 1}^{N_n} B_{1 / n}(x_i)\),
  \[\mu(K^c) = \mu\left(\bigcup_{n = 1}^\infty \left(\bigcup_{i = 1}^{N_n} B_{1 / n}(x_i)\right)^c\right)
    \le \sum_{n = 1}^\infty \mu\left(\bigcup_{i = 1}^{N_n} B_{1 / n}(x_i)\right)^c
    \le \sum_{n = 1}^\infty \frac{\epsilon}{2^n} = \epsilon,\]
  implying \(\mu(K) \ge 1 - \epsilon\) as required. Finally, \(K\) is totally bounded 
  as for all \(\delta > 0\), there exists some \(n\) such that \(1 / n < \delta\), 
  and so, \(\{B_{1 / n}(x_i) \mid i = 1, \cdots, N_n\}\) is a finite cover of 
  \(K\) with each element having radius \(1 / n < \delta\).
\end{proof}

This lemma motivates the definition of tightness (note the analogy with uniform 
integrability).

\begin{definition}[Tight]
  Let \(M \subseteq P(\mathcal{X})\). Then, \(M\) is said to be tight if for 
  all \(\epsilon > 0\), there exists some compact \(K \subseteq \mathcal{X}\) 
  such that 
  \[\mu(K) \ge 1 - \epsilon\]
  for all \(\mu \in M\).
\end{definition}

\begin{theorem}[Prokhorov]
  Let \(\mathcal{X}\) be a separable complete metric space. Then a family 
  \(M \subseteq P(X)\) is tight if and only if \(M\) is relatively compact
  (i.e. for all \((\mu_n) \subseteq M\), there exists some \(\mu \in P(\mathcal{X})\)
  such that \(\mu_n \to \mu\) weakly). 
\end{theorem}
\begin{proof}
  See probability theory notes.
\end{proof}

\subsection{Invariant Measures and Lyapunov Function Test}

\begin{theorem}[Krylov-Bogoliubov]
  Let \(P\) be Feller on the complete separable metric space \(\mathcal{X}\). 
  If there exists some \(x_0 \in \mathcal{X}\) such that the family of measures
  \(\{P^n(x_0, \cdot) \mid n \in \mathbb{N}\} \subseteq P(\mathcal{X})\) is 
  tight, then \(P\) has an invariant probability measure.
\end{theorem}
\begin{proof}
  Define 
  \[\mu_N(A) := \frac{1}{N} \sum_{n = 1}^N P^n(x_0, A)\]
  for all \(A \in \mathcal{B}(\mathcal{X})\). Then, \(\{\mu_N\}\) is tight 
  (by choosing the same \(K\) as \(\{P^n(x_0, \cdot)\}\))
  and by Prokhorov's theorem, there exists some \(\pi \in P(\mathcal{X})\) 
  such that \(\mu_{N_k} \to \pi\) weakly.

  As mentioned previously, \(T\pi = \pi\) if \(\int f \dd(T\pi) = \int f \dd \pi\) 
  for all bounded continuous functions \(f\), and so, it suffices to show the latter.
  Indeed, by noting \(P(\cdot, A)\) is continuous as \(T\) is Feller, 
  \[\begin{split}
    T\pi(A) & = \int P(y, A) \pi(\dd y) = \lim_{k \to \infty} \int P(y, A) \mu_{N_k}(\dd y)\\
    & = \lim_{k \to \infty} \int P(y, A) \frac{1}{N_k} \sum_{n = 1}^{N_k} P^n(x_0, \dd y)\\
    & = \lim_{k \to \infty} \frac{1}{N_k} \sum_{n = 1}^{N_k} \int P(y, A) P^n(x_0, \dd y)\\
    & = \lim_{k \to \infty} \frac{1}{N_k} \sum_{n = 1}^{N_k}P^{n + 1}(x_0, A)\\
    & = \lim_{k \to \infty} \mu_{N_k}(A) + \frac{1}{N_k}(P^{N_k + 1}(x_0, A) - P(x_0, A)).
  \end{split}\]
  Thus, for all bounded continuous \(f\), as 
  \(\int f(y) P^{N_k + 1}(x_0, \dd y) \le \|f\|_\infty\), 
  \[\begin{split}
    \int f \dd(T\pi) & = 
    \lim_{k \to \infty} \int f(y) \mu_{N_k}(\dd y) 
    + \frac{1}{N_k} \int f(y) P^{N_k + 1}(x_0, \dd y) 
    - \frac{1}{N_k} \int f(y) P(x_0, \dd y)\\
    & = \lim_{k \to \infty} \int f(y) \mu_{N_k}(\dd y) = \int f \dd \pi
  \end{split}\]
  as required.
\end{proof}

\begin{corollary}
  If \(\mathcal{X}\) is compact, any Feller transition probability operator 
  has an invariant probability measure.
\end{corollary}

\begin{corollary}
  If \((x_n)\) is a Markov chain with \(\mathcal{L}(x_0) = \delta_{x_0}\) 
  on \(\mathbb{R}^n\) with Feller transition probability \(P\). Then, there exist 
  an invariant probability measure if any of the following holds:
  \begin{itemize}
    \item \(\sup_n \mathbb{E}|x_n|^p < \infty\) for some \(p > 0\);
    \item \(\sup_n \mathbb{E} \log(|x_n| + 1) < \infty\).
  \end{itemize}
\end{corollary}
\begin{proof}
By definition, \(P^n(x_0, \cdot) = \mathcal{L}(x_n)\), and so, for all \(M\), 
\[P^n(x_0, \overline{B_M(0)}^c) = \mathbb{P}(|x_n| > M) \le 
\frac{\sup_n \mathbb{E}\log(|x_n| + 1)}{\log(M + 1)} \to 0,\]
by Markov's inequality. Thus, \(\{P^n(x_0, \cdot)\}\) is tight implying the 
existence of an invariant measure with Krylov-Bogoliubov.

Similar proof for the first case.
\end{proof}

\begin{proposition}
Let \(P\) be a transition function on \(\mathcal{X}\) and let \(V : \mathcal{X} \to \mathbb{R}_+\)
be Borel measurable. Then, if there exists some \(\gamma \in (0, 1)\) and \(c > 0\)
such that 
\[TV(x) \le \gamma V(x) + c,\]
then, \(T^n V(x) \le \gamma^n V(x) + \frac{c}{1 - \gamma}\).
\end{proposition}
\begin{proof}
\[\begin{split}
  T^nV(x) & = \int_{\mathcal{X}} V(y) P^n(x, \dd y) 
    = \int_{\mathcal{X}} \int_{\mathcal{X}} V(y) P(z, \dd yy)P^{n - 1}(x, \dd z)\\
  & = \int_{\mathcal{X}} TV(z)P^{n - 1}(x, \dd z)
    \le \gamma \int_{\mathcal{X}} V(z) P^{n - 1}(x, \dd y) + c\\ 
  & \le \cdots \le \gamma^n V(x) + \frac{c}{1 - \gamma}.
\end{split}\]
\end{proof}

\begin{definition}[Lyapunov Function]
  Let \(\mathcal{X}\) be a complete separable metric space and \(P\) a transition 
  probability on \(\mathcal{X}\). Then, a Borel measurable function 
  \(V : \mathcal{X} \to \overline{\mathbb{R}_+}\) is a Lyapunov function for 
  \(P\) if 
  \begin{itemize}
    \item \(V^{-1}(\mathbb{R}_+) \neq \varnothing\);
    \item \(V^{-1}([0, a])\) is compact for all \(a \in \mathbb{R}\);
    \item there exists some \(\gamma < 1\) and \(c\) such that 
    \(TV(x) \le \gamma V(x) + c\) for all \(x\) which \(V(x) \neq \infty\).
  \end{itemize}
\end{definition}

\begin{theorem}[Lyapunov Function Test]
  If a transition function \(P\) is Feller and admits a Lyapunov function \(V\),
  then, it has an invariant probability measure \(\pi\).
\end{theorem}
\begin{proof}
  Let \(x_0 \in \mathcal{X}\) with \(V(x_0) < \infty\) and let \(a > 0\) and 
  define \(K_a := V^{-1}[0, a]\) which is compact. Then, 
  \[\begin{split}
    P^n(x_0, K_a^c) & = \int_{V(y) > a} P^n(x_0, \dd y) \le \int \frac{V(y)}{a} P^n(x_0, \dd y)\\
    & = \frac{1}{a} T^n V(x_0) \le \frac{1}{a}\left(\frac{c}{1 - \gamma} + \gamma^n V(x_0)\right).
  \end{split}\]
  Thus, for all \(\epsilon > 0\), choosing \(a > \frac{1}{\epsilon}\left(\frac{c}{1 - \gamma} + V(x_0)\right)\),
  we have \(P^n(x_0, K_a) > 1 - \epsilon\) for all \(n\) implying \(\{P^n(x_0, \cdot)\}\) 
  is tight which implies the existence of an invariant probability measure 
  by Krylov-Bogoliubov.
\end{proof}

\begin{proposition}
  Let \(P\) be a trnasition function on \(\mathcal{X}\) and let \(V : \mathcal{X} \to \mathbb{R}_+\)
  be a Borel measurable function. Then, if there exists some \(\gamma \in (0, 1)\), 
  \(c > 0\) such that \(TV(x) \le \gamma V(x) + c\), every invariant probability 
  measure \(\pi\) for \(P\) satisfies 
  \[\int_{\mathcal{X}} V \dd \pi \le \frac{c}{1 - \gamma}.\]
\end{proposition}
\begin{proof}
  Let \(M > 0\), then 
  \[\int V \wedge M \dd \pi = \int T^n(V \wedge M) \dd \pi \le \int 
    \gamma^n  V \wedge M + \frac{c}{1 - \gamma} \dd \pi.\]
  By dominated convergence, by taking \(n \to \infty\),
  \[\in V \wedge M \dd \pi \le \frac{c}{1 - \gamma}\]
  for all \(M\). Thus, taking \(M \uparrow \infty\), allows us to conclude the 
  inequality.
\end{proof}

\begin{corollary}
  Let \(F : \mathcal{X} \times \mathcal{Y} \to \mathcal{X}\) be Borel measurable and 
  let \((\xi_n)\) be i.i.d. on \(\mathcal{Y}\) all of which are independent of 
  \(x_0\) on \(\mathcal{X}\). Then, defining \(x_{n + 1} := F(x_n, \xi_{n + 1})\),
  we have \(TV(x) = \mathbb{E}V(F(x, \xi_n))\). 
  
  Now, if \(F(\cdot, \xi_n(\omega))\) 
  is continuous for all \(\omega \in A\) where \(A\) is some set of probability 1, 
  and there exists a Borel measurable function \(V : \mathcal{X} \to \mathbb{R}_+\) 
  with compact level sets such that there exists some \(\gamma \in (0, 1), c \ge 0\), 
  \[\mathbb{E}V(F(x, \xi_n)) \le \gamma V(x) + c,\]
  then \((x_n)\) is Feller and \((x_\cdot)\) has at least one invariant probability
  measure.
\end{corollary}
\begin{proof}
  The first claim follows by sequential continuity while the second follows straight 
  away by the Lyapunov function test.
\end{proof}

\subsection{Deterministic Contraction and Minorisation}

So far, with the Lyapunov function test, we have provided a sufficient condition
for the existence of an invariant probability measure. We will now consider their 
uniqueness. 

Suppose \(\pi_1, \pi_2\) are two probability measures on a complete separable 
space \(\mathcal{X}\). Let \(\mu\) be the coupling of \(\pi_1\) and \(\pi_2\), 
namely, \(\mu \in P(\mathcal{X}^2)\) and \((\text{pr}_1)_*\mu = \pi_1\) and 
\((\text{pr}_2)_*\mu = \pi_2\) where \(\text{pr}_1, \text{pr}_2\) are the two 
projection maps.

\begin{lemma}
  If there exists a coupling \(\mu\) of \(\pi_1\) and \(\pi_2\) such that 
  \(\mu(\Delta) = 1\) where \(\Delta = \{(x, x) \mid x \in \mathcal{X}\}\), 
  then \(\pi_1 = \pi_2\). In particular, \(\pi_1 = \pi_2\) if 
  \[\int_{\mathcal{X}^2} 1 \wedge d(x, y) \mu(\dd x, \dd y) = 0.\]
\end{lemma}
\begin{proof}
  Let \(A \in \mathcal{B}(\mathcal{X})\), we have 
  \[\begin{split}
    \pi_1(A) & = \mu(A \times \mathcal{X}) = \mu((A \times \mathcal{X}) \cap \Delta)\\
    & = \mu((\mathcal{X} \times A) \cap \Delta) = \mu(\mathcal{X} \times A) = \pi_2(A)
  \end{split}\]
  where the second equality follows as \(\mu(\Delta) = 1\). Thus, \(\pi_1 = \pi_2\) 
  as required.

  Now, by observing that \(\{(x, y) \mid 1 \wedge d(x, y) = 0\} = \Delta\), if 
  \[\int_{\mathcal{X}^2} 1 \wedge d(x, y) \mu(\dd x, \dd y) = 0\]
  then \(1 \wedge d(x, y) \mu(\dd x, \dd y) = 0\) almost everywhere, implying 
  \(1 = \mu(\{1 \wedge d(x, y) \mu(\dd x, \dd y) = 0\}) = \mu(\Delta)\).
\end{proof}

\begin{lemma}
  Let \(\{\mu_n\}\) be a family of couplings of \(\pi_1\) and \(\pi_2\). Then 
  \(\{\mu_n\}\) is tight.
\end{lemma}
\begin{proof}
  As \(\pi_1, \pi_2\) are probability measures, they are themselves tight. Thus, 
  for all \(\epsilon > 0\), there exists some compact \(K_1, K_2\) such that 
  \(\pi_i(K_i^c) < \epsilon / 2\). Then, as \((K_1 \times K_2)^c \subseteq 
  K_1^c \times \mathcal{X} \cup \mathcal{X} \times K_2^c\), we have 
  \[\mu_i((K_1 \times K_2)^c) \le \mu(K_1^c \times \mathcal{X}) + \mu(\mathcal{X} \times K_2^c)
    = \pi_1(K_1^c) + \pi_2(K_2^c) < \epsilon.\]
  Hence, as \(K_1 \times K_2\) is compact, we have \(\{\mu_n\}\) is tight as required.
\end{proof}

\begin{lemma}
  If \(\{\mu_n\}\) are couplings of \(\pi_1\) and \(\pi_2\), then so is any of 
  its accumulation points (also known as limit/cluster points).
\end{lemma}
\begin{proof}
  Suppose \(\mu_{n_k} \to \mu\) weakly. Then, as the projection map is continuous, 
  \[\int f \dd \pi_i = \lim_{n \to \infty} \int f \circ \text{pr}_i \dd \mu_n
    = \int f \circ \text{pr}_i \dd \mu,\]
  for all bounded continuous \(f\). Thus, \(\int f \dd \pi_i = \int f \dd (\text{pr}_i)_* \mu\)
  implying \(\pi_i = (\text{pr}_i)_* \mu\) as required.
\end{proof}

\begin{lemma}
  Let \(x_{n + 1} = F(x_n, \xi_{n + 1}), y_{n + 1} = F(y_n, \xi_{n + 1})\) be 
  Markov chains where \(\xi_i\) are i.i.d.
  where \(x_0, y_0\) are independent and independent from \(\xi_i\) and let 
  \(\mu_n = \mathcal{L}((x_n, y_n))\). Then, if there exists some constant \(\gamma \in (0, 1)\) such that 
  \[\mathbb{E} d(F(x, \xi_1), (y, \xi_1)) \le \gamma d(x, y),\] 
  we have
  \[\lim_{n \to \infty} \mathbb{E}(1 \wedge d(x_n, y_n)) = 
    \lim_{n \to \infty} \int_{\mathcal{X}} 1 \wedge d \dd \mu_n = 0\]
\end{lemma}
\begin{proof}
  Define \(\phi(t) = 1 \wedge t\). By noting that \(\phi\) is convex, we may apply the 
  conditional Jensen's inequality, namely
  \[\begin{split}
    \mathbb{E}(1 \wedge d(x_n, y_n)) & = \mathbb{E} (\mathbb{E} \phi(d(x_n, y_n)) \mid x_{n - 1}, y_{n - 1})\\
    & \le \mathbb{E} \phi(\mathbb{E}(d(x_n, y_n) \mid x_{n - 1}, y_{n - 1}))\\
    & = \mathbb{E}\phi(\mathbb{E} d(F(x_{n - 1}, \xi_n), F(y_{n - 1}, \xi_n)))\\
    & \le \mathbb{E}\phi(\gamma d(x_{n - 1}, y_{n - 1})) = \mathbb{E}(1 \wedge \gamma d(x_{n - 1}, y_{n - 1})).
  \end{split}\]
  By iterating this inequality, we obtain \(\mathbb{E}(1 \wedge d(x_n, y_n)) \le 
  \mathbb{E}(1 \wedge \gamma^n d(x_0, y_0))\). Thus, as \(1 \wedge \gamma^n d(x_0, y_0) \to 0\) 
  as \(n\ to \infty\) almost everywhere, by dominated convergence 
  \[\lim_{n \to \infty}\mathbb{E}(1 \wedge d(x_n, y_n)) = 0\]
  as required.
\end{proof}

\begin{theorem}[Deterministic Contraction]
  Let \(x_{n + 1} = F(x_n, \xi_{n + 1})\) be a Markov chain where \(\xi_i\) are i.i.d.
  Then, if there exists some constant \(\gamma \in (0, 1)\) such that 
  \[\mathbb{E} d(F(x, \xi_1), (y, \xi_1)) \le \gamma d(x, y)\]
  for all \(x, y \in \mathcal{X}\), \((x_n)\) has at most one invariant probability 
  measure.  
\end{theorem}
\begin{proof}
  Let \(\pi_1, \pi_2\) be invariant probability measures and let \(x_0, y_0\) 
  be independent random variables both independent from \(\xi_i\) such that 
  \(\mathcal{L}(x_0) = \pi_1\) and \(\mathcal{L}(y_0) = \pi_2\). Then, as \(\pi_i\) 
  are invariant, \(x_n, y_n\) has distribution \(\pi_1, \pi_2\) respectively
  for all \(n\). 
  
  Now, defining \(\mu_i = \mathcal{L}((x_n, y_n))\), \(\{\mu_n\}\) is a 
  coupling of \(\pi_1\) and \(\pi_2\). By the above lemma, \(\{\mu_n\}\) is tight 
  and so, by Prokhorov's theorem, there exists a weakly convergent subsequence 
  \(\mu_{n_k}\) with limit \(\mu\) which is also a coupling of \(\pi_1\) and \(\pi_2\).
  Thus, as by the above lemma,
  \[\int 1 \wedge d \dd \mu = \lim_{k \to \infty} \int 1 \wedge d \dd \mu_{n_k} = 0,\]
  we have \(\pi_1 = \pi_2\) as required.
\end{proof}

\begin{definition}[Minorisation]
  Let \(\eta \in P(\mathcal{X})\). We say a family of transition probabilities 
  \(P = (P(x, \cdot))\) is minorised by \(\eta\) if there exists some \(a > 0\) 
  such that for all \(x \in \mathcal{X}\),
  \[P(x, \cdot) \ge a \eta.\]
\end{definition}

In the finite state case, minorisation is saying that \(P(i, j) \ge a \eta(j)\) 
for all \(i, j \in \mathcal{X}\). Thus, if we take \(\eta\) to be the vector 
with 1 in the \(j_0\)-th position and 0 everywhere else, \(P\) is minorised by 
\(\eta\) if and only if \(P(i, j_0) \ge a\) for all \(i\).

Before moving on, let us introduce another alternative definition for the total 
variation of measures which will be helpful.

\begin{proposition}
  Let \(\mu, \nu\) be positive measures on \(\Omega\). Let \(\eta\) be 
  a positive measure such that \(\mu \ll \eta\) and \(\nu \ll \eta\). Then, 
  \[\|\mu - \nu\|_{TV} = \int \left|\dv{\mu}{\eta} - \dv{\nu}{\eta}\right| \dd \eta.\]
  We note that such an \(\eta\) always exists by simply taking \(\eta = \mu + \nu\).
\end{proposition}

We note that this formulation is independent of the choice of \(\eta\). 
Indeed,
\[\begin{split}
  \int \left|\dv{\mu}{\eta} - \dv{\nu}{\eta}\right| \dd \eta & = 
  \int \dv{(\mu + \nu)}{\eta} \left|\dv{\mu}{(\mu + \nu)} - \dv{\nu}{(\mu + \nu)}\right| \dd \eta\\
  & = \int \left|\dv{\mu}{(\mu + \nu)} - \dv{\nu}{(\mu + \nu)}\right| \dd (\mu + \nu).
\end{split}\]

\begin{definition}
  Given measures \(\mu, \nu\), we define 
  \[\mu \wedge \nu := \left(\dv{\mu}{\eta} \wedge \dv{\nu}{\eta}\right)\eta\]
  where \(\mu, \nu \ll \eta\). This definition is independent of the choice of \(\eta\).
\end{definition}

\begin{lemma}
  Given measures \(\mu, \nu\), 
  \[\|\mu - \nu\|_{TV} = \mu(\Omega) + \nu(\Omega) - 2 \mu \wedge \nu(\Omega)\]
  which equals \(2(1 - \mu \wedge \nu(\Omega))\) if \(\mu, \nu \in P(\Omega)\).
\end{lemma}

\begin{lemma}
  The space \(P(\mathcal{X})\) is complete under \(\|\cdot\|_{TV}\).
\end{lemma}
\begin{proof}
  Let \((\mu_n)\) be a Cauchy sequence of probability measures and let 
  \[\eta := \sum_{n = 1}^\infty \frac{1}{2^n} \mu_n,\]
  so that \(\mu_n \ll \eta\) for all \(n\).Thus, 
  \[\|\mu_n - \mu_m\|_{TV} = \int \left|\dv{\mu_n}{\eta} - \dv{\mu_m}{\eta}\right| \dd\eta.\]
  So, \((\mu_n)\) is Cauchy if and only if \((\dd \mu_n / \dd \eta)\) is Cauchy 
  in \(L^1\). As \(L^1\) is complete, there exists some \(f \in L^1\) such that 
  \(\dd \mu_n / \dd \eta \to f\) in \(L^1\). So, \(\mu_n \to \mu\) in total variation 
  where \(\mu = f \eta \in P(\mathcal{X})\).
\end{proof}

\begin{lemma}
  Let \(\mu, \nu\) be probability measures on \(\mathcal{X}\). Then, denoting 
  \[\bar \mu := \frac{\mu - \mu \wedge \nu}{\frac{1}{2}\|\mu - \nu\|_{TV}},\]
  and 
  \[\bar \nu := \frac{\nu - \mu \wedge \nu}{\frac{1}{2}\|\mu - \nu\|_{TV}},\]
  \(\bar \mu, \bar \nu\) are probability measures and 
  \[\mu - \nu = \frac{1}{2}\|\mu - \nu\|_{TV} (\bar \mu - \bar \nu).\]
\end{lemma}
\begin{proof}
  Clear.
\end{proof}

\begin{corollary}
  Let \(\mu, \nu\) be probability measures on \(\mathcal{X}\) and \(T\) a transition 
  operator. Then 
  \[\|T\mu - T\nu\|_{TV} = \frac{1}{2}\|\mu - \nu\|_{TV} \|T\bar \mu - T \bar \nu\| 
    \le \|\mu - \nu\|_{TV}.\]
\end{corollary}

\begin{theorem}[Geometric Convergence Theorem]
  Suppose \(P\) is a transition probability on \(\mathcal{X}\) minorised by a 
  probability measure \(\eta\) (i.e. \(P(x, \cdot) \ge a\eta\) for some \(a \in (0, 1)\)).
  Then, \(P\) has a unique invariant probability measure \(\pi\).  

  Furthermore, if \(\mu, \nu \in P(\mathcal{X})\), we have 
  \[\|T^{n + 1}\mu - T^{n + 1}\nu\|_{TV} \le (1 - a)^{n + 1} \|\mu - \nu\|_{TV}.\]
\end{theorem}
\begin{proof}
  If \(m\) is a probability measure on \(\mathcal{X}\), then 
  \[Tm = \int_{\mathcal{X}} P(x, \cdot) m(\dd x) \ge a \eta.\]
  Furthermore, \((Tm - a \eta)(\mathcal{X}) = 1 - a\). So, 
  \[\begin{split}
    \|Tm - T\tilde m\|_{TV} & = \|(Tm - a \eta) - (T\tilde m - a \eta)\|_{TV}\\
    & \le (1 - a) \left\|\frac{Tm - a \eta}{1 - a} - \frac{T\tilde m - a \eta}{1 - a}\right\|_{TV} 
      \le 2(1 - a).
  \end{split}\]
  Hence, for \(\mu, \nu \in P(\mathcal{X})\), using the above lemma
  \[\begin{split}
    \|T\mu - T\nu\|_{TV} = \frac{1}{2}\|\mu - \nu\|_{TV} \|T\bar \mu - T \bar \nu\| 
    \le \frac{1}{2}\|\mu - \nu\|_{TV} 2 (1 - a) = (1 - a)\|\mu - \nu\|_{TV}.
  \end{split}\]
  Thus, by the Banach fixed point theorem, \(T\) has a unique fixed point, 
  namely \(P\) has a unique invariant probability measure.
\end{proof}

\begin{corollary}
  If \(\pi\) is the invariant probability measure for \(T\), 
  \[\|T^n \mu - \pi\|_{TV} \le (1 - a)^n\|\mu - \pi\|_{TV}.\]
\end{corollary}

We note that we may generalise the convergence theorem such that \(P\) has 
a unique invariant probability measure if there exists some \(n_0\), \(a \in (0, 1)\) 
\(\eta \in P(\mathcal{X})\) such that \(P^{n_0}(x, \cdot) \ge a \eta\) by considering 
the the more general Banach fixed point theorem which only require \(T^n\) to be 
a strict contraction for some \(n\).

\subsection{Strong Feller Property}

\begin{definition}[Support]
  Let \(\mu\) be a measure on the separable metric space \(\mathcal{X}\). Then, 
  the support of \(\mu\) is the closed set \(A\) such that \(A\) is the smallest  
  closed set of full-measure, i.e. 
  \[\text{supp}(\mu) := \bigcap_{\substack{\mu(A^c) = 0\\A \text{ closed}}}A.\]
  Alternatively, the support is the set \(A\) such that any open set containing 
  it has positive measure. 
\end{definition}

\begin{theorem}
  If \(\mu, \nu\) are mutually singular probability measures, and is 
  invariant for a transition operator \(T\). Then, if \(T\) has the strong 
  Feller property, 
  \[\text{supp}(\mu) \cap \text{supp}(\nu) = \varnothing.\]
\end{theorem}
\begin{proof}
  As \(\mu \perp \nu\), there exists some measurable \(F \subseteq \mathcal{X}\) 
  such that \(\mu(F) = 1\) and \(\nu(F) = 0\). Then, as \(T\) is strong Feller, 
  \(T\mathbf{1}_{F}(x) = P(x, F) \in [0, 1]\) is continuous. Now, as \(\nu\) 
  is invariant, 
  \[0 = \nu(F) = \int \mathbf{1}_F \dd \nu = \int \mathbf{1}_F \dd T\nu = 
  \int T\mathbf{1}_F \dd \nu.\]
  Since, \(T\mathbf{1}_{F}(x) = P(x, F) \ge 0\), 
  \(\nu(T\mathbf{1}_F^{-1}(\{0\})) = \nu(\{T\mathbf{1}_F = 0\}) = 1\). Similarly, 
  we have \(\mu(T\mathbf{1}_F^{-1}(\{1\})) = 1\). Thus, as 
  \(T\mathbf{1}_F^{-1}(\{0\}), T\mathbf{1}_F^{-1}(\{1\})\) are closed as 
  \(T\mathbf{1}_F\) is continuous, we have 
  \[\text{supp}(\mu) \cap \text{supp}(\nu) \subseteq 
    T\mathbf{1}_F^{-1}(\{1\}) \cap T\mathbf{1}_F^{-1}(\{0\}) = \varnothing.\]
\end{proof}

\begin{proposition}
  Let \(g : \mathbb{R}^n \to \mathbb{R}_+\) be measurable such that 
  \(\int g \dd \lambda = 1\). If \(Tf(x) = \int f(y)g(x - y) \lambda(\dd y) = f * g(x)\),
  then \(T\) has strong Feller property.
\end{proposition}
\begin{proof}
  This follows from the fact \(f : \mathbb{R}^n \to \mathbb{R}\) is bounded measurable 
  and \(g : \mathbb{R}^n \to \mathbb{R}\) is in \(L^1\), then \(f * g\) is a bounded 
  continuous function. 
\end{proof}

\begin{proposition}
  Let \(P : \mathcal{X}^2 \to \mathbb{R}\) be measurable such that 
  \(P(x, \dd y) = P(x, y)\mu\) for some measure \(\mu\) on \(\mathcal{X}\). 
  Then, if either (1) and (2) or (1) and (3) holds, \(P\) has the strong 
  Feller property, where 
  \begin{enumerate}
    \item \(P(\cdot, y)\) is continuous for all \(y\).
    \item for all \(x\), there exists some \(a > 0\) such that 
      \[\sup_{z \in B_a(x)} P(z, \cdot) \in L^1(\mu).\]
    \item for all \(x\), there exists some \(a > 0\) such that 
      \(\{P(z, y) \mid z \in B_a(x)\}\) is uniformly integrable.
  \end{enumerate}
  We note that (2) implies (3).
\end{proposition}

\subsection{Invariant Sets}

\begin{definition}[Invariant Sets]
  Let \(P\) be a family of transition probabilities. A Borel set \(A\) 
  is \(P\)-invariant if \(P(x, A) = 1\) for all \(x \in A\).
\end{definition}

An easy example of an invariant set is a communication class. 

It is easy to see that if \(A\) is an invariant set of the Markov chain 
\((x_n)\), then 
\[\mathbb{P}(x_0 \in A, \cdots, x_n \in A) = \pi(A)\]
where \(\pi\) is the initial distribution. Furthermore, if \(\mathbb{P}_\pi\) 
is the stationary distribution on \(\mathcal{X}^N\) where \(\pi\) is the 
initial distribution. Then, \(\mathbb{P}_\pi(A^n) = \pi(A)\).

Since for an invariant set \(A\), \(P(x, A) = 1\) for all \(x \in A\), 
\(P|_A\) provides a family of transition probabilities on \(A\). 
As we in general work with \textbf{complete} separable metric space, 
in order for the restriction to also be complete, we prefer to consider 
closed invariant sets such that Krylov-Bogoliubov can be applied.

\begin{lemma}
  Let \(A\) be \(P\)-invariant and let \(\pi^0\) be a probability measure 
  on \(A\), we can define a probability measure \(\pi\) on \(X\) with 
  \[\pi(B) := \pi^0(B \cap A).\]
  Then, \(\pi^0\) is invariant for \(P|_A\) if and only if \(\pi\) is invariant 
  for \(P\).
\end{lemma}
\begin{proof}
  Denoting \(T\) the transition operator, for all \(B\), 
  \[T\pi(B) = \int_{\mathcal{X}} P(x, B) \pi(\dd x) = 
    \int_A P(x, B) \pi(\dd x) = \int_A P(x, B \cap A) \pi(\dd x) 
    = \pi^0(B \cap A) = \pi(B),\]
  where the second equality is due to \(\pi(A) = 1\) and the third equality 
  follows as for all \(x \in A\), \(P(x, B) = P(x, B \cap A)\).

  Reverse direction is clear.
\end{proof}

\begin{theorem}
  Let \(P\) be Feller and suppose there exists a compact \(P\)-invariant set 
  \(A\). Then, there exists an invariant probability measure for \(P\).
\end{theorem}
\begin{proof}
  Let \(P^0\) be the restriction of \(P\) to \(A\). Then, as \(A\) is compact, 
  \(P^0\) is tight. Then, for all \(f : A \to \mathbb{R}\) bounded continuous, 
  by Tietze's lemma, it extends to a bounded continuous function 
  \(\bar f : \mathcal{X}\to \mathbb{R}\). Thus, \(P^0\) is Feller. Hence, 
  as \(P^0\) has an invariant probability measure by Krylov-Bogoliubov,
  the above lemma allows us to conclude \(P\) has an invariant probability
  measure.
\end{proof}

We can also use invariant sets to show the uniqueness of the invariant measure 
provided the invariant set is sufficiently absorbing. Consider the following 
sequence. Let \(A\) be invariant, \(A_0 = A\) and 
\(A_{n + 1} = \{x \mid P(x, A_n) > 0\}\). We see that \((A_n)\) is a sequence 
such that elements of \(A_{n + 1}\) can reach inside \(A_n\) in 1 time step 
with positive probability.

\begin{lemma}
  \((A_n)\) is increasing.
\end{lemma}
\begin{proof}
  We will show by induction \(A_n \subseteq A_{n + 1}\). Clearly 
  \(A_0 \subseteq A_1\) as \(A_0 = A\) and so, for all \(x \in A_0\), 
  \(P(x, A_0) = 1 > 0\) implying \(x \in A_1\). Now, for all \(n\), 
  \(x \in A_n\), by the inductive hypothesis \(A_{n - 1} \subseteq A_n\) and 
  so, \(P(x, A_n) \ge P(x, A_{n - 1}) > 0\) implying \(x \in A_{n + 1}\) as 
  required.
\end{proof}

\begin{lemma}
  Let \(A\) be \(P\)-invariant, then for any \(n \ge 1\), for any \(x \in A_n\), 
  \(P^n(x, A) > 0\).
\end{lemma}
\begin{proof}
  Clear by Chapman-Kolmogorov.
\end{proof}

\begin{proposition}
  Let \(A\) be \(P\)-invariant. Then, if \(\bigcup_{n = 0}^\infty A_n = \mathcal{X}\),
  every invariant probability measure \(\pi\) of \(P\) is an invariant probability 
  measure of \(P\) on \(A\). 
\end{proposition}
\begin{proof}
  If \(\pi(A) < 1\), then there exists some \(A_{n_0}\) with 
  \(\pi(A_{n_0} \setminus A) > 0\) (as \(\lim_{n \to \infty} \pi(A_n) = 1\)).
  Thus, 
  \[\begin{split}
    \pi(A) & = T^{n_0}\pi(A) = \int P^{n_0}(x, A) \pi(\dd x) 
      \ge \int_{A_{n_0}} P^{n_0}(x, A) \pi(\dd x) \\
    & = \int_A P^{n_0}(x, A) \pi(\dd x) + \int_{A_{n_0} \setminus A}
      P^{n_0}(x, A) \pi(\dd x) > \pi(A)
  \end{split}\]
  which is a contradiction. Hence, \(\pi\) is a probability measure on \(A\). 
  Now as \(\pi\) is invariant on \(A\) as it is invariant on \(\mathcal{X}\), 
  we conclude the claim.
\end{proof}

\begin{corollary}
  If \(A\) is compact, \(P\)-invariant where \(P\) is Feller and 
  \(\bigcup_{n = 0}^\infty A_n = \mathcal{X}\). Then, if there exists some 
  \(\gamma < 1\) such that 
  \[\mathbb{E}d(F(x, \xi_1), F(y, \xi_1)) \le \gamma d(x, y)\]
  for all \(x, y \in A\), there exists a unique invariant probability measure 
  for \(P\).
\end{corollary}
\begin{proof}
  Applying the deterministic contraction theorem on \(P\) restricted to \(A\), 
  we obtain that \(P\) has a unique invariant measure on \(A\). Now, by the 
  above lemmas, this invariant measure can be extended to \(\mathcal{X}\) 
  implying the existence of an invariant probability measure. On the other 
  hand, if we can another invariant measure, it restricted on \(A\) is a 
  invariant measure on \(A\) and hence, by uniqueness, they are equal. 
  Thus, we obtain the uniqueness of the invariant measure.
\end{proof}

\subsection{Random Dynamical System}

We recall the definition of initial value problem. Given 
\(g : \mathbb{R}^n \to \mathbb{R}^n\), \(f : \mathbb{R} \to \mathbb{R}^n\) 
measurable, a initial value problem 
\[\begin{cases}
  \dot x(t) & = g(x(t)) + f(t),\\
  x(t_0) & = x
\end{cases}\]
is said to have solution \(x \in C(a, b)\) if \(t_0 \in (a, b)\) and 
for all \(t \in (a, b)\), 
\[x(t) = x + \int_{t_0}^t g(x(s)) \dd s + \int_{t_0}^t f(s) \dd s.\]
If \(g\) is locally Lipschitz and \(f\) is continuous, \(\dot x\) exists.

We recall the existence and uniqueness of solutions from second year.

\begin{proposition}[Picard-Lindelöf]
  If \(g\) is locally Lipschitz and \(f\) is locally bounded, then for every 
  initial value, there exists a unique maximal solution. Furthermore, if 
  there exists some \(c\) such that 
  \[\langle x, g(x) \rangle \le c(1 + |x|^2)\]
  for all \(x\), the the solution exists on \(\mathbb{R}\). This condition 
  is called the one sided linear growth condition.
\end{proposition}

\begin{proposition}[Flow Condition]
  Suppose \(\phi_{t_0, t}(x)\) is the global solution to the IVT with initial 
  values \(x(t_0) = x\), then, 
  \[\phi_{u, t}(x) = \phi_{s, t}(\phi_{u, s}(x))\]
  for all \(u < s < t\). We denote \(\phi_t(x) = \phi_{0, t}(x)\).
\end{proposition}

\begin{theorem}
  Suppose \(g\) is globally Lipschitz and \(f\) is bounded measurable. Then 
  there exists a unique global solution \(\phi_{t_0, t}(x)\). 
  Furthermore, the map \((t, x) \mapsto \phi_t(x)\) is continuous (with repsect 
  to both \(t, x\)) and the map \(x \mapsto \phi_t(x)\) is differentiable. 
  Denoting \(V_t = (D\phi_t)_{x_0}(v_0)\) for all \(x_0, v_0 \in \mathbb{R}^n\),
  \(V_t\) is the unique solution to 
  \[\begin{cases}
    \dot V(t) & = (Dg)_{\phi_t(x_0)}(v(t)),\\
    V(0) & = v_0. 
  \end{cases}\] 
  Finally, \((x, v) \mapsto (Dg)_x(v)\) is continuous and Lipschitz.
\end{theorem}

\begin{corollary}
  If \(\langle Dg(x)(v), v\rangle \le -c(x)|v|^2\) for some \(c\). Then, 
  \[|v_t| \le e^{-\int_0^t c(\phi_s(x)) \dd s}\]
  where \(v_t = (D\phi_t)_{\phi_t(x)}(v)\).
\end{corollary}
\begin{proof}
  \[\begin{split}
    \dv{t}|v_t|^2 & = 2\langle v_t, \dv{t}v_t\rangle 
      = 2 \langle v_t, (Dg)_{\phi_t(x)}(v_t)\rangle\\
      & \le -2c(\phi_t(x))|v_t|^2,
  \end{split}\]
  which implies the result by integrating both sides.
\end{proof}

Thus, if \(c(x) \ge c > 0\), then 
\[|\phi_t(x) - \phi_t(y)| \le \|D\phi_t\|_\infty |x - y| \le e^{-ct}|x - y|.\]

We also introduce a more general system where we allow \(g\) to evolve with 
time. Namely, \(g : \mathbb{R}_+ \times \mathbb{R}^d \to \mathbb{R}^d\) 
and consider 
\[\begin{cases}
  \dot x(t) & = g(t, x(t)),\\
  x(t_0) & = x.
\end{cases}\]

\begin{proposition}
  If \(|g(t, x) - g(t, y)| \le K|x - y|\) i.e. \(g\) is Lipschitz with respect to 
  the second argument, then, for any initial value, there exists a unique 
  global solution which is differentiable in time. In particular, 
  \[x(t) = x_0 + \int_{t_0}^t g(\gamma, x(\gamma)) \dd \gamma.\]
  Also, if \(\phi_{t_0, t}(x)\) denotes this solution, the map 
  \(x \mapsto \phi_{t_0, t}(x)\) is differentiable (and hence also continuous).
\end{proposition}

With the above in mind, we come back to Markov processes by constructing a 
Markov process with a dynamical system. Namely, we will vary \(f\). 
Denote \(\phi_t(x, f)\) the solution to 
\[\begin{cases}
  \dot x(t) & = g(x(t)) + f(t),\\
  x(0) & = x,
\end{cases}\]
and assume that there exists a unique global solution for any initial value. 
Denoting the solution at time 1 as \(\Phi\), i.e. \(\Phi(x, f) = \phi_1(x, f)\), 
if \((\xi_n : \Omega \to C_b(\mathbb{R}))\) is a sequence of continuous iid. 
random variables (where \(C_b(\mathbb{R})\) is the space of bounded continuous 
functions on \(\mathbb{R}\))
\[\begin{cases}
  \dot x(t) & = g(x(t)) + \xi_n(t, \omega),\\
  x(0) & = x,
\end{cases}\]
we define \(x_0 := x, x_1 := \Phi(x, \xi_1), \cdots, x_n := \Phi(x_{n - 1}, \xi_n), \cdots\).
Indeed, \((x_n)\) is a Markov chain as \((\xi_n)\) are independent.

See official notes for a detailed example of such a random process and a proof 
that such an process has an invariant measure using \(P\)-invariant sets.

\end{document}
