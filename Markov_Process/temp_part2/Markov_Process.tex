% Options for packages loaded elsewhere
\PassOptionsToPackage{unicode}{hyperref}
\PassOptionsToPackage{hyphens}{url}
\PassOptionsToPackage{dvipsnames,svgnames*,x11names*}{xcolor}
%
\documentclass[]{article}
\usepackage{lmodern}
\usepackage{amssymb,amsmath}
\usepackage{ifxetex,ifluatex}
\ifnum 0\ifxetex 1\fi\ifluatex 1\fi=0 % if pdftex
  \usepackage[T1]{fontenc}
  \usepackage[utf8]{inputenc}
  \usepackage{textcomp} % provide euro and other symbols
\else % if luatex or xetex
  \usepackage{unicode-math}
  \defaultfontfeatures{Scale=MatchLowercase}
  \defaultfontfeatures[\rmfamily]{Ligatures=TeX,Scale=1}
\fi
% Use upquote if available, for straight quotes in verbatim environments
\IfFileExists{upquote.sty}{\usepackage{upquote}}{}
\IfFileExists{microtype.sty}{% use microtype if available
  \usepackage[]{microtype}
  \UseMicrotypeSet[protrusion]{basicmath} % disable protrusion for tt fonts
}{}
\makeatletter
\@ifundefined{KOMAClassName}{% if non-KOMA class
  \IfFileExists{parskip.sty}{%
    \usepackage{parskip}
  }{% else
    \setlength{\parindent}{0pt}
    \setlength{\parskip}{6pt plus 2pt minus 1pt}}
}{% if KOMA class
  \KOMAoptions{parskip=half}}
\makeatother
\usepackage{xcolor}\pagecolor[RGB]{28,30,38} \color[RGB]{213,216,218}
\IfFileExists{xurl.sty}{\usepackage{xurl}}{} % add URL line breaks if available
\IfFileExists{bookmark.sty}{\usepackage{bookmark}}{\usepackage{hyperref}}
\hypersetup{
  pdftitle={Markov Process},
  pdfauthor={Kexing Ying},
  colorlinks=true,
  linkcolor=Maroon,
  filecolor=Maroon,
  citecolor=Blue,
  urlcolor=red,
  pdfcreator={LaTeX via pandoc}}
\urlstyle{same} % disable monospaced font for URLs
\usepackage[margin = 1.5in]{geometry}
\usepackage{graphicx}
\makeatletter
\def\maxwidth{\ifdim\Gin@nat@width>\linewidth\linewidth\else\Gin@nat@width\fi}
\def\maxheight{\ifdim\Gin@nat@height>\textheight\textheight\else\Gin@nat@height\fi}
\makeatother
% Scale images if necessary, so that they will not overflow the page
% margins by default, and it is still possible to overwrite the defaults
% using explicit options in \includegraphics[width, height, ...]{}
\setkeys{Gin}{width=\maxwidth,height=\maxheight,keepaspectratio}
% Set default figure placement to htbp
\makeatletter
\def\fps@figure{htbp}
\makeatother
\setlength{\emergencystretch}{3em} % prevent overfull lines
\providecommand{\tightlist}{%
  \setlength{\itemsep}{0pt}\setlength{\parskip}{0pt}}
\setcounter{secnumdepth}{5}
\usepackage{tikz}
\usepackage{physics}
\usepackage{amsthm}
\usepackage{mathtools}
\usepackage{esint}
\usepackage[ruled,vlined]{algorithm2e}
\theoremstyle{definition}
\newtheorem{theorem}{Theorem}
\newtheorem{definition*}{Definition}
\newtheorem{prop}{Proposition}
\newtheorem{corollary}{Corollary}[theorem]
\newtheorem*{remark}{Remark}
\theoremstyle{definition}
\newtheorem{definition}{Definition}[section]
\newtheorem{lemma}{Lemma}[section]
\newtheorem{proposition}{Proposition}[section]
\newtheorem{example}{Example}[section]
\newcommand{\diag}{\mathop{\mathrm{diag}}}
\newcommand{\Arg}{\mathop{\mathrm{Arg}}}
\newcommand{\hess}{\mathop{\mathrm{Hess}}}

% the redefinition for the missing \setminus must be delayed
\AtBeginDocument{\renewcommand{\setminus}{\mathbin{\backslash}}}

\title{Markov Process}
\author{Kexing Ying}

\begin{document}
\maketitle

{
\hypersetup{linkcolor=}
\setcounter{tocdepth}{2}
\tableofcontents
}
\newpage
\section{Invariant Measures in General State Space}

\subsection{Weak Convergence and Feller}

We recall the transition operator \(T^* : \mu \mapsto (A \mapsto \int P(x, A) \mu(\dd x))\) 
and the dual transition operator \(T_* : f \mapsto (x \mapsto \int f(y) P(x, \dd y))\),
and the relation 
\[\int f \dd T^* \mu = \int T_* f \dd \mu.\]
We note that one may deduce \(P, T^*\) and \(T_*\) from one another and 
in general, we will denote \(T\) for both \(T^*\) and \(T_*\).

\begin{definition}[Weak Convergence of Measures]
  A sequence of measures \((\mu_n)\) is said to converge weakly to \(\mu\) 
  if for any bounded continuous real-valued function \(\phi\), 
  \[\lim_{n \to \infty} \int \phi \dd \mu_n = \int \phi \dd \mu.\]
\end{definition}

The definition of weak convergence is inspired by the following lemma.

\begin{lemma}
  Let \(\mu, \nu\) be measures on a separable complete metric space \(\mathcal{X}\). 
  Then, \(\mu = \nu\) if for every bounded real-value uniformly continuous function
  \(f\), we have 
  \[\int f \dd \mu = \int f \dd \nu.\]
  Furthermore, the space of measures \(P(\mathcal{X})\) can be equipped with 
  a topology known as the weak topology which is metrizable in which 
  \(\mu_n \to \mu\) weakly if and only if \(d(\mu_n, \mu) \to 0\).
\end{lemma}

\begin{proposition}
  If \(\mathcal{X}\) is discrete, then any function is continuous. So, 
  \(\mu_n \to \mu\) weakly if and only if \(\mu_n(A) \to \mu(A\) for all 
  measurable \(A\) (choosing \(\phi = \mathbf{1}_A\)). 
\end{proposition}

\begin{proposition}
  If \(x_n \to x\) in \(\mathcal{X}\), then \(\delta_{x_n} \to \delta_x\) weakly.
\end{proposition}

\begin{proposition}
  If \(\mathcal{X} = \mathbb{R}\), defining \(F_n(x) = \mu((-\infty, x])\) 
  and \(F(x) = \mu((-\infty, x])\), \(\mu_n \to \mu\) weakly if and only if 
  \(F_n(x) \to F(x)\) at all points of continuity of \(F\).
\end{proposition}

It is easy to check that the above holds, except perhaps the last proposition 
for which a more general proof is presented in the probability theory notes.

\begin{definition}[Feller]
  A time homogeneous Markov process with transition operator \(T\) is Feller 
  if \(Tf\) is continuous whenever \(f\) is bounded continuous.
\end{definition}

We note that \(T \phi(x) = \int \phi(y) P(x, \dd y)\) and so, \(T\) is Feller 
if and only if \(x \mapsto P(x, \cdot)\) is continuous in the weak topology of 
\(P(\mathcal{X})\).

\begin{definition}[Strong-Feller]
  A time homogeneous Markov process with transition operator \(T\) is Strong-Feller 
  if \(Tf\) is continuous whenever \(f\) is bounded measurable.
\end{definition}

\begin{lemma}
  Let \(\mu\) be a probability measure on a complete separable metric space. Then 
  for every \(\epsilon > 0\), there exists a compact set \(K\) such that 
  \(\mu(K) \ge 1 - \epsilon\).
\end{lemma}
\begin{proof}
  Recall that totally bounded + complete implies compact. So, as \(\mathcal{X}\) 
  is complete, it suffices to find a totally bounded \(K\) satisfying 
  \(\mu(K) \ge 1 - \epsilon\).

  As \(\mathcal{X}\) is separable, there exists some 
  \(\{x_i\}_{i = 1}^\infty \subseteq \mathcal{X}\) dense. So, for all \(n \in \mathcal{N}\), 
  \(\mathcal{X} = \bigcup_{i = 1}^\infty B_{1 / n}(x_i)\). Then, by the continuity 
  of measures, there exists some \(N_n\) such that 
  \[\mu\left(\bigcup_{i = 1}^{N_n} B_{1 / n}(x_i)\right) \ge 1 - \frac{\epsilon}{2^n}.\]
  Thus, defining \(K := \bigcap_{n = 1}^\infty \bigcup_{i = 1}^{N_n} B_{1 / n}(x_i)\),
  \[\mu(K^c) = \mu\left(\bigcup_{n = 1}^\infty \left(\bigcup_{i = 1}^{N_n} B_{1 / n}(x_i)\right)^c\right)
    \le \sum_{n = 1}^\infty \mu\left(\bigcup_{i = 1}^{N_n} B_{1 / n}(x_i)\right)^c
    \le \sum_{n = 1}^\infty \frac{\epsilon}{2^n} = \epsilon,\]
  implying \(\mu(K) \ge 1 - \epsilon\) as required. Finally, \(K\) is totally bounded 
  as for all \(\delta > 0\), there exists some \(n\) such that \(1 / n < \delta\), 
  and so, \(\{B_{1 / n}(x_i) \mid i = 1, \cdots, N_n\}\) is a finite cover of 
  \(K\) with each element having radius \(1 / n < \delta\).
\end{proof}

This lemma motivates the definition of tightness (note the analogy with uniform 
integrability).

\begin{definition}[Tight]
  Let \(M \subseteq P(\mathcal{X})\). Then, \(M\) is said to be tight if for 
  all \(\epsilon > 0\), there exists some compact \(K \subseteq \mathcal{X}\) 
  such that 
  \[\mu(K) \ge 1 - \epsilon\]
  for all \(\mu \in M\).
\end{definition}

\begin{theorem}[Prokhorov]
  Let \(\mathcal{X}\) be a separable complete metric space. Then a family 
  \(M \subseteq P(X)\) is tight if and only if \(M\) is relatively compact
  (i.e. for all \((\mu_n) \subseteq M\), there exists some \(\mu \in P(\mathcal{X})\)
  such that \(\mu_n \to \mu\) weakly). 
\end{theorem}
\begin{proof}
  See probability theory notes.
\end{proof}

\subsection{Invariant Measures and Lyapunov Function Test}

\begin{theorem}[Krylov-Bogoliubov]
  Let \(P\) be Feller on the complete separable metric space \(\mathcal{X}\). 
  If there exists some \(x_0 \in \mathcal{X}\) such that the family of measures
  \(\{P^n(x_0, \cdot) \mid n \in \mathbb{N}\} \subseteq P(\mathcal{X})\) is 
  tight, then \(P\) has an invariant probability measure.
\end{theorem}
\begin{proof}
  Define 
  \[\mu_N(A) := \frac{1}{N} \sum_{n = 1}^N P^n(x_0, A)\]
  for all \(A \in \mathcal{B}(\mathcal{X})\). Then, \(\{\mu_N\}\) is tight 
  (by choosing the same \(K\) as \(\{P^n(x_0, \cdot)\}\))
  and by Prokhorov's theorem, there exists some \(\pi \in P(\mathcal{X})\) 
  such that \(\mu_{N_k} \to \pi\) weakly.

  As mentioned previously, \(T\pi = \pi\) if \(\int f \dd(T\pi) = \int f \dd \pi\) 
  for all bounded continuous functions \(f\), and so, it suffices to show the latter.
  Indeed, by noting \(P(\cdot, A)\) is continuous as \(T\) is Feller, 
  \[\begin{split}
    T\pi(A) & = \int P(y, A) \pi(\dd y) = \lim_{k \to \infty} \int P(y, A) \mu_{N_k}(\dd y)\\
    & = \lim_{k \to \infty} \int P(y, A) \frac{1}{N_k} \sum_{n = 1}^{N_k} P^n(x_0, \dd y)\\
    & = \lim_{k \to \infty} \frac{1}{N_k} \sum_{n = 1}^{N_k} \int P(y, A) P^n(x_0, \dd y)\\
    & = \lim_{k \to \infty} \frac{1}{N_k} \sum_{n = 1}^{N_k}P^{n + 1}(x_0, A)\\
    & = \lim_{k \to \infty} \mu_{N_k}(A) + \frac{1}{N_k}(P^{N_k + 1}(x_0, A) - P(x_0, A)).
  \end{split}\]
  Thus, for all bounded continuous \(f\), as 
  \(\int f(y) P^{N_k + 1}(x_0, \dd y) \le \|f\|_\infty\), 
  \[\begin{split}
    \int f \dd(T\pi) & = 
    \lim_{k \to \infty} \int f(y) \mu_{N_k}(\dd y) 
    + \frac{1}{N_k} \int f(y) P^{N_k + 1}(x_0, \dd y) 
    - \frac{1}{N_k} \int f(y) P(x_0, \dd y)\\
    & = \lim_{k \to \infty} \int f(y) \mu_{N_k}(\dd y) = \int f \dd \pi
  \end{split}\]
  as required.
\end{proof}

\begin{corollary}
  If \(\mathcal{X}\) is compact, any Feller transition probability operator 
  has an invariant probability measure.
\end{corollary}

\begin{corollary}
  If \((x_n)\) is a Markov chain with \(\mathcal{L}(x_0) = \delta_{x_0}\) 
  on \(\mathbb{R}^n\) with Feller transition probability \(P\). Then, there exist 
  an invariant probability measure if any of the following holds:
  \begin{itemize}
    \item \(\sup_n \mathbb{E}|x_n|^p < \infty\) for some \(p > 0\);
    \item \(\sup_n \mathbb{E} \log(|x_n| + 1) < \infty\).
  \end{itemize}
\end{corollary}
\begin{proof}
By definition, \(P^n(x_0, \cdot) = \mathcal{L}(x_n)\), and so, for all \(M\), 
\[P^n(x_0, \overline{B_M(0)}^c) = \mathbb{P}(|x_n| > M) \le 
\frac{\sup_n \mathbb{E}\log(|x_n| + 1)}{\log(M + 1)} \to 0,\]
by Markov's inequality. Thus, \(\{P^n(x_0, \cdot)\}\) is tight implying the 
existence of an invariant measure with Krylov-Bogoliubov.

Similar proof for the first case.
\end{proof}

\begin{proposition}
Let \(P\) be a transition function on \(\mathcal{X}\) and let \(V : \mathcal{X} \to \mathbb{R}_+\)
be Borel measurable. Then, if there exists some \(\gamma \in (0, 1)\) and \(c > 0\)
such that 
\[TV(x) \le \gamma V(x) + c,\]
then, \(T^n V(x) \le \gamma^n V(x) + \frac{c}{1 - \gamma}\).
\end{proposition}
\begin{proof}
\[\begin{split}
  T^nV(x) & = \int_{\mathcal{X}} V(y) P^n(x, \dd y) 
    = \int_{\mathcal{X}} \int_{\mathcal{X}} V(y) P(z, \dd yy)P^{n - 1}(x, \dd z)\\
  & = \int_{\mathcal{X}} TV(z)P^{n - 1}(x, \dd z)
    \le \gamma \int_{\mathcal{X}} V(z) P^{n - 1}(x, \dd y) + c\\ 
  & \le \cdots \le \gamma^n V(x) + \frac{c}{1 - \gamma}.
\end{split}\]
\end{proof}

\begin{definition}[Lyapunov Function]
  Let \(\mathcal{X}\) be a complete separable metric space and \(P\) a transition 
  probability on \(\mathcal{X}\). Then, a Borel measurable function 
  \(V : \mathcal{X} \to \overline{\mathbb{R}_+}\) is a Lyapunov function for 
  \(P\) if 
  \begin{itemize}
    \item \(V^{-1}(\mathbb{R}_+) \neq \varnothing\);
    \item \(V^{-1}([0, a])\) is compact for all \(a \in \mathbb{R}\);
    \item there exists some \(\gamma < 1\) and \(c\) such that 
    \(TV(x) \le \gamma V(x) + c\) for all \(x\) which \(V(x) \neq \infty\).
  \end{itemize}
\end{definition}

\begin{theorem}[Lyapunov Function Test]
  If a transition function \(P\) is Feller and admits a Lyapunov function \(V\),
  then, it has an invariant probability measure \(\pi\).
\end{theorem}
\begin{proof}
  Let \(x_0 \in \mathcal{X}\) with \(V(x_0) < \infty\) and let \(a > 0\) and 
  define \(K_a := V^{-1}[0, a]\) which is compact. Then, 
  \[\begin{split}
    P^n(x_0, K_a^c) & = \int_{V(y) > a} P^n(x_0, \dd y) \le \int \frac{V(y)}{a} P^n(x_0, \dd y)\\
    & = \frac{1}{a} T^n V(x_0) \le \frac{1}{a}\left(\frac{c}{1 - \gamma} + \gamma^n V(x_0)\right).
  \end{split}\]
  Thus, for all \(\epsilon > 0\), choosing \(a > \frac{1}{\epsilon}\left(\frac{c}{1 - \gamma} + V(x_0)\right)\),
  we have \(P^n(x_0, K_a) > 1 - \epsilon\) for all \(n\) implying \(\{P^n(x_0, \cdot)\}\) 
  is tight which implies the existence of an invariant probability measure 
  by Krylov-Bogoliubov.
\end{proof}

\begin{proposition}
  Let \(P\) be a trnasition function on \(\mathcal{X}\) and let \(V : \mathcal{X} \to \mathbb{R}_+\)
  be a Borel measurable function. Then, if there exists some \(\gamma \in (0, 1)\), 
  \(c > 0\) such that \(TV(x) \le \gamma V(x) + c\), every invariant probability 
  measure \(\pi\) for \(P\) satisfies 
  \[\int_{\mathcal{X}} V \dd \pi \le \frac{c}{1 - \gamma}.\]
\end{proposition}
\begin{proof}
  Let \(M > 0\), then 
  \[\int V \wedge M \dd \pi = \int T^n(V \wedge M) \dd \pi \le \int 
    \gamma^n  V \wedge M + \frac{c}{1 - \gamma} \dd \pi.\]
  By dominated convergence, by taking \(n \to \infty\),
  \[\in V \wedge M \dd \pi \le \frac{c}{1 - \gamma}\]
  for all \(M\). Thus, taking \(M \uparrow \infty\), allows us to conclude the 
  inequality.
\end{proof}

\begin{corollary}
  Let \(F : \mathcal{X} \times \mathcal{Y} \to \mathcal{X}\) be Borel measurable and 
  let \((\xi_n)\) be i.i.d. on \(\mathcal{Y}\) all of which are independent of 
  \(x_0\) on \(\mathcal{X}\). Then, defining \(x_{n + 1} := F(x_n, \xi_{n + 1})\),
  we have \(TV(x) = \mathbb{E}V(F(x, \xi_n))\). 
  
  Now, if \(F(\cdot, \xi_n(\omega))\) 
  is continuous for all \(\omega \in A\) where \(A\) is some set of probability 1, 
  and there exists a Borel measurable function \(V : \mathcal{X} \to \mathbb{R}_+\) 
  with compact level sets such that there exists some \(\gamma \in (0, 1), c \ge 0\), 
  \[\mathbb{E}V(F(x, \xi_n)) \le \gamma V(x) + c,\]
  then \((x_n)\) is Feller and \((x_\cdot)\) has at least one invariant probability
  measure.
\end{corollary}
\begin{proof}
  The first claim follows by sequential continuity while the second follows straight 
  away by the Lyapunov function test.
\end{proof}

\subsection{Deterministic Contraction and Minorisation}

So far, with the Lyapunov function test, we have provided a sufficient condition
for the existence of an invariant probability measure. We will now consider their 
uniqueness. 

Suppose \(\pi_1, \pi_2\) are two probability measures on a complete separable 
space \(\mathcal{X}\). Let \(\mu\) be the coupling of \(\pi_1\) and \(\pi_2\), 
namely, \(\mu \in P(\mathcal{X}^2)\) and \((\text{pr}_1)_*\mu = \pi_1\) and 
\((\text{pr}_2)_*\mu = \pi_2\) where \(\text{pr}_1, \text{pr}_2\) are the two 
projection maps.

\begin{lemma}
  If there exists a coupling \(\mu\) of \(\pi_1\) and \(\pi_2\) such that 
  \(\mu(\Delta) = 1\) where \(\Delta = \{(x, x) \mid x \in \mathcal{X}\}\), 
  then \(\pi_1 = \pi_2\). In particular, \(\pi_1 = \pi_2\) if 
  \[\int_{\mathcal{X}^2} 1 \wedge d(x, y) \mu(\dd x, \dd y) = 0.\]
\end{lemma}
\begin{proof}
  Let \(A \in \mathcal{B}(\mathcal{X})\), we have 
  \[\begin{split}
    \pi_1(A) & = \mu(A \times \mathcal{X}) = \mu((A \times \mathcal{X}) \cap \Delta)\\
    & = \mu((\mathcal{X} \times A) \cap \Delta) = \mu(\mathcal{X} \times A) = \pi_2(A)
  \end{split}\]
  where the second equality follows as \(\mu(\Delta) = 1\). Thus, \(\pi_1 = \pi_2\) 
  as required.

  Now, by observing that \(\{(x, y) \mid 1 \wedge d(x, y) = 0\} = \Delta\), if 
  \[\int_{\mathcal{X}^2} 1 \wedge d(x, y) \mu(\dd x, \dd y) = 0\]
  then \(1 \wedge d(x, y) \mu(\dd x, \dd y) = 0\) almost everywhere, implying 
  \(1 = \mu(\{1 \wedge d(x, y) \mu(\dd x, \dd y) = 0\}) = \mu(\Delta)\).
\end{proof}

\begin{lemma}
  Let \(\{\mu_n\}\) be a family of couplings of \(\pi_1\) and \(\pi_2\). Then 
  \(\{\mu_n\}\) is tight.
\end{lemma}
\begin{proof}
  As \(\pi_1, \pi_2\) are probability measures, they are themselves tight. Thus, 
  for all \(\epsilon > 0\), there exists some compact \(K_1, K_2\) such that 
  \(\pi_i(K_i^c) < \epsilon / 2\). Then, as \((K_1 \times K_2)^c \subseteq 
  K_1^c \times \mathcal{X} \cup \mathcal{X} \times K_2^c\), we have 
  \[\mu_i((K_1 \times K_2)^c) \le \mu(K_1^c \times \mathcal{X}) + \mu(\mathcal{X} \times K_2^c)
    = \pi_1(K_1^c) + \pi_2(K_2^c) < \epsilon.\]
  Hence, as \(K_1 \times K_2\) is compact, we have \(\{\mu_n\}\) is tight as required.
\end{proof}

\begin{lemma}
  If \(\{\mu_n\}\) are couplings of \(\pi_1\) and \(\pi_2\), then so is any of 
  its accumulation points (also known as limit/cluster points).
\end{lemma}
\begin{proof}
  Suppose \(\mu_{n_k} \to \mu\) weakly. Then, as the projection map is continuous, 
  \[\int f \dd \pi_i = \lim_{n \to \infty} \int f \circ \text{pr}_i \dd \mu_n
    = \int f \circ \text{pr}_i \dd \mu,\]
  for all bounded continuous \(f\). Thus, \(\int f \dd \pi_i = \int f \dd (\text{pr}_i)_* \mu\)
  implying \(\pi_i = (\text{pr}_i)_* \mu\) as required.
\end{proof}

\begin{lemma}
  Let \(x_{n + 1} = F(x_n, \xi_{n + 1}), y_{n + 1} = F(y_n, \xi_{n + 1})\) be 
  Markov chains where \(\xi_i\) are i.i.d.
  where \(x_0, y_0\) are independent and independent from \(\xi_i\) and let 
  \(\mu_n = \mathcal{L}((x_n, y_n))\). Then, if there exists some constant \(\gamma \in (0, 1)\) such that 
  \[\mathbb{E} d(F(x, \xi_1), (y, \xi_1)) \le \gamma d(x, y),\] 
  we have
  \[\lim_{n \to \infty} \mathbb{E}(1 \wedge d(x_n, y_n)) = 
    \lim_{n \to \infty} \int_{\mathcal{X}} 1 \wedge d \dd \mu_n = 0\]
\end{lemma}
\begin{proof}
  Define \(\phi(t) = 1 \wedge t\). By noting that \(\phi\) is convex, we may apply the 
  conditional Jensen's inequality, namely
  \[\begin{split}
    \mathbb{E}(1 \wedge d(x_n, y_n)) & = \mathbb{E} (\mathbb{E} \phi(d(x_n, y_n)) \mid x_{n - 1}, y_{n - 1})\\
    & \le \mathbb{E} \phi(\mathbb{E}(d(x_n, y_n) \mid x_{n - 1}, y_{n - 1}))\\
    & = \mathbb{E}\phi(\mathbb{E} d(F(x_{n - 1}, \xi_n), F(y_{n - 1}, \xi_n)))\\
    & \le \mathbb{E}\phi(\gamma d(x_{n - 1}, y_{n - 1})) = \mathbb{E}(1 \wedge \gamma d(x_{n - 1}, y_{n - 1})).
  \end{split}\]
  By iterating this inequality, we obtain \(\mathbb{E}(1 \wedge d(x_n, y_n)) \le 
  \mathbb{E}(1 \wedge \gamma^n d(x_0, y_0))\). Thus, as \(1 \wedge \gamma^n d(x_0, y_0) \to 0\) 
  as \(n\ to \infty\) almost everywhere, by dominated convergence 
  \[\lim_{n \to \infty}\mathbb{E}(1 \wedge d(x_n, y_n)) = 0\]
  as required.
\end{proof}

\begin{theorem}[Deterministic Contraction]
  Let \(x_{n + 1} = F(x_n, \xi_{n + 1})\) be a Markov chain where \(\xi_i\) are i.i.d.
  Then, if there exists some constant \(\gamma \in (0, 1)\) such that 
  \[\mathbb{E} d(F(x, \xi_1), (y, \xi_1)) \le \gamma d(x, y)\]
  for all \(x, y \in \mathcal{X}\), \((x_n)\) has at most one invariant probability 
  measure.  
\end{theorem}
\begin{proof}
  Let \(\pi_1, \pi_2\) be invariant probability measures and let \(x_0, y_0\) 
  be independent random variables both independent from \(\xi_i\) such that 
  \(\mathcal{L}(x_0) = \pi_1\) and \(\mathcal{L}(y_0) = \pi_2\). Then, as \(\pi_i\) 
  are invariant, \(x_n, y_n\) has distribution \(\pi_1, \pi_2\) respectively
  for all \(n\). 
  
  Now, defining \(\mu_i = \mathcal{L}((x_n, y_n))\), \(\{\mu_n\}\) is a 
  coupling of \(\pi_1\) and \(\pi_2\). By the above lemma, \(\{\mu_n\}\) is tight 
  and so, by Prokhorov's theorem, there exists a weakly convergent subsequence 
  \(\mu_{n_k}\) with limit \(\mu\) which is also a coupling of \(\pi_1\) and \(\pi_2\).
  Thus, as by the above lemma,
  \[\int 1 \wedge d \dd \mu = \lim_{k \to \infty} \int 1 \wedge d \dd \mu_{n_k} = 0,\]
  we have \(\pi_1 = \pi_2\) as required.
\end{proof}

\begin{definition}[Minorisation]
  Let \(\eta \in P(\mathcal{X})\). We say a family of transition probabilities 
  \(P = (P(x, \cdot))\) is minorised by \(\eta\) if there exists some \(a > 0\) 
  such that for all \(x \in \mathcal{X}\),
  \[P(x, \cdot) \ge a \eta.\]
\end{definition}

In the finite state case, minorisation is saying that \(P(i, j) \ge a \eta(j)\) 
for all \(i, j \in \mathcal{X}\). Thus, if we take \(\eta\) to be the vector 
with 1 in the \(j_0\)-th position and 0 everywhere else, \(P\) is minorised by 
\(\eta\) if and only if \(P(i, j_0) \ge a\) for all \(i\).

Before moving on, let us introduce another alternative definition for the total 
variation of measures which will be helpful.

\begin{proposition}
  Let \(\mu, \nu\) be positive measures on \(\Omega\). Let \(\eta\) be 
  a positive measure such that \(\mu \ll \eta\) and \(\nu \ll \eta\). Then, 
  \[\|\mu - \nu\|_{TV} = \int \left|\dv{\mu}{\eta} - \dv{\nu}{\eta}\right| \dd \eta.\]
  We note that such an \(\eta\) always exists by simply taking \(\eta = \mu + \nu\).
\end{proposition}

We note that this formulation is independent of the choice of \(\eta\). 
Indeed,
\[\begin{split}
  \int \left|\dv{\mu}{\eta} - \dv{\nu}{\eta}\right| \dd \eta & = 
  \int \dv{(\mu + \nu)}{\eta} \left|\dv{\mu}{(\mu + \nu)} - \dv{\nu}{(\mu + \nu)}\right| \dd \eta\\
  & = \int \left|\dv{\mu}{(\mu + \nu)} - \dv{\nu}{(\mu + \nu)}\right| \dd (\mu + \nu).
\end{split}\]

\begin{definition}
  Given measures \(\mu, \nu\), we define 
  \[\mu \wedge \nu := \left(\dv{\mu}{\eta} \wedge \dv{\nu}{\eta}\right)\eta\]
  where \(\mu, \nu \ll \eta\). This definition is independent of the choice of \(\eta\).
\end{definition}

\begin{lemma}
  Given measures \(\mu, \nu\), 
  \[\|\mu - \nu\|_{TV} = \mu(\Omega) + \nu(\Omega) - 2 \mu \wedge \nu(\Omega)\]
  which equals \(2(1 - \mu \wedge \nu(\Omega))\) if \(\mu, \nu \in P(\Omega)\).
\end{lemma}

\begin{lemma}
  The space \(P(\mathcal{X})\) is complete under \(\|\cdot\|_{TV}\).
\end{lemma}
\begin{proof}
  Let \((\mu_n)\) be a Cauchy sequence of probability measures and let 
  \[\eta := \sum_{n = 1}^\infty \frac{1}{2^n} \mu_n,\]
  so that \(\mu_n \ll \eta\) for all \(n\).Thus, 
  \[\|\mu_n - \mu_m\|_{TV} = \int \left|\dv{\mu_n}{\eta} - \dv{\mu_m}{\eta}\right| \dd\eta.\]
  So, \((\mu_n)\) is Cauchy if and only if \((\dd \mu_n / \dd \eta)\) is Cauchy 
  in \(L^1\). As \(L^1\) is complete, there exists some \(f \in L^1\) such that 
  \(\dd \mu_n / \dd \eta \to f\) in \(L^1\). So, \(\mu_n \to \mu\) in total variation 
  where \(\mu = f \eta \in P(\mathcal{X})\).
\end{proof}

\begin{lemma}
  Let \(\mu, \nu\) be probability measures on \(\mathcal{X}\). Then, denoting 
  \[\bar \mu := \frac{\mu - \mu \wedge \nu}{\frac{1}{2}\|\mu - \nu\|_{TV}},\]
  and 
  \[\bar \nu := \frac{\nu - \mu \wedge \nu}{\frac{1}{2}\|\mu - \nu\|_{TV}},\]
  \(\bar \mu, \bar \nu\) are probability measures and 
  \[\mu - \nu = \frac{1}{2}\|\mu - \nu\|_{TV} (\bar \mu - \bar \nu).\]
\end{lemma}
\begin{proof}
  Clear.
\end{proof}

\begin{corollary}
  Let \(\mu, \nu\) be probability measures on \(\mathcal{X}\) and \(T\) a transition 
  operator. Then 
  \[\|T\mu - T\nu\|_{TV} = \frac{1}{2}\|\mu - \nu\|_{TV} \|T\bar \mu - T \bar \nu\| 
    \le \|\mu - \nu\|_{TV}.\]
\end{corollary}

\begin{theorem}[Geometric Convergence Theorem]
  Suppose \(P\) is a transition probability on \(\mathcal{X}\) minorised by a 
  probability measure \(\eta\) (i.e. \(P(x, \cdot) \ge a\eta\) for some \(a \in (0, 1)\)).
  Then, \(P\) has a unique invariant probability measure \(\pi\).  

  Furthermore, if \(\mu, \nu \in P(\mathcal{X})\), we have 
  \[\|T^{n + 1}\mu - T^{n + 1}\nu\|_{TV} \le (1 - a)^{n + 1} \|\mu - \nu\|_{TV}.\]
\end{theorem}
\begin{proof}
  If \(m\) is a probability measure on \(\mathcal{X}\), then 
  \[Tm = \int_{\mathcal{X}} P(x, \cdot) m(\dd x) \ge a \eta.\]
  Furthermore, \((Tm - a \eta)(\mathcal{X}) = 1 - a\). So, 
  \[\begin{split}
    \|Tm - T\tilde m\|_{TV} & = \|(Tm - a \eta) - (T\tilde m - a \eta)\|_{TV}\\
    & \le (1 - a) \left\|\frac{Tm - a \eta}{1 - a} - \frac{T\tilde m - a \eta}{1 - a}\right\|_{TV} 
      \le 2(1 - a).
  \end{split}\]
  Hence, for \(\mu, \nu \in P(\mathcal{X})\), using the above lemma
  \[\begin{split}
    \|T\mu - T\nu\|_{TV} = \frac{1}{2}\|\mu - \nu\|_{TV} \|T\bar \mu - T \bar \nu\| 
    \le \frac{1}{2}\|\mu - \nu\|_{TV} 2 (1 - a) = (1 - a)\|\mu - \nu\|_{TV}.
  \end{split}\]
  Thus, by the Banach fixed point theorem, \(T\) has a unique fixed point, 
  namely \(P\) has a unique invariant probability measure.
\end{proof}

\begin{corollary}
  If \(\pi\) is the invariant probability measure for \(T\), 
  \[\|T^n \mu - \pi\|_{TV} \le (1 - a)^n\|\mu - \pi\|_{TV}.\]
\end{corollary}

We note that we may generalise the convergence theorem such that \(P\) has 
a unique invariant probability measure if there exists some \(n_0\), \(a \in (0, 1)\) 
\(\eta \in P(\mathcal{X})\) such that \(P^{n_0}(x, \cdot) \ge a \eta\) by considering 
the the more general Banach fixed point theorem which only require \(T^n\) to be 
a strict contraction for some \(n\).

\subsection{Strong Feller Property}

\begin{definition}[Support]
  Let \(\mu\) be a measure on the separable metric space \(\mathcal{X}\). Then, 
  the support of \(\mu\) is the closed set \(A\) such that \(A\) is the smallest  
  closed set of full-measure, i.e. 
  \[\text{supp}(\mu) := \bigcap_{\substack{\mu(A^c) = 0\\A \text{ closed}}}A.\]
  Alternatively, the support is the set \(A\) such that any open set containing 
  it has positive measure. 
\end{definition}

\begin{theorem}
  If \(\mu, \nu\) are mutually singular probability measures, and is 
  invariant for a transition operator \(T\). Then, if \(T\) has the strong 
  Feller property, 
  \[\text{supp}(\mu) \cap \text{supp}(\nu) = \varnothing.\]
\end{theorem}
\begin{proof}
  As \(\mu \perp \nu\), there exists some measurable \(F \subseteq \mathcal{X}\) 
  such that \(\mu(F) = 1\) and \(\nu(F) = 0\). Then, as \(T\) is strong Feller, 
  \(T\mathbf{1}_{F}(x) = P(x, F) \in [0, 1]\) is continuous. Now, as \(\nu\) 
  is invariant, 
  \[0 = \nu(F) = \int \mathbf{1}_F \dd \nu = \int \mathbf{1}_F \dd T\nu = 
  \int T\mathbf{1}_F \dd \nu.\]
  Since, \(T\mathbf{1}_{F}(x) = P(x, F) \ge 0\), 
  \(\nu(T\mathbf{1}_F^{-1}(\{0\})) = \nu(\{T\mathbf{1}_F = 0\}) = 1\). Similarly, 
  we have \(\mu(T\mathbf{1}_F^{-1}(\{1\})) = 1\). Thus, as 
  \(T\mathbf{1}_F^{-1}(\{0\}), T\mathbf{1}_F^{-1}(\{1\})\) are closed as 
  \(T\mathbf{1}_F\) is continuous, we have 
  \[\text{supp}(\mu) \cap \text{supp}(\nu) \subseteq 
    T\mathbf{1}_F^{-1}(\{1\}) \cap T\mathbf{1}_F^{-1}(\{0\}) = \varnothing.\]
\end{proof}

\begin{proposition}
  Let \(g : \mathbb{R}^n \to \mathbb{R}_+\) be measurable such that 
  \(\int g \dd \lambda = 1\). If \(Tf(x) = \int f(y)g(x - y) \lambda(\dd y) = f * g(x)\),
  then \(T\) has strong Feller property.
\end{proposition}
\begin{proof}
  This follows from the fact \(f : \mathbb{R}^n \to \mathbb{R}\) is bounded measurable 
  and \(g : \mathbb{R}^n \to \mathbb{R}\) is in \(L^1\), then \(f * g\) is a bounded 
  continuous function. 
\end{proof}

\begin{proposition}
  Let \(P : \mathcal{X}^2 \to \mathbb{R}\) be measurable such that 
  \(P(x, \dd y) = P(x, y)\mu\) for some measure \(\mu\) on \(\mathcal{X}\). 
  Then, if either (1) and (2) or (1) and (3) holds, \(P\) has the strong 
  Feller property, where 
  \begin{enumerate}
    \item \(P(\cdot, y)\) is continuous for all \(y\).
    \item for all \(x\), there exists some \(a > 0\) such that 
      \[\sup_{z \in B_a(x)} P(z, \cdot) \in L^1(\mu).\]
    \item for all \(x\), there exists some \(a > 0\) such that 
      \(\{P(z, y) \mid z \in B_a(x)\}\) is uniformly integrable.
  \end{enumerate}
  We note that (2) implies (3).
\end{proposition}

\end{document}
