% Options for packages loaded elsewhere
\PassOptionsToPackage{unicode}{hyperref}
\PassOptionsToPackage{hyphens}{url}
\PassOptionsToPackage{dvipsnames,svgnames*,x11names*}{xcolor}
%
\documentclass[]{article}
\usepackage{lmodern}
\usepackage{amssymb,amsmath}
\usepackage{ifxetex,ifluatex}
\ifnum 0\ifxetex 1\fi\ifluatex 1\fi=0 % if pdftex
  \usepackage[T1]{fontenc}
  \usepackage[utf8]{inputenc}
  \usepackage{textcomp} % provide euro and other symbols
\else % if luatex or xetex
  \usepackage{unicode-math}
  \defaultfontfeatures{Scale=MatchLowercase}
  \defaultfontfeatures[\rmfamily]{Ligatures=TeX,Scale=1}
\fi
% Use upquote if available, for straight quotes in verbatim environments
\IfFileExists{upquote.sty}{\usepackage{upquote}}{}
\IfFileExists{microtype.sty}{% use microtype if available
  \usepackage[]{microtype}
  \UseMicrotypeSet[protrusion]{basicmath} % disable protrusion for tt fonts
}{}
\makeatletter
\@ifundefined{KOMAClassName}{% if non-KOMA class
  \IfFileExists{parskip.sty}{%
    \usepackage{parskip}
  }{% else
    \setlength{\parindent}{0pt}
    \setlength{\parskip}{6pt plus 2pt minus 1pt}}
}{% if KOMA class
  \KOMAoptions{parskip=half}}
\makeatother
\usepackage{xcolor} \pagecolor[RGB]{28,30,38} \color[RGB]{213,216,218}
\IfFileExists{xurl.sty}{\usepackage{xurl}}{} % add URL line breaks if available
\IfFileExists{bookmark.sty}{\usepackage{bookmark}}{\usepackage{hyperref}}
\hypersetup{
  pdftitle={Martingales},
  pdfauthor={Kexing Ying},
  colorlinks=true,
  linkcolor=Maroon,
  filecolor=Maroon,
  citecolor=Blue,
  urlcolor=red,
  pdfcreator={LaTeX via pandoc}}
\urlstyle{same} % disable monospaced font for URLs
\usepackage[margin = 1.5in]{geometry}
\usepackage{graphicx}
\makeatletter
\def\maxwidth{\ifdim\Gin@nat@width>\linewidth\linewidth\else\Gin@nat@width\fi}
\def\maxheight{\ifdim\Gin@nat@height>\textheight\textheight\else\Gin@nat@height\fi}
\makeatother
% Scale images if necessary, so that they will not overflow the page
% margins by default, and it is still possible to overwrite the defaults
% using explicit options in \includegraphics[width, height, ...]{}
\setkeys{Gin}{width=\maxwidth,height=\maxheight,keepaspectratio}
% Set default figure placement to htbp
\makeatletter
\def\fps@figure{htbp}
\makeatother
\setlength{\emergencystretch}{3em} % prevent overfull lines
\providecommand{\tightlist}{%
  \setlength{\itemsep}{0pt}\setlength{\parskip}{0pt}}
\setcounter{secnumdepth}{5}
\usepackage{tikz}
\usepackage{physics}
\usepackage{amsthm}
\usepackage{mathtools}
\usepackage{esint}
\usepackage[ruled,vlined]{algorithm2e}
\theoremstyle{definition}
\newtheorem{theorem}{Theorem}
\newtheorem{definition*}{Definition}
\newtheorem{prop}{Proposition}
\newtheorem{corollary}{Corollary}[theorem]
\newtheorem*{remark}{Remark}
\theoremstyle{definition}
\newtheorem{definition}{Definition}[section]
\newtheorem{lemma}{Lemma}[section]
\newtheorem{proposition}{Proposition}[section]
\newtheorem{example}{Example}[section]
\newcommand{\diag}{\mathop{\mathrm{diag}}}
\newcommand{\Arg}{\mathop{\mathrm{Arg}}}
\newcommand{\hess}{\mathop{\mathrm{Hess}}}

\title{Martingales}
\author{Kexing Ying}
\date{July 24, 2021}

\begin{document}
\maketitle

{
\hypersetup{linkcolor=}
\setcounter{tocdepth}{2}
\tableofcontents
}
\newpage

\section{Discrete Time Martingales}

\begin{definition}[Filtration]
  Let \((X, \mathcal{A}, \mu)\) be a measure space, then a filtration on this 
  measure space is a sequence \((\mathcal{A}_n)_{n \in \mathbb{N}}\) of 
  increasing sub-\(\sigma\)-algebras of \(\mathcal{A}\). If 
  \((X, \mathcal{A}_0, \mu)\) is \(\sigma\)-finite, then we call 
  \((X, \mathcal{A}, (\mathcal{A}_n), \mu)\) a \(\sigma\)-finite filtered 
  measure space.
\end{definition}

\begin{definition}[Natural Filtration]
  Given a sequence of functions \((f_n : X \to Y)\) where \(Y\) is a measurable 
  space, the natural filtration is the measure space 
  \((X, \sigma(f_n \mid n \in \mathbb{N}), \mu)\) equipped with the filtration
  \[(\mathcal{A}_n) = (\sigma(f_1, \cdots, f_n)).\]
\end{definition}

It is clear that the natural filtration is the least filtration such that \(f_n\) 
is \(\mathcal{A}_n\)-measurable (and we call this property \((f_n)\) is adapted 
to the filtration) . As we shall see in the continuous case, one 
may define filtrations more generally with a pre-order index. 

\begin{definition}[Discrete Time Martingale]
  Let \((X, \mathcal{A}, (\mathcal{A}_n), \mu)\) be a \(\sigma\)-finite filtered 
  measure space. Then a sequence of \(\mathcal{A}\)-measurable functions 
  \((f_n)_{n \in \mathbb{N}}\) is a martingale if 
  \(f_n \in \mathcal{L}^1(\mathcal{A}_n)\) and 
  \[\int_A f_{n + 1} \dd \mu = \int_A f_n \dd \mu\]
  for all \(n \in \mathbb{N}\) and \(A \in \mathcal{A}_n\).
\end{definition}

Using probabilistic notation, the above property for martingales is simply 
\(\mathbb{E}[f_{n + 1} \mid \mathcal{A_n}] = f_n\).

\begin{definition}[Sub/Super-martingale]
  The sequence of \(\mathcal{A}\)-measurable functions \((f_n)_{n \in \mathbb{N}}\) 
  is a submartingale if \(f_n \in \mathcal{L}^1(\mathcal{A}_n)\) and 
  \[\int_A f_{n + 1} \dd \mu \ge \int_A f_n \dd \mu\]
  for all \(n \in \mathbb{N}\) and \(A \in \mathcal{A}_n\). Similarly, \((f_n)\) 
  is a supermartingale if \(f_n \in \mathcal{L}^1(\mathcal{A}_n)\) and 
  \[\int_A f_{n + 1} \dd \mu \le \int_A f_n \dd \mu\]
  for all \(n \in \mathbb{N}\) and \(A \in \mathcal{A}_n\).
\end{definition}

It is clear that a martingale is both a submartingale and a supermartingale and 
the negative of a submartingale is a supermartingale (and vice-versa).

\begin{proposition}
  The sequence of \(\mathcal{L}^1\) functions \((f_n)_{n \in \mathbb{N}}\) is a 
  submartingale if and only if 
  \[\int \phi f_{n + 1} \dd \mu \ge \int \phi f_n \dd \mu\]
  for all \(\phi \in \mathcal{L}^\infty(\mathcal{A}_n) \cap \ge 0\). 
  Similar statements hold for martingales and supermartingales by the same proof.
\end{proposition}
\begin{proof}
  The backwards implication is clear by choosing \(\phi = \mathbb{1}_A\) so 
  let us consider the forward direction. By approximation by simple functions, 
  there exists a sequence of simple functions 
  \(\left(s_N = \sum_{i = 0}^N \alpha_i \mathbb{1}_{A_i}\right)_{N \in \mathbb{N}}\) 
  such that \(s_N \uparrow \phi\). Then, as \(\phi \in \mathcal{L}^\infty\), 
  there exists some \(M \in \mathbb{R}\) such that \(M \ge \phi \ge s_N\). 
  Thus, by dominated convergence theorem, 
  \[\begin{split}
    \int \phi f_{n + 1} = \int \lim_{N \to \infty} s_N f_{n + 1} 
    & = \lim_{N \to \infty} \int s_N f_{n + 1} 
      = \lim_{N \to \infty} \sum_{i = 0}^N \int_{A_i} f_{n + 1}\\
    & \ge \lim_{N \to \infty} \sum_{i = 0}^N \int_{A_i} f_n 
      = \lim_{N \to \infty} \int s_N f_n = \int \lim_{N \to \infty} s_N f_n 
      = \int \phi f_n.
    \end{split}\] 
\end{proof}

\begin{proposition}
  If \((f_n)\) and \((g_n)\) are martingales and \(\alpha, \beta \in \mathbb{R}\),
  then \(\alpha f_n + \beta g_n\) is also a martingale. Similar statements 
  hold for submartingales and supermartingales (with consideration to the  
  sign of \(\alpha\) and \(\beta\)).
\end{proposition}

Let us recall some definitions from probability theory.

\begin{definition}[Independence]
  Let \((X, \mathcal{A}, \mu)\) be a probability space. A sequence of 
  functions \((f_n)_{n \in \mathbb{N}} \subseteq \mathcal{L}^1\) is said to be 
  independent if 
  \[\mu\left(\bigcap_{n = 1}^N f_n^{-1}(B_n)\right) = 
    \prod_{n = 1}^N\mu(f_n^{-1}(B_n)),\]
  for all \((B_n)_{n = 1}^N \subseteq \mathcal{B}(\mathbb{R})\).
\end{definition}

\begin{lemma}
  Given a sequence of independent functions \((f_n)_{n = 1}^{N + 1} 
  \subseteq \mathcal{L}^1\), for all \(A \in \sigma(f_1, \cdots, f_N)\),
  \[\int_A f_{N+ 1} \dd \mu = \mu(A) \int f_{N + 1} \dd \mu,\]
  and for all \(\phi \in \mathcal{L}^1(\sigma(f_1, \cdots, f_N))\),  
  \[\int \phi f_{N + 1} \dd\mu = \int \phi \dd\mu \cdot\int f_{N + 1}\dd\mu.\]
  Furthermore, 
  \[\int \prod_{n = 1}^k f_n \dd \mu = \prod_{n = 1}^k \int f_n \dd \mu,\]
  for all \(k = 1, \cdots, N\).
\end{lemma}

\begin{proposition}
  Given a sequence of independent functions \((f_n)_{n \in \mathbb{N}} 
  \subseteq \mathcal{L}^1\), \((s_n) := \left(\sum_{i = 1}^n f_i\right)\) is 
  a submartingale with respect to the natural filtration if and only if 
  \(\int f_n \ge 0\) for all \(n\).
\end{proposition}
\begin{proof}
  The statement follows by the following chain of equalities,
  \[\int_A s_{n + 1} \dd \mu = \int_A s_n + f_{n + 1} \dd \mu = 
    \int_A s_n \dd \mu + \int_A f_{n + 1} \dd \mu = 
    \int_A s_n \dd \mu + \mu(A) \int f_{n + 1} \dd \mu.\] 
\end{proof}

\end{document}
